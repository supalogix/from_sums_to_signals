\section{Integrating Neuro-Symbolic GNNs with Inclusion Maps and Vector Clocks for High-Frequency Market Event Modeling}

This approach combines Graph Neural Networks (GNNs), symbolic temporal logic, and measure-theoretic constructs to model and analyze high-frequency market events. It treats market data---such as trades, price updates, and macroeconomic signals---as events embedded in a measurable event space, using \emph{inclusion maps} to structure their relationships and \emph{vector clocks} to capture partial orderings across distributed agents. The system is designed to address the limitations of purely numerical or purely symbolic systems by integrating the strengths of both paradigms.

\subsection{Modeling Framework}

We define a measurable event space $(\Omega, \mathcal{F})$, where:
\begin{itemize}
    \item $\Omega$: The sample space of all possible market events.
    \item $\mathcal{F}$: A $\sigma$-algebra over $\Omega$, containing measurable subsets such as the set of all trades at time $t$, or all events involving a specific security.
\end{itemize}

Each \emph{agent} (e.g., trader, exchange, algorithm) observes a subset of $\mathcal{F}$, denoted $\mathcal{F}_i \subseteq \mathcal{F}$, and each such subset is connected to others via \emph{inclusion maps} $i_{ij}: \mathcal{F}_i \hookrightarrow \mathcal{F}_j$, forming a presheaf over the category of agents. This structure models partial observability and information flow in the market.

\subsection{Temporal Logic with Vector Clocks}

Let each event $e$ carry a \emph{vector clock} $v(e) \in \mathbb{N}^n$, where $n$ is the number of agents. Define the partial ordering $e \rightarrow e'$ (i.e., ``event $e$ happened before $e'$'') as:
\[
v(e) < v(e') \iff \forall i,\ v(e)_i \leq v(e')_i \ \text{and} \ \exists j \ \text{s.t.} \ v(e)_j < v(e')_j
\]
This captures \emph{causality} in the system---crucial for determining arbitrage paths, information flow, or detecting violations of market fairness.

\subsection{Graph Neural Network Integration}

We define a graph $G_t = (V_t, E_t)$ at time $t$, where:
\begin{itemize}
    \item $V_t$: Vertices represent market agents or securities.
    \item $E_t$: Edges represent causal or informational dependencies between entities, computed via inclusion maps and vector clock comparisons.
\end{itemize}

Each node $v \in V_t$ has a feature vector $x_v$, incorporating:
\begin{itemize}
    \item Observed events (trades, quotes, news),
    \item Historical state (price velocity, momentum, etc.),
    \item Vector clock embeddings (relative to other agents).
\end{itemize}

The GNN message-passing function $\phi$ respects the partial ordering induced by the vector clocks:
\[
h_v^{(k+1)} = \text{UPDATE}^{(k)}\left(h_v^{(k)}, \sum_{\substack{u \in \mathcal{N}_v^{(k)} \\ v(u) < v(v)}} \text{MESSAGE}^{(k)}(h_u^{(k)}, h_v^{(k)}, e_{uv})\right)
\]
This ensures that information from temporally prior (but possibly concurrent) events is correctly propagated, enabling \emph{non-linear temporal reasoning} across the network.

\subsection{Symbolic-Probabilistic Fusion}

Each GNN layer outputs a distribution over measurable subsets of $\mathcal{F}$, representing updated \emph{beliefs} about latent variables like volatility spikes, manipulation risk, or systemic shocks. These subsets can be mapped back to symbolic expressions in first-order temporal logic, e.g.:

\begin{quote}
``There exists a trade sequence from agent A to B such that the price impact exceeds threshold $\delta$ within $\tau$ ms.''
\end{quote}

Thus, predictions are not just pointwise but \emph{measurable set-valued}, and interpretable.

\subsection{Applications}

\begin{itemize}
    \item \textbf{Arbitrage Detection}: Detect cycles in the event graph that violate causality or information asymmetry constraints.
    \item \textbf{Latency Arbitrage Immunization}: Identify vulnerable edges (slow links) and symbolically verify resilience via inclusion-preserving rewrites.
    \item \textbf{Market Manipulation Detection}: Capture spoofing or layering patterns as fixed points or anomalies in vector-clock-ordered subgraphs.
    \item \textbf{Macro-Micro Signal Fusion}: Merge top-down macroeconomic signals (symbolic events) with bottom-up tick-level GNN predictions via conditional inclusion.
\end{itemize}


\subsection{Belief Space Dynamics via Information-Geometric Lagrangians}

Extending beyond classical physics-informed models, we explore a formulation where the system evolves in a \emph{belief space}, guided not by physical laws, but by their formal structure---specifically, Lagrangian and Hamiltonian mechanics on information-geometric manifolds. In this setting, the market is interpreted as a distributed inference system, and each agent’s evolving internal model of the world (e.g., volatility estimates, risk exposure, latent order flow) is treated as a generalized coordinate \( q(t) \) evolving through time.

We define a Lagrangian over this belief manifold as:
\[
L(q, \dot{q}) = \frac{1}{2} \dot{q}^T G(q) \dot{q} - D_{\mathrm{KL}}(q \parallel q_{\mathrm{target}})
\]
where \( G(q) \) is the Fisher information metric and \( D_{\mathrm{KL}} \) penalizes divergence from a target belief distribution (e.g., equilibrium pricing or consensus sentiment). This mirrors the classical \( L = T - V \) structure, with ``kinetic energy’’ interpreted as the \emph{rate of inferential change}, and ``potential energy’’ as the \emph{informational tension} between an agent’s internal model and the observable market consensus.

The belief dynamics are governed by the Euler-Lagrange equation:
\[
\frac{d}{dt} \left( \frac{\partial L}{\partial \dot{q}} \right) - \frac{\partial L}{\partial q} = 0
\]
whose residual can be interpreted as the extent to which an agent's inference violates information-theoretic consistency over time. This residual defines a constraint loss:
\[
\mathcal{L}_{\text{lagrange}} = \sum_{t_i} \left\| \frac{d}{dt} \left( \frac{\partial L}{\partial \dot{q}} \right) - \frac{\partial L}{\partial q} \right\|^2
\]
which can be backpropagated through a neural network approximator \( f_\theta(t, x) \approx q(t) \). The neural architecture effectively learns a trajectory through belief space that respects both temporal consistency and information-geometric structure.

This formulation enables richer control-theoretic interpretations as well. If \( u(t) \) denotes the control input (e.g., trading decision, model adjustment), we can alternatively define:
\[
L(q, u) = \frac{1}{2} u^T R u + D(q \parallel \hat{q})
\]
with \( R \) penalizing effort and \( D \) quantifying divergence from a desired belief state \( \hat{q} \). This reflects a Hamilton-Jacobi-Bellman interpretation of inference under uncertainty: an agent minimizes informational regret and control cost over time.

In this setting, the symbolic-probabilistic fusion is enhanced: measurable subsets of market events are not just passively inferred, but actively \emph{navigated} in a way that balances computational effort, belief evolution, and consistency with macrostructure.

Applications of this Lagrangian-informed architecture include:
\begin{itemize}
    \item \textbf{Trajectory Forecasting in Belief Space}: Learn agent-specific inference paths aligned with latent structure in high-frequency data.
    \item \textbf{Interventional Market Modeling}: Model counterfactuals where control inputs \( u(t) \) simulate regulatory constraints, sentiment shifts, or liquidity injections.
    \item \textbf{Policy Synthesis via Optimal Belief Control}: Derive control policies (e.g., execution strategies) that minimize divergence from desired system behavior while respecting belief dynamics.
\end{itemize}



\subsection{High-Level PINN Architecture for Information-Geometric Belief Dynamics}

To implement the Lagrangian-based belief evolution described above, the Physics-Informed Neural Network (PINN) is structured around modular components that mirror the functional form of the Euler-Lagrange equations. These components work together to encode information-theoretic structure into the network's training dynamics.

\vspace{0.5em}
\noindent\textbf{(1) \texttt{compute\_derivative(q, t)}}

This function estimates the time derivative of the belief vector $q(t)$, either through:

\begin{itemize}
    \item \textbf{Automatic differentiation}: Using \texttt{torch::autograd::grad} or its C++ equivalent to compute $\dot{q}$ directly from the network output.
    \item \textbf{Finite differences}: For structured belief evolution over discrete time grids, a central or forward difference scheme can be applied.
\end{itemize}

\textbf{Purpose:} Captures the rate of inferential change, which serves as the kinetic term in the Lagrangian.

\vspace{0.5em}
\noindent\textbf{(2) \texttt{compute\_lagrangian(q, dq\_dt)}}

This function implements the Lagrangian:
\[
L(q, \dot{q}) = \frac{1}{2} \dot{q}^T G(q) \dot{q} - D(q \parallel q_{\text{target}})
\]

\begin{itemize}
    \item \textbf{Fisher metric term} $\dot{q}^T G(q) \dot{q}$: Represents information-theoretic motion over the manifold. $G(q)$ may be approximated empirically or learned as a differentiable module.
    \item \textbf{Divergence term} $D(q \parallel q_{\text{target}})$: Encourages alignment with an ideal or reference belief distribution (e.g., empirical market prior).
\end{itemize}

\textbf{Purpose:} Encodes belief dynamics into a unified scalar quantity, suitable for variational principles and gradient flow.

\vspace{0.5em}
\noindent\textbf{(3) \texttt{compute\_residual(L, q, dq\_dt, t)}}

This function applies the Euler-Lagrange operator:
\[
\mathcal{R}(t) = \frac{d}{dt} \left( \frac{\partial L}{\partial \dot{q}} \right) - \frac{\partial L}{\partial q}
\]

\begin{itemize}
    \item Uses automatic differentiation to compute both \texttt{dL/dq} and \texttt{dL/ddq}, and their time derivative.
    \item Supports backpropagation through time to enable learning of optimal inferential trajectories.
    \item Can be wrapped into a custom autograd function for better control and interpretability.
\end{itemize}

\textbf{Purpose:} Measures deviation from information-consistent dynamics, forming the constraint loss.

\vspace{0.5em}
\noindent\textbf{(4) \texttt{loss = mean(residual\^{}2)}}

The final scalar loss is computed as:
\[
\mathcal{L}_{\text{lagrange}} = \sum_{t_i} \left\| \mathcal{R}(t_i) \right\|^2
\]

\begin{itemize}
    \item Can be combined with other domain-specific losses (e.g., prediction error, classification loss).
    \item May be weighted by uncertainty or belief entropy to regularize against overconfident inference.
\end{itemize}

\textbf{Purpose:} Guides network training toward belief trajectories that are smooth, consistent, and information-conservative.

\vspace{0.5em}

Together, these components form a symbolic-neural system that enforces structured belief evolution across time. Unlike standard PINNs that enforce physical constraints, this framework imposes informational and inferential constraints, grounded in a variational and control-theoretic formalism. The result is a network that learns both to infer and to evolve in a way that respects its own logic of belief.


\subsection{Class and Function Signatures for Python Implementation}

The following outlines the high-level architecture of a Python implementation for the Lagrangian-based, information-geometric PINN. Each class and function is given as a structural stub, showing how the system might be modularized across components such as belief dynamics, metric learning, divergence calculation, and the training loop.

\begin{lstlisting}[language=Python]
import torch
import torch.nn as nn

# Main PINN model
class InfoGeoPINN(nn.Module):
    def __init__(self, network_config, metric_module, divergence_module):
        super().__init__()
        self.f_net = self.build_network(network_config)
        self.metric = metric_module
        self.divergence = divergence_module

    def build_network(self, config):
        pass

    def forward(self, t, x):
        pass

    def compute_derivative(self, q, t):
        pass

    def compute_lagrangian(self, q, dq_dt):
        pass

    def compute_residual(self, L, q, dq_dt, t):
        pass

    def compute_loss(self, residual):
        pass

# Fisher information metric or Riemannian metric module
class FisherMetric(nn.Module):
    def __init__(self, input_dim):
        super().__init__()
    
    def forward(self, q):
        pass

# Divergence module (e.g., KL divergence)
class Divergence(nn.Module):
    def __init__(self, target_distribution):
        super().__init__()

    def forward(self, q):
        pass

# Utility for time integration or time sampling
def generate_time_grid(t_min, t_max, steps):
    pass

# Loss wrapper for Euler-Lagrange residuals
def euler_lagrange_residual(L, q, dq_dt, t):
    pass

# Training loop
def train(model, dataloader, optimizer, num_epochs):
    pass

# Dataset wrapper for time-dependent inputs
class TimeSeriesDataset(torch.utils.data.Dataset):
    def __init__(self, time_points, features):
        pass

    def __len__(self):
        pass

    def __getitem__(self, idx):
        pass
\end{lstlisting}

This modular layout allows each component---from metric evaluation to divergence estimation and residual computation---to be independently tested, optimized, or swapped with alternate formulations. The structure supports experimentation with both classical and learned metrics, symbolic divergences, and multi-scale time dynamics.


