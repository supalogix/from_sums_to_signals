\section{High-Frequency Trading: Where Math Moves Faster Than Money}

\subsection{Beyond Timestamps: Causal Inference in the Relativistic World of HFT}

High-frequency trading (HFT) is a battlefield of speed, precision, and mathematical sophistication. A single trading machine can execute thousands of trades per second, but an HFT firm operates at an even greater scale—deploying \textbf{thousands or even tens of thousands of machines}, all responding to market changes in milliseconds. 

At this scale, fundamental problems emerge: 

\begin{itemize}
    \item \textbf{How do we determine which trades influenced which?}
    \item \textbf{How do we filter meaningful signals from noise?}
    \item  \textbf{And how do we avoid trading on illusions rather than real market structure?} 
\end{itemize}


Traditional financial models rely on timestamps to reconstruct trade sequences, but this approach fails under HFT conditions for three major reasons:

\begin{itemize}
    \item There is no single global clock—trades occur asynchronously.
    \item Network latency distorts timestamps, making event ordering unreliable.
    \item Market activity is distributed across multiple exchanges and thousands of independent trading machines.
\end{itemize}

If we naively assume that timestamps determine order, we run into paradoxes where a trade appears to happen "before" its own cause—simply due to differences in latency.

To solve this, HFT systems use \textbf{vector clocks}—a tool from distributed computing that, much like \textbf{Einstein’s worldlines in spacetime}, tracks causality rather than absolute time. Instead of treating trades as stationary points in time, we view them as moving along their own trajectories, each with its own velocity in the "market spacetime."

\subsection{Vector Clocks: Tracking the Flow of Market Influence in Spacetime}

In relativity, objects do not merely exist at fixed positions in space—they follow \textbf{worldlines} through spacetime. Their movement through time is governed by the \textbf{four-velocity}, which describes their motion relative to an observer. The four-velocity of an object with proper time \( \tau \) is given by:

\[
U^\mu = \frac{dx^\mu}{d\tau}
\]

where \( x^\mu \) is the four-position in spacetime, and \( \tau \) is the proper time of the object.

Much like moving objects in spacetime, trades in an HFT system are not stationary—they move through market time, affected by delays, dependencies, and network latencies. Two trades that appear simultaneous in one exchange may not be so in another. Just as different observers in relativity disagree on the timing of events but still agree on causal structure, vector clocks ensure that market events respect causality.

\subsection{Vector Clocks as Market Four-Velocities}

If a trade \( A \) influences trade \( B \), then \( A \) should appear before \( B \) in our system, even if the raw timestamps suggest otherwise. The vector clock ensures this by recording a \textbf{dependency structure}, rather than relying on a single flawed timeline.

Mathematically, for each trade \( T_i \), we maintain a vector \( V(T_i) \) where each element counts how many events have occurred in a given market stream. When one trade influences another, its vector is updated accordingly:

\[
V(T_j) = \max(V(T_i), V(T_j)) + 1
\]

This is analogous to how an object’s four-velocity changes as it moves through spacetime—it accounts for \textbf{relative movement} rather than absolute positioning. Just as trains moving at different speeds separate over time in different frames of reference, trades move apart in "market spacetime" due to latency and execution time differences.

\subsection{Financial Relativity: The Moving Trades}

In relativistic mechanics, simultaneity is relative. If two trains are moving at different speeds, an observer on one train sees events in a different order than an observer on the other. Likewise, in HFT, each trading venue has its own clock, and no single timeline can capture the true order of trades.

Vector clocks, like four-velocity, track the \textbf{relative motion} of trades, ensuring that causality is preserved even when timestamps seem contradictory. Instead of assuming a fixed "Newtonian" market clock, we recognize that every trade has its own motion through time—just as every object in relativity has its own frame of reference.

\newpage


\begin{center}
\begin{tikzpicture}[scale=1.2]

    % Define x-coordinates for worldlines
    \newcommand{\Xone}{0}
    \newcommand{\Xtwo}{3}
    \newcommand{\Xthree}{6}
    \newcommand{\Xexchange}{9}

    % Define total height
    \newcommand{\Height}{10}

    % Machine labels at the top
    \node[above] at (\Xone,\Height) {\small Machine 1};
    \node[above] at (\Xtwo,\Height) {\small Machine 2};
    \node[above] at (\Xthree,\Height) {\small Machine 3)};
    \node[above] at (\Xexchange,\Height) {\small Exchange A};

    % Draw vertical worldlines
    \draw[dotted, ->] (\Xone,\Height) -- (\Xone,0);
    \draw[dotted, ->] (\Xtwo,\Height) -- (\Xtwo,0);
    \draw[dotted, ->] (\Xthree,\Height) -- (\Xthree,0);
    \draw[dotted, ->] (\Xexchange,\Height) -- (\Xexchange,0);

    % Define tick mark spacing manually
    \newcommand{\TickSpacing}{0.5}

    % Define 20 tick marks starting from the **top** and going down
    \newcommand{\TickZero}{\Height}  % Top tick
    \newcommand{\TickOne}{\Height - 0.5}
    \newcommand{\TickTwo}{\Height - 1.0}
    \newcommand{\TickThree}{\Height - 1.5}
    \newcommand{\TickFour}{\Height - 2.0}
    \newcommand{\TickFive}{\Height - 2.5}
    \newcommand{\TickSix}{\Height - 3.0}
    \newcommand{\TickSeven}{\Height - 3.5}
    \newcommand{\TickEight}{\Height - 4.0}
    \newcommand{\TickNine}{\Height - 4.5}
    \newcommand{\TickTen}{\Height - 5.0}
    \newcommand{\TickEleven}{\Height - 5.5}
    \newcommand{\TickTwelve}{\Height - 6.0}
    \newcommand{\TickThirteen}{\Height - 6.5}
    \newcommand{\TickFourteen}{\Height - 7.0}
    \newcommand{\TickFifteen}{\Height - 7.5}
    \newcommand{\TickSixteen}{\Height - 8.0}
    \newcommand{\TickSeventeen}{\Height - 8.5}
    \newcommand{\TickEighteen}{\Height - 9.0}
    \newcommand{\TickNineteen}{\Height - 9.5}

    % Add tick marks with named variables
    \foreach \x in {\Xone,\Xtwo,\Xthree,\Xexchange} {
        \foreach \tick in {\TickZero,\TickOne,\TickTwo,\TickThree,\TickFour,\TickFive,
                           \TickSix,\TickSeven,\TickEight,\TickNine,\TickTen,\TickEleven,
                           \TickTwelve,\TickThirteen,\TickFourteen,\TickFifteen,
                           \TickSixteen,\TickSeventeen,\TickEighteen,\TickNineteen} {
            \draw[dotted] (\x-0.2,\tick) -- (\x+0.2,\tick);
        }
    }

    % Define trade event positions using named tick variables
    \newcommand{\YA}{\TickTwo}
    \newcommand{\YB}{\TickSix}
    \newcommand{\YC}{\TickTen}
    \newcommand{\YD}{\TickFourteen}
    \newcommand{\YE}{\TickEighteen}
    
    

    % Trade A
    \filldraw[black] (\Xthree,\TickOne) circle (2pt) node[left=4pt, fill=white, inner sep=5pt] {\parbox{2cm}{\raggedleft \tiny Buy REQ\\ \small $[0,0,1]$}};
    \draw[->, >=triangle 45, shorten >=15pt, shorten <=15pt] (\Xthree,\TickOne) -- (\Xexchange,\TickOne); 
   
    \filldraw[black] (\Xexchange,\TickOne) circle (2pt) node[right=4pt, fill=white, inner sep=5pt] {\parbox{2cm}{\small Trade A }}; 
   
    \draw[->, >=triangle 45, shorten >=15pt, shorten <=15pt] (\Xexchange,\TickOne) -- (\Xthree,\TickTen); 
    \filldraw[black] (\Xthree,\TickTen) circle (2pt) node[left=4pt, fill=white, inner sep=5pt] {\parbox{2cm}{\raggedleft \tiny Buy AWK \\ \small $[0,3,1]$}};
    
    \filldraw[black] (\Xthree,\TickEleven) circle (2pt) node[right=4pt, fill=white, inner sep=5pt] {\parbox{2cm}{\tiny Broadcast \\ \small $[0,3,2]$}}; 
    
    \filldraw[black] (\Xone,\TickFourteen) circle (2pt) node[left=4pt, fill=white, inner sep=5pt] {\parbox{2cm}{\raggedleft \tiny Receive\\ \small $[1,3,3]$}};
    \draw[dashed, ->, >=triangle 45, shorten >=15pt, shorten <=15pt] (\Xthree,\TickEleven) -- (\Xone,\TickFourteen); 
    
    \filldraw[black] (\Xtwo,\TickSixteen) circle (2pt) node[right=4pt, fill=white, inner sep=5pt] {\parbox{2cm}{\tiny Receive\\ \small $[0,1,3,0]$}};
    \draw[dashed, ->, >=triangle 45, shorten >=15pt, shorten <=15pt] (\Xthree,\TickEleven) -- (\Xtwo,\TickSixteen); 
   
    
    
    % Trade B
    \filldraw[black] (\Xtwo,\TickOne) circle (2pt) node[left=4pt, fill=white, inner sep=5pt] {\parbox{2cm}{\raggedleft \tiny Buy REQ\\ \small $[0,1,0]$}};
    \draw[->, >=triangle 45, shorten >=15pt, shorten <=15pt] (\Xtwo,\TickOne) -- (\Xexchange,\TickThree); 
   
    \filldraw[black] (\Xexchange,\TickThree) circle (2pt) node[right=4pt, fill=white, inner sep=5pt] {\parbox{2cm}{\small Trade B }}; 
    
    \draw[->, >=triangle 45, shorten >=15pt, shorten <=15pt] (\Xexchange,\TickThree) -- (\Xtwo,\TickFour); 
    \filldraw[black] (\Xtwo,\TickFour) circle (2pt) node[left=4pt, fill=white, inner sep=5pt] {\parbox{2cm}{\raggedleft \tiny Buy AWK \\ \small $[0,2,0]$}};

    
     \filldraw[black] (\Xtwo,\TickFive) circle (2pt) node[anchor=north west,  fill=white, inner sep=5pt, xshift=6pt, yshift=-6pt] {\parbox{2cm}{\tiny Broadcast \\ \small $[0,3,0]$}}; 
    
    \filldraw[black] (\Xone,\TickNine) circle (2pt) node[left=4pt, fill=white, inner sep=5pt] {\parbox{2cm}{\raggedleft \tiny Receive\\ \small $[2,3,0]$}};
    \draw[dashed, ->, >=triangle 45, shorten >=15pt, shorten <=15pt] (\Xtwo,\TickFive) -- (\Xone,\TickNine); 
    
    \filldraw[black] (\Xthree,\TickFive) circle (2pt) node[right=4pt, fill=white, inner sep=5pt] {\parbox{1cm}{\tiny Receive\\ \small $[0,1,3,0]$}};
    \draw[dashed, ->, >=triangle 45, shorten >=15pt, shorten <=15pt] (\Xtwo,\TickFive) -- (\Xthree,\TickFive); 
    
    
    
    % Trade C
     \filldraw[black] (\Xone,\TickOne) circle (2pt) node[left=4pt, fill=white, inner sep=5pt] {\parbox{2cm}{\raggedleft \tiny Buy REQ\\ \small $[1,0,0]$}};
    \draw[->, >=triangle 45, shorten >=15pt, shorten <=15pt] (\Xone,\TickOne) -- (\Xexchange,\TickFive); 
   
    \filldraw[black] (\Xexchange,\TickFive) circle (2pt) node[right=4pt, fill=white, inner sep=5pt] {\parbox{2cm}{\small Trade C}}; 
    
    \draw[->, >=triangle 45, shorten >=15pt, shorten <=15pt] (\Xexchange,\TickFive) -- (\Xone,\TickEleven); 
    \filldraw[black] (\Xone,\TickEleven) circle (2pt) node[left=4pt, fill=white, inner sep=5pt] {\parbox{2cm}{\raggedleft \tiny Buy AWK \\ \small $[0,0,2,0]$}};
    
     \filldraw[black] (\Xone,\TickTwelve) circle (2pt) node[right=4pt, fill=white, inner sep=5pt] {\parbox{2cm}{\tiny Broadcast \\ \small $[0,0,3,0]$}}; 
    
    \filldraw[black] (\Xtwo,\TickEighteen) circle (2pt) node[right=4pt, fill=white, inner sep=5pt] {\parbox{2cm}{\tiny Receive\\ \small $[1,0,3,0]$}};
    \draw[dashed, ->, >=triangle 45, shorten >=15pt, shorten <=15pt] (\Xone,\TickTwelve) -- (\Xtwo,\TickEighteen); 
    
    \filldraw[black] (\Xthree,\TickSeventeen) circle (2pt) node[right=4pt, fill=white, inner sep=5pt] {\parbox{2cm}{\tiny Receive\\ \small $[1,0,3,0]$}};
    \draw[dashed, ->, >=triangle 45, shorten >=15pt, shorten <=15pt] (\Xone,\TickTwelve) -- (\Xthree,\TickSeventeen); 
    
   % \filldraw[black] (\Xtwo,\TickSixteen) circle (2pt) node[left=4pt, fill=white, inner sep=5pt] {\parbox{2cm}{\raggedleft \tiny Receive\\ \small $[0,1,3,0]$}};
    %\draw[dashed, ->, >=triangle 45, shorten >=15pt, shorten <=15pt] (\Xthree,\TickEleven) -- (\Xtwo,\TickSixteen); 

     

\end{tikzpicture}
\end{center}




\subsection{Mathematical Background: Modeling Causal Direction with the von Mises--Fisher Distribution}

To ground our Keplerian analogy of causal directionality in mathematical tools, we introduce a formal model for representing and analyzing directional influence in distributed systems: the \\textbf{von Mises--Fisher (vMF) distribution}. This section outlines how vector clocks, inclusion maps, and vMF can be unified into a structured mathematical framework.

\subsubsection{Causal Geometry in Distributed Systems}
In a distributed high-frequency trading system:
\begin{itemize}
  \item \textbf{Vector clocks} assign partial temporal orderings to events, capturing their causal timestamps across machines.
  \item \textbf{Inclusion maps} (e.g., $e_1 \subseteq e_2$) encode structural containment: which events include or causally depend on others.
  \item \textbf{Each event} can be embedded as a high-dimensional vector, representing its direction of influence in a causality space.
\end{itemize}

These vectors describe \emph{direction}, not magnitude. We want to compare the \emph{angular similarity} between events: how aligned they are in their propagation across the system.

\subsubsection{The von Mises--Fisher Distribution}
The vMF distribution is the natural analog of the Gaussian for directional data on the unit hypersphere:
\[
  f(x \mid \mu, \kappa) = C_p(\kappa) \exp(\kappa \mu^T x)
\]
Where:
\begin{itemize}
  \item $x \in \mathbb{R}^p$, $\|x\| = 1$ is a data point on the unit hypersphere,
  \item $\mu$ is the mean direction (also a unit vector),
  \item $\kappa$ is the concentration parameter (higher = tighter cluster),
  \item $C_p(\kappa)$ is a normalization constant depending on dimension $p$.
\end{itemize}

\subsubsection{Causal Vector Construction}
To model causality:
\begin{itemize}
  \item Let $VC_i \in \mathbb{R}^N$ be the vector clock for event $i$.
  \item Compute causal gradients: $x_i = VC_i - VC_{\text{parent}(i)}$, where the parent is determined via inclusion maps.
  \item Normalize each $x_i$ to unit length: $x_i := \frac{x_i}{\|x_i\|}$.
\end{itemize}

These unit vectors represent the \textbf{direction of causal propagation}. The collection $\{x_i\}$ can be modeled using a vMF distribution.

\subsubsection{Why vMF?}
This distribution is ideal because:
\begin{itemize}
  \item It respects \textbf{rotational symmetry} and lives on the hypersphere.
  \item It captures \textbf{concentration of directionality} around a mean.
  \item It supports \textbf{anomaly detection}: vectors with low vMF likelihood may represent outliers or novel causal regimes.
\end{itemize}

\subsubsection{Application to Causal Dynamics}
Once the vMF is fit to the set of causal direction vectors $\{x_i\}$:
\begin{itemize}
  \item \textbf{Clusters} of aligned events emerge as concentrated regions on the sphere.
  \item \textbf{Divergence} from the mean direction signals shifts in the causal geometry (e.g., regime changes).
  \item \textbf{Directional flow} becomes quantifiable, interpretable, and comparable over time.
\end{itemize}

\subsubsection{Connecting Back to Kepler}
Just as Kepler’s laws interpreted planetary orbits via geometry and conserved quantities, this approach models causal motion via \emph{directional vectors} and \emph{angular distributions}. The hypersphere becomes our celestial map — and vMF becomes our algebraic telescope.

\subsubsection{Summary Table}
\begin{center}
\begin{tabular}{ll}
\toprule
\textbf{Concept} & \textbf{ML/Causal Modeling Equivalent} \\
\midrule
Vector clock deltas & Direction vectors in $\mathbb{R}^d$ \\
Inclusion maps & Structural encodings of causality \\
Direction of influence & Unit vectors on the hypersphere \\
Modeling orientation & von Mises--Fisher distribution \\
Goal & Cluster, track, or detect changes in causality direction \\
\bottomrule
\end{tabular}
\end{center}




\subsection{Keplerian Causality: An Analogy for Directional Data in Distributed Systems}

Modeling directional data in high-frequency trading systems using tools like vector clocks, inclusion maps, and the von Mises--Fisher (vMF) distribution can feel abstract and ungrounded. But there is a powerful analogy that can help make it intuitive: the celestial geometry of Kepler’s laws.

\subsubsection{The Setting: Causal Orbits in Event Space}

Imagine every event in a distributed trading system as a planet orbiting a central force. The "sun" is not a gravitational body but a shared source of causality: a market signal, a time sync, or a liquidity impulse. Events propagate outward, influencing other machines. Vector clocks give us the temporal location of these events; inclusion maps trace their structural influence.

\subsubsection{Kepler’s First Law: Causal Ellipses}

Kepler said: planets move in ellipses with the sun at one focus. In our analogy, events don’t always follow uniform or symmetric causal paths. Some propagate tightly and quickly (like perihelion), others meander and delay (like aphelion). Inclusion maps define these "orbits" — how far a causal effect extends through the network. Vector clocks help describe their shape.

\subsubsection{Kepler’s Second Law: Equal Causal Areas in Equal Time}

Kepler's second law is about conservation of angular momentum. Planets sweep equal areas in equal time. In the trading system, the rate at which influence propagates may vary, but there's a conserved quantity: the consistency of directional flow. Events propagate faster when closer to the causal core (tight consensus, fast messaging) and slower when farther (latency, fragmentation). 

The directional vectors of causality — gradients of vector clock changes — can be modeled as unit vectors on the hypersphere. The von Mises--Fisher distribution lets us statistically model the orientation of this flow, just as Kepler used geometry to understand motion.

\subsubsection{Kepler’s Third Law: Causal Radius and Influence Duration}

Kepler related the orbital period of a planet to its distance from the sun. Analogously, the "radius" of an event — how many machines it touches, how long its causal shadow persists — is related to its time footprint. High-impact trades have wide, slow orbits. Smaller trades stay tight. This structure defines the temporal geometry of causality.

\subsubsection{Directional Modeling as Angular Momentum}

In Kepler’s world, angular momentum gives you the direction and speed of rotation. In your system, vector clock deltas provide a direction of causal change. By normalizing these vectors and modeling them with the vMF distribution, we focus not on how long an influence is, but \emph{where} it points. 

\begin{quote}
Just as Kepler studied angular velocity and conserved areas, we study causal directionality and conserved influence. The hypersphere becomes our celestial map.
\end{quote}

\subsubsection{Summary Table: From Celestial Mechanics to Distributed Causality}

\begin{center}
\begin{tabular}{ll}
\textbf{Kepler Concept} & \textbf{Causal System Analogy} \\
\toprule
Orbit (ellipse) & Inclusion map: causal envelope of an event \\
Equal areas in equal time & Conserved rate of influence (angular flow) \\
Period-radius relation & Causal span vs. temporal footprint \\
Angular velocity & Vector clock delta (direction of change) \\
Angular momentum $L$ & vMF-modeled directional flow on hypersphere \\
\bottomrule
\end{tabular}
\end{center}

\subsubsection{Conclusion}

The causal geometry of your distributed system echoes the celestial mechanics of Kepler. Instead of gravity, you have causality. Instead of orbits, you have inclusion maps. And instead of planetary motion, you model directional flow with vector clocks and the von Mises--Fisher distribution. What Kepler saw in the heavens, we now trace in the microsecond paths of trading logic.
