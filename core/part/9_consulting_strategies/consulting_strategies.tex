\part{Consulting Strategies: How to Sell Magic to People Who Don’t Know Math}

\section{The Innovation Vortex: How to Drown an Executive in Buzzwords Before They Can Ask a Technical Question}
\begin{quote}
“Disruption, at scale, in the cloud, with blockchain synergies.” \textbf{\small Translation: I Googled “AI trends” and copy-pasted everything into one slide.}
\end{quote}
\ExecutiveChecklist{high}{Surviving the Innovation Vortex}{
  \item Ask for a clear definition of terms—especially “AI,” “synergy,” and “disruption.”
  \item Demand a one-paragraph plain-English explanation of the proposal.
  \item If the answer sounds like startup Mad Libs, pause the meeting.
  \item Test them: ask how the same solution applies to a non-technical industry.
}

\section{The Algorithm That Cried Alpha: Forecasting Everything Except Reality}
\begin{quote}
Build a model that sort-of works, rename it something Greek, and pray no one asks what a hyperparameter is.
\end{quote}
\ExecutiveChecklist{high}{Spotting Overpromised Forecasts}{
  \item Ask: “What happened the last time this model made a prediction?”
  \item Require real-world backtesting results with timeframes and baselines.
  \item Demand error bounds—not just the mean.
  \item Reject anything with Greek names and no source code.
}

\section{Digital Transformation Theater: Where Pilot Projects Go to Die}
\begin{quote}
Build a model that sort-of works, rename it something Greek, and pray no one asks what a hyperparameter is. You don’t need a working model—you need a dashboard that looks expensive.
\end{quote}
\ExecutiveChecklist{medium}{Avoiding Pilot Purgatory}{
  \item Before funding, define what “success” looks like.
  \item Ask what happens after the demo—who owns, maintains, and scales it?
  \item Require a production deployment plan (not just a Jupyter notebook and vibes).
  \item Never confuse a PowerPoint for a product.
}

\section{ROI as a Feeling: Why Consultants Never Quantify Success (and You Shouldn’t Ask)}
\begin{quote}
If they ask for metrics, pivot to “brand impact.” If they insist, say it’s “too early to tell.”
\end{quote}
\ExecutiveChecklist{high}{When ROI Is Just a Feeling}{
  \item Insist on pre-defined ROI metrics—quantitative, not qualitative.
  \item Ask, “If this fails, how will we know?”
  \item Watch for evasive language like “brand uplift” or “strategic alignment.”
  \item Demand post-mortems on past client engagements.
}

\section{The Slide Deck Ouroboros: Selling Strategy That Refers to Strategy That Refers to Strategy}
\begin{quote}
\textit{Strategy} is the product. Implementation is for the poor.
\end{quote}
\ExecutiveChecklist{medium}{Escaping the Slide Deck Ouroboros}{
  \item Ask if this is a meta-strategy (i.e., a pitch for more pitches).
  \item Demand one slide titled: “Here’s What We Will Actually Build.”
  \item If the deliverables are more slides, cancel the contract.
  \item Strategy should end in product, not PowerPoint recursion.
}

\section{Confuse and Conquer: Jargon as a Service (JaaS)}
\begin{quote}
Teach the client just enough ML lingo to feel smart, but not enough to realize you haven’t built anything.
\end{quote}
\ExecutiveChecklist{high}{Jargon as a Service (JaaS)}{
  \item Ask them to explain the proposal to a smart intern.
  \item Bring in a technical reviewer unaffiliated with the consultant.
  \item Flag any phrase containing “synergistic AI fabric” or “neural hypercloud.”
  \item If they name-drop GPT but can’t explain a transformer, it’s theater.
}

\section{Vendor Lock-In, the Long Con: How to Make Dependency Look Like Vision}
\begin{quote}
Replace open source with “partner ecosystem.” Charge extra for integration. Never let them leave.
\end{quote}
\ExecutiveChecklist{high}{Avoiding the Lock-In Trap}{
  \item Ask: “Can we migrate away from this without rebuilding everything?”
  \item Require open standards and APIs.
  \item Audit their contract for integration penalties and proprietary traps.
  \item Get an exit strategy before you sign anything.
}

\section{The AI\texttrademark{} Branding Play: If It Does Math, It’s Now Artificial Intelligence}
\begin{quote}
Linear regression? AI. Logistic regression? Also AI. CountIf in Excel? Close enough—call it “edge computing.”
\end{quote}
\ExecutiveChecklist{medium}{AI Branding Detox}{
  \item Ask: “Would this still work if we removed the word ‘AI’?”
  \item Require a comparison to traditional methods (regression, heuristics, etc.).
  \item Don’t pay for branding—pay for results.
  \item If Excel could do it, don’t call it disruption.
}

\section{The Ghost in the Spreadsheet: Build Something That Sort of Works, Then Blame the Data}
\begin{quote}
When the model fails, blame dirty data. When it succeeds, call it AI magic.
\end{quote}
\ExecutiveChecklist{high}{Preventing the Spreadsheet Ghost}{
  \item Ask what happens when the model fails—who’s accountable?
  \item Demand pre-analysis of the dataset before model promises.
  \item Require a plan for data hygiene, not just a shrug and a dashboard.
  \item If “garbage in, AI out” is the fallback—walk away.
}

\section{The Myth of the One-Click Model: Why You Don’t Need Data Scientists—Until You Do}
\begin{quote}
“No-code AI” tools that generate models with no context or domain knowledge. But hey, it’s got a drag-and-drop UI!
\end{quote}
\ExecutiveChecklist{medium}{The One-Click Mirage}{
  \item Ask who built the model, and who will maintain it post-deployment.
  \item Require transparency on architecture and tuning.
  \item Check if domain expertise was involved—or if it’s just AutoML roulette.
  \item If the product promises “no-code AI,” ask what happens when something breaks.
}

\section{Postmodern Performance Metrics: Measuring Whatever You Already Improved}
\begin{quote}
“We increased user engagement by 300\%!” (Relative to what? Silence.)
\end{quote}
\ExecutiveChecklist{medium}{Debunking Postmodern Metrics}{
  \item Ask: “Improved compared to what baseline?”
  \item Demand before-and-after comparisons with control groups.
  \item Watch for cherry-picked metrics and shifting KPIs mid-project.
  \item If all metrics are success stories, it’s marketing, not analysis.
}

\section{The Infinite Pivot: When One Use Case Fails, Rename It and Pitch a New Vertical}
\begin{quote}
Didn’t work for HR? Call it a finance tool. Didn’t work for finance? It’s healthcare-ready now. Scale = rename.
\end{quote}
\ExecutiveChecklist{medium}{Preventing the Infinite Pivot}{
  \item Ask how many industries this tool has pivoted through.
  \item Require proof of domain-specific customization.
  \item Don’t fund a product looking for a problem.
  \item If it failed in fintech, it probably won’t work in agriculture either.
}

\section{The Black Box as Security Blanket: How Ambiguity Sells Better Than Accuracy}
\begin{quote}
Clients don’t want to know how it works. They want to know it \textit{sounds} like it works.
\end{quote}
\ExecutiveChecklist{high}{Escaping the Black Box Security Blanket}{
  \item Require model explainability reports or interpretability tools.
  \item Ask: “What features drove this prediction?” If they can’t answer, walk.
  \item Avoid tools where “proprietary” = “you’re not allowed to ask.”
  \item Don’t accept “it just works” as a justification.
}

\section{The Eternal Proof of Concept: Why the Model Never Quite Reaches Production}
\begin{quote}
Because once it’s in production, someone might find out it’s just a boosted decision tree from 2015.
\end{quote}
\ExecutiveChecklist{medium}{Ending the Eternal Proof of Concept}{
  \item Ask when the model will be in production—and who’s responsible for that.
  \item Require infrastructure compatibility checks before greenlighting development.
  \item If it’s still in POC after 6 months, shut it down or ship it.
  \item No model is better than a broken model in production.
}

\section{PowerPoint-Driven Development: Engineering by Executive Imagination}
\begin{quote}
Build what's on the slide, not what’s technically feasible. Pivot later. Or never.
\end{quote}
\ExecutiveChecklist{high}{Escaping PowerPoint-Driven Development}{
  \item Ask if anyone with technical background reviewed the slides before building started.
  \item Ensure feasibility assessments are done by engineers, not marketers.
  \item Kill features born in slide decks but unsupported by the tech stack.
  \item If the feature can’t be prototyped, don’t fund it.
}
