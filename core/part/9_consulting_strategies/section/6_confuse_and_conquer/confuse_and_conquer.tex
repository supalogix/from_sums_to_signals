\section{Confuse and Conquer: Jargon as a Service (JaaS)}

\begin{quote}
Teach the client just enough ML lingo to feel smart, but not enough to realize you haven’t built anything.
\end{quote}

  \textbf{Jargon as a Service (JaaS)} isn’t about explaining—it’s about obscuring.
  
  \medskip
  
  \textbf{Law 4} from \textit{The 48 Laws of Power} is the cornerstone of every jargon-fueled pitch:
  \begin{quote}
  ``The more you say, the more common you appear. The less you say, the more intimidating and profound you seem.''
  \end{quote}
  
  \medskip
  
  That’s why consultants sprinkle terms like \textit{``synergistic AI fabric''} and \textit{``neural hypercloud''}—not to clarify, but to create an aura of untouchable expertise.
  
  \medskip
  
  The trick is to say \textbf{just enough} to make the client feel included, but never enough for them to realize that under all the buzzwords is… nothing.
  
  \medskip
  
  If every question is met with more layered terminology—and never a straight answer—you’re not being informed. \\
  You’re being \textbf{out-worded}.
  
  \medskip
  
  \textbf{Remember:} Real engineers simplify. JaaS consultants complicate—because clarity kills the mystique (and the retainer).
  
  

\ExecutiveChecklist{high}{Jargon as a Service (JaaS)}{
  \item Ask them to explain the proposal to a smart intern.
  \item Bring in a technical reviewer unaffiliated with the consultant.
  \item Flag any phrase containing “synergistic AI fabric” or “neural hypercloud.”
  \item If they name-drop GPT but can’t explain a transformer, it’s theater.
}