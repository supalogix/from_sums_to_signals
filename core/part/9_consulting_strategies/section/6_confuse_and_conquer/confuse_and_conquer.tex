\section{Confuse and Conquer: Jargon as a Service (JaaS)}

\begin{quote}
Teach the client just enough ML lingo to feel smart, but not enough to realize you haven’t built anything.
\end{quote}

  \textbf{Jargon as a Service (JaaS)} isn’t about explaining—it’s about obscuring.
  
  \medskip
  
  \textbf{Law 4} from \textit{The 48 Laws of Power} is the cornerstone of every jargon-fueled pitch:
  \begin{quote}
  ``The more you say, the more common you appear. The less you say, the more intimidating and profound you seem.''
  \end{quote}
  
  \medskip
  
  That’s why consultants sprinkle terms like \textit{``synergistic AI fabric''} and \textit{``neural hypercloud''}—not to clarify, but to create an aura of untouchable expertise.
  
  \medskip
  
  The trick is to say \textbf{just enough} to make the client feel included, but never enough for them to realize that under all the buzzwords is… nothing.
  
  \medskip
  
  If every question is met with more layered terminology—and never a straight answer—you’re not being informed. \\
  You’re being \textbf{out-worded}.
  
  \medskip
  
  \textbf{Remember:} Real engineers simplify. JaaS consultants complicate—because clarity kills the mystique (and the retainer).
  
  

\ExecutiveChecklist{high}{Jargon as a Service (JaaS)}{
  \item Ask them to explain the proposal to a smart intern.
  \item Bring in a technical reviewer unaffiliated with the consultant.
  \item Flag any phrase containing “synergistic AI fabric” or “neural hypercloud.”
  \item If they name-drop GPT but can’t explain a transformer, it’s theater.
}


Here's the LaTeX version of the historical sidebar formatted to match your tone and style:

\begin{HistoricalSidebar}{Red Lobster’s Shrimpocalypse: When Jargon Devoured Strategy}
In the early 2020s, Red Lobster’s executive leadership was caught in a whirlpool of buzzwords and bad decisions. Thai Union—its parent company—appointed Paul Kenny, a seafood executive, as interim CEO. Under his tenure, Red Lobster rolled out the infamous “Ultimate Endless Shrimp” promotion as a \textbf{permanent} offering.

The phrase sounded delicious in press releases and pitch decks—“value-forward positioning,” “customer retention via loyalty-based indulgence,” “menu democratization.” But what it meant in practice was that people were eating shrimp faster than the company could afford to serve them.

Behind the scenes, decision-making relied on strategic frameworks and cost models that looked sophisticated—but often obscured basic business logic. Kenny, who had financial ties to Red Lobster’s shrimp supplier (Thai Union itself), also pushed operational changes that slashed staff and restructured supplier deals—reducing resilience while maximizing jargon-heavy justifications like “vertical integration” and “supply-side synergy.”

Executives blamed the data. Then the economy. Then customer “volume anomalies.” But the truth was simpler: they mistook a spreadsheet ghost for a strategy.

\medskip

\textit{Lesson: You can’t AI-wash your way out of bad shrimp economics.}
\end{HistoricalSidebar}