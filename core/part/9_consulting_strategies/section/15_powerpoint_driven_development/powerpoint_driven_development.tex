\section{PowerPoint-Driven Development: Engineering by Executive Imagination}

\begin{quote}
Build what's on the slide, not what’s technically feasible. Pivot later. Or never.
\end{quote}

  \textbf{PowerPoint-Driven Development} doesn’t start with system architecture—it starts with \textit{confidence and a clicker}.
  
  \medskip
  
  \textbf{Law 28} from \textit{The 48 Laws of Power} explains why executive imagination so often overrides technical feasibility:
  \begin{quote}
  ``If you are unsure of a course of action, do not attempt it. Your doubts will become apparent. Boldness makes you seem destined to succeed.''
  \end{quote}
  
  \medskip
  
  That’s why:
  \begin{itemize}
    \item Features appear in slide decks before anyone checks if they’re possible.
    \item Timelines are set by enthusiasm, not engineering estimates.
    \item Words like \textit{``seamless AI integration''} and \textit{``real-time blockchain analytics''} get approved—because they \textbf{sound} visionary.
  \end{itemize}
  
  \medskip
  
  In PDD, it’s not about what can be built—it’s about who can present it with enough swagger to get buy-in. \\
  The pivot (or failure) comes later—quietly.
  
  \medskip
  
  \textbf{Remember:} When bold promises outpace technical reviews, you’re not leading innovation—you’re funding a \textbf{confidence campaign} disguised as product development.
  
  

\ExecutiveChecklist{high}{Escaping PowerPoint-Driven Development}{
  \item Ask if anyone with technical background reviewed the slides before building started.
  \item Ensure feasibility assessments are done by engineers, not marketers.
  \item Kill features born in slide decks but unsupported by the tech stack.
  \item If the feature can’t be prototyped, don’t fund it.
}


\begin{tcolorbox}[colback=blue!5!white, colframe=blue!50!black,
  title={Historical Sidebar: Boeing 737 MAX — When Slides Overruled Safety}]

In the early 2010s, \textbf{Boeing} faced intense competition from Airbus's A320neo. To respond swiftly, CEO \textbf{James McNerney} decided to upgrade the existing 737 model instead of developing a new aircraft. This decision prioritized cost savings and expedited timelines over engineering innovations.

\medskip

\begin{quote}
\textbf{The implied promise:} \textit{We can modernize the 737 quickly and economically without compromising safety.}
\end{quote}

\medskip

This approach led to two tragic crashes, resulting in 346 fatalities, and the grounding of the 737 MAX fleet. Investigations revealed that cost-cutting and schedule pressures compromised safety protocols.

\medskip

\textbf{The Lesson?} Prioritizing executive ambitions and market pressures over engineering expertise can have catastrophic consequences.

\end{tcolorbox}


\subsection{Hypothetical Case Study: BlueVine AI — When Buzzwords Outrank Compliance}

BlueVine AI wasn’t just a fintech platform—it was pitched as “the future of algorithmic credit underwriting.” Its promise? Real-time loan approvals powered by a proprietary “AI-driven, blockchain-secured, omnichannel credit engine,” built using principles from \textbf{SAFe} (Scalable Agile Framework) to “ensure lean, continuous value delivery.”

The original concept was simple: accelerate small business lending decisions. But by the third executive offsite, the slide decks had morphed into something else entirely:  

\begin{quote}
“A full-stack, cloud-native, self-optimizing credit platform integrating predictive AI, federated learning, zero-trust security, and decentralized identity management.”  
\end{quote}

It sounded incredible. It sounded inevitable.  
No one in the room could explain how it worked—but no one wanted to be the first to ask.

The consultant driving the roadmap was a rising star from Apex Strategies, a boutique firm known for blending agile methodologies with “transformational leadership.” Coincidentally, he was also a classmate of the CEO at Wharton. His strategy didn’t focus on engineering constraints—it focused on maintaining “momentum narratives” to sustain investor confidence.

Every engineering review flagged risks:  

\begin{itemize}
    \item The AI models were trained on incomplete, biased datasets.
    \item The blockchain layer created immutable logs that violated customer data correction rights under GDPR.
    \item The “real-time credit decisions” bypassed traditional underwriting controls required by the Equal Credit Opportunity Act.
    \item The “federated learning” implementation lacked differential privacy, exposing user-level data across partners.
\end{itemize}

Each red flag was met with the same reply:

\begin{quote}
“Let’s not get bogged down in compliance. We’ll layer governance in a later sprint.”
\end{quote}

When the CTO raised objections, the CEO waved them off. “We’re innovating faster than regulators can keep up,” he said. “Besides, Alex”—the consultant—“has seen this work at scale. Trust him.”

Meanwhile, every board update recycled the same buzzwords: AI, blockchain, zero-trust, omnichannel, continuous delivery. The roadmap shifted weekly, but the marketing never faltered. Each pivot was framed as “agile responsiveness to market signals.”

By the time the platform launched, it was an architecture of contradictions:

\begin{itemize}
    \item “Omnichannel” meant user data was stitched together from disparate sources without customer consent.
    \item “Blockchain-secured” meant customer loan decisions were immutable even when errors were discovered.
    \item “AI-driven credit” meant opaque model outputs no one could explain—especially to regulators.
\end{itemize}

No external audits had been performed. No stress tests had been conducted. But the PowerPoint animations looked clean. The consultant was charismatic. And the board didn’t want to disrupt momentum.

When the first wave of lawsuits arrived—class actions for discriminatory lending, fines for data privacy violations, inquiries into algorithmic bias—the same executives who had fast-tracked the roadmap couldn’t understand what went wrong.

After all, every status update had been “green.”

\medskip

\textbf{The Lesson?}  
In PowerPoint-Driven Development, it’s not the architecture diagrams that matter—it’s the confidence of the presenter holding the clicker.
