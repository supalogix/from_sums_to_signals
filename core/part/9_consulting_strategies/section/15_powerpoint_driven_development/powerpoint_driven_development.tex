\section{PowerPoint-Driven Development: Engineering by Executive Imagination}

\begin{quote}
Build what's on the slide, not what’s technically feasible. Pivot later. Or never.
\end{quote}

  \textbf{PowerPoint-Driven Development} doesn’t start with system architecture—it starts with \textit{confidence and a clicker}.
  
  \medskip
  
  \textbf{Law 28} from \textit{The 48 Laws of Power} explains why executive imagination so often overrides technical feasibility:
  \begin{quote}
  If you are unsure of a course of action, do not attempt it. Your doubts will become apparent. Boldness makes you 
  seem destined to succeed.
  \end{quote}
  
  \medskip
  
  That’s why:
  \begin{itemize}
    \item Features appear in slide decks before anyone checks if they’re possible.
    \item Timelines are set by enthusiasm, not engineering estimates.
    \item Words like \textit{``seamless AI integration''} and \textit{``real-time blockchain analytics''} get approved—because they \textbf{sound} visionary.
  \end{itemize}
  
  \medskip
  
  In PDD, it’s not about what can be built—it’s about who can present it with enough swagger to get buy-in. \\
  The pivot (or failure) comes later—quietly.
  
  \medskip
  
  \textbf{Remember:} When bold promises outpace technical reviews, you’re not leading innovation—you’re funding a \textbf{confidence campaign} disguised as product development.
  
  

\ExecutiveChecklist{high}{Escaping PowerPoint-Driven Development}{
  \item Ask if anyone with technical background reviewed the slides before building started.
  \item Ensure feasibility assessments are done by engineers, not marketers.
  \item Kill features born in slide decks but unsupported by the tech stack.
  \item If the feature can’t be prototyped, don’t fund it.
}


\begin{HistoricalSidebar}{Boeing 737 MAX — When Slides Overruled Safety}

In the early 2010s, \textbf{Boeing} faced intense competition from Airbus's A320neo. To respond swiftly, CEO \textbf{James McNerney} decided to upgrade the existing 737 model instead of developing a new aircraft. This decision prioritized cost savings and expedited timelines over engineering innovations.

\medskip

\begin{quote}
\textbf{The implied promise:} \textit{We can modernize the 737 quickly and economically without compromising safety.}
\end{quote}

\medskip

This approach led to two tragic crashes, resulting in 346 fatalities, and the grounding of the 737 MAX fleet. Investigations revealed that cost-cutting and schedule pressures compromised safety protocols.

\medskip

\textbf{The Lesson?} Prioritizing executive ambitions and market pressures over engineering expertise can have catastrophic consequences.

\end{HistoricalSidebar}


\subsection{Imaginary Case Study: Heritage Bank \& Trust and ModernEdge Consulting}

It was a brisk Monday morning when Heritage Bank \& Trust’s board assembled in a  
mahogany-paneled room. The marble tellers had never seen such anxiety: retail  
foot traffic had fallen by 20\%, while digital-only challengers racked up new  
accounts and Heritage’s lobby sat half-empty.

\textbf{Robert Santiago (CEO):} “Our lobby is half-empty, and digital challengers  
are stealing 30\% of our customers.”

A tall man in a charcoal suit rose from the head of the table, his polished  
Oxfords tapping out a measured rhythm on the hardwood floor. Santiago’s sleeves  
were immaculately pressed, the crisp cuff of his white shirt peeking from  
beneath a tailored jacket—every detail hinting at the discipline he demanded  
of himself. He clasped his hands behind his back and leaned forward, gaze fixed  
on the empty chairs at the far end of the room as if personally responsible  
for each vacant seat.

When he spoke again, his voice was calm but edged with quiet urgency—every word  
chosen as precisely as the broad stripes on his silk tie. A faint crease formed  
between his steel-gray eyes, betraying the late nights spent poring over  
spreadsheets and customer surveys. Yet his lips curved in a half-smile whenever  
someone dared to suggest a bold new idea, signaling that beneath the banker’s  
reserve lay a restless appetite for innovation.

He ran a hand through closely cropped salt-and-pepper hair and glanced at the  
stock ticker scrolling on the side wall, posture unflinching. In that moment,  
Robert Santiago was equal parts strategist and sentinel—steady of nerve,  
relentless in preparation, and already plotting how to fill those empty chairs.

ModernEdge Consulting’s partner Elena Morales burst in, clicker in hand, and  
skipped the pleasantries. With a single keystroke, she spun up a serverless  
core-banking API across global data centers, auto-scaling to a thousand  
transactions per second, while Heritage’s mainframe groaned through its daily  
batch.

\textbf{Elena Morales (ModernEdge Consultant):} “Imagine launching a new lending  
service in seconds instead of waiting through nightly batches.”

A lithe figure in sharply creased trousers pivoted by the projection  
screen, fingernails tapping the remote’s buttons with precise urgency.  
Elena’s dark-carpet hair was pulled into a high ponytail, each strand  
catching the room’s fluorescent glow like polished wire. She flicked through  
diagrams of microservices and ledger shards with a smooth confidence born of  
hundreds of boardroom battles. As Strategy Consultant at ModernEdge Consulting,  
she spoke of latency and throughput as if reciting lines from memory.

Her glasses caught the light as she scanned the table’s occupants with an  
electric smile that promised both disruption and collaboration. Beneath her  
tranquil exterior lay a mind mapping every contingency—cloud failovers,  
compliance callbacks, and the migration of monolithic jobs to ephemeral  
functions. When she leaned in to underscore her final point, the hum of the  
data centers became not noise, but the living pulse of her vision.  

\textbf{Margaret Liu (CIO):} “Our mainframe has powered this bank for decades.  
It’s stable, secure, and battle-tested.”

A dignified presence in a tailored navy blazer, Liu’s poised composure betrayed  
her decades in banking technology. With an MBA from Wharton and a computer  
science degree from MIT, she rose through Heritage’s ranks by leading three  
major system upgrades with zero downtime. She recalls the ’07 crisis, when she  
orchestrated a seamless failover that kept critical payments flowing under  
extreme market stress. Her hands rest lightly on the conference table, wrists  
adorned with the founder’s watch—an heirloom she wears as both charm and  
challenge. When she narrows her dark-brown eyes, every consultant in the room  
knows she’s ready to defend stability against untested ambition.  

\textbf{Elena Morales (smiling):} “Old systems are costly sandbags. Everyone’s on  
the cloud now, and clinging to mainframes only sinks you deeper. Consider that  
  (A) 30\% of customers are migrating to mobile-first rivals as determined by 
  the Q1 2024 Forrester Global  Banking Report,   
  (B) A forecast by Basel IV  
        compliance briefing of mandatory real-time fraud reporting which is 
        something your  
        nightly batch processes simply weren’t built to handle, and
  (C) fintech startups poaching your top engineers with  
        promises of modern stacks, rapid feature deployment, and equity 
        based on your own exit interviews and HR attrition survey.  
These insights aren’t hypothetical. We pulled these from industry analyses,  
regulatory roadmaps, and your own talent feedback. If you don’t adapt now, you'll
be left explaining why you watched our best customers and engineers walk away.”  

A low murmur rippled through the room as Elena set the clicker down. This wasn’t  
improvisation—it was the culmination of weeks of preparation. ModernEdge’s  
research cell, a dozen analysts armed with financial filings, system logs, and  
HR exit interviews, had dissected Heritage Bank’s every report and compliance  
memo. They poured over latency metrics from your nightly batches, reverse-engineered  
your infrastructure diagrams, and even mapped organizational attrition corridors.

Part of Elena’s remit as Strategy Consultant was to know the bank better than its  
own executives. She carried with her an annotated binder of anticipated  
objections—complete with data-backed rebuttals to every claim of stability,  
security, and sunk cost. While Margaret Liu leaned on institutional memory, Elena  
wielded competitive intelligence. It was an unfair fight: one side armed with  
conviction and a hundred pages of evidence, the other hoping that decades of  
proven uptime would suffice.  

In the hush that followed, the board grasped the unspoken verdict: the mainframe  
was a relic. By the final slide, Robert Santiago leaned forward.

\textbf{Robert Santiago (CEO):} “Alright. Scrap the mainframe. Full cloud techstack  
reboot. Cloud or Bust.”

\subsection{Narrative Aside: The Calm Before the Cloud Storm}

The board leans forward, dazzled by talk of cost cuts and elastic scaling.  
(It almost sounds sensible.) They’re about to swap a battle-tested mainframe  
that hasn’t missed a cycle in decades for a labyrinth of microservices and  
managed services. No one mentions that lift-and-shift can become  
lift-and-shipwreck once missing dependencies surface.

Phil Buckellew, veteran of real mainframe battles, once warned that these  
workloads “are not toys.” But warnings tend to vanish between slides on  
“cloud-native” and the next round of buzzword bingo.

Soon enough, whispers of hybrid-cloud sanity will be drowned out, master data  
will lose its master, and that fateful cutover date will loom like a guillotine.  
But hey, at least everybody’s on the cloud now, right?  

\subsection{Nightmare \#1: Event-Oriented Architecture (a.k.a.\ “Real-Time” Chaos)}

Fast forward twenty-four months. What began as a chaotic, event-driven meltdown  
had cascaded into a series of corporate overhauls:

\begin{itemize}
  \item A bank-wide reorganization carved out a Digital Transformation Office  
        and absorbed FinTech startup AccelePay;  
  \item Leadership churn saw three new CFOs, two COOs, and an expanded Risk  
        \& Compliance Council;  
  \item Cross-functional “agile squads” sprouted—then withered—as “agile”  
        became another checkbox in a growing bureaucracy;  
  \item ModernEdge Consulting was acquired by Titan Financial Group, spawning  
        an inescapable slate of playbooks and vendor certifications.  
\end{itemize}

The ops war room, once lit by alarm beacons, now flickered under the glow of  
murderously dense process documents. Alice scrolled through her fifth 
“Change Advisory Board” dashboard of the  
day. Her squint was trained on compliance templates rather than error logs.

As a seasoned cloud architect, Alice was hired almost immediately after the  
decision to transition to the cloud—recruited from Amazon Web Services for her  
hands-on experience building event-driven systems at scale. She’d spent the  
last five years designing data pipelines that processed billions of events per  
day, optimizing schemas on Avro and Kafka across multiple regions.

\textbf{Alice (Lead Dev):} “I just deployed the new event schema. The JSON to  
Avro conversion is live.”

For the past twenty-four months, the engineering organization has been locked  
into a massive format migration—rewriting every message payload to Avro and  
backing it with Kafka topics. The project kicked off as a “one-year sprint,”  
a timeline set not by technical reality but by the CEO’s impatience and a  
management culture too timid to push back. What was supposed to be twelve  
months of schema design and topic roll-outs stretched into two years of  
version compatibility hacks, failed deployments, and an endless parade of  
“urgent” migration tickets.  

A broad-shouldered presence with a weary edge, Bob had joined the bank at the  
same time as Alice—lured away from Facebook by promises of true work–life balance  
(that never materialized) and equity packages touted to match his Silicon Valley  
compensation. He’d moved his wife and three young children across the country,  
convinced that “easier company” meant more time at home. Instead, three rounds  
of reorganization landed him on the Data Engineering team, doing the identical  
schema migrations he’d mastered at Facebook—only now sandwiched between twice  
as many approval layers and committee sign-offs.

\textbf{Bob (Data Engineer):} “Which schema version? I merged v1.2 yesterday,  
but Jenkins built v1.1.”

Continuous Integration (CI) and Continuous Delivery (CD) are simply ways to make 
sure software changes flow smoothly from a developer’s laptop into the live 
system—without surprises.

Imagine a bakery where every new recipe must be tested, baked, and taste-approved 
before it reaches customers. In CI, every time Bob or Alice “drops” a new 
recipe (their code changes) into the shared kitchen, an automated oven 
(the build server) tests that recipe on a dummy batch. If the cookies burn 
or the loaf collapses (a failing test), the oven immediately flags the 
mistake so the baker can fix it.

CD takes it one step further: once the recipes pass those initial tests, 
they’re automatically packaged and delivered to the bakery’s display case 
(the production environment), ready for sale at any moment. Bakers don’t 
have to manually box up each loaf; instead, the packaging line runs 
continuously, ensuring fresh bread appears on the shelves without delay.

In Bob’s case, he merged schema v1.2 into the shared “recipe book,” but 
the CI oven mistakenly pulled the old v1.1 flour mix. Maybe the oven’s 
cache wasn’t cleared, or the packaging conveyor was still using 
last week’s labels. The result: Jenkins baked the wrong batch. In real 
life, this can happen when build servers reuse old artifacts or when 
multiple “build lines” aren’t synchronized—so the fix is to ensure 
each merge triggers a clean build and that all pipelines are aligned 
before delivery.  

Logs scrolled by as Service Gamma retried every batch of “real-time” events  
in thirty-minute intervals. Downstream, Service Delta threw parse errors in  
bright red:

\textbf{Service Delta (automated alert):} “ERROR: Unknown field ‘customerTier’  
in Avro message.”

Meanwhile, in a corner of the corporate Slack workspace, Carla and Dev Patel—now on  
separate teams after the latest reorg—argued about the messaging format. Institutional  
knowledge was splintered across ever-shifting org charts, and because the schema  
lived in an external registry rather than in the codebase, “ghost events” began  
haunting their Kafka topics: messages conforming to outdated or unpublished schemas  
that no service recognized.

\textbf{Carla (Backend Team):} “We agreed on JSON—Avro adds too much overhead.”

\textbf{Dev Patel (Platform Team):} “Avro guarantees schema registry validation!”

Yet beneath their bickering lay a deeper problem: with development and platform now  
siloed, no single team owned the end-to-end event format, and the missing link between  
schema and source code meant every new reorg risked erasing someone’s tribal knowledge.  

By lunchtime, tracing a single transaction meant hopping through eight services,  
four logging systems, and a six-month-old Git branch. The team froze, eyes on  
the blinking red pipeline:

\textbf{Alice (muttering):} “I miss old mainframe...”


\subsection{Nightmare \#2: Data Modeling From Scratch (a.k.a.\ "Let's Redo 30 Years of Work in Six Months")}

The Monday after the event chaos, the whiteboard room was packed with  
diagrams and caffeine-stained coffee cups. Sarah from Business Intelligence  
pounded her fist on the table:

\medskip

\textbf{Sarah (BI Lead):} "We need a schema that evolves dynamically with  
our business. No rigid tables allowed."

\medskip

Tom, the senior DBA, cleared his throat:

\medskip

\textbf{Tom (DBA):} "A normalized model with versioned migrations could  
ensure data integrity and performance..."

\medskip

But nobody listened. Instead, Luke, the solutions architect, waved a data lake  
infocard:

\medskip

\textbf{Luke (Consultant):} "Data Lake plus raw JSON means no more schema  
bottlenecks. Trust me, it's best practice."

\medskip

Over the next two days, thirty-five meetings debated whether customer\_id  
should be a UUID or an integer. Each session ended with a new slide deck  
and no decision.

\medskip

Late on Friday, CFO Jane wandered in:

\medskip

\textbf{Jane (CFO):} "I run our P\&L in spreadsheets. Are you telling me all  
this modeling even matters?"

\medskip

As blank stares filled the room, the new schema lay tangled across ten  
whiteboards—each drawn with a different colored marker. Alice exhaled quietly:

\medskip

\textbf{Alice (Lead Dev):} "I miss our reliable mainframe more than ever..."

\subsection{Nightmare \#3: Project Managing This Disaster to Death}

Chad, Director of Agile Transformation, swaggered into the war room,  
tablet in hand, and declared:

\medskip

\textbf{Chad (Director of Agile Transformation):} “From now on, everything  
—every feature, bug, and idea—must live in Jira. No exceptions.”

\medskip

By week two, the boards had multiplied:

\medskip

\textbf{Emily (Marketing):} “We’ve got five status columns: Backlog, To Do,  
In Review, Approved, Promoted.”

\medskip

\textbf{Ravi (Engineering):} “We need 27 columns to track Dev, Test, QA,  
Staging, Pre-Prod… and ‘Code Archaeology.’”

\medskip

\textbf{Felicia (Finance):} “Our Jira board looks like this now,” (points to  
projected spreadsheet)—“so can we just stick to Excel?”

\medskip

Three months in, sprint planning meetings stretched all morning. The backlog  
swelled to 5,000 tickets—half were duplicates, a quarter had no description,  
and a shocking 200 still referred to mainframe jobs.

\medskip

\textbf{Dev (Senior Engineer):} “I’m DM’ing you on Slack because I can’t bear  
editing tickets anymore.”

\medskip

At the six-month mark, Chad admitted defeat and called in reinforcements:

\medskip

\textbf{Chad (sheepish):} “We need an Agile coach. Someone to optimize our  
workflow.”

\medskip

Enter Alex, the SAFe consultant, certified and bright-eyed:

\medskip

\textbf{Alex (SAFe Consultant):} “Welcome to PI Planning! We’ll replace sprints  
with Program Increments, introduce ART syncs, and align teams with lean  
portfolio management.”

\medskip

Two days later, everyone was exhausted, velocity charts still flat, and the  
backlog had sprouted new “epics” with no owners. The finance team stared blankly:

\medskip

\textbf{Jane (CFO):} “So, what exactly did PI planning fix?”

\medskip

No one answered. Chad glanced at the mainframe console in the corner—still  
humming, still untouched.

\medskip

\textbf{Chad (muttering):} “Old is bad… unless it actually works.”

\subsection{Nightmare \#4: The Add-On Feature That Nobody Asked For (But Someone’s Bonus Depends on It)}

It was Wednesday of week eight when Ethan stormed into the steering committee  
meeting, phone in one hand and buzzword bingo card in the other.

\medskip

\textbf{Ethan (VP of Innovation \& Buzzwords):} “I just read a killer LinkedIn  
post—blockchain payments are the future. We need our own system, built on-chain!”

\medskip

No one spoke. Ethan’s grin faded only when the CEO, Robert Santiago, leaned in.

\medskip

\textbf{Robert (CEO):} “Can’t we just plug in Apple Pay or Stripe?”

\medskip

\textbf{Ethan:} “Because… uh… Web3! Decentralization! Token economies!  
It’ll be huge.”

\medskip

And so the team dutifully scaffolded smart contracts and launched a private  
ledger—only to discover:

\medskip

\textbf{Emma (Payments Engineer):} “Chargebacks? The chain can’t reverse entries.”

\medskip

\textbf{Jamal (QA Lead):} “Our TPS is under 10. Real credit card networks move  
thousands per second.”

\medskip

\textbf{Sara (Compliance):} “Regulators need audit trails and dispute flows.  
This… doesn’t comply.”

\medskip

Panic rippled back to Ethan’s office. He’d already promised the board a demo.  
The solution? Pivot on the fly.

\medskip

\textbf{Ethan (hovering over whiteboard):} “It’s no longer a blockchain system—it’s  
a credit card gateway now.”

\medskip

A week later, the “new” payment API failed end-to-end tests. Transactions  
timed out. Funds vanished into exceptions. Users bailed.

\medskip

And Ethan? He still walked away with his bonus—and a promotion—because in  
corporate America, executing on buzzwords is more important than shipping  
working software.  

\subsection{Nightmare \#5: The Multi-Cloud Mirage (a.k.a.\ “Oops, Turns Out We Still Need the Mainframe”)}

Two weeks into the credit-card pivot, CIO Margaret Liu strode into the war room,  
handing out branded multi-cloud badges.

\medskip

\textbf{Margaret (CIO):} “One cloud is a single point of failure. We need AWS for  
databases, GCP for compute, Azure for payments—and why not sprinkle in Oracle  
Cloud for data analytics?”

\medskip

\textbf{Robert (CEO):} “Isn’t that… a lot of clouds?”

\medskip

\textbf{Margaret:} “Vendor independence is our new mantra. Plus, it looks great in slides.”

\medskip

By Wednesday, the team flitted between consoles:

\medskip

\textbf{Emma (Payments):} “Our Azure endpoint is timing out—no fallback to AWS.”

\medskip

\textbf{Jamal (DevOps):} “Cross-cloud replication lag just spiked to 45 seconds.  
Transactions are appearing out of order.”

\medskip

\textbf{Sara (Security):} “I have three separate IAM policies in flight. One typo  
and our production data is public in two regions.”

\medskip

Meanwhile, Finance discovered their “cloud cost forecast” looked like a rocket  
launch graph—easier to read than our budget, but infinitely more terrifying.

\medskip

\textbf{Alice (Lead Dev, whispering):} “I miss the mainframe’s hum… at least it  
was predictable.”


\subsection{Nightmare \#6: The Hybrid Cloud Ruse (a.k.a.\ “Oops, Turns Out We Still Need the Mainframe”)}

After months of burning through budget and caffeine, the team gathered at the  
monitor again.

\medskip

\textbf{Alice (Lead Dev):} “The cloud system can’t file shareholder reports—  
it still pulls numbers from the mainframe.”

\medskip

\textbf{Bob (Data Engineer):} “So every transaction we process in AWS, GCP,  
and Azure now syncs back to that same mainframe?”

\medskip

\textbf{Emma (Payments):} “Yep. And when reconciliation jobs fail, someone  
has to patch the CSV by hand and re-upload. Then we rerun the sync.”

\medskip

On the call with Finance, Jane’s voice cracked:

\medskip

\textbf{Jane (CFO):} “Our SEC filings must match. The cloud data is off by  
thousands of dollars.”

\medskip

\textbf{Margaret (CIO, smoothing over):} “That’s why we’re adopting a hybrid  
cloud strategy—best of both worlds.”

\medskip

\textbf{Robert (CEO):} “Hybrid cloud? So the mainframe stays?”

\medskip

\textbf{Margaret:} “Absolutely. We’ll call it ‘operational resilience.’”

\medskip

And just like that, the mainframe lived on—buried deeper under cross-cloud  
scripts and manual patches—while the execs touted “hybrid innovation” in  
their next investor deck.  

\subsection{Nightmare \#7: The CFO Discovers “Unnecessary” Costs}

Jane, the CFO whose soul is forged of Excel rows, leaned back and muttered  
to herself as she scanned the cloud cost breakdown.

\medskip

\textbf{Jane (CFO):} “Why are we running workloads on four clouds? This is  
more expensive than our old mainframe.”

\medskip

Her eyes locked on one line item: “Staging Environment.”

\medskip

\textbf{Jane (pointing at spreadsheet):} “Wait. What’s this staging  
environment charge?”

\medskip

David, the CTO, already dead inside, sighed.

\medskip

\textbf{David (CTO):} “That’s where we test deployments before they hit  
production.”

\medskip

Jane’s expression hardened.

\medskip

\textbf{Jane (exasperated):} “We’re paying for tests? Tests that should  
already pass? I can’t afford this. Tell your engineers to write perfect  
code—no staging, no excuses.”

\medskip

David nodded, defeated.

\medskip

\textbf{David (softly):} “Okay.”

\medskip

Within days, the staging environment vanished. Engineers now pushed  
straight to production.

\medskip

\textbf{Alice (Lead Dev, whispering):} “Remember the blockchain payment  
system that never worked? We never tested it.”

\medskip

And so the grand cost-cutting initiative hit its mark:

\begin{itemize}
  \item Less spending!
  \item Fewer environments!
  \item More customer-facing outages!
\end{itemize}

\noindent Ironically, this spawned new costs:

\begin{itemize}
  \item Payments failing in prod.
  \item Engineers racing to debug in live systems.
  \item Consultants summoned to “fix” it by adding a fifth cloud.
\end{itemize}

Through it all, Jane stared at her spreadsheet—still believing the numbers  
proved they were saving money.  

\subsection{Nightmare \#8: The Ingenious Tax Hack That Never Happened}

\medskip

\textbf{Kyle (Cloud Engineer):} “Look, I know how we can restore our staging  
environment—and save millions. We set up ‘Heritage Cloud Services Ltd.’ in  
a low-tax jurisdiction. That unit bills us for cloud usage, and it’s a  
deductible expense.”

\medskip

\textbf{Maria (COO):} “A shadow corporation?”

\medskip

\textbf{Kyle:} “Exactly. It’s transfer pricing—you charge between related  
entities at arm’s length. AWS, Twitch, Amazon.com… they all do it. It’s  
basic financial engineering.”

\medskip

\textbf{Robert (CEO, leaning in):} “So we’d essentially own our own cloud,  
expense it, and reduce our taxable income?”

\medskip

\textbf{Kyle (shrugging):} “That’s the idea. No staging costs in the P\&L—just  
internal chargebacks.”

\medskip

The executives stared, jaws slack.

\medskip

\textbf{Margaret (CIO, quietly):} “We could have avoided this entire multi-
cloud nightmare…”

\medskip

For a fleeting moment, a dangerous spark of realization ignited: they could  
have written off their own cloud, kept staging, and never touched the  
mainframe. Then the moment passed—buried under the next buzzword-laden  
slide deck.  

\subsection{Nightmare \#9: Investor Relations Spin Cycle}

Jane, the CFO, storms into the IR war room—where glossy decks live and breathe.

\medskip

\textbf{Jane (CFO):} “Mark, what do we tell investors about our migration?”

\medskip

\textbf{Mark (Head of IR):} “Uh… we highlight our ‘multi-cloud resilience’ story.  
Focus on agility and cost savings.”

\medskip

\textbf{Kyle (Cloud Engineer, whispering):} “We could’ve expensed our own cloud  
and kept staging…”

\medskip

Mark shoots Kyle a glare: no one admits the staging hack in front of investors.

\medskip

\textbf{Mark (clearing throat):} “We’ll emphasize our seamless hybrid strategy  
and market differentiation. Investors love buzzwords.”

\medskip

\textbf{Robert (CEO, nodding):} “Perfect. Emphasize innovation, ignore the rest.”

\medskip

In the next earnings call, the slide reads:  
\emph{“Heritage Bank \& Trust delivers unmatched hybrid-cloud agility and  
operational excellence.”}

\medskip

Investors applaud. Share price ticks up. No one mentions the mainframe still  
processing their SEC filings at 3 AM.

\medskip

And Kyle? He quietly updates his resume—just in case “operational excellence”  
turns into “another thrilling tech saga.”  


\subsection{Epilogue: Kyle’s Redemption and Déjà Vu with ModernEdge}

Kyle had landed at Meridian Insurance Co., an enterprise whose 40-year-old  
mainframe still crunched policy renewals overnight without a hiccup. At his  
new desk, he found his old Heritage colleagues—Alice, Bob, and Emma—scattered  
across the org chart, each nursing a twinge of trauma from the Great Rewrite.

\medskip

\textbf{Kyle (smiling wryly):} “You know, over 60\% of Fortune 100 firms still  
trust IBM mainframes. They clock 99.99999\% uptime—just 3 seconds of downtime  
per year.”

\medskip

Alice glanced up from her terminal:

\medskip

\textbf{Alice (curious):} “Really? Our nightly batch here takes six hours,  
not three days. And security’s rock solid.”

\medskip

Bob chimed in, tapping a stack of spreadsheets:

\medskip

\textbf{Bob (grinning):} “Forty-five of the top 50 banks still run on  
mainframes. Once you measure total cost of ownership, the distributed mess  
is usually pricier.”

\medskip

Just as Kyle began explaining incremental modernization, the familiar click  
of heels echoed in the hallway. ModernEdge’s partner, Elena Morales, swept in,  
tablet aloft.

\medskip

\textbf{Elena (brightly):} “Kyle! So glad you joined Meridian. Let’s talk cloud  
migration—microservices, event-driven pipelines, the whole nine yards.”

\medskip

Emma rolled her eyes:

\medskip

\textbf{Emma (under her breath):} “Here we go again…”

\medskip

Kyle stood, arms folded:

\medskip

\textbf{Kyle (calmly):} “Before we rewrite mission-critical workloads, remember  
this: if it works, don’t break it. Airlines, telecoms, banks—they didn’t ditch  
mainframes in a frenzy. They modernized around them.”

\medskip

Elena hesitated, cursor hovering over “Migrate Now” slide. Behind her, the  
screen showed stats: 8 of 10 telcos, 4 of 5 airlines, and Meridian’s own  
99.99999 % uptime on System z.

\medskip

For a beat, the boardroom held its breath. Then Elena cleared her throat:

\medskip

\textbf{Elena (sheepish smile):} “Maybe… a hybrid approach?”

\medskip

Kyle nodded:

\medskip

\textbf{Kyle:} “Exactly. Keep what works, wrap new services around it, and  
deliver value without burning the house down.”

\medskip

And with that, Kyle and his colleagues sidestepped another rewrite disaster—  
armed with data, hard-won lessons, and the simple truth: \emph{If it ain’t broke,  
don’t rewrite it.}  


