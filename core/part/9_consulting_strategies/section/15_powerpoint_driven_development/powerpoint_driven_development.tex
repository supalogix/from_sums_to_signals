\section{PowerPoint-Driven Development: Engineering by Executive Imagination}

\begin{quote}
Build what's on the slide, not what’s technically feasible. Pivot later. Or never.
\end{quote}

  \textbf{PowerPoint-Driven Development} doesn’t start with system architecture—it starts with \textit{confidence and a clicker}.
  
  \medskip
  
  \textbf{Law 28} from \textit{The 48 Laws of Power} explains why executive imagination so often overrides technical feasibility:
  \begin{quote}
  ``If you are unsure of a course of action, do not attempt it. Your doubts will become apparent. Boldness makes you seem destined to succeed.''
  \end{quote}
  
  \medskip
  
  That’s why:
  \begin{itemize}
    \item Features appear in slide decks before anyone checks if they’re possible.
    \item Timelines are set by enthusiasm, not engineering estimates.
    \item Words like \textit{``seamless AI integration''} and \textit{``real-time blockchain analytics''} get approved—because they \textbf{sound} visionary.
  \end{itemize}
  
  \medskip
  
  In PDD, it’s not about what can be built—it’s about who can present it with enough swagger to get buy-in. \\
  The pivot (or failure) comes later—quietly.
  
  \medskip
  
  \textbf{Remember:} When bold promises outpace technical reviews, you’re not leading innovation—you’re funding a \textbf{confidence campaign} disguised as product development.
  
  

\ExecutiveChecklist{high}{Escaping PowerPoint-Driven Development}{
  \item Ask if anyone with technical background reviewed the slides before building started.
  \item Ensure feasibility assessments are done by engineers, not marketers.
  \item Kill features born in slide decks but unsupported by the tech stack.
  \item If the feature can’t be prototyped, don’t fund it.
}


\begin{tcolorbox}[colback=blue!5!white, colframe=blue!50!black,
  title={Historical Sidebar: Boeing 737 MAX — When Slides Overruled Safety}]

In the early 2010s, \textbf{Boeing} faced intense competition from Airbus's A320neo. To respond swiftly, CEO \textbf{James McNerney} decided to upgrade the existing 737 model instead of developing a new aircraft. This decision prioritized cost savings and expedited timelines over engineering innovations.


\medskip

\begin{quote}
\textbf{The implied promise:} \textit{We can modernize the 737 quickly and economically without compromising safety.}
\end{quote}

\medskip

This approach led to two tragic crashes, resulting in 346 fatalities, and the grounding of the 737 MAX fleet. Investigations revealed that cost-cutting and schedule pressures compromised safety protocols.

\medskip

\textbf{The Lesson?} Prioritizing executive ambitions and market pressures over engineering expertise can have catastrophic consequences.

\end{tcolorbox}
