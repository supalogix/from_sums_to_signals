\section{The Infinite Pivot: When One Use Case Fails, Rename It and Pitch a New Vertical}

\begin{quote}
Didn’t work for HR? Call it a finance tool. Didn’t work for finance? It’s healthcare-ready now. Scale = rename.
\end{quote}

  \textbf{The Infinite Pivot} isn’t about product-market fit—it’s about \textit{identity management}.
  
  \medskip
  
  \textbf{Law 25} from \textit{The 48 Laws of Power} captures why some products seem to ``scale across industries'' overnight:
  \begin{quote}
  ``Do not accept the roles that society foists on you. Recreate yourself by forging a new identity—one that commands attention and secures power.''
  \end{quote}
  
  \medskip
  
  When a tool fails in one sector, smart consultants don’t call it failure—they \textbf{rebrand}:
  
  \begin{itemize}
    \item HR software becomes a ``financial compliance platform.''
    \item A failed fintech app is suddenly ``healthcare-ready.''
    \item The same dashboard now solves ``logistics challenges at scale.''
  \end{itemize}
  
  \medskip
  
  This isn’t adaptability—it’s a \textbf{chameleon strategy}. \\
  The product doesn’t evolve; only the pitch does.
  
  \medskip
  
  \textbf{Remember:} True scalability comes from solving core problems across domains—not from slapping a new label on last quarter’s unsellable prototype.
  
  \medskip
  
  If a company’s greatest strength is its ability to rename itself, you’re not investing in technology—you’re funding a game of \textbf{buzzword musical chairs}.
  


\ExecutiveChecklist{medium}{Preventing the Infinite Pivot}{
  \item Ask how many industries this tool has pivoted through.
  \item Require proof of domain-specific customization.
  \item Don’t fund a product looking for a problem.
  \item If it failed in fintech, it probably won’t work in agriculture either.
}