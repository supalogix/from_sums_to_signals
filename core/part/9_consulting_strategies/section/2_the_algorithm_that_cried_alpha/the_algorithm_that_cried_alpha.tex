\section{The Algorithm That Cried Alpha: Forecasting Everything Except Reality}


\begin{quote}
Build a model that sort-of works, rename it something Greek, and pray no one asks what a hyperparameter is.
\end{quote}


  In the world of overpromised forecasts, \textbf{attention is currency}.
  
  \medskip
  
  A mediocre model wrapped in a Greek letter—\textit{AlphaPredict\texttrademark}, anyone?—isn’t about accuracy. It’s about making sure executives remember the name.
  
  \medskip
  
  \textbf{Law 6} from \textit{The 48 Laws of Power} explains this perfectly:
  \begin{quote}
  ``Court attention at all costs. The more mysterious and complex you appear, the more people assume value.''
  \end{quote}
  
  \medskip
  
  That’s why these models rarely come with source code or error bounds—\textit{mystique} sells better than math.
  
  \medskip
  
  If the pitch focuses more on branding than backtesting, you're not being offered a solution—you’re being dazzled into forgetting to ask if it works.
  
  


\ExecutiveChecklist{high}{Spotting Overpromised Forecasts}{
  \item Ask: “What happened the last time this model made a prediction?”
  \item Require real-world backtesting results with timeframes and baselines.
  \item Demand error bounds—not just the mean.
  \item Reject anything with Greek names and no source code.
}