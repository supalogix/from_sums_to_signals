\section{The Eternal Proof of Concept: Why the Model Never Quite Reaches Production}

\begin{quote}
Because once it’s in production, someone might find out it’s just a boosted decision tree from 2015.
\end{quote}

  \textbf{The Eternal Proof of Concept} isn’t a stalled project—it’s a \textit{timing strategy}.
  
  \medskip
  
  \textbf{Law 35} from \textit{The 48 Laws of Power} reveals why some models never leave the sandbox:
  \begin{quote}
  ``Never seem to be in a hurry. Waiting allows you to control the narrative and choose the perfect moment—if that moment ever needs to arrive.''
  \end{quote}
  
  \medskip
  
  Consultants and internal teams know that once a model hits production, reality starts asking hard questions—like whether that \textit{``AI solution''} is just a recycled decision tree from a 2015 Kaggle competition.
  
  \medskip
  
  So instead, they \textbf{perfect the art of delay}:
  \begin{itemize}
    \item Endless ``refinements.''
    \item New stakeholder feedback cycles.
    \item Shifting requirements that conveniently reset the clock.
  \end{itemize}
  
  \medskip
  
  By staying in POC purgatory, they avoid accountability while maintaining the illusion of progress.
  
  \medskip
  
  \textbf{Remember:} If a project is always \textit{``almost ready,''} it’s probably designed to be that way. \\
  \textbf{Delays aren’t accidental—they’re billable.}
  


\ExecutiveChecklist{medium}{Ending the Eternal Proof of Concept}{
  \item Ask when the model will be in production—and who’s responsible for that.
  \item Require infrastructure compatibility checks before greenlighting development.
  \item If it’s still in POC after 6 months, shut it down or ship it.
  \item No model is better than a broken model in production.
}