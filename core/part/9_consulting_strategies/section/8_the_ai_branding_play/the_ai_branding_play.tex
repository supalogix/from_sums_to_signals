\section{The AI\texttrademark{} Branding Play: If It Does Math, It’s Now Artificial Intelligence}

\begin{quote}
Linear regression? AI. Logistic regression? Also AI. CountIf in Excel? Close enough—call it “edge computing.”
\end{quote}

  In the age of \textbf{AI\texttrademark{} Branding}, it’s not about what the tool \textit{does}—it’s about what it \textit{looks like} it does.
  
  \medskip
  
  \textbf{Law 37} from \textit{The 48 Laws of Power} explains why every spreadsheet formula is now ``AI-powered'':
  \begin{quote}
  ``Striking imagery and grand symbolic gestures create the aura of power. People will believe what they see before they believe what is rational.''
  \end{quote}
  
  \medskip
  
  That’s why simple regression models are rebranded as \textit{``machine learning solutions''}, and automated Excel sheets become \textit{``AI-driven decision platforms.''}
  
  \medskip
  
  It’s not about algorithms—it’s about the \textbf{spectacle}: \\
  Glossy dashboards, futuristic buzzwords, and just enough mystery to make stakeholders nod in awe.
  
  \medskip
  
  If removing the word ``AI'' from the pitch makes it sound ordinary, that’s because it \textbf{is} ordinary. \\
  But ordinary doesn’t close deals—\textbf{spectacle} does.
  
  \medskip
  
  \textbf{Remember:} When consultants or vendors lean heavily on AI branding, they’re not selling intelligence. They’re selling \textbf{the illusion of innovation}.
  


\ExecutiveChecklist{medium}{AI Branding Detox}{
  \item Ask: “Would this still work if we removed the word ‘AI’?”
  \item Require a comparison to traditional methods (regression, heuristics, etc.).
  \item Don’t pay for branding—pay for results.
  \item If Excel could do it, don’t call it disruption.
}