\section{The Myth of the One-Click Model: Why You Don’t Need Data Scientists—Until You Do}

\begin{quote}
“No-code AI” tools that generate models with no context or domain knowledge. But hey, it’s got a drag-and-drop UI!
\end{quote}

  The appeal of \textbf{One-Click AI} isn’t technical—it’s psychological.
  
  \medskip
  
  \textbf{Law 32} from \textit{The 48 Laws of Power} explains why ``no-code AI'' pitches work so well:
  \begin{quote}
  ``People will believe in the fantasy because the truth is too ugly or complicated. Give them what they want to hear.''
  \end{quote}
  
  \medskip
  
  And what do executives want to hear? That they can skip hiring data scientists, avoid messy model tuning, and generate enterprise-grade AI with a drag-and-drop UI—just like building a PowerPoint.
  
  \medskip
  
  The reality? Behind every ``no-code AI'' success story is usually a team of engineers cleaning up after the magic button fails.
  
  \medskip
  
  But fantasy sells:
  \begin{itemize}
    \item No data pipelines to manage.
    \item No domain expertise required.
    \item No mention of what happens when the data drifts—or when regulators ask for explainability.
  \end{itemize}
  
  \medskip
  
  \textbf{Remember:} The easier it sounds, the more likely you’re being sold a \textbf{dream}, not a deployable system. \\
  And in tech, dreams without maintenance plans turn into nightmares.
  
  


\ExecutiveChecklist{medium}{The One-Click Mirage}{
  \item Ask who built the model, and who will maintain it post-deployment.
  \item Require transparency on architecture and tuning.
  \item Check if domain expertise was involved—or if it’s just AutoML roulette.
  \item If the product promises “no-code AI,” ask what happens when something breaks.
}