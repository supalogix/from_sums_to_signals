\section{Fortune’s Fools: Gullibility’s Price}

Chen Wei arrived in Oakland on a humid spring morning in 2005, clutching a single suitcase and a one-year 
business visa stamped “Hong Kong” by the United States Embassy. He had pitched his plan artfully: a humble 
Cantonese restaurant in a struggling part of Northern California—an endorsement, embassy officials said, 
that would bring commerce and jobs to a neighborhood long starved for both. What they did not know was 
that Chen Wei’s true ambition lay not in noodle soup or dim sum but in the backroom tables of an illegal 
gambling den—what some locals, half-mocking and half-admiring, called a “fish house.”

He settled on a shabby storefront at the edge of Fruitvale, where rents were cheap and few outsiders 
bothered to look. By day, the place—“Golden Lotus Kitchen”—served congee, roast pork, and shrimp dumplings 
to aging Filipino grandmothers and warehouse laborers craving familiarity. At night, though, when the 
paper lanterns outside glowed crimson and most diners had gone home, a secret door behind the kitchen 
wall swung open. Inside, wooden fan-tan tables were set up under a fluorescent bulb, and a handful of 
shadowed figures crowded around, quietly stashing their cash beneath teapots. The Cantonese-speaking 
dealers called the suckers “shui yu”—“water fish”—and Chen Wei took a discreet half-cut of every pot.

Not long after opening, Chen Wei noticed a lanky, scruffy nineteen-year-old hanging around the back alley, 
pushing a rusted shopping cart of recyclables. He introduced himself as Brian Miller, a recent dropout 
from Modesto, trying to earn money for community college. Chen Wei offered the boy a job washing dishes 
inside Golden Lotus. Brian showed up the next day in a grimy T-shirt and grabbed a mop without question. 
Over weeks, Chen Wei watched him: how Brian’s bright blue eyes followed the gamblers when they trickled 
in at dusk, how his lean frame hummed with restless ambition. In return for a few extra dollars, Brian 
agreed to run errands—delivering nefarious envelopes of cash to a neighboring dry cleaner, skulking out 
to buy new decks of marked cards, even keeping watch when the cops did surprise patrols. Chen Wei called 
him “Xiao Bai” (“Little White”) behind closed doors.

As months passed, Chen Wei entrusted Brian with greater roles. First, he allowed him to tally nightly 
revenues. Then he gave Brian the nickname “business partner” in private: when a second outpost opened in 
Stockton, official papers labeled Brian Miller as owner/operator of “Lotus Express,” while Chen Wei stayed 
in the shadows, funding the renovation, training the staff, and quietly installing concealed cameras in 
the restrooms to catch any talk of police snitches. When Golden Lotus Kitchen’s gambling profits ballooned, 
Chen Wei opened a third location—a “noodle house” near Milpitas—again putting Brian’s name on the lease. 
Brian hired day-shift cooks of Filipino and Latino descent, posted neon signs in English and Chinese, and 
submitted all permits under his own Social Security number. Chen Wei moved cross-country to Frisco one 
winter, scouting even wealthier neighborhoods where new “fishhouses” could flourish, while Brian handled 
the day-to-day grind of hiring cooks, ordering takeout trays, and fending off extra-furious health inspectors.

For years, the scheme grew: twenty-seven seats in the dining room, thirty-two seats in the backroom fan-tan 
parlor, a side door for known gamblers only, off-hours poker tournaments promoted via flyers slipped under 
windshield wipers. Chen Wei’s plan was working—soon he would open a flagship “restaurant” in South of Market, 
and every high-roller from Silicon Valley to the Tenderloin would risk thousands on “fish” games, fanning 
red envelopes across grimy tables. But local police had begun to notice patterns: repeated gambling alarms 
at Golden Lotus locations, muddy footprints in back alleys, suspicious night-time deliveries. An anonymous 
tip from a neighborhood watch meeting reported “a slaphouse run by an Asian fellow and a white kid.”

In April 2014, officers in plain clothes raided the Stockton branch just as a hand of cards was being dealt. 
They found contraband chips, cash stashes taped under poker tables, and a trove of Excel spreadsheets logging 
wagers, payoffs, and “protection fees.” Brian Miller was cuffed in handcuffs, blinking in disbelief as 
detectives seized his laptop. Chen Wei, long gone to San José, was still several hundred miles away, changing 
addresses and phone numbers. In court, prosecutors confronted Brian with lease agreements, bank statements, 
and emails—the mountain of paper proving he owned every “restaurant” and operated the entire chain. Chen Wei’s 
name never appeared on a single bank record.

At sentencing, Brian sat at the defense table—thin, hollow-eyed, wearing an ill-fitted suit—while U.S. Attorney 
Marisol Aguilar painted the picture: a naïve kid who “ought to know better” than to front for an immigrant 
criminal mastermind. Brian tearfully insisted he was “like a son” to Chen Wei, that he’d believed the 
businesses were aboveboard. The judge, unmoved, sentenced him to four years in federal prison for conspiracy 
to operate an illegal gambling enterprise. Meanwhile, Chen Wei quietly slipped back to Hong Kong using a 
friend’s passport, already planning the next venture—perhaps a “boba house” that would mask a more sophisticated 
slaphouse for high-stakes fish games.

And so the cycle continued: a Chinese restaurant in a poor neighborhood, fish-faced gamblers at back-room tables, 
a white frontman taking the fall in Washington’s surprise raids. The “fish,” it seemed, were always someone 
else's “shui yu”—eager, vulnerable, and quick to be netted.

https://lbpost.com/news/crime/illegal-gambling-slaphouse-raids-police/