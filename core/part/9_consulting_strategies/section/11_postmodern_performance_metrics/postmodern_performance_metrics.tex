\section{Postmodern Performance Metrics: Measuring Whatever You Already Improved}

\begin{quote}
“We increased user engagement by 300\%!” (Relative to what? Silence.)
\end{quote}

  \textbf{Postmodern Performance Metrics} aren’t designed to inform—they’re designed to make you feel good about approving the budget.
  
  \medskip
  
  \textbf{Law 13} from \textit{The 48 Laws of Power} reveals why every metric you see is a success story:
  \begin{quote}
  ``When asking for help or support, appeal to people’s self-interest, never to their mercy or sense of honesty.''
  \end{quote}
  
  \medskip
  
  In metric-land, the truth is irrelevant if it doesn’t serve the agenda. \\
  Consultants and internal teams know that executives don’t want raw data—they want validation that their \textit{``strategic investments''} are paying off.
  
  \medskip
  
  That’s why:
  \begin{itemize}
    \item Baselines mysteriously vanish.
    \item KPIs shift mid-project to highlight whatever looks best.
    \item Engagement jumps by 300\%—because last month’s numbers were conveniently excluded.
  \end{itemize}
  
  \medskip
  
  It’s not analysis—it’s \textbf{narrative management}.
  
  \medskip
  
  \textbf{Remember:} If every chart points upward and every metric is a win, you're not looking at performance data—you're looking at a \textbf{self-esteem report} designed to secure the next round of funding.
  


\ExecutiveChecklist{medium}{Debunking Postmodern Metrics}{
  \item Ask: “Improved compared to what baseline?”
  \item Demand before-and-after comparisons with control groups.
  \item Watch for cherry-picked metrics and shifting KPIs mid-project.
  \item If all metrics are success stories, it’s marketing, not analysis.
}