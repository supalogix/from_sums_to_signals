\section{Vendor Lock-In, the Long Con: How to Make Dependency Look Like Vision}

\begin{quote}
Replace open source with “partner ecosystem.” Charge extra for integration. Never let them leave.
\end{quote}

  \textbf{Vendor Lock-In} isn’t hidden—it’s presented as a \textit{strategic advantage}.
  
  \medskip
  
  \textbf{Law 12} from \textit{The 48 Laws of Power} reveals the tactic at play:
  \begin{quote}
  ``One sincere move will cover over dozens of deceptive ones. Use selective honesty to cloak your intentions.''
  \end{quote}
  
  \medskip
  
  That’s why vendors openly promote their \textit{``partner ecosystem''}, \textit{``seamless integration''}, and \textit{``optimized proprietary solutions''}. \\
  They’re being ``honest'' about how well everything works together—what they don’t mention is that it only works together inside their walled garden.
  
  \medskip
  
  By highlighting convenience and performance today, they distract you from the \textbf{dependency} you’ll face tomorrow.
  
  \medskip
  
  The sincerity: \\
  \textit{``We’re committed to your success within our ecosystem.''}
  
  \medskip
  
  The deception: \\
  You’ll need a small fortune—and several months—to escape that ecosystem.
  
  \medskip
  
  \textbf{Remember:} When a vendor is unusually transparent about how ``perfectly'' their system fits your needs, check how easily it lets go. \\
  \textbf{Selective honesty is the lock; your signature is the key.}
  
  

\ExecutiveChecklist{high}{Avoiding the Lock-In Trap}{
  \item Ask: “Can we migrate away from this without rebuilding everything?”
  \item Require open standards and APIs.
  \item Audit their contract for integration penalties and proprietary traps.
  \item Get an exit strategy before you sign anything.
}