\section{The Black Box as Security Blanket: How Ambiguity Sells Better Than Accuracy}

\begin{quote}
Clients don’t want to know how it works. They want to know it \textit{sounds} like it works.
\end{quote}

  \textbf{The Black Box} doesn’t thrive because it’s effective—it thrives because it’s \textit{impressive}.
  
  \medskip
  
  \textbf{Law 6} from \textit{The 48 Laws of Power} explains why ambiguity often outsells accuracy:
  \begin{quote}
  ``Mystery creates awe. The more obscure and complex you appear, the more attention and authority you command.''
  \end{quote}
  
  \medskip
  
  A clear, explainable model invites scrutiny. \\
  A black box wrapped in terms like \textit{``proprietary AI engine''} or \textit{``deep learning core''} shuts down questions before they start.
  
  \medskip
  
  After all, if no one knows how it works, no one can prove that it doesn’t.
  
  \medskip
  
  Consultants and vendors know that clients often prefer \textbf{the illusion of sophistication} over transparency. \\
  The less you understand, the more you're inclined to trust the expert behind the curtain.
  
  \medskip
  
  \textbf{Remember:} If the only explanation is \textit{``It just works''}, what you're buying isn’t technology—it’s a \textbf{confidence trick in a sleek UI}.
  


\ExecutiveChecklist{high}{Escaping the Black Box Security Blanket}{
  \item Require model explainability reports or interpretability tools.
  \item Ask: “What features drove this prediction?” If they can’t answer, walk.
  \item Avoid tools where “proprietary” = “you’re not allowed to ask.”
  \item Don’t accept “it just works” as a justification.
}