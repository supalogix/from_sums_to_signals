\section{ROI as a Feeling: Why Consultants Never Quantify Success (and You Shouldn’t Ask)}

\begin{quote}
If they ask for metrics, pivot to “brand impact.” If they insist, say it’s “too early to tell.”
\end{quote}

  When it comes to \textbf{ROI as a Feeling}, consultants have mastered the art of turning the absence of measurable success into a strategic advantage.
  
  \medskip
  
  \textbf{Law 22} from \textit{The 48 Laws of Power} explains this maneuver:
  \begin{quote}
  ``When you are weaker, never fight for honor's sake; instead, surrender. Surrender gives you time to recover, time to undermine, and time to wait for your opponent's power to wane.''
  \end{quote}
  
  \medskip
  
  Translation in consultant-speak: \\
  \textit{``We don’t have the numbers yet—but that’s because true impact takes time.''}
  
  \medskip
  
  By \textbf{surrendering} to the fact that ROI can't be shown today, they shift the narrative from failure to \textit{``long-term strategic value.''} \\
  The longer they delay concrete metrics, the harder it becomes for you to hold them accountable—because now you're ``invested in the journey.''
  
  \medskip
  
  If every request for data is met with phrases like \textit{``brand uplift''}, \textit{``market positioning''}, or \textit{``too early to quantify''}, you’re watching Law 22 in action.
  
  \medskip
  
  \textbf{Remember:} A real ROI isn’t afraid of being measured. \\
  If success is always just over the horizon, you’re being managed—not informed.
  


\ExecutiveChecklist{high}{When ROI Is Just a Feeling}{
  \item Insist on pre-defined ROI metrics—quantitative, not qualitative.
  \item Ask, “If this fails, how will we know?”
  \item Watch for evasive language like “brand uplift” or “strategic alignment.”
  \item Demand post-mortems on past client engagements.
}