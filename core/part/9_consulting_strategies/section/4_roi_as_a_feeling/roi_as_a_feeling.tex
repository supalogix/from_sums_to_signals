\section{ROI as a Feeling: Why Consultants Never Quantify Success (and You Shouldn’t Ask)}

\begin{quote}
If they ask for metrics, pivot to “brand impact.” If they insist, say it’s “too early to tell.”
\end{quote}

  When it comes to \textbf{ROI as a Feeling}, consultants have mastered the art of turning the absence of measurable success into a strategic advantage.
  
  \medskip
  
  \textbf{Law 22} from \textit{The 48 Laws of Power} explains this maneuver:
  \begin{quote}
  ``When you are weaker, never fight for honor's sake; instead, surrender. Surrender gives you time to recover, time to undermine, and time to wait for your opponent's power to wane.''
  \end{quote}
  
  \medskip
  
  Translation in consultant-speak: \\
  \textit{``We don’t have the numbers yet—but that’s because true impact takes time.''}
  
  \medskip
  
  By \textbf{surrendering} to the fact that ROI can't be shown today, they shift the narrative from failure to \textit{``long-term strategic value.''} \\
  The longer they delay concrete metrics, the harder it becomes for you to hold them accountable—because now you're ``invested in the journey.''
  
  \medskip
  
  If every request for data is met with phrases like \textit{``brand uplift''}, \textit{``market positioning''}, or \textit{``too early to quantify''}, you’re watching Law 22 in action.
  
  \medskip
  
  \textbf{Remember:} A real ROI isn’t afraid of being measured. \\
  If success is always just over the horizon, you’re being managed—not informed.
  


\ExecutiveChecklist{high}{When ROI Is Just a Feeling}{
  \item Insist on pre-defined ROI metrics—quantitative, not qualitative.
  \item Ask, “If this fails, how will we know?”
  \item Watch for evasive language like “brand uplift” or “strategic alignment.”
  \item Demand post-mortems on past client engagements.
}


\begin{tcolorbox}[colback=blue!5!white, colframe=blue!50!black,
  title={Historical Sidebar: When ROI Becomes Loyalty --- China's Real Estate Blackmail Scandals}]

In the 2000s and early 2010s, China's booming real estate market created intense pressure on developers to win government contracts.  
For some, competitive bids and transparent negotiations were too slow—or too uncertain.

\medskip

Instead, a darker strategy emerged:  
\textbf{Shift the measure of value from public results to private loyalty}.

\medskip

\begin{itemize}
    \item Developers orchestrated sexual blackmail schemes against Communist Party officials.
    \item Bribery, favors, and secret relationships replaced competitive pricing or tangible outcomes.
    \item Officials granted favorable land deals not based on performance, but on personal compromise.
\end{itemize}

\medskip

The most famous example was Lei Zhengfu, a party secretary secretly filmed in a hotel by operatives working for a developer.  
The resulting scandal exposed dozens of officials and shattered public trust—but only after years of "success" built on hidden incentives.

\medskip

\begin{quote}
The implicit pitch: \textbf{You don't have to show public results if you privately secure loyalty}.
\end{quote}

\medskip

\textbf{The Lesson?} When "value" becomes something you can't measure—and aren't supposed to measure—real failure is already underway. It just hasn't surfaced yet.
\end{tcolorbox}

\subsection{Case Study: The Pipeline That Knew Too Much (TitanBank, 2023)}

TitanBank had a problem—not with fraud detection, but with fraud detection that worked too well.

When the Risk Engineering team greenlit a multi-phase project titled \textit{Adaptive ML for Internal Anomaly Auditing}, the board saw it as a forward-thinking investment. On paper, it looked like a best-in-class pipeline for catching financial irregularities. In practice, it became a \$4.2 million ghost system—never deployed, never challenged, never explained.

At the center of the project were two people: a consultant from a boutique AI firm, and the bank’s own VP of Risk Engineering. Their relationship was professional in name only. The contract renewals weren’t tied to KPIs or deployment milestones—but to a more implicit agreement:

\begin{itemize}
    \item The consultant would provide the VP with compromising material—screenshots, audio, or favors—that could be used as collateral.
    \item In return, the VP would ensure funding was continuous and no technical reviewer ever saw the full pipeline output.
    \item The pipeline itself was deliberately misconfigured: enough buzzwords to pass a board review, but never enough rigor to detect real fraud in TitanBank’s books.
\end{itemize}

This wasn’t oversight—it was strategic surrender. The consultant had no power to enforce the pipeline’s adoption, so they leaned into Law 22: \textit{“When you are weaker, never fight for honor’s sake… surrender gives you time to undermine.”} In this case, surrender meant giving the VP what he wanted—not a solution, but a liability.

By removing performance as the metric, the project became something else entirely: a loyalty engine.  
Each year it was renewed, it created a deeper trail of complicity.  
ROI couldn’t be quantified, because quantification was the threat.  

And so, a shadow economy bloomed inside the bank—founded not on results, but on mutual exposure.

\medskip

\textbf{The Takeaway:}  
This is why the checklist matters. It's not a formality—it’s your early warning system.  
If a project can’t be evaluated, it can’t be trusted.  
And if it can’t be trusted, you’d better ask who benefits from the ambiguity.
