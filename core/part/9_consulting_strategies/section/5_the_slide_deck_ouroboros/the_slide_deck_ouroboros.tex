\section{The Slide Deck Ouroboros: Selling Strategy That Refers to Strategy That Refers to Strategy}

\begin{quote}
\textit{Strategy} is the product. Implementation is for the poor.
\end{quote}

  In the world of the \textbf{Slide Deck Ouroboros}, delivery is dangerous—because once you deliver, the client doesn’t need you anymore.
  
  \medskip
  
  \textbf{Law 11} from \textit{The 48 Laws of Power} explains why strategy consultants never stop strategizing:
  \begin{quote}
  ``To maintain your independence, you must always be needed and wanted. The more reliant someone is on you, the more freedom you have.''
  \end{quote}
  
  \medskip
  
  That’s why every \textit{``strategy session''} leads to… another \textit{``strategy session.''} \\
  The product isn’t implementation—it’s the illusion that you’re \textit{almost there}, but need just one more framework, one more alignment meeting, one more deck.
  
  \medskip
  
  By keeping deliverables vague and perpetually in-progress, consultants ensure they remain the \textbf{trusted guide} through a labyrinth they designed.
  
  \medskip
  
  If the only thing growing is the number of slides—not the number of shipped features—you’re not in a strategy partnership. \\
  You’re in a \textbf{dependency loop with nice formatting}.
  
  \medskip
  
  \textbf{Remember:} Real strategy points to an exit. The Ouroboros points back to itself.
  


\ExecutiveChecklist{medium}{Escaping the Slide Deck Ouroboros}{
  \item Ask if this is a meta-strategy (i.e., a pitch for more pitches).
  \item Demand one slide titled: “Here’s What We Will Actually Build.”
  \item If the deliverables are more slides, cancel the contract.
  \item Strategy should end in product, not PowerPoint recursion.
}