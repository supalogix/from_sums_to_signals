\subsection{Fail-Safes: Because You Shouldn’t Need a War Room at 3AM}

\textbf{Act I: The Launch}

At a large e-commerce company (let’s call it \textit{Rainforest Prime}), a new recommender model was greenlit for production.

It had passed all the checks — in staging.\\
CI was green. Dashboards were calm. Champagne was metaphorically uncorked.

Five minutes after deployment, every product page was proudly showcasing:\\
\textbf{\$700 HDMI cables} as “Most Relevant.”

\vspace{1em}
\textbf{Act II: The Root Cause}

Turns out, a bug in the price normalization logic had skewed the entire ranking layer.\\
High prices = high scores.\\
Cables were the new meta.

By the time someone traced it:
\begin{itemize}
    \item Customer trust had taken a hit
    \item Twitter had opinions
    \item A VP was angrily refreshing the dashboard and yelling “\textsc{ROLL IT BACK}” into Zoom
\end{itemize}

\vspace{1em}
\textbf{Act III: The Void}

There was no:
\begin{itemize}
    \item Fallback model
    \item Circuit breaker
    \item Shadow deployment
\end{itemize}

No gradual rollout. No kill switch.\\
Just a full send into the void.

The only contingency plan was raw panic.

\vspace{1em}
\textbf{Act IV: The 3AM War Room}

An emergency Zoom was spun up.\\
Everyone was paged. Everyone was tired.

DevOps raced to find the last working model.\\
Marketing drafted an apology tweet.\\
Legal hovered nearby.

By sunrise, the issue was fixed.\\
But the memory would linger.

\vspace{1em}
\textbf{Epilogue: Why Fail-Safes Matter}

Even the best models go sideways.

Maybe it’s a bad feature.\\
Maybe it’s upstream data corruption.\\
Maybe it’s just Tuesday.

But without an escape hatch — a known-good backup, a rollback plan, a sanity check on output distributions —\\
you’re one bug away from a PR disaster and an all-nighter.

Plan for failure. So you don’t have to wake up at 3AM to handle it.
