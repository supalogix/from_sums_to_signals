\subsection{Psychic Structure and System Structure: Jung Meets Conway}

If the DSM gives us a map of dysfunction, it tells us \textit{what} is happening—not \textit{why}. For that, we need theories of structure. Enter Carl Jung and Melvin Conway.

Jung gave us a model of the mind built on archetypes—recurring symbols and roles that emerge across cultures, myths, and yes, dysfunctional data teams. Conway, on the other hand, showed us that these patterns don’t just live in the mind—they shape our codebases. His law is simple: \textbf{the structure of any system will reflect the structure of the organization that built it}.

Together, they explain a terrifying truth: your pipeline is not just broken. It is broken in a way that mirrors your team. It contains forgotten scripts, unmonitored models, orphaned feature stores—not randomly, but ritually. These aren’t exceptions. They’re archetypes.

\begin{tcolorbox}[
  title=Jungian Archetypes and the Data Team Unconscious,
  colback=gray!5,
  colframe=black,
  fonttitle=\bfseries,
  sharp corners=south,
  boxrule=0.5pt,
  enhanced,
  breakable
]
Carl Jung believed that beneath our individual quirks lies a collective unconscious—shared symbols, myths, and roles that recur across cultures. The same, it turns out, is true for machine learning teams.

Every dysfunctional pipeline is haunted by its own cast of recurring characters:

\begin{itemize}
  \item \textbf{The Hero (a.k.a. “The One Who Knows the Configs”)}\\
  Lone savior of the production branch. Writes code at 3AM. Is also the reason no one else knows how anything works.

  \item \textbf{The Shadow (a.k.a. “That Bash Script from 2019”)}\\
  It still runs. No one understands it. It controls 40\% of your deploy process.

  \item \textbf{The Trickster (a.k.a. “The Dashboard”)}\\
  Looks beautiful. Shows results from two years ago. Still cited in quarterly meetings.

  \item \textbf{The Wise Old Man (a.k.a. “Former Employee on Slack”)}\\
  Knows everything. Left the company. Still gets pinged. Never responds.

  \item \textbf{The Anima/Animus (a.k.a. “The Project Manager”)}\\
  Oscillates between order and chaos. Tries to align business goals with what the model can’t do. Slowly loses grip on reality.

  \item \textbf{The Orphan (a.k.a. “The Unused Feature Store”)}\\
  Built with care. Abandoned without ceremony. Still eats up \$600/month in AWS costs.
\end{itemize}

Jung argued that individuation—the path to wholeness—requires integrating these archetypes. Likewise, healing a pipeline begins with naming the dysfunction. Because these aren’t mistakes. They’re patterns. And they’re unconscious until you see them.
\end{tcolorbox}

\begin{tcolorbox}[
  title=Sidebar: Conway’s Law — The Org Chart as Source Code,
  colback=gray!5,
  colframe=black,
  fonttitle=\bfseries,
  sharp corners=south,
  boxrule=0.5pt,
  enhanced,
  breakable
]
In 1967, Melvin Conway submitted a paper with a radical claim: \textbf{“Any organization that designs a system will produce a design whose structure is a copy of the organization’s communication structure.”}

The paper was rejected—too informal, they said. But Fred Brooks cited it anyway, and Conway’s Law became gospel in software circles.

Today, it’s prophetic. Machine learning systems reflect the very human dysfunctions that build them. Microservices mimic turf wars. Metrics map to reporting chains. Unused features reflect abandoned teams. 

Conway’s Law isn’t a warning. It’s a mirror. Flip your org chart sideways—you’re looking at your pipeline’s DAG. It’s not just spaghetti code. It’s structural anthropology.
\end{tcolorbox}

