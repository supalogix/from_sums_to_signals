\subsection{When Systems Fail: Jung, Addiction, and the Danger of Being Too Systematic}

In engineering, when something breaks, the instinct is simple: apply a system to fix it. Define the problem, standardize the solution, automate where possible, and sleep better knowing it’s now ``under control.'' But when the problem isn’t purely technical—when it’s tangled with human behavior, culture, and psychology—that approach doesn’t just fall short. It can make things worse.

Carl Jung understood this long before DevOps was a thing.

\begin{tcolorbox}[
  title=Jung’s Warning: Not Every Problem Has a Systematic Fix,
  colback=gray!5,
  colframe=black,
  fonttitle=\bfseries,
  sharp corners=south,
  boxrule=0.5pt,
  enhanced,
  breakable
]

Before \textit{DevOps} became a buzzword slapped onto cloud services and overpriced workshops, it was an observation—a psychological insight disguised as an IT strategy.

\medskip

Gene Kim, one of the founding thinkers behind DevOps, didn’t set out to invent a new framework. He simply noticed a pattern:  \textbf{the biggest dysfunctions in software teams weren’t technical—they were organizational.}  

\medskip

Companies loved to separate Development from Operations. One team wrote the code, the other kept it running. On paper, this looked efficient. In practice, it created silos, blame games, slow deployments, and a toxic culture where “success” meant tossing problems over the wall.

\medskip

DevOps—short for \textbf{Development + Operations}—was never about tools, pipelines, or automation scripts. It was about breaking down those artificial barriers and treating software delivery as a collaborative, continuous process. At its core, DevOps was a humanistic response to the dehumanizing effects of rigid corporate procedures.

\medskip

\textbf{The irony?}  Modern consulting took this sociotechnical philosophy and did exactly what it warned against:  They turned DevOps into a \textit{product}: a bundle of technologies, processes, and procedures you could \textit{buy}.

\medskip

Instead of fostering trust, autonomy, and collaboration, companies were sold dashboards, KPIs, and ``DevOps as a Service.''  Humans became cogs again—this time in a CI/CD pipeline.

\medskip

In the end, what began as a rebellion against mechanistic thinking was repackaged as… more mechanistic thinking.  Because it’s always easier to install a tool than to fix a culture.

\end{tcolorbox}

\medskip

When Jung told Bill Wilson—the co-founder of Alcoholics Anonymous—that \textbf{``the only known cure for alcoholism is a spiritual awakening''}, he wasn’t prescribing religion. He was warning against the illusion that human dysfunction can be solved by checklists and protocols. Jung saw addiction not as a behavioral bug to patch, but as a symptom of something far deeper: an existential void, a misdirected search for meaning, wholeness, and transcendence.

Addiction, in Jung’s view, was the psyche’s distorted attempt to fill that void—a struggle that no systematic therapy could fully address. It wasn’t a problem you could ``engineer'' away. It demanded something personal, transformative, and beyond rational control.


Now, swap out ``addiction'' for ``organizational dysfunction'' or ``ML pipeline chaos.'' The lesson holds.

Too often, teams respond to deep-rooted dysfunction by doubling down on systems: new frameworks, stricter processes, mandatory standups, or yet another layer of monitoring. But if the problem is rooted in team culture, misaligned incentives, or unconscious archetypes playing out across your infrastructure, no amount of Jira tickets will save you.

Jung disliked systems that were too systematic because they pretended humans were predictable machines. He believed that when dealing with people—and by extension, the systems they build—rules should be seen as \textbf{guidelines}, not gospel. Healing comes not from rigid structure, but from confronting the underlying psychological forces at play.

\begin{tcolorbox}[
    title=Jung’s Warning: Not Every Problem Has a Systematic Fix,
    colback=gray!5,
    colframe=black,
    fonttitle=\bfseries,
    sharp corners=south,
    boxrule=0.5pt,
    enhanced,
    breakable
]
Jung told Bill Wilson that addiction couldn’t be cured by methodical treatment alone—it required a profound inner transformation. Why? Because addiction wasn’t just a habit. It was a misplaced search for meaning.

\medskip

The same applies to dysfunctional ML systems. When your pipeline collapses under tech debt and miscommunication, it’s tempting to reach for a new framework or enforce stricter processes. But sometimes, the issue isn’t technical at all.

\medskip

It’s cultural. It’s psychological. It’s the Shadow running your Bash scripts and the Hero hoarding config knowledge.

\medskip

\textbf{You can’t refactor your way out of an existential crisis.} Some problems require less engineering—and more organizational introspection.
\end{tcolorbox}

Jung’s frustration with programs like AA (and later, the oversimplification of his typology into things like Myers-Briggs) stemmed from this exact point: complex human struggles aren’t solved by one-size-fits-all systems. They require engaging with the unique, messy, unconscious dynamics beneath the surface.

So before you launch that ``Pipeline Governance Initiative'' or adopt yet another MLOps tool promising order, ask yourself:

\begin{quote}
\textit{Is this really a system problem?} \textit{Or is it a human problem disguised as one?}
\end{quote}

Because sometimes, the fix isn’t a framework—it’s a conversation no one wants to have.

\textbf{Jung didn’t offer systems. He offered mirrors.} And when dealing with human-built systems, that might be the better place to start.

