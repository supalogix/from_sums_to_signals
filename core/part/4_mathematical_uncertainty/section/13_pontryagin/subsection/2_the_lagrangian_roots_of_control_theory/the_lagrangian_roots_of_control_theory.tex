\subsection{The Lagrangian Roots of Control Theory}

As the 20th century unfolded, a remarkable pattern emerged: nearly every domain that sought to formalize uncertainty, structure, or prediction ended up converging on the same mathematical tools—measure theory, variational principles, and entropy.

From Carnot’s falling caloric to Boltzmann’s microstates, from Shannon’s bits to Kolmogorov’s probability spaces, a unifying theme took shape: systems evolve not arbitrarily, but in ways that maximize or minimize something under constraint. In physics, this something was often energy. In information theory, it was uncertainty. And in control theory, it became the cost of a path.

Yet in the Soviet Union, this convergence was more than intellectual—it was ideological. Control theory, especially in its formative Soviet incarnation, was not merely a mathematical innovation. It was a tool of governance, production, and discipline. At the center of this synthesis stood \textbf{Lev Pontryagin}, whose \textbf{Pontryagin Maximum Principle (PMP)} provided a geometric formalism for optimizing action under constraint. Developed in the 1950s, PMP became the mathematical bedrock of the Soviet command economy: it enabled engineers to design missile trajectories, regulators to fine-tune power grids, and planners to manage resource allocation with surgical precision.

Pontryagin’s mathematics was never just about curves and manifolds—it was about control in the fullest sense: control of motion, of systems, of labor, of history. His work exemplified the union of mathematics and state power. Where Andrey Kolmogorov sought to insulate mathematics from political ideology through abstraction and probabilistic universality, Pontryagin embedded it in the mechanisms of centralized planning and technological command. Soviet mathematics, in this light, was not merely divided by subject—it was divided by vision.

What follows is a story of that vision: how variational principles once born in mechanics were retooled into instruments of optimization, governance, and control.