\section{Boole}

\subsection{Boole and the Algebra of Thought: When Probability Met Logic}

If Laplace showed that ignorance could be calculated, then \textbf{George Boole} asked a deeper question:  
\textit{What if reasoning itself could be mathematized?}

Boole (1815–1864), working in the shadow of Enlightenment determinism and the emerging machinery of probability, sought not just to quantify uncertainty — but to formalize the very structure of thought. Where Laplace turned uncertainty into a ratio, Boole turned logic into an algebra.

And he didn’t see these as separate pursuits.

\begin{quote}
\textit{“The laws we here propose to investigate are the laws of one of the most important of our faculties — that of reasoning.”} \\
— \textbf{George Boole}, \textit{The Laws of Thought} (1854)
\end{quote}

\subsection{Logic as Symbolic Structure}

Boole’s insight was that logical propositions — like “All men are mortal” or “If A then B” — could be represented symbolically and manipulated algebraically. He replaced syllogisms with symbols, logical deduction with operations.

In Boole’s system:
\begin{itemize}
    \item Logical \textbf{AND} became multiplication.
    \item Logical \textbf{OR} became addition.
    \item Negation became subtraction from unity: \(1 - x\).
\end{itemize}

In short, Boole created what we now call **Boolean algebra** — a system that underlies modern computing, digital circuits, and formal logic. But for Boole, this was never just about machines. It was a philosophy of mind.

\subsection{Probability as Logic with Uncertainty}

Boole didn’t stop at pure logic. In fact, he explicitly treated **probability as an extension of logical reasoning** — a way to generalize truth when certainty isn’t possible. Where Laplace gave us:

\[
p(e) = \frac{\text{favorable outcomes}}{\text{possible outcomes}},
\]

Boole asked: what if we don’t know the outcomes, but we still want to reason about propositions?

His solution was a new kind of **algebraic probability** — where the truth values of logical statements could be uncertain, yet still obey structural rules. He investigated how partial knowledge constrained possible inferences, and even developed early versions of **probabilistic inequalities** — bounding what must be true even in the absence of full information.

\begin{quote}
\textit{Where Laplace reduced uncertainty to ratio, Boole embedded it in logic itself.}
\end{quote}

\subsection{The Algebra of Ignorance}

Boole admired Laplace but was not content to merely count cases. He sought a system that could handle **vagueness**, **incomplete data**, and **interrelated propositions** — without sacrificing rigor.

His algebra of thought:
\begin{itemize}
    \item Unified logic and probability into a single symbolic system.
    \item Laid the groundwork for modern propositional calculus.
    \item Prefigured the development of probabilistic logic, information theory, and even AI.
\end{itemize}

Where Laplace brought mathematics to ignorance, Boole brought structure to uncertainty. And in doing so, he turned the act of reasoning into a system that could be computed, constrained, and — eventually — programmed.

\begin{quote}
\textit{Laplace gave us the calculus of the unknown. \\
Boole gave us the algebra of knowing.}
\end{quote}

\medskip

In the next section, we’ll see how Boole’s symbolic formalism seeded a deeper transformation — one where probability would no longer live on finite sets or algebraic tricks, but on measure spaces, σ-algebras, and the logic of infinite collections.

\begin{tcolorbox}[colback=gray!5!white, colframe=black!75!white, title={Historical Sidebar: Kant and the Dream of Formal Thought}]

    \textbf{Immanuel Kant} (1724–1804) never wrote a line of code or built a logic gate — but his fingerprints are all over George Boole’s project.
    
    In the \emph{Critique of Pure Reason}, Kant proposed that knowledge isn’t just received from the world — it’s structured by the mind. Space, time, and causality aren’t just facts out there; they are part of the human operating system. Reason itself, Kant argued, follows formal laws. These laws of logic and inference don’t come from experience — they make experience possible.
    
    \medskip
    
    Boole, writing half a century later, took that idea and ran with it — straight into algebra.
    
    Where Kant described the structure of thought in philosophical terms, Boole sought to \textbf{symbolize it mathematically}. His “algebra of logic” wasn’t just about truth tables — it was about encoding how the mind moves from one proposition to another, even under uncertainty.
    
    \medskip
    
    In this sense, Boole can be seen as a kind of applied Kantian: taking the formal structure of reason and making it calculable.
    
    \begin{quote}
    \emph{Kant said the mind imposes order.  
    Boole made that order explicit.}
    \end{quote}
    
    Just as Kant imagined a logic that underlies all reasoning, Boole tried to distill that logic into equations — paving the way for everything from symbolic AI to formal verification in computer science.
    
    \medskip
    
    \textbf{Kant gave us the form of thought. Boole gave it syntax.}

\end{tcolorbox}
