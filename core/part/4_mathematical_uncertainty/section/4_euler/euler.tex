\section{Euler and the Expansion of Combinatorics}

\subsection{From Probability to Pure Structure}

While de Moivre grounded probability in combinatorial methods, Leonhard Euler (1707--1783) elevated combinatorics itself to a central and independent branch of mathematics.

Euler's work was broader in scope: he was less focused on direct applications to probability and more interested in the structural properties of counting, arrangements, and sequences. Yet, his innovations would later prove indispensable to the development of probability, statistics, and random processes.

\subsection{Generating Functions: Encoding Counting into Algebra}

One of Euler’s most profound contributions was the introduction of \textbf{generating functions}. By encoding sequences into formal power series, he created a powerful method for solving complex counting problems.

For example, a sequence \( a_0, a_1, a_2, \dots \) could be encoded as the series:

\[
G(x) = a_0 + a_1x + a_2x^2 + a_3x^3 + \cdots
\]

This algebraic framework allowed problems of enumeration to be transformed into problems of manipulating series — a revolutionary shift that greatly expanded the toolkit of mathematicians.

Today, generating functions are a cornerstone in fields such as stochastic processes, random walks, and combinatorial probability.

\subsection{Partitions, Permutations, and Expansions}

Euler also tackled the problem of \textbf{partitions}: determining how many ways an integer can be expressed as a sum of other integers. His insights into partition functions would later influence number theory, combinatorics, and statistical mechanics.

In addition, he advanced the theory of \textbf{binomial and multinomial expansions}, laying foundations for counting principles that would later be formalized into the study of \textbf{enumerative combinatorics}.

\subsection{Legacy for Probability and Beyond}

Although Euler himself did not focus directly on probability applications, the combinatorial methods he developed became essential tools for later probabilistic models:

\begin{itemize}
    \item Generating functions became standard in analyzing random processes and Markov chains.
    \item Partition functions influenced statistical mechanics and probability distributions.
    \item His work on permutations provided a foundation for modern combinatorial probability.
\end{itemize}

Euler's vision transformed combinatorics from an ad hoc collection of techniques into a structured mathematical discipline — one whose influence would eventually permeate probability theory, physics, and computer science.

\begin{quote}
    \textit{``Mathematics, rightly viewed, possesses not only truth, but supreme beauty.''} — (attributed to Euler's mathematical spirit)
\end{quote}
