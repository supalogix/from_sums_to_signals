\subsection{Why This Mattered — Especially After Weierstrass}

This wasn’t an idle metaphysical game. Analysis in the 1800s had uncovered functions so wild they seemed to defy everything calculus had promised.

\textbf{Weierstrass} had created functions that were continuous everywhere but differentiable nowhere — wild, erratic, mathematically legal monsters. But to even define such functions, you needed a solid concept of continuity. That meant you needed limits. And to define limits, you needed a complete number system.

Dedekind gave Weierstrass the ground to stand on.

By showing how to construct the reals from within the rationals — using nothing more than logical partitioning — Dedekind made it possible to treat even the strangest functions with mathematical precision.

\begin{tcolorbox}[colback=gray!5!white, colframe=black!80!white, title={Historical Sidenote: Filling the Gaps with Logic}]
When Dedekind introduced his cuts, some mathematicians resisted. The idea that a number could be defined by a gap — a hole between two sets — felt too abstract. But Dedekind insisted: it’s not the digits that matter, it’s the position.

As he wrote:
\begin{quote}
\textit{"The essence of number lies entirely in its being a part of a simply ordered system... determined solely by its position in that system."}
\end{quote}

With that philosophy, the real number line became not a collection of decimal strings, but a complete and ordered structure — no gaps, no guessing, no infinity without grounding.
\end{tcolorbox}
