\section{The History of How Simon Stevin Turned Fractions into Decimals}

\subsection{Background: Before Stevin}

Decimal fractions existed long before Simon Stevin. They appear in the works of Islamic mathematicians like Al-Kashi (15th century) and were known to European astronomers such as Regiomontanus and Copernicus. But these decimals were usually limited to specialized contexts — like astronomical tables — and often written in sexagesimal (base-60) notation.

In daily life before decimals became widespread, most people used \textbf{common fractions} like \( \frac{1}{2} \), \( \frac{3}{4} \), or \( \frac{5}{8} \) to represent quantities. These worked fine for simple tasks—like splitting a pie or measuring fabric—but they quickly became a headache in more complicated calculations.

Imagine you're a merchant at a 17\textsuperscript{th}-century market trying to total the cost of three purchases:

\begin{itemize}
    \item \( \frac{3}{4} \) of a pound of pepper,
    \item \( \frac{5}{8} \) of a yard of cloth,
    \item \( \frac{7}{10} \) of a gallon of wine.
\end{itemize}

To add these or compare them, you'd need to find a common denominator, convert each fraction, perform the addition, and then reduce the result—all by hand. It's easy to make mistakes and difficult to keep track.

Now imagine the same problem using decimals:

\begin{itemize}
    \item 0.75 pounds of pepper,
    \item 0.625 yards of cloth,
    \item 0.7 gallons of wine.
\end{itemize}

You can now add them directly:
\[
0.75 + 0.625 + 0.7 = 2.075
\]

No common denominators. No reducing. Just addition.

This was the kind of simplicity Simon Stevin championed. Decimals weren’t just easier—they made math more \textit{mechanical, reliable, and accessible}, especially for tradespeople, navigators, and engineers who couldn’t afford mistakes.


In short: fractions might be fine for cutting pies, but decimals built bridges, sailed ships, and balanced ledgers.

\subsection{Stevin’s Decimal Revolution: Powers of Ten for the People}

In 1585, Simon Stevin published a slim but radical treatise titled \textit{De Thiende} (“The Tenth”). Though modest in appearance, its message was revolutionary: Stevin argued that decimals shouldn’t just be an occasional convenience—they should be the default language of arithmetic. His vision was simple but transformative: any number, no matter how clumsy or irregular, could be written as a sum of tenths, hundredths, thousandths, and so on.

Stevin was a huge proponent of \textbf{decimal fractions} and \textbf{positional notation}, both of which depend on the \textbf{Hindu-Arabic numeral system}—a base-10 system using the digits 0–9. While he didn’t introduce Arabic numerals to Europe (that credit goes to earlier figures like Fibonacci), Stevin’s work in \textit{De Thiende} helped normalize their use in everyday calculation, especially in commerce and science.

In contrast to the cumbersome Roman numerals, Stevin’s notation allowed for streamlined, modular computation. Consider the number 1987:

\begin{center}
\begin{tabular}{|c|c|}
\hline
\textbf{Hindu-Arabic Numerals} & \textbf{Roman Numerals} \\
\hline
1987 & MCMLXXXVII \\
2024 & MMXXIV \\
753 & DCCLIII \\
49 & XLIX \\
8 & VIII \\
\hline
\end{tabular}
\end{center}

Each digit in the Hindu-Arabic system occupies a place value—units, tens, hundreds, thousands—making operations like addition, multiplication, and decimal extension algorithmic. Roman numerals, by contrast, lacked a concept of place value and zero, and were better suited for inscriptions than for calculations.

Stevin’s proposal wasn’t just a convenience—it was a philosophical stance. Where most people used fractions like \( \frac{1}{2} \) or \( \frac{3}{4} \), Stevin proposed writing \( 0.5 \) and \( 0.75 \), eliminating the tedious need to find common denominators or reduce awkward fractions by hand. His decimal notation made arithmetic easier to teach, easier to mechanize, and far more scalable for real-world tasks like surveying land, calculating taxes, navigating oceans, or designing engineering systems.

Although he didn’t invent the decimal point—that refinement would come later from Napier and others—Stevin used superscripts to denote decimal places: \( 2^{0}1^{1}4^{2} \) was his way of writing what we now call 2.14. The superscripts indicated the place value: units, tenths, hundredths, and so forth. It looked unusual, but beneath it lay a stunningly modern insight: any number can be decomposed into powers of ten.

In \textit{De Thiende}, Stevin declared, ``There is no number which cannot be written by using only the unit, ten, and its powers.'' To him, fractions were not just outdated—they were obstacles to progress. Decimals, on the other hand, opened the door to algorithmic thinking. They made arithmetic systematic, modular, and ripe for automation. And in a time when Latin was still the language of scholars, Stevin wrote \textit{De Thiende} in Dutch, a deliberate act of democratization. Just as the Reformation brought Scripture to the people, Stevin brought mathematics out of the cloister and into the marketplace.



\begin{tcolorbox}[title=Historical Sidebar: Fibonacci and the Numbers That Changed Everything, colback=gray!5, colframe=black]
    In 1202, a young Italian mathematician named \textbf{Leonardo of Pisa}—better known today as \textbf{Fibonacci}—published a book that would quietly reshape European mathematics. Titled \textit{Liber Abaci} (The Book of Calculation), it introduced the Hindu-Arabic numeral system to a Latin-reading audience. 
    
    This system, which used just ten symbols (0–9) in a positional format, was already widespread in the Islamic world. Fibonacci had encountered it while traveling through North Africa, where he observed merchants and mathematicians using these symbols with ease and efficiency. Compared to Roman numerals—clumsy for arithmetic and practically useless for algebra—the Hindu-Arabic system was compact, versatile, and profoundly scalable.
    
    Fibonacci's book explained how to use these numerals for practical tasks: bookkeeping, currency exchange, weights and measures, and interest calculations. But adoption was slow. Many merchants and clerics were skeptical, even suspicious. Roman numerals were familiar; abacuses were trusted. The zero was especially controversial—seen by some as heretical or even “dangerous” due to its foreign (and non-Christian) origins.
    
    It wasn’t until the 15th and 16th centuries—thanks to printing, accounting needs, and advocates like \textbf{Simon Stevin}—that Arabic numerals truly took hold. By Stevin’s time, they were poised to become the default. But it was Fibonacci who had first opened the door.
\end{tcolorbox}





\subsection{From Theory to Practice: Stevin’s Decimal Arithmetic in the Real World}

Simon Stevin wasn’t content with proposing a new number system—he wanted to put it to work. In \textit{De Thiende}, he didn’t just theorize about decimals; he demonstrated how they could simplify daily calculations in trade, engineering, and administration.

At the time, most practical arithmetic was done with common fractions like \( \frac{3}{8} \) or \( \frac{5}{12} \), which were hard to add, subtract, or multiply—especially without a chalkboard or abacus. Stevin showed that by converting everything into decimals, computations could be made far more straightforward.

\textbf{In trade}, merchants constantly calculated interest rates, exchanged currencies, and converted units of measurement. Instead of juggling awkward fractions like \( \frac{7}{16} \), Stevin’s system let them write \( 0.4375 \) and compute with simple rules of place value.

\textbf{In land surveying}, Stevin demonstrated how to calculate areas and distances using decimals, making it easier to divide plots and compute taxes. Surveyors could now work with decimal feet or decimal acres, rather than struggling with traditional units like “rods” or “chains.”

\begin{table}[H]
    \centering
    \caption{Land Tax Calculation Using Stevin’s Decimal Notation}
    \renewcommand{\arraystretch}{1.3}
    \begin{tabular}{|c|c|c|c|}
    \hline
    \textbf{Parcel} & \textbf{Area (acres)} & \textbf{Tax Rate (per acre)} & \textbf{Total Tax} \\
    \hline
    A & $1^{0}5^{1}$ (1.5 acres) & $2^{0}0^{1}$ (2.0 units) & $3^{0}0^{1}$ (3.0 units) \\
    B & $2^{0}7^{1}5^{2}$ (2.75 acres) & $1^{0}6^{1}$ (1.6 units) & $4^{0}4^{1}0^{2}$ (4.40 units) \\
    C & $0^{0}6^{1}2^{2}5^{3}$ (0.625 acres) & $3^{0}0^{1}$ (3.0 units) & $1^{0}8^{1}7^{2}5^{3}$ (1.875 units) \\
    \hline
    \multicolumn{3}{|r|}{\textbf{Total Tax:}} & $9^{0}2^{1}7^{2}5^{3}$ (9.275 units) \\
    \hline
    \end{tabular}
\end{table}

\textbf{In engineering and construction}, decimals enabled precise measurements and tolerances. Builders no longer needed to reduce fractions of an inch; they could work with values like \( 0.125 \) inches instead of \( \frac{1}{8} \), streamlining design and fabrication.

\textbf{In navigation}, where precision meant safety, Stevin’s decimals allowed sailors to work with fractional degrees of latitude and longitude more easily than ever before. His ideas laid the groundwork for more accurate maps and instruments.

To support this practical agenda, Stevin provided worked examples of real problems solved with decimals—everything from computing interest on a loan to estimating timber yields from a forest. His tables and methods were designed to be usable by shopkeepers, ship captains, and surveyors—not just scholars. In doing so, Stevin turned arithmetic from an academic exercise into an everyday tool.

\begin{quote}
For Stevin, decimals weren’t just a mathematical convenience. They were a technology of clarity—bringing precision to trade, fairness to taxation, and order to the physical world.
\end{quote}


\begin{tcolorbox}[colback=gray!5!white, colframe=black!80!white, title={Historical Sidenote: Stevin’s Theology of Decimal Math}, fonttitle=\bfseries, arc=1.5mm, boxrule=0.4pt]

    Simon Stevin didn’t just believe that decimals were useful; he believed they were \textit{right}. Behind his mathematical reforms was a deeply Calvinist conviction: that knowledge should be accessible to all, and not hoarded by elites. Just as Reformers like Luther argued that Scripture should be available in the language of the people, Stevin insisted that mathematics should be written in the vernacular; so, he published all his work in Dutch.

    \medskip

    At the time, Latin was the language of the educated class: the default tongue of scholars, clergy, and scientific elites. To write math in Dutch was, for Stevin, a theological rebellion. It wasn’t just a pedagogical choice. It was a statement: truth does not require translation.

    \medskip
    
    Stevin believed that mathematics was a gift from God meant for everyone... not just scholars cloistered in Latin. In his eyes, decimals weren’t simply more efficient than fractions; they reflected a divine clarity embedded in creation itself.

    \medskip
    
    His motto, \textit{“Wiskunde voor de Gemeene Man”} (“Mathematics for the Common Man”), wasn’t a slogan. It was a mission. Decimal notation, written in the language of bakers and shipbuilders, was part of a broader vision: that truth --- mathematical and spiritual --- belonged to the people.

\end{tcolorbox}


