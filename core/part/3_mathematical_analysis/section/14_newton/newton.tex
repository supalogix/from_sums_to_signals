\section{Newton: Infinite Series and the Emerging Calculus (1670s)}

If Barrow wielded geometry like a draftsman’s compass, Newton picked up algebra like a scalpel.

While studying Barrow’s visual constructions, Newton began to reframe them in terms of power series—expressing curves as infinite polynomials. His goal was to tame curved motion and varying forces with the rigor of algebra, and he succeeded spectacularly.

Newton discovered that many functions—square roots, logarithms, sines, cosines—could be expressed as infinite sums. These expansions gave him a new kind of calculus, one that didn't rely on geometry at all:

\[
\sqrt{1 + x} = 1 + \frac{1}{2}x - \frac{1}{8}x^2 + \frac{1}{16}x^3 - \cdots
\]

\[
\log(1 + x) = x - \frac{x^2}{2} + \frac{x^3}{3} - \frac{x^4}{4} + \cdots
\]

\[
\sin x = x - \frac{x^3}{3!} + \frac{x^5}{5!} - \cdots \qquad \cos x = 1 - \frac{x^2}{2!} + \frac{x^4}{4!} - \cdots
\]

These weren’t just numerical tricks. They let Newton \textit{differentiate and integrate} without ever drawing a curve. He could now solve problems of motion, area, and force entirely through algebraic manipulation.

\vspace{1em}

Interestingly, Newton’s work also brought him close to a profound mathematical constant:

\[
e^x = 1 + x + \frac{x^2}{2!} + \frac{x^3}{3!} + \cdots
\]

But he never singled out the number \( e \) itself. For Newton, \( e^x \) was just another series—useful, yes, but not yet sacred. It would take Euler, nearly a century later, to elevate \( e \) to the centerpiece of exponential growth, complex numbers, and continuous change.

\vspace{1em}

\begin{center}
\textit{Barrow saw calculus in the shapes of curves. Newton saw it in the symmetries of series.}
\end{center}
