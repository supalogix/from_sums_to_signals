\subsection{From Fields to Continuum: What the Reals Really Are}

By the late 19\textsuperscript{th} century, mathematics wasn’t just solving equations — it was defining the very terrain in which equations lived. One key terrain was the \textbf{field}: a system where addition and multiplication behave predictably. Specifically, a field satisfies:

\begin{itemize}
    \item \textbf{Commutativity}: \( a + b = b + a \), and \( ab = ba \)
    \item \textbf{Associativity}: \( (a + b) + c = a + (b + c) \)
    \item \textbf{Distributivity}: \( a(b + c) = ab + ac \)
    \item \textbf{Inverses}: Every element (except 0 for multiplication) has an inverse
\end{itemize}

Rational numbers \( \mathbb{Q} \) form such a field. But a field, even a well-behaved one, isn’t enough to define continuity.

To make sense of limits, convergence, and the idea of getting arbitrarily close to something, you need more than a field. You need an \textbf{ordered field} — a field with a total ordering \( < \), compatible with its arithmetic.

\begin{itemize}
  \item If \( a < b \), then \( a + c < b + c \)
  \item If \( 0 < a \) and \( 0 < b \), then \( 0 < ab \)
\end{itemize}

And yet even this isn’t enough.

Because neither a field nor an ordered field guarantees completeness. Only the real numbers \( \mathbb{R} \) do — and what makes them special is the least upper bound property.
