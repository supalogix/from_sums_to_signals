\subsection{Carnap’s Categorical Pivot: When Truth Became a System Setting}

While Gödel shattered the dream of complete formal foundations, not everyone gave up the quest. One figure—less tragic than Gödel, more pragmatic than Wittgenstein—quietly rerouted the entire conversation.

\textbf{Rudolf Carnap}, founding member of the Vienna Circle and high priest of logical positivism, had once believed that science and mathematics could be encoded in a single universal language of logic. But after Gödel’s incompleteness theorems detonated that dream, Carnap didn’t retreat into mysticism or linguistic relativism.

He pivoted.

Instead of demanding a single, true foundation, Carnap proposed something far more flexible—and more radical: \textbf{truth is internal to a system}. What matters isn’t whether a statement is “true” in some absolute sense, but whether it’s \textit{derivable} from the axioms of a chosen formal framework.

In other words: 

\begin{quote}
Mathematics is a build-your-own-universe kit. You pick your axioms. You pick your rules. And the truths that follow are yours—relative to that world.
\end{quote}

This shift wasn’t a retreat from rigor. It was an architectural revolution. Carnap wasn’t abandoning logic—he was modularizing it. He introduced the idea of \textbf{logical frameworks}: meta-systems within which you can define other systems. Instead of asking “Is this statement true?”, you ask:

\begin{quote}
Given these rules, these axioms, and this inference system—what follows?
\end{quote}

This modular, relativized approach to logic set the stage for modern formal semantics, model theory, and even type theory. And it fits remarkably well with the category-theoretic mindset that emerged decades later.

\vspace{1em}

\begin{tcolorbox}[colback=gray!5!white, colframe=black, title=\textbf{Sidebar: Category Theory as Carnap’s Heir}, fonttitle=\bfseries, arc=1.5mm, boxrule=0.4pt]

Category theory doesn’t ask “What is this object?”  
It asks: \textbf{How does it relate to everything else?}

And that’s pure Carnap.

Where set theorists seek absolutes and constructivists seek purity, category theorists are the agnostic engineers of mathematics. For them, a concept is meaningful only insofar as it participates in a structure—via \textbf{morphisms}, not membership.

\medskip

\textbf{Carnap:} Truth is relative to a formal language.  
\textbf{Category theorists:} Language is just a category. Pick one.

\medskip

In this view, a “foundation” isn’t an unshakeable ground—it’s a user-defined context. You don’t ask whether an object exists. You ask: \textit{Can I define a morphism to it?}

\end{tcolorbox}

\vspace{1em}

Carnap’s modular view of truth influenced generations of logicians, from Tarski to Kripke. But its deepest resonance may be with category theory, where even logic becomes functorial—something that maps between worlds.

When Gödel broke the absolute, Carnap didn’t mourn it.  
He just made truth configurable.

\begin{quote}
Truth is no longer a mountain to climb. It’s a coordinate system to choose.
\end{quote}



\section{Certainty Without Knowing: The Rise of Concentration Inequalities}
\begin{tcolorbox}[title={\textbf{Historical Sidebar: After Gödel — Carnap and Wittgenstein at the Crossroads}}, colback=gray!5, colframe=black, fonttitle=\bfseries]

    \textbf{1931:} Gödel proves that no consistent formal system powerful enough to do arithmetic can be both complete and consistent. The foundational dreams of Hilbert and the logicists begin to unravel.
    
    \medskip
    
    \textbf{Carnap's Response:} Rather than abandon formalism, Carnap pivots. In works like \textit{The Logical Syntax of Language}, he proposes that truth is not absolute but \textit{internal to a formal system}. His solution: \textbf{framework pluralism}. Instead of one universal logic, we choose a language system and derive within it. For Carnap, metaphysical questions like “Do numbers exist?” are meaningless unless defined within a linguistic framework.
    
    \medskip
    
    \textbf{Wittgenstein's Response:} Wittgenstein, having already broken with his earlier logical atomism in the \textit{Tractatus}, takes a more radical route. He abandons the search for absolute formal precision. In his later philosophy, especially \textit{Philosophical Investigations}, meaning arises not from symbolic structure but from \textbf{use in everyday contexts}. Language is a set of “games” — each with its own rules, moves, and meanings.
    
    \medskip
    
    \textbf{The Philosophical Break:} Carnap still believed in formal systems — he just didn’t believe in any single one. Wittgenstein gave up on formalism entirely. Where Carnap saw language as a logic engine, Wittgenstein saw it as a playground of practices. Gödel closed one door; Carnap reinforced the hinges, Wittgenstein walked away.
    
\end{tcolorbox}
