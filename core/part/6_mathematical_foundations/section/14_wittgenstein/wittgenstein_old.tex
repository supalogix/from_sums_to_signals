\section{Emergent Structure in Deep Learning: A Pragmatist Interpretation}

\subsection{Wittgenstein’s Later Turn: From Logic to Use}

In his early work, \textit{Tractatus Logico-Philosophicus}, \textbf{Ludwig Wittgenstein} sought to describe the world through logical propositions that mirrored reality. But in his later work, especially \textit{Philosophical Investigations}, he radically changed course. Meaning, he argued, is not a matter of internal correspondence to some abstract essence—it is a function of \emph{use}.

This gave rise to his famous concept of \textbf{language games}: the idea that the meaning of a word arises from the context in which it is used, embedded in a form of life, not from some intrinsic or symbolic mapping.

\begin{quote}
    ``For a large class of cases—though not for all—in which we employ the word 'meaning' it can be defined thus: the meaning of a word is its use in the language.''\\
    — \textit{Philosophical Investigations} \S43
\end{quote}

This shift from definition to use, from symbol to social behavior, echoes in modern deep learning. Neural networks do not learn meanings in any ontological sense. They learn \emph{patterns of use}—statistical regularities across massive corpora of human activity.

\subsection{From Symbolic Representations to Emergent Concepts}

Traditional approaches to AI treated knowledge as a system of symbols and rules. Concepts were handcrafted, encoded explicitly, and manipulated via logic. But deep learning abandoned this model. Instead of programming knowledge, we let the model \emph{discover} it.

A neural network is not given an explicit structure for ``cat,'' ``strategy,'' or ``trust.'' It forms distributed representations—vectors in high-dimensional space—that acquire functional meaning by how they behave in the network’s training regime.

These internal representations are not interpretable in the classical sense. There is no fixed symbol-to-meaning map. Instead, their ``meaning'' is determined pragmatically: through the transformations they enable, the predictions they improve, and the losses they minimize.

\subsection{Implementation: How Emergence Happens}

In practice, emergent structure arises from the interaction of three core elements:

\begin{itemize}
    \item \textbf{Architecture}: Layered, compositional systems (e.g., transformers, CNNs) allow abstraction to emerge hierarchically.
    \item \textbf{Objective Function}: The model’s ``game''—what it’s trying to win. This includes losses like cross-entropy, contrastive learning objectives, or even reward functions in reinforcement learning.
    \item \textbf{Optimization Dynamics}: Gradient descent and stochastic updates iteratively sculpt the weight space to encode functional structure.
\end{itemize}

These components jointly define an \textbf{information ecology} where structure is not imposed from above but \emph{discovered through use}. Features like part-whole hierarchies, syntactic trees, or causal variables may emerge—not because the network is told to look for them, but because those structures are statistically useful for minimizing error.

\subsection{Wittgenstein and the Neural Epistemology}

Just as Wittgenstein denied the possibility of a ``private language'' with no social grounding, we may ask: does a neural network's internal representation have meaning if it cannot be interpreted?

Perhaps not in the symbolic sense. But in the pragmatist sense—does it work? Does it generalize? Does it let the model win its language game?

Then yes. It means.

\vspace{0.5em}
\begin{quote}
    \emph{In the end, meaning is not given—it is earned.}
\end{quote}













\section{Emergent Structure in Deep Learning: A Pragmatist Interpretation}

\subsection{Wittgenstein’s Later Turn: From Logic to Use}

In his early work, \textit{Tractatus Logico-Philosophicus}, \textbf{Ludwig Wittgenstein} sought to describe the world through logical propositions that mirrored reality. But in his later work, especially \textit{Philosophical Investigations}, he radically changed course. Meaning, he argued, is not a matter of internal correspondence to some abstract essence—it is a function of \emph{use}.

This gave rise to his famous concept of \textbf{language games}: the idea that the meaning of a word arises from the context in which it is used, embedded in a form of life, not from some intrinsic or symbolic mapping.

\begin{quote}
    ``For a large class of cases—though not for all—in which we employ the word 'meaning' it can be defined thus: the meaning of a word is its use in the language.''\\
    — \textit{Philosophical Investigations} \S43
\end{quote}

This shift from definition to use, from symbol to social behavior, echoes in modern deep learning. Neural networks do not learn meanings in any ontological sense. They learn \emph{patterns of use}—statistical regularities across massive corpora of human activity.

\subsection{Wiener and McLuhan: Meaning in Feedback and Form}

Wittgenstein’s later philosophy suggested that meaning arises from use, not definition. In the mid-20th century, this insight found an unexpected parallel in the work of two thinkers who approached language and communication not from philosophy, but from engineering and media theory.

\textbf{Norbert Wiener}, the founder of cybernetics, reframed communication as a dynamic, self-regulating process. In his view, systems—biological, mechanical, or social—maintain their stability not through fixed codes, but through \emph{feedback}: the constant exchange of signals, corrections, and responses. Information was not a passive label but an active component in behavior. Meaning emerged through interaction, not representation.

\medskip

\textbf{Marshall McLuhan}, writing two decades later, turned attention to the role of media in shaping perception and meaning. His famous aphorism—\textit{``the medium is the message''}—challenged the assumption that content was primary. Instead, he argued that the form of communication itself (print, television, radio, etc.) alters how people think, speak, and organize society. Just as the rules of a game shape what moves are possible, the medium sets the grammar for cultural expression.

\begin{quote}
``It is the medium that shapes and controls the scale and form of human association and action.''\\
— \textit{Marshall McLuhan}, \textit{Understanding Media}
\end{quote}

\noindent
Where Wittgenstein saw meaning as practice, Wiener and McLuhan saw it as \emph{process and infrastructure}. Meaning becomes less a thing to be extracted and more a \emph{relation}—a function of feedback loops, systems constraints, and the architecture through which signals travel.

\medskip

\noindent
Together, their work suggests a broader view of language and thought: not as fixed structures built from definitions, but as evolving systems shaped by tools, media, and continuous interaction with their environments.

\subsection{Lakoff and Johnson: Metaphors We Live By}

If Wittgenstein taught us that meaning arises from use, and McLuhan argued that media shape meaning, then \textbf{George Lakoff} and \textbf{Mark Johnson} took the next step: showing that even our most basic conceptual structures are metaphorical.

In their influential 1980 book, \textit{Metaphors We Live By}, Lakoff and Johnson argued that metaphors are not just poetic devices—they are foundational to human thought. We don't simply \emph{use} metaphors to describe the world; we \emph{understand} the world through them.

\begin{quote}
    ``Our ordinary conceptual system, in terms of which we both think and act, is fundamentally metaphorical in nature.''\\
    — \textit{Metaphors We Live By}
\end{quote}

For example:
\begin{itemize}
    \item \textbf{Argument is war}: ``He shot down my argument.'' ``She attacked my position.''
    \item \textbf{Time is money}: ``You're wasting my time.'' ``How do you spend your days?''
    \item \textbf{Ideas are objects}: ``She gave me a lot to think about.'' ``That's a weighty idea.''
\end{itemize}

These metaphors are not decorative—they structure our reasoning, frame our judgments, and guide our actions. In doing so, they reveal something deeper: our conceptual system is not purely abstract. It is shaped by our embodied experience of space, time, motion, and interaction.

Lakoff and Johnson’s work extends the pragmatist lineage of Wittgenstein and McLuhan. Where Wittgenstein saw meaning in use, and McLuhan saw meaning in media, Lakoff and Johnson show that meaning is embedded in metaphor—often unconsciously, and always culturally grounded.


This shift has profound implications: if our reasoning is metaphorical, then our most abstract systems—logic, morality, mathematics—are not exempt. They, too, are shaped by the metaphors we live by.

\subsection{Wittgenstein as a Proto-Postmodernist: Truth as a Social Construction}

Wittgenstein’s later philosophy, especially in \textit{Philosophical Investigations}, is often viewed as a precursor to postmodern thought. His turn away from the rigid logical structures of his early work toward a view of language as inherently contextual and use-driven anticipated many of the central claims of postmodernism.

In particular, his insistence that \textbf{meaning is use} undermined the notion of an objective, correspondence-based truth. If language does not mirror reality, but rather constructs it through practice, then truth itself becomes something that is \emph{socially embedded}. It does not float freely above discourse—it emerges from the way communities speak, act, and interpret.

\begin{quote}
Truth is not discovered—it is negotiated. It arises within a form of life, embedded in shared practices and contingent language games.
\end{quote}

This shift away from objective foundations led to entire methodologies designed to model this socially constructed truth. One of the most prominent is \textbf{grounded theory}—a research approach that does not begin with hypotheses, but allows theories to emerge from patterns in data through iterative coding and interpretation.

At its heart, grounded theory is a post-positivist way of building ontologies. Rather than assuming a pre-existing structure of the world waiting to be uncovered, it acknowledges that \textbf{ontologies are constructed}—and must reflect the categories and distinctions used by participants within a community or practice.

This stands in contrast to classical ontology schemes, such as \textbf{Linnaean classification} in biology, which presuppose an objective world segmented into natural kinds that language can cleanly represent. In a Wittgensteinian world, such taxonomies are themselves language games—historically contingent, culturally embedded, and constantly evolving.

Thus, disciplines like information science, qualitative sociology, and cognitive anthropology began to shift: moving from mirror theories of language to \textbf{use-based models} of conceptual structure. What once were thought to be universal categories became seen as \emph{situated ontologies}, co-constructed by observers and participants alike.

\begin{quote}
If there is no view from nowhere, then every ontology is local. Every truth is entangled in language. Every classification is a reflection of use.
\end{quote}

\subsection{Information Ecologies and Eigenvector Centrality: Language in Motion}

If truth is not fixed but constructed through use, then the structures we build to navigate knowledge must adapt to the shifting contours of language. This is precisely what modern search engines attempt to do—not by imposing static hierarchies of knowledge, but by continuously reorganizing information based on how people use it.

A prime example is \textbf{eigenvector centrality}, the mathematical backbone of early search algorithms like Google’s PageRank. Rather than rank websites based on their content alone, PageRank considers how other sites link to them. In essence, a site is important if it is linked to by other important sites. This recursive metric reflects not some fixed notion of truth, but a constantly evolving \textbf{information ecology}, shaped by use.

\begin{quote}
Information doesn’t just live in documents—it flows through a web of references, shaped by how people search, link, click, and write.
\end{quote}

In this model, relevance is not defined a priori—it is emergent. It reflects collective behavior over time, embedding public language games into the algorithmic structure of the web. A search engine, then, becomes a \textbf{dynamic ontology}: a system that models the world not as it is, but as it is described and used.

This is deeply Wittgensteinian. Like a language game, the meaning of a term on the web arises not from its definition but from its network of usage. Pages gain meaning through interaction, context, and shared reference—mirroring the social construction of concepts.

\begin{itemize}
    \item A rarely linked page disappears from view—regardless of its “objective” value.
    \item A term’s importance rises or falls with the language trends of users.
    \item Ontologies evolve not through debate, but through usage patterns and statistical inference.
\end{itemize}

\medskip

\noindent
In this sense, search engines do not merely reflect what we know—they shape how we know. By embedding usage patterns into ranking algorithms, they instantiate a living, computational version of Wittgenstein’s world: one in which \textbf{meaning is always being negotiated}, and where the map of knowledge is always under revision.

\begin{quote}
Every query is a move in a language game. Every link is a vote in a decentralized ontology.
\end{quote}
