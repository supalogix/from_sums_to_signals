\subsection{Revenge of the Unmeasurable: Infinity Strikes Back}

Gödel’s theorem didn’t just upend logic; it threw a Molotov cocktail into the already-smoldering debate about infinity. At the heart of the firestorm was the \textbf{Continuum Hypothesis (CH)}:

\begin{quote}
Is there an infinity between the size of the natural numbers () and the real numbers ()?
\end{quote}

Cantor had suspected the answer was no. Hilbert made proving or disproving CH the first challenge of the 20th century. But then Gödel and Cohen came along and pulled the rug out:

\begin{itemize}
\item Gödel showed that CH \textbf{cannot be disproven} using standard set theory (ZFC).
\item Cohen then showed that CH \textbf{cannot be proven} either.
\end{itemize}

Which meant that the truth of CH wasn’t just unknown—it was \textit{unknowable}. You could assume CH was true. You could assume it was false. \textbf{Both were equally valid.} Welcome to the multiverse of mathematical reality.
