\subsection{The Problem with Aristotle’s Motion: Forced Motion Needs a Constant Push}  

Aristotle’s framework was elegant—and deeply explanatory. He wasn’t just trying to describe motion; he wanted to define what it *was*. In his view, motion (\textit{kinesis}) was the actualization of a potential—something that could change, beginning to change, under the right conditions. This included not just movement through space, but also changes in quality, quantity, and even being.  

\begin{quote}
\textbf{Motion, to Aristotle, was a process of becoming.}  And it had rules. Everything had a “natural state,” and when something moved in defiance of that state—like a rock flying through the air—it needed help. A cause. A push. A reason. The minute that push stopped, the object would stop too.  
\end{quote}

This view fit neatly into his broader philosophy of nature, with its hierarchy of causes—material, formal, efficient, and final. Motion wasn’t random or spontaneous. It was explainable. Predictable. Anchored in purpose.

\textbf{And underneath it all was a powerful idea:}  Something can have the *capacity* to move (a potential), even before that motion is realized.  

That distinction—between \textbf{potentiality and actuality}—would echo forward, centuries later, as physicists started describing objects in terms of **stored energy** (what they could do) and **active motion** (what they were doing).  

But Aristotle didn’t go there. For him, the motion wasn’t something you could measure quantitatively like energy. It was a transformation in the nature of a thing.  

Still, a core question remained: \textbf{If something is moving, but there’s nothing pushing it, why doesn’t it stop?}

To Aristotle, this wasn’t a problem. Resistance (like air or friction) would always slow things down. Without constant input, motion would fizzle out.  

\begin{figure}[H]
\centering
\begin{tikzpicture}[every node/.style={font=\footnotesize}]

% Panel 1 — questioning modern student
\comicpanel{0}{4}
  {Student}
  {Aristotle}
  {Wait, you're saying heavy things fall because... they want to?}
  {(0,-0.5)}

% Panel 2 — Aristotle explains earnestly
\comicpanel{6.5}{4}
  {Student}
  {Aristotle}
  {Yes. Everything strives toward its natural place. Things fall because that’s where they belong.}
  {(0,-0.5)}

% Panel 3 — Student raises eyebrow
\comicpanel{0}{0}
  {Student}
  {Aristotle}
  {So if I drop a rock, it’s just fulfilling its destiny?}
  {(0,0.8)}

% Panel 4 — Aristotle drops the mic
\comicpanel{6.5}{0}
  {Student}
  {Aristotle}
  {Yes. And stars move in circles because they are divine. Don’t hate the motion: respect the purpose.}
  {(0,0.8)}

\end{tikzpicture}
\caption{Aristotle’s teleological physics: where everything moves with meaning.}
\end{figure}
