\subsection{On Not Being Able to Even: Gödel, Wittgenstein, and the Existential Comedy of Math}

By now, you might be wondering: how did Gödel feel about all this? Did he sit back and chuckle while mathematicians ran headfirst into walls he built with pure logic?

Not exactly. Gödel was deeply serious, even spiritual, about mathematics. He believed mathematical truths were as real as stars or atoms—just harder to find. To him, incompleteness wasn’t a flaw in math; it was a revelation. A warning label stamped on the foundations of thought: “There are truths you will never reach, no matter how clever you are.”

\noindent Wittgenstein, on the other hand, read Gödel’s paper and famously declared it nonsense. He thought it was all a linguistic trick—that Gödel hadn’t discovered anything deep about mathematics, just about the weird ways we talk about it.

\medskip

\noindent This clash wasn’t just academic; it was personal and philosophical. Wittgenstein had once been seen as a kind of prophet by the logical positivists. His early work, the \emph{Tractatus Logico-Philosophicus}, outlined a vision of the world built from logical ``atoms'': a theory known as \textbf{logical atomism}. The Vienna Circle revered him. They saw in his work the blueprint for a purified language of science and mathematics, stripped of ambiguity.

But Wittgenstein changed his mind. In his later years, he rejected the idea of rigid logical structures as the foundation of meaning. Instead, he proposed that meaning comes from \textbf{language games}: the social, contextual use of language in specific settings. In this view, mathematics wasn’t a realm of Platonic truth, but just one more game we play with symbols and rules.

Gödel, meanwhile, had been loosely connected to the Vienna Circle but stood apart in temperament and conviction. While the Circle emphasized empiricism and verification, Gödel believed in a deep, Platonic reality behind mathematics --- truths that existed independently of our proofs or perceptions.

Gödel didn’t just publish a theorem—he detonated a philosophical bomb under the foundations of logic itself. His incompleteness results so thoroughly shattered the dreams of the logical positivists that the entire landscape of 20th-century thought shifted in response. The Vienna Circle had hoped for a crystalline, complete language of science grounded in logic, but Gödel showed that even arithmetic—logic’s supposed bedrock—contained truths that logic alone could never reach. The fallout was existential. Wittgenstein, once their intellectual lodestar, pivoted from logical atomism to language games, drifting toward a postmodernist view where meaning was social, not structural. Popper, seeking stability, offered falsification in place of verification. Others, like Quine and later pragmatists, abandoned the idea of foundational knowledge altogether, embracing the contingency of belief systems and the shifting utility of models. 

\begin{quote}
And so we’re left, even today, staring at the statue of science and wondering if its gleaming form stands on feet of clay.
\end{quote}


This sparked one of the most quietly savage standoffs in intellectual history. Gödel, the shy logician who could barely give a lecture without collapsing from stress. Wittgenstein, the brooding philosopher who once waved a fireplace poker at Karl Popper while debating the merits of falsification. Each thought the other was missing the point. 

And the rest of the math world? It fractured into four tribes:

\begin{itemize}
    \item The \textbf{Formalists} shrugged and said, “Let’s just keep proving stuff and hope the roof doesn’t collapse.”
    \item The \textbf{Intuitionists} muttered, “Told you so,” and went off to live in a hut where infinity doesn’t exist.
    \item The \textbf{Platonists} raised a glass to Gödel, muttering, “The Truth is out there,” like mathematical Mulder cultists.
    \item The \textbf{Pragmatists} said, “Does it run on silicon? Good enough for us.”
\end{itemize}


\medskip

\begin{tcolorbox}[colback=blue!5!white, colframe=blue!50!black, title=Historical Sidebar: Truth vs Games]

  \textbf{Kurt Gödel} believed that mathematical truths were \emph{real}. Not “useful” or “agreed upon” or “symbolically convenient”—real, as in floating somewhere in a timeless Platonic realm. His incompleteness theorems weren’t just technical results; they were existential declarations: \emph{Some truths are forever beyond formal proof.}
  
  \medskip
  
  \textbf{Ludwig Wittgenstein} wasn’t impressed.  He read Gödel’s theorem and decided it wasn’t metaphysical insight—it was a linguistic misunderstanding. To him, mathematics was a \textbf{language game}: a structured way we talk and reason, shaped by context and use. If your system can’t prove something, maybe you’re just playing the wrong game.
  
  \medskip
  
  Where Gödel saw revelation, Wittgenstein saw confusion. Gödel said, “There are truths you’ll never prove.” Wittgenstein replied, “Define ‘truth.’”
  
  \medskip
  
  Their philosophical tension came to a head during a now-legendary seminar in Cambridge, where Wittgenstein dismissed Gödel’s theorem \emph{without reading it fully}, claiming it was a trick of syntax masquerading as depth. Gödel, shy and soft-spoken, never responded publicly. But privately? He was horrified. He thought Wittgenstein had missed the point entirely.
  
  \medskip
  
  \textbf{The result?} A stalemate worthy of the ages.  Gödel thought mathematics was a window into the eternal.  Wittgenstein thought it was a mirror reflecting our rules.  And neither could convince the other that they were even playing the same game.
  
  \medskip
  
  \textbf{Legacy:}  
  If you're a Platonist, Gödel is your tragic prophet.  If you're a Post-Modernist, Wittgenstein is your knight in shinning armor. If you're a working mathematician, you probably just want your grant proposals approved and your journal articles accepted.
  
\end{tcolorbox}

\medskip

As for me?

There’s something poetic about it. We built an entire science to measure uncertainty, only to discover that some uncertainties measure us back. We wanted to calculate everything, and found a ceiling made of questions. And somewhere in the basement of the universe, a Vitali set is laughing quietly, unmeasurable and smug.

With his Incompleteness Theorems, Gödel shattered the Enlightenment dream from within. He proved that even in the most rigorous mathematical systems, there exist truths that cannot be proven within those systems. Mathematics, it turned out, wasn’t the airtight cathedral of reason the Enlightenment thinkers had hoped for. Reason had gaps, and Gödel wasn’t shy about where he thought those gaps pointed: toward a transcendent realm beyond human reason.

\begin{tcolorbox}[colback=blue!5!white, colframe=blue!50!black, title=Historical Sidebar: Truth vs Games]

While most logicians were busy tightening the screws of formal systems, \textbf{Kurt Gödel} was quietly sketching a proof\ldots{} of \textbf{God}.

\medskip

As a committed \textbf{Lutherian}, Gödel didn’t see belief in God as just a matter of faith or tradition. It was, for him, the inevitable outcome of pure reason. Inspired by \textbf{Leibniz’s ontological argument}, Gödel set out to do what Enlightenment thinkers would have found either audacious or absurd: Use \textbf{modal logic} to demonstrate that a \emph{perfect, necessary being} must exist.

\medskip

Gödel’s \textbf{Ontological Proof} wasn’t theological rhetoric—it was a minimalist, axiomatic structure worthy of Euclid:

\medskip

\begin{itemize}
    \item Define ``God'' as a being possessing all \textbf{positive properties}.
    \item Assert that existence is among those positive properties.
    \item Conclude that such a being must exist in at least one possible world; and, by necessity, in all possible worlds.
\end{itemize}

\medskip

Simple. Elegant. The kind of thing only a mind like Gödel’s would dare to take seriously.

\medskip

But Gödel, ever cautious, \textbf{never published} his proof during his lifetime. It was discovered among his papers after his death -- perhaps because proving God’s existence with modal logic felt like intellectual heresy in academic circles.

\medskip

For Gödel, it was mathematics applied to metaphysics. Where others saw an unbridgeable gap between logic and God, Gödel calmly built a formal system across it.

\begin{quote}
    \emph{I don’t believe in empirical science. I only believe in a priori science.} — \textbf{Kurt Gödel}
\end{quote}

\end{tcolorbox}

\medskip

In response, Wittgenstein --- once the architect of logical certainty --- abandoned the quest for absolute truth altogether. If Gödel cracked the Enlightenment's foundation, Wittgenstein simply walked off the construction site. His later philosophy questioned whether ``truth'' was even the right thing to seek, replacing it with language games, context, and use.

In this sense, Gödel and Wittgenstein became unlikely partners in ending the Enlightenment project:

\begin{itemize}
  \item Gödel reached backwards, and sought refuge in a Platonic, perhaps even theological, grounding for mathematics.
  \item Wittgenstein pushed forward, dismissed grand narratives of truth, and embraced post-modernism.
\end{itemize}

Together, they forced the intellectual world to confront an uncomfortable reality: the Enlightenment's vision of reason or empiricism as pillars of a self-sufficient compass had failed. Not only was pure logic unable to guarantee truth, but experience alone could offer no rescue when the very structure of knowledge was full of holes.

%But paradoxically, in exposing these limits, they also cleared the way for a new kind of secular knowledge -- one that no longer claimed divine sanction or absolute certainty. Mathematics, and by extension all knowledge, became foundational not because it revealed eternal truths, but because it offered structural coherence within human-made systems.

%Gödel killed the Enlightenment ideal, and Wittgenstein ensured no one could resurrect it. What followed wasn’t a return to mysticism, but a humbler, self-aware rationality—one that accepted its boundaries and moved forward anyway.



\begin{figure}[H]
\centering

% === First row ===
\begin{subfigure}[t]{0.45\textwidth}
\centering
\begin{tikzpicture}
  \comicpanel{0}{0}
    {Gödel}
    {}
    {\footnotesize There are truths no system can prove. We can’t cage the truth.}
    {(0,-0.6)}
\end{tikzpicture}
\caption*{The incompleteness bomb drops.}
\end{subfigure}
\hfill
\begin{subfigure}[t]{0.45\textwidth}
\centering
\begin{tikzpicture}
  \comicpanel{0}{0}
    {Popper}
    {}
    {\footnotesize Then let’s stop proving and start falsifying. We learn by failure.}
    {(0,-0.6)}
\end{tikzpicture}
\caption*{Science pivots to survival.}
\end{subfigure}

\vspace{1em}

% === Second row ===
\begin{subfigure}[t]{0.45\textwidth}
\centering
\begin{tikzpicture}
  \comicpanel{0}{0}
    {Quine}
    {}
    {\footnotesize Meaning? Truth? Those are just nodes in a semantic web.}
    {(0,-0.6)}
\end{tikzpicture}
\caption*{Goodbye foundations. Hello networks.}
\end{subfigure}
\hfill
\begin{subfigure}[t]{0.45\textwidth}
\centering
\begin{tikzpicture}
  \comicpanel{0}{0}
    {Wittgenstein}
    {}
    {\footnotesize If truth is a game then it’s one I no longer wish to play.}
    {(0,-0.6)}
\end{tikzpicture}
\caption*{Resignation at the edge of reason.}
\end{subfigure}

\caption{The wreckage of certainty: Gödel broke the system, and everyone else rewrote the rules.}
\end{figure}






