\section{From Calculus to Axioms: Gibbs Built with Vectors, Peano Built Vector Spaces}

\subsection{From Operator to Object: Gibbs Used Vectors, Peano Defined Them}

By the time Gibbs had finished refining the calculus of fields, the vector had become a working tool of the physicist — a symbol of rotation, force, and flux.  
But there was still an unspoken assumption: that these objects “made sense” because they matched our physical intuition.

Enter \textbf{Giuseppe Peano}.

While Gibbs wielded vectors as concrete entities in \( \mathbb{R}^3 \), Peano took a different route: he asked what vectors \textit{are}, independent of dimension, coordinates, or physical interpretation.

\bigskip

Where Gibbs developed vector calculus to express physical laws like:
\[
\vec{F} = m\vec{a}, \quad \vec{r} \times \vec{v} = \vec{L}
\]
Peano introduced a formal structure in which such expressions could live. He defined a **vector space** not in terms of axes or coordinates, but through axioms:

\begin{itemize}
    \item Vectors form an abelian group under addition.
    \item Scalar multiplication distributes over vector addition.
    \item Scalars come from a field (like \( \mathbb{R} \) or \( \mathbb{C} \)).
\end{itemize}

In Peano’s hands, a vector was no longer a directed arrow — it was an element \( v \in V \), where \( V \) satisfies the axioms of linearity.  
This allowed vectors to exist in infinite dimensions, over abstract fields, or as purely symbolic constructs.

\bigskip

\begin{tcolorbox}[colback=gray!5!white, colframe=black, title=\textbf{Sidebar: Gibbs vs. Peano — Action vs. Abstraction}, fonttitle=\bfseries, arc=1.5mm, boxrule=0.4pt]

\textbf{Gibbs:}  
Vectors are arrows in \( \mathbb{R}^3 \). Use them to encode rotation, flow, and force.

\textbf{Peano:}  
Vectors are elements of an abstract space \( V \), defined only by axioms of addition and scalar multiplication.

\medskip

\textbf{Gibbs:}  
Cross products and dot products are geometric operations with physical meaning.

\textbf{Peano:}  
Linear structure comes first. Geometry is optional.

\medskip

\textbf{Gibbs:}  
Invented tools for physics.

\textbf{Peano:}  
Invented the space those tools live in.
\end{tcolorbox}

\bigskip

This shift from operational to structural thinking is what allowed later mathematicians and physicists — from Hilbert to Noether — to generalize mechanics, develop quantum theory, and build differential geometry. But it also lets us look back and reinterpret classical laws like Kepler’s through a new lens.

Where Gibbs translated Kepler’s Second Law into an expression of angular momentum conservation,  
\[
\vec{L} = \vec{r} \times \vec{v} = \text{constant}
\]
Peano gave us the tools to ask: what kind of space must \( \vec{r} \) and \( \vec{v} \) inhabit for that identity to make sense?

And that brings us to a deeper interpretation of Kepler’s law—not as a geometric story, or even a physical one, but as a statement about \textbf{structure in a vector space}.

\bigskip

This is where Peano’s abstraction becomes more than just a philosophical exercise.  
Once vectors are defined independently of coordinates, we can start building new kinds of objects out of them — antisymmetric ones, multilinear ones, geometric ones that go beyond length and direction.

From Peano’s vector spaces, it’s a short conceptual leap to the algebra of \textbf{exterior products}, where area and volume are no longer hand-drawn or inferred from geometry but constructed directly from algebraic rules. This framework lets us formalize Kepler’s areal velocity not just as a cross product in \( \mathbb{R}^3 \), but as a \textbf{2-form} — an antisymmetric bilinear object that encodes rotation and orientation in any dimension.

With Peano’s tools, Kepler’s Second Law becomes more than a physical fact:  
It becomes an invariant of linear structure.

And that sets the stage for a deeper reformulation.


\subsection{Peano, Planetary Motion, and the Algebra of Swept Areas}

Giuseppe Peano didn’t write about Kepler’s laws—but he gave us the language to rewrite them.

In the early 20th century, Peano helped formalize the notion of a \textbf{vector space}: a structure defined not by geometric intuition, but by abstract operations. Vectors became not arrows in space, but elements in an algebraic system, governed by axioms of addition and scalar multiplication. It was a shift from pictures to principles.

\bigskip

\textbf{Kepler’s Second Law} says that a planet sweeps out equal areas in equal times. Geometrically, that looks like this:

\[
\frac{dA}{dt} = \text{constant}
\]

But Newton translated it into the language of physics:

\[
\vec{r} \times \vec{v} = \text{constant}
\]

This is a cross product—an operation on vectors in \( \mathbb{R}^3 \). But thanks to Peano’s work, we can go further.

\bigskip

\textbf{Reinterpreting Kepler Algebraically:}

Let \( \vec{r}(t) \) be the planet’s position vector, and \( \vec{v}(t) = \frac{d\vec{r}}{dt} \) its velocity. Then the infinitesimal area swept in time \( dt \) is given by the antisymmetric product (a 2-form in modern language):

\[
dA = \frac{1}{2} \| \vec{r}(t) \wedge \vec{v}(t) \| \, dt
\]

In Peano’s formalism—especially his work on differential geometry and exterior algebra—this becomes a linear-algebraic invariant. The conservation of angular momentum isn’t just a geometric observation; it’s a statement about the structure of vector spaces and the preservation of a bilinear, antisymmetric quantity.

\bigskip

\begin{tcolorbox}[colback=gray!5!white, colframe=black, title=\textbf{Sidebar: From Peano to Kepler — Sweeping in Vector Space}, fonttitle=\bfseries, arc=1.5mm, boxrule=0.4pt]

\textbf{Peano’s Contributions:}
\begin{itemize}
  \item Abstract vector spaces (axioms of linearity)
  \item Differential forms and exterior algebra (antisymmetric products)
  \item Foundations of differential geometry (manifolds, tangent vectors)
\end{itemize}

\textbf{Reframing Kepler’s Law:}
\begin{itemize}
  \item Planetary motion becomes a curve in a vector space.
  \item The area swept out is a 2-form: \( \vec{r} \wedge \vec{v} \).
  \item Conservation becomes invariance under linear transformation.
\end{itemize}

\textbf{In short:} Peano abstracted the universe just enough for Kepler’s geometry to become algebra.

\end{tcolorbox}

\bigskip

Peano laid the groundwork for the modern treatment of conserved quantities—not as mere accidents of geometry, but as \textbf{structural invariants} in a vector space. If Newton made Kepler analytic, Peano made Newton abstract.
