\section{Regiomontanus and the Birth of Trigonometry as a Discipline}

By the 15th century, European mathematics was beginning to reawaken — and one of its clearest signs was the appearance of a book that, for the first time, treated trigonometry as an independent mathematical subject.

That book was \textit{De Triangulis Omnimodis} (On Triangles of Every Kind), written around 1464 by the German mathematician \textbf{Johann Müller}, better known as \textbf{Regiomontanus}.

Unlike earlier astronomers who embedded trigonometry within celestial models or tables, Regiomontanus organized it as a coherent mathematical system. His work drew heavily from Islamic sources — particularly the chord-based trigonometry of Al-Battani and Al-Tusi — but restructured them using the evolving notational tools of Renaissance Europe.

\textbf{What made \textit{On Triangles} groundbreaking?}

\begin{itemize}
  \item It was the first European text to treat \textbf{trigonometry as a branch of mathematics} — not just a tool for astronomy.
  \item It gave explicit attention to both \textbf{plane} and \textbf{spherical triangles}, covering the general rules for solving them.
  \item It included clear statements of what we now call the \textbf{Law of Sines} and \textbf{Law of Cosines}.
  \item It synthesized Greek geometry and Islamic trigonometric techniques using Latin terminology and notation.
\end{itemize}

Regiomontanus was also a practicing astronomer and instrument-maker, and his goal was practical as well as theoretical: to provide accurate methods for celestial navigation, eclipse prediction, and positional astronomy — all through a unified geometric lens.

\textbf{Legacy:} Though his book would not be printed until 1533 (after his death), \textit{On Triangles} became the template for trigonometry in the Latin West — a text that signaled the transition from geometry as pure form to geometry as computational science.

\begin{tcolorbox}[colback=gray!5!white, colframe=black, title=\textbf{TL;DR: The First Book of Trigonometry}, fonttitle=\bfseries, arc=1.5mm, boxrule=0.4pt]
\textit{De Triangulis Omnimodis} by Regiomontanus (1464) was the first European work to define trigonometry as a standalone discipline.  
It unified the spherical methods of Islamic astronomy with classical geometry, and turned trigonometry into a proper branch of mathematics — not just a set of tables.
\end{tcolorbox}



\subsection{Regiomontanus’ Trigonometry: Measuring the Firmament with Triangles}

For Regiomontanus, trigonometry wasn’t just abstract mathematics — it was a way to map the heavens.

In \textit{De Triangulis Omnimodis}, he systematically organized the rules for solving both \textbf{plane} and \textbf{spherical triangles}. While today these distinctions are part of standard curricula, in the 15\textsuperscript{th} century, this was revolutionary — especially because spherical trigonometry was the key to unlocking the geometry of the cosmos.

\subsubsection*{The Core of Regiomontanus’ Trigonometry}

Regiomontanus built his system around fundamental relationships we now recognize as:

\begin{itemize}
  \item The \textbf{Law of Sines}:
  \[
  \frac{\sin A}{a} = \frac{\sin B}{b} = \frac{\sin C}{c}
  \]
  for plane triangles, and its spherical counterpart adapted for arcs on a sphere.
  
  \item The \textbf{Law of Cosines} for both plane and spherical triangles, allowing computation of unknown sides or angles when direct measurement was impossible.
\end{itemize}

He replaced the older chord-based methods inherited from Ptolemy and Islamic astronomers with a more streamlined sine-based approach, which made calculations simpler and more versatile.

But this wasn’t math for math’s sake. Regiomontanus had his eyes on the sky.

\subsubsection*{Applying Trigonometry to the Firmament}

Like his predecessors, Regiomontanus operated within a geocentric cosmology. The firmament — the outermost celestial sphere containing the fixed stars — was still considered a real, rotating structure. To predict planetary positions, chart eclipses, or navigate by the stars, one had to compute angles and distances on this imagined sphere.

Spherical trigonometry was the language of the heavens.

Regiomontanus applied his methods to:

\begin{itemize}
  \item Calculate the apparent positions of celestial bodies on the rotating celestial sphere.
  \item Determine angular separations between stars — critical for astronomical tables.
  \item Predict events like eclipses by modeling intersections of celestial circles (the ecliptic, equator, and horizon).
\end{itemize}

For him, solving triangles wasn’t confined to parchment — it was how you navigated the cosmos, both intellectually and literally.

\subsubsection*{Did Regiomontanus Believe in the Firmament?}

Yes — but with Renaissance nuance.

Regiomontanus inherited the medieval vision of a layered, spherical cosmos:

\begin{itemize}
  \item Earth at the center.
  \item Surrounding spheres carrying the Moon, planets, Sun.
  \item The firmament as the sphere of fixed stars, rotating daily.
  \item Beyond that, the \textit{Primum Mobile} and the Empyrean — the realm of the divine.
\end{itemize}

While his mathematics advanced beyond Ptolemaic methods, his cosmology remained largely traditional. The firmament was still viewed as a structured, measurable reality — a crystalline sphere that could be charted with precision.

\begin{quote}
For Regiomontanus, every star had a place — not just metaphorically, but geometrically, fixed upon the rotating vault of heaven.
\end{quote}

His trigonometry provided the tools to compute these positions, reinforcing belief in a harmonious, ordered cosmos where divine architecture could be decoded through angles and ratios.

\begin{tcolorbox}[colback=blue!5!white, colframe=blue!50!black, title={Triangles: The Surveyor’s Tool for the Heavens}]
Regiomontanus didn’t just invent trigonometry as a discipline — he weaponized it against celestial uncertainty.

In a universe built from spheres and circles, triangles became the key to understanding the firmament’s design. Every astronomical problem was, at its heart, a geometric puzzle waiting to be solved.
\end{tcolorbox}

\subsubsection{Legacy: From Firmament to Framework}

Ironically, while Regiomontanus' trigonometry was crafted to serve a geocentric, firmament-bound cosmos, his mathematical clarity laid the groundwork for future astronomers — like Copernicus — to question that very structure.

But in his own time, Regiomontanus wasn’t dismantling the firmament. He was measuring it — one triangle at a time.


\begin{tcolorbox}[colback=gray!5, colframe=black, title=\textbf{Historical Sidebar: Regiomontanus and the Measurable Firmament}, fonttitle=\bfseries, arc=1.5mm, boxrule=0.4pt]

  In the 15th century, the \textbf{firmament}—the outermost celestial sphere holding the fixed stars—was still taken as a real, physical part of the cosmos.
  
  Regiomontanus didn’t question its existence. For him, the firmament wasn’t a metaphor or abstraction: it was a structured, rotating vault, whose geometry could be charted, angles computed, and positions predicted.
  
  But unlike earlier thinkers who treated the heavens as mystical or theological symbols, Regiomontanus approached the firmament as a field of measurable coordinates. His trigonometry provided tools not just to describe, but to \emph{calculate} its structure.
  
  \medskip
  
  \textbf{Key insight:} The heavens were no longer just divine storytelling—they were a geometric system, legible through triangles.
  
  \begin{center}
  \emph{If medieval thinkers gazed at the stars, Regiomontanus measured them.}
  \end{center}
  
  This mathematical approach would quietly set the stage for later astronomers to abstract the firmament into a coordinate framework—even as its physical reality faded from cosmology.
  
  \end{tcolorbox}
