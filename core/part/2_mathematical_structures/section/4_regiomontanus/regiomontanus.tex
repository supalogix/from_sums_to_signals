\section{Regiomontanus and the Birth of Trigonometry as a Discipline}

By the 15th century, European mathematics was beginning to reawaken — and one of its clearest signs was the appearance of a book that, for the first time, treated trigonometry as an independent mathematical subject.

That book was \textit{De Triangulis Omnimodis} (On Triangles of Every Kind), written around 1464 by the German mathematician \textbf{Johann Müller}, better known as \textbf{Regiomontanus}.

Unlike earlier astronomers who embedded trigonometry within celestial models or tables, Regiomontanus organized it as a coherent mathematical system. His work drew heavily from Islamic sources — particularly the chord-based trigonometry of Al-Battani and Al-Tusi — but restructured them using the evolving notational tools of Renaissance Europe.

\textbf{What made \textit{On Triangles} groundbreaking?}

\begin{itemize}
  \item It was the first European text to treat \textbf{trigonometry as a branch of mathematics} — not just a tool for astronomy.
  \item It gave explicit attention to both \textbf{plane} and \textbf{spherical triangles}, covering the general rules for solving them.
  \item It included clear statements of what we now call the \textbf{Law of Sines} and \textbf{Law of Cosines}.
  \item It synthesized Greek geometry and Islamic trigonometric techniques using Latin terminology and notation.
\end{itemize}

Regiomontanus was also a practicing astronomer and instrument-maker, and his goal was practical as well as theoretical: to provide accurate methods for celestial navigation, eclipse prediction, and positional astronomy — all through a unified geometric lens.

\textbf{Legacy:} Though his book would not be printed until 1533 (after his death), \textit{On Triangles} became the template for trigonometry in the Latin West — a text that signaled the transition from geometry as pure form to geometry as computational science.

\begin{tcolorbox}[colback=gray!5!white, colframe=black, title=\textbf{TL;DR: The First Book of Trigonometry}, fonttitle=\bfseries, arc=1.5mm, boxrule=0.4pt]
\textit{De Triangulis Omnimodis} by Regiomontanus (1464) was the first European work to define trigonometry as a standalone discipline.  
It unified the spherical methods of Islamic astronomy with classical geometry, and turned trigonometry into a proper branch of mathematics — not just a set of tables.
\end{tcolorbox}



