\section{Jacobi and the Equations of Surfaces: From Flow to Geometry}

If Hamilton showed that motion could be described as a flow in phase space,  
then Carl Gustav Jacob Jacobi showed that this flow could be traced back to a hidden landscape:  
a surface whose geometry encoded every possible motion at once.

Hamilton had reduced mechanics to a paired set of first-order equations:  
differential equations that described how positions and momenta evolved over time.

But Jacobi saw something deeper lurking beneath those equations—  
something that Hamilton himself had glimpsed, but had not fully unfolded.

\bigskip

\subsection*{The Leap from Hamilton to Jacobi}

Hamilton’s equations governed the flow of trajectories in phase space.  
But Jacobi asked:  
Could we find a single function whose gradients described this entire flow?

In Hamilton’s formalism, mechanics was a system of differential equations.  
In Jacobi’s hands, it became a problem of solving a **partial differential equation (PDE)**—  
a single equation whose solution contained all the dynamical information.

That equation would come to be known as the \textbf{Hamilton–Jacobi equation}:

\[
H\left( q_i, \frac{\partial S}{\partial q_i}, t \right) + \frac{\partial S}{\partial t} = 0
\]

where:

\begin{itemize}
  \item \( S(q,t) \) is the \textbf{Hamilton’s principal function}, also known as the action.
  \item \( H \) is the Hamiltonian.
\end{itemize}

\bigskip

This was a profound shift:

✅ Instead of solving differential equations for \( q(t) \) and \( p(t) \) directly,  
✅ one could solve a **PDE for \( S \)**, and then recover trajectories by differentiation:

\[
p_i = \frac{\partial S}{\partial q_i}
\]

\bigskip

\subsection*{From Trajectories to Surfaces}

In Hamilton’s formulation, each trajectory was a curve.  
In Jacobi’s, all trajectories were encoded as level sets of a function \( S \)—  
each trajectory corresponding to moving along a contour of constant \( S \) in configuration space.

Where Hamilton gave us the flow of points,  
Jacobi gave us the **geometry of surfaces** from which those flows could be derived.

The mechanics of curves became the geometry of hypersurfaces.

\bigskip

\begin{tcolorbox}[colback=gray!5!white, colframe=black, title=\textbf{Sidebar: The Shift from Hamilton to Jacobi}, fonttitle=\bfseries, arc=1.5mm, boxrule=0.4pt]

\begin{tabular}{>{\raggedright}p{4cm} >{\raggedright}p{5.5cm} >{\raggedright\arraybackslash}p{5.5cm}}
 & \textbf{Hamilton} & \textbf{Jacobi} \\
\midrule
State description & Trajectories as integral curves in phase space & Trajectories as level sets of \( S(q,t) \) \\
Key equation & Hamilton's equations (ODEs) & Hamilton–Jacobi equation (PDE) \\
Solution form & Paths \( (q(t), p(t)) \) & Generating function \( S(q,t) \) whose gradient gives \( p \)
\end{tabular}

\end{tcolorbox}

\bigskip

\subsection*{The Geometry Hidden in Motion}

Jacobi’s insight wasn’t just technical; it was conceptual.

He realized that mechanics could be understood as a problem of finding a special surface \( S(q,t) = \text{const} \),  
where each solution to the equations of motion corresponded to moving along such a surface.

The action \( S \) became a kind of \textbf{optical wavefront} sweeping through configuration space—  
an analogy that would one day lead directly into wave mechanics and quantum theory.

\bigskip

Where Hamilton gave us a flow on a phase space,  
Jacobi showed that flow could be understood as the gradient of a function—  
that mechanics was not merely the evolution of coordinates, but the geometry of a landscape in which trajectories were steepest paths.

\bigskip

\begin{quote}
In Euler, we calculated forces.  
In Lagrange, we minimized action.  
In Hamilton, we traced flows.  
In Jacobi, we found the surfaces those flows glide along.
\end{quote}

\subsection*{Setting the Stage for Geometry Itself}

Jacobi’s leap was to recast mechanics as a problem in geometry:  
not the geometry of shapes, but the geometry of functions—  
where the dynamics of a system were encoded in the curvature and gradients of \( S \).

This was a foreshadowing of something deeper:

\begin{itemize}
    \item That the laws of physics might not be about forces and trajectories at all,  
    \item but about the structure of the geometric objects beneath them.
\end{itemize}

The next step would require moving beyond functions defined on flat spaces,  
into functions defined on curved manifolds.

It would take Gauss, Riemann, Christoffel, Ricci, and Levi-Civita to write that geometry.  
And it would take Einstein to realize that the geometry was spacetime itself.

\subsection*{Reinterpreting Kepler’s Second Law Through Jacobi’s Lens}

Kepler’s Second Law—\textit{``A planet sweeps out equal areas in equal times''}—was originally a geometric observation about ellipses and time.  
But through Jacobi’s lens, it reveals a deeper truth:  

Kepler wasn’t just describing orbits.  
He was hinting at the presence of an \textbf{underlying geometry}:  
a surface over which planetary motion unfolds, with structure encoded not in force laws,  
but in \textbf{gradients and symmetries}.

\bigskip

\begin{tcolorbox}[colback=blue!5!white, colframe=blue!80!black, title=\textbf{The Jacobi Interpretation of Kepler's Second Law}]
If we describe motion through the Hamilton–Jacobi function \( S(q,t) \),  
then the conserved area Kepler observed corresponds to a conserved quantity arising from symmetry:  
\textbf{the flow along a level set of \( S \)} that preserves area in configuration space.
\end{tcolorbox}

\bigskip

\paragraph{From Areas to Actions.}

In Jacobi's language, Kepler’s law is no longer about areas in space,  
but about the \textbf{invariance of motion under transformations}—a consequence of a \textbf{symplectic structure} that preserves volume in phase space.

But more than that:

\begin{itemize}
  \item The motion of a planet traces out level sets of \( S(q,t) \),
  \item And Kepler’s equal area law becomes a statement about how those contours are crossed uniformly in time,
  \item With the action \( S \) acting like a wavefront sweeping through configuration space,  
  carrying planets along steepest-descent trajectories.
\end{itemize}

\bigskip

\paragraph{Geometry as Conservation.}

This reframing ties Kepler’s Second Law to \textbf{Noether’s Theorem} (before Noether):  
the conservation of areal velocity arises from \textbf{rotational symmetry}—  
a fact encoded geometrically in the shape of the \( S \)-surface,  
and dynamically in the conserved angular momentum.

\bigskip

\begin{quote}
Kepler saw a sweep of areas.\\
Jacobi saw a sweep of gradients.\\
And geometry was the bridge between the two.
\end{quote}
