\subsection{(\textit{logos}): The Relationship Between Magnitudes}

A \textbf{ratio} is how one magnitude relates to another (i.e. a comparison by division):

\[
\text{Ratio of } A \text{ to } B = A : B
\]

But here's the catch: For the Greeks, you could only \emph{express} this ratio numerically if the two magnitudes were \textbf{rational} (\textgreek{λογος}).

Two magnitudes are said to be \textbf{rational} (\textgreek{λογος}) if there exists a common measure; that is, a smaller unit that can measure both of them exactly. For example, a stick of 6 units and another of 9 units are rational (\textgreek{λογος}) because their common unit is 3.

\begin{quote}
The common measure for the diagonal and side of a square they called \textit{alogos} (\textgreek{ἄλογος}) which literally means ``without reason'' because you can never land on a clean division for both.
\end{quote}

This was the Pythagorean nightmare. They believed \textit{"All is number,"} but then they discovered (likely via the diagonal of a square) that some ratios could not be expressed as a ratio of whole numbers.

They tried to find a fraction \( \frac{a}{b} \) that equals \( \sqrt{2} \), but failed. Eventually, they proved that no such rational number exists.  And, this shattered the belief that all ratios of magnitudes could be represented by numbers (i.e., ratios of whole numbers).  It meant:

\begin{quote}
Some magnitudes have a ratio. You can say "this is longer than that", but that ratio \emph{cannot} be numerically expressed.
\end{quote}

And so, to the ancient Greek mind — especially after the discovery of incommensurables — ratio emerged as more fundamental than number. While numbers could count discrete objects, ratios could express relationships between continuous magnitudes. They captured something deeper: the way one thing relates to another in proportion, in balance, in harmony.

Before there was number, there was magnitude — length, area, time — the raw material of the world. Numbers only entered the picture when those magnitudes happened to align neatly, when their relationship could be pinned down by a shared unit. In that sense, number wasn’t the foundation of mathematics. It was the exception.
Magnitude came first. \textbf{Number was what happened when magnitude behaved.}

\medskip

\begin{tcolorbox}[title=Historical Sidebar: Ratio as Revelation, colback=gray!5, colframe=black, fonttitle=\bfseries]

  For the Pythagoreans, a ratio wasn't just a mathematical convenience... it was divine revelation.

  \medskip
  
  The Greek word for ratio, \textit{logos}, meant far more than just a fraction or comparison. It also meant \textit{reason}, \textit{word}, \textit{principle}, even \textit{cosmic order}. When a Pythagorean studied the ratio of string lengths that produced harmonious notes — like 2:1 for the octave or 3:2 for the perfect fifth — they weren't just doing music theory.

  \medskip
  
  They believed they were glimpsing the hidden architecture of the cosmos.
  
  \medskip
  
  In their worldview, everything — the motion of the stars, the balance of the soul, the justice of the city — was governed by \textbf{logos}. When two magnitudes stood in a harmonious ratio, that was more than a measurement. It was a moment of divine clarity: the universe revealing itself through proportion.
  
  \medskip
  
  This is why the discovery of incommensurable magnitudes (like the diagonal of a square relative to its side) was such a crisis. If some magnitudes had no logos — no ratio that could be expressed as a whole number relationship — then part of reality lay beyond reason.
  
  \medskip

  It meant that not everything could be brought into harmony.  And to the Pythagorean mind, that was nothing short of heresy.
  
\end{tcolorbox}
