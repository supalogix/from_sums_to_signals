\subsection{Math, Before It Got Complicated}

Mathematics wasn’t always an ivory-tower intellectual pursuit. In its early days, it was mostly about not starving. People needed to know things like:  

\begin{enumerate}
	\item How much grain do I have left before I die?  
	\item How much land do I own before my neighbor starts sharpening his spear?  
	\item How do I make sure my cows don’t wander off and become someone else’s cows?  
\end{enumerate}

And thus, math was born. It started as a practical tool — a survival mechanism for ancient civilizations. But before there were numbers, there were \textbf{magnitudes}.

\begin{quote}
A magnitude is a continuous quantity: not how many things you have, but how much. Not \emph{five cows}, but how \emph{long} the fence is. Not \emph{seven jars}, but how \emph{full} each jar is. Length, area, weight, time — these were the raw materials of early math.
\end{quote}

People compared these magnitudes by saying things like “this is longer than that” or “this field is twice as big.” These weren’t numbers in the modern sense — they were \emph{relations} between physical realities. You couldn’t always count them, but you could compare them. And when two magnitudes could be measured by the same unit, they were said to be \emph{commensurable}.

So before math was about equations, functions, and formal proofs, it was about comparing stuff: more, less, equal. That was the beginning — not numbers, but magnitude.

\begin{tcolorbox}[title=Historical Sidebar: \textit{Will Work for Beer}, colback=gray!5, colframe=black, fonttitle=\bfseries]

  \textbf{Uruk IV Period, c. 3000 BCE — Southern Mesopotamia}
  
  One of the earliest known uses of writing wasn’t to record philosophy or poetry — it was to make sure nobody got shorted on their daily brew.
  
  A small clay tablet from the city of Uruk, dating to around 3000 BCE, displays a pictogram of a human head eating from a bowl, followed by the symbol for beer. Alongside these images are early numerical signs, showing how much beer was distributed to different workers.
  
  \medskip
  
  In plain terms, it reads something like:
  
  \begin{quote}
  Bob gets 2 silas of beer. Sarah gets 3. Dave only gets 1 because he’s new.
  \end{quote}
  
  \medskip
  
  This wasn’t just an ancient lunch order — it was logistics and justice by volume. These early scribes were tracking not things but \textbf{magnitudes} — continuous quantities like volume — and recording them in units like the \textit{sila} (about one liter).
  
  \medskip
  
  It’s the oldest surviving evidence of a universal economic truth: \textbf{will work for beer}. And so begins the history of math — not with numbers, but with the fair distribution of fermented wages.
  
\end{tcolorbox}
  



\begin{figure}[H]
\centering
\begin{tikzpicture}[every node/.style={font=\footnotesize}]

% Panel 1 — Ancient farmer doing practical math
\comicpanel{0}{4}
  {Worker}
  {Scribe}
  {I expected 10 silas of beer, and I only got 7. That means someone is about to have a bad day.}
  {(0,-0.5)}

% Panel 2 — Land division
\comicpanel{6.5}{4}
  {Worker 1}
  {Worker 2}
  {If they reduce my allocation of beer to pay the new guy then I'm gonna murder him.}
  {(0,-0.5)}

% Panel 3 — Priest or scribe doing math
\comicpanel{0}{0}
  {Scribe 1}
  {Scribe 2}
  {If we list the missing beer as a “ritual offering,” the gods get drunk and we stay employed.}
  {(0,0.8)}

% Panel 4 — Philosopher enters with a math headache
\comicpanel{6.5}{0}
  {Philosopher}
  {Scribe}
  {What even *is* a number? Can it exist without beer?}
  {(0,0.8)}

\end{tikzpicture}
\caption{Math: originally invented to track livestock. Now used to question the fabric of reality.}
\end{figure}