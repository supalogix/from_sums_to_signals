\subsection{From Passive Observation to Dialectical Agency}

The Lagrangian and Hamiltonian formalisms of classical mechanics are built on a set of foundational assumptions that reflect Enlightenment ideals: determinism, objectivity, and mathematical neutrality. In this worldview:

\begin{itemize}
	\item The system evolves independently of the observer.  
	\item Mathematics is a mirror, not a lever—it reflects reality rather than acting upon it.  
	\item Optimization (like extremizing the action) reveals nature’s underlying logic, but it does not alter its course.
\end{itemize}

This is a vision of science rooted in Newtonian empiricism and Laplacian determinism: the universe is a grand, clockwork mechanism, and we are spectators with equations.

The Pontryagin Maximum Principle, which arises in optimal control theory, inherits this lineage but introduces a subtle twist. While it retains the structure of deterministic evolution, it allows for *control inputs*—a way to steer the system. And yet, the philosophical backdrop often remains unchanged: the control designer is “outside” the system, prescribing inputs but not entangled in its dynamics.

This framework assumes a stable distinction between system and observer, dynamics and intention. But what if that line isn't stable? What if the mathematics of dynamics can encode not only what is observed, but what is desired—what is *possible*?