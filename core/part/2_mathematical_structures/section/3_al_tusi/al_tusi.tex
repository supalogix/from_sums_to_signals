\section{Al-Tusi and the Geometry of the Heavens: When Circles Spoke Algebra}

By the 13th century, the elegance of Ptolemy’s model had grown cumbersome. Epicycles upon epicycles cluttered the sky, each a patch on a system trying to match observation without abandoning Aristotle’s geocentric stage. Enter \textbf{Nasir al-Din al-Tusi} (1201–1274)—astronomer, mathematician, and architect of one of the most subtle revolutions in premodern astronomy.

From his observatory in \textbf{Maragha}, funded by the Mongol empire and equipped with some of the finest astronomical instruments of the time, Al-Tusi composed his \textit{Zīj-i Īlkhānī} (the \textit{Memoir on Astronomy}). There, he did more than calculate positions—he reimagined the mechanics behind them.

\subsection{The Tusi Couple: Straight Lines from Circles}

One of Al-Tusi’s most ingenious inventions was the \textbf{Tusi Couple}, a purely geometric device: two circles, one rotating inside the other, producing linear oscillation. With this elegant mechanism, he could generate linear motion from uniform circular motion—without breaking the Aristotelian rule that the heavens must move in circles.

\begin{quote}
\textit{A small circle rolls inside a larger circle exactly twice its size. A point on the circumference of the smaller circle traces a straight line back and forth along the diameter of the larger one.}
\end{quote}

The result? A mathematical tool that let Al-Tusi refine lunar and planetary models while eliminating unnecessary epicycles. And he did it using only geometry—yet with the precision of algebra.

\subsection{Trigonometry as Algebra in Disguise}

Al-Tusi wasn’t content with geometric intuition alone. He formalized the rules of \textbf{spherical trigonometry} and treated \textbf{sine and cosine} not as geometrical accidents, but as objects with their own algebraic laws.

\begin{itemize}
  \item He defined the law of sines for spherical triangles.
  \item He tabulated trigonometric functions with high precision.
  \item He expressed trigonometric identities in a quasi-algebraic form.
\end{itemize}

While he didn’t use symbolic algebra in the modern sense, his work embodied the algebraic mindset: decomposing relationships, manipulating rules, and treating quantities abstractly. He merged the computational logic of algebra with the spatial insight of geometry—exactly what future mathematicians like Viète and Descartes would aim for centuries later.

\subsection{Algebra Meets the Cosmos}

In Al-Tusi’s hands, the heavens were not just objects to describe—they were systems to model, structures to compute. He wasn’t solving for \( x \) in a symbolic equation, but he was solving for Mars.

\begin{tcolorbox}[colback=blue!5!white, colframe=blue!50!black, title={Al-Tusi’s Astronomical Legacy}]
Al-Tusi introduced algebraic reasoning into astronomical modeling by:

\begin{itemize}
  \item Replacing geometrical epicycles with algebraic-geometric devices
  \item Using trigonometric identities to compute planetary positions
  \item Developing mathematical tools later echoed by Copernicus and Kepler
\end{itemize}

His Tusi Couple whispered something radical:
\\[0.5em]
\textit{Linear motion can emerge from circles.  
The heavens obey rules—  
but the rules can evolve.}
\end{tcolorbox}
