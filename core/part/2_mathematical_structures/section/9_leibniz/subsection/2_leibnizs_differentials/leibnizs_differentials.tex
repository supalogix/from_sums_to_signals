\subsection{Leibniz: Infinitesimals, Harmony, and the Algebra of Change}

While Newton conceived calculus through the lens of motion and absolute time, \textbf{Gottfried Wilhelm Leibniz} envisioned something different:  
A symbolic language of relations, a grammar for understanding the infinitesimal structure of change itself.

For Leibniz, mathematics was not merely a description of physical motion; it was a reflection of divine rationality—the unfolding of a universe optimized through the simplest possible laws.

\subsection{A Symbolic View of Calculus}

Where Newton defined fluxions as rates of physical change flowing through absolute time, Leibniz defined differentials as symbolic quantities: infinitely small changes, whose ratios revealed the hidden structure of curves, motions, and forces.

He introduced the notation:

\[
\frac{dy}{dx}
\]

—an abstract ratio of infinitesimal differences. Unlike Newton’s fluxions, which were anchored to real-world time, Leibniz's differentials existed independently, manipulated through symbolic rules.

At the heart of Leibniz’s vision was the idea that:

\begin{itemize}
    \item \( dx \) is an infinitesimal change in the input.
    \item \( dy \) is the corresponding infinitesimal change in the output.
    \item Their ratio \( \frac{dy}{dx} \) captures the local behavior of a function at a point.
\end{itemize}

\begin{tcolorbox}[colback=gray!5!white, colframe=black, title=\textbf{Historical Sidebar: Leibniz’s Notation vs Newton’s Fluxions}, fonttitle=\bfseries, arc=1.5mm, boxrule=0.4pt]
Where Newton viewed calculus as describing motion through absolute time, Leibniz saw it as a symbolic algebra of infinitesimal relations.

Newton's fluxions (\( \dot{x} \)) depended on the notion of a cosmic timekeeper—an invisible river against which change was measured.

Leibniz’s differentials (\( dx, dy \)) needed no such clock. They were purely algebraic, describing how quantities changed in relation to each other, without reference to any external temporal flow.

Today, calculus textbooks overwhelmingly use Leibniz’s notation—not because Newton was wrong, but because Leibniz’s abstraction made the machinery of change universally portable.
\end{tcolorbox}

\subsection{Leibniz’s Infinitesimals: Bridging the Finite and the Infinite}

In Leibniz’s mind, infinitesimals were not imaginary fictions—they were real mathematical entities that bridged the gap between finite and infinite quantities.

He imagined curves not as smooth wholes, but as stitched together from infinitely many infinitesimal linear segments. Zoom close enough, and any curve would look like a straight line.

Thus, differentiation became the study of these infinitesimal triangles, and integration became their recombination into wholes.

He expressed this beautifully:

\[
dy = v\,dt
\quad \text{and} \quad
v = \frac{ds}{dt}
\]

The world, for Leibniz, was a seamless fabric of change woven from vanishing threads.

\subsection{The Metaphysical Vision: Harmony and Optimization}

Leibniz's calculus was part of a much larger philosophical program.

He believed that God, being perfect, would create the \textbf{best of all possible worlds}: the simplest causes yielding the most complex and beautiful effects.

In this vision, nature operated according to \textbf{optimization principles}—minimizing effort, maximizing harmony. Change was rational, structured, elegant.

Calculus, in Leibniz's hands, was not just a mathematical tool. It was a theological insight, a window into the divine algorithm underlying reality.

\begin{tcolorbox}[colback=blue!5!white, colframe=blue!50!black, 
  title={Historical Sidebar: Leibniz—Calculus, Harmony, and the Divine Algorithm}]
  
  \textbf{Gottfried Wilhelm Leibniz} didn’t just invent calculus—he invented it as part of a larger metaphysical project: to show that the universe itself was the optimal solution to a divine equation.

  For Leibniz, mathematics was philosophy in symbolic form.  
  His obsession with optimization—simple laws, rich outcomes—foreshadowed the later development of variational principles in physics.

  \medskip

  His differential notation (\( dx, dy \)) reflected this view: nature changes not by abrupt leaps, but by smooth, rational transitions—gliding through possibility space with divine efficiency.

  \medskip

  To Leibniz, every mathematical expression was a hint, a whisper of God's mind.

  \medskip

  \textbf{Quote from Leibniz (1686):}
  \begin{quote}
  “The present is pregnant with the future; the future could be calculated from it, if we had sufficient knowledge of all causes.”
  \end{quote}
  
\end{tcolorbox}

\subsection{Contrast with Newton: Time vs Structure}

While Leibniz built a calculus of structure and relationship, Newton grounded his in absolute time and physical flux.

\begin{center}
\begin{tabular}{c|c}
\textbf{Newton (contrast)} & \textbf{Leibniz (focus)} \\
\hline
Fluxions: change over absolute time & Differentials: ratios of change \\
Geometry tied to physical motion & Algebra abstracted from motion \\
Divine time as universal backdrop & Divine optimization as structural law \\
Physical causality (force) & Structural relations (infinitesimals) \\
\end{tabular}
\end{center}

Newton’s calculus flowed with time; Leibniz’s calculus floated above it.

Both contributed essential tools to modern science, but it was Leibniz’s abstract, symbolic vision that made calculus portable—extending it beyond planets and physics to fields as diverse as economics, statistics, and machine learning.

\subsection{Leibniz’s Enduring Legacy}

Today, whenever we write:

\[
\frac{dy}{dx}
\quad \text{or} \quad
\int f(x)\,dx
\]

—whenever we study derivatives as relationships, or integrals as accumulated changes—we walk the road that Leibniz paved.

His calculus is not a description of physical flow, but a universal language of structured change.

Where Newton saw the river of time, Leibniz built the bridge of symbols.

\begin{tcolorbox}[colback=gray!5!white, colframe=black, title=\textbf{Philosophical Footnote: Einstein and the Reunion of Newton and Leibniz}, fonttitle=\bfseries, arc=1.5mm, boxrule=0.4pt]

  In the 20th century, \textbf{Albert Einstein} would confront a strange destiny: to reconcile the two divergent visions born in the calculus of Newton and Leibniz.

  \medskip
  
  Einstein’s theory of \textbf{relativity} demanded both:

  \medskip
  
  \begin{itemize}
      \item \textbf{Newton’s dynamical realism} — that change is real, physical, and law-governed.
      \item \textbf{Leibniz’s structural relationalism} — that space, time, and motion are not absolute stages, but networks of relationships between events.
  \end{itemize}

  \medskip
  
  When Einstein bent space and time into a single fabric, he inherited Newton’s faith in the objective structure of the universe—but stripped away Newton’s absolute time.  When he described motion as the geometry of spacetime itself, he realized Leibniz’s dream: that relations, not substances, are primary.
  
  \medskip
  
  In a way, relativity is where Newton’s flowing clock and Leibniz’s invisible algebra finally shook hands—two rival visions, fused into the strange, flexible, curved continuum we now call spacetime.
  
  \medskip
  
  \textbf{In the beginning, there was motion. In the end, there were relations.}

\end{tcolorbox}

\subsection{A New Kind of Equation: Relations of Change}

Leibniz didn’t just invent notation—he opened the door to a new kind of equation.

By symbolizing derivatives as \( \frac{dy}{dx} \), he made it possible to write \textbf{equations where the unknown wasn’t a number, but a function whose rate of change followed certain rules}. These became known as \textbf{differential equations}: equations that relate a function to its derivatives.

For Leibniz, differential equations were a natural extension of calculus. If differentiation measured change, and integration summed change, then a differential equation was a statement about how a quantity \emph{must change}—a recipe for motion, growth, or interaction.

This wasn’t just symbolic cleverness. It created a new mathematical language for describing the world. Forces, orbits, velocities, flows—anything changing over time or space could now be encoded as an equation involving derivatives.

Leibniz’s framework laid the foundation for later generations of mathematicians and physicists, who would turn these symbolic expressions into tools for solving real-world problems.

\vspace{1em}

\begin{center}
\textit{Leibniz gave calculus its grammar—and with it, a way to write the laws of change themselves.}
\end{center}
