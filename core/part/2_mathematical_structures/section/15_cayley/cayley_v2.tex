\section{Cayley and the Algebra of Geometry: From Surfaces to Structure}

If Jacobi showed that motion could be described by sweeping surfaces in configuration space,  
then Arthur Cayley turned geometry itself into an algebra—  
an algebra not merely of points and curves, but of the very transformations that connected them.

Jacobi had taken Hamilton’s flow and reframed it as a problem of finding a special function \( S(q,t) \),  
whose gradients traced out every possible trajectory.  
In Jacobi’s hands, mechanics became a geometry of surfaces.

But Cayley asked a deeper question:

\begin{quote}
“What lies behind the geometry of those surfaces themselves?”
\end{quote}

Could geometry itself be reduced to operations?  
Could distance, angle, and curvature be encoded not in the figures drawn, but in the algebra of relations among them?

\bigskip

\subsection*{The Leap from Jacobi to Cayley}

Jacobi’s Hamilton–Jacobi equation transformed mechanics into a problem of finding level sets in a function space.  
Each trajectory became a path along such a surface.

But Cayley saw geometry not as a study of shapes on a backdrop,  
but as a study of the algebraic relationships that defined the backdrop itself.

Where Jacobi worked with differential equations to describe curves,  
Cayley introduced matrices to describe transformations.

He realized that many geometric properties could be expressed through systems of equations—  
and that those systems could be represented, manipulated, and studied through arrays of numbers.

Thus was born the **algebra of matrices**.

\bigskip

In a world where Euclid had given us points and lines,  
and where Hamilton and Jacobi had given us flows and surfaces,  
Cayley gave us a way to describe transformations between spaces algebraically.

He introduced the \textbf{matrix product}, the \textbf{determinant}, and the notion of an abstract algebraic structure  
where geometry and linear transformations were two faces of the same coin.

\bigskip

\begin{tcolorbox}[colback=gray!5!white, colframe=black, title=\textbf{Sidebar: The Shift from Jacobi to Cayley}, fonttitle=\bfseries, arc=1.5mm, boxrule=0.4pt]

\begin{tabular}{>{\raggedright}p{4cm} >{\raggedright}p{5.5cm} >{\raggedright\arraybackslash}p{5.5cm}}
 & \textbf{Jacobi} & \textbf{Cayley} \\
\midrule
Key object & Level sets \( S(q,t) \) encoding trajectories & Matrices encoding linear transformations \\
Geometry as & Surfaces in configuration space & Relations between spaces under transformation \\
Equation type & Partial differential equation (Hamilton–Jacobi) & Algebraic equations of matrices and determinants
\end{tabular}

\end{tcolorbox}

\bigskip

\subsection*{Transforming Geometry Itself}

Cayley’s work foreshadowed a profound shift:

✅ Geometry was no longer just the study of shapes,  
✅ but the study of **invariants under transformation**.

Instead of focusing on points and lines, Cayley turned attention to the operations that moved, stretched, or rotated those objects—  
and to the quantities that remained unchanged under such operations.

This idea of invariance would seed the future field of **projective geometry**,  
and later become foundational in the work of Felix Klein’s Erlangen Program.

\bigskip

Where Jacobi had used surfaces to encode motion,  
Cayley showed that such surfaces could themselves be encoded in algebraic structures.

And while Jacobi’s equation governed how trajectories unfolded,  
Cayley’s matrices governed how spaces themselves were connected.

\bigskip

\begin{quote}
In Euler, we pushed.  
In Lagrange, we minimized.  
In Hamilton, we flowed.  
In Jacobi, we found surfaces.  
In Cayley, we discovered the algebra beneath those surfaces.
\end{quote}

\subsection*{Laying the Groundwork for Abstract Spaces}

Cayley’s leap turned geometry inward—  
from the visible shapes to the hidden structures that defined their relations.

His algebra of matrices would become the language of linear transformations,  
and his theory of invariants would lay the groundwork for understanding geometry not as a fixed set of figures,  
but as a flexible, transforming system governed by symmetries.

The world of vectors, tensors, and transformations had been hinted at by Peano, Jacobi, and Hamilton—  
but it was Cayley who formalized the idea that geometry could be studied as algebra.

And with that move, he opened the door for geometry to be rewritten not on flat planes,  
but on spaces whose very fabric could bend, twist, and stretch—  
a door that Riemann would soon step through.

