\subsection{The Controversy: Kepler vs. The Universe}

Why was this a problem for Aristotle?

Because Aristotle’s physics said motion required a constant force. If planets were moving faster and slower without anyone pushing them, something was terribly wrong. \textbf{Kepler had just killed another fundamental assumption of Aristotelian motion}.

Which brings us to the real controversy.

Kepler didn’t just break the rules of celestial motion: he broke the rules of polite intellectual society.

\begin{itemize} 
	\item \textbf{The Catholics hated him} because he was Protestant. 
	\item \textbf{The Protestants hated him} because he wasn’t their \textit{kind} of Protestant. 
	\item \textbf{The Aristotelians hated him} because he said their perfect heavens were lurching around like drunkards. 
\end{itemize}

And just when things couldn’t get worse, his personal life problems were compounded by misfortune: \textbf{his mother was put on trial for witchcraft}. (Because obviously, if your son can predict planetary motion, your whole family must be in league with Satan.)

Kepler barely saved her from being burned at the stake, but he couldn’t save his own career. His insights, like the idea that you could use area to describe motion, were simply too strange, too early, and too threatening to the world’s tidy beliefs.

Kepler’s use of areas to describe motion was a game-changer. It wasn’t just about astronomy: it was about \textbf{introducing the idea that motion and change could be calculated dynamically}.

Of course, Kepler didn’t get to see himself become a legend. He died broke, excommunicated, and largely ignored. 


\begin{tcolorbox}[colback=blue!5!white, colframe=blue!50!black, breakable, title=Historical Sidebar: Kepler’s Theology—Excommunicated From All Sides]

  \textbf{Johannes Kepler} (1571–1630) didn’t just believe in harmony among the planets: he believed in harmony between the physical world and scripture. To him, discovering mathematical laws of the universe was a way of reading God’s mind. He once wrote, ``Geometry is co-eternal with God.''

  \medskip

  But the churches of his time were not as enthusiastic.  Kepler was a \textbf{Lutheran in a Catholic empire} so the Catholics didn’t trust him, especially once he started defending heliocentrism with Scripture. It got really complicated for him when he rejected key Lutheran doctrines like consubstantiation after reading the Bible himself, which got him \textbf{excommunicated from his own church}.  

  \medskip
  
  He was basically too Protestant for the Catholics, and too Protestant for the Protestants.

  \medskip

  He didn't care, though. Theology was not a side hustle for him: it shaped his entire worldview... including his scientific method. When Kepler discovered that planetary orbits were elliptical, he assumed it was part of God’s design and set off to discover what that design could be.


  \begin{quote}
  I give myself over to my rapture. I tremble, my blood leaps... I have stolen the golden vessels of the Egyptians to build a tabernacle for my God from them... The book is written. It will be read either by the present age or by posterity—I care not which. It can wait a century for a reader, as God has waited six thousand years for a witness.
  --- \textit{Harmonices Mundi} (1619)
  \end{quote}


\end{tcolorbox}


\begin{figure}[H]
\centering

% === First row ===
\begin{subfigure}[t]{0.45\textwidth}
\centering
\begin{tikzpicture}
  \comicpanel{0}{0}
    {Catholic Priest}
    {Kepler}
    {\footnotesize Heresy! You read the Bible... without a priest!}
    {(0,-0.6)}
\end{tikzpicture}
\caption*{Theological autonomy: forbidden.}
\end{subfigure}
\hfill
\begin{subfigure}[t]{0.45\textwidth}
\centering
\begin{tikzpicture}
  \comicpanel{0}{0}
    {Lutheran Pastor}
    {Kepler}
    {\footnotesize Heresy! You read the Bible... wrong.}
    {(0,-0.6)}
\end{tikzpicture}
\caption*{Interpretation: not a solo sport.}
\end{subfigure}

\vspace{1em}

% === Second row ===
\begin{subfigure}[t]{0.45\textwidth}
\centering
\begin{tikzpicture}
  \comicpanel{0}{0}
    {Cosmologist}
    {Kepler}
    {\footnotesize Heresy! You believe in ellipses and not circles.}
    {(0,-0.6)}
\end{tikzpicture}
\caption*{Orthodoxy dies in elliptical orbits.}
\end{subfigure}
\hfill
\begin{subfigure}[t]{0.45\textwidth}
\centering
\begin{tikzpicture}
  \comicpanel{0}{0}
    {Tycho Brahe}
    {Kepler}
    {\footnotesize I once got in a duel over math and lost my nose. You’ll be fine.}
    {(0,-0.6)}
\end{tikzpicture}
\caption*{Mentorship: Viking edition.}
\end{subfigure}

\caption{Kepler’s crime: using math to describe the universe. Everyone else’s crime: being alive in the 1600s.}
\end{figure}

