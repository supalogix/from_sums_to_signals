\section{Joseph-Louis Lagrange: From Inertia to the Lagrangian}

If Euler gave us the mathematics of rotation, then \textbf{Joseph-Louis Lagrange} gave us a new way to think about mechanics entirely: not as forces acting on objects, but as the unfolding of a system’s \textit{configuration over time}, driven by underlying principles of extremality.

Where Newton focused on forces, and Euler on rotational inertia, Lagrange asked:

\begin{quote}
\textit{What if we didn’t think in terms of forces at all? What if we described motion by how a system balances itself along paths of least effort?}
\end{quote}

\subsection{Generalizing Inertia: The Kinetic Energy View}

Euler’s moment of inertia measured a body’s resistance to changes in rotation:

\[
I = \int r^2 \, dm
\]

But Lagrange saw that inertia wasn’t just a property of spinning bodies—it was woven into the \textbf{kinetic energy} of a system:

\[
T = \frac{1}{2} m v^2
\]

For a rotating rigid body:

\[
T = \frac{1}{2} I \omega^2
\]

In other words, moment of inertia was a special case of a broader idea: kinetic energy expressed in generalized coordinates.

Lagrange took this insight and extended it to \textbf{any system}, with \textbf{any constraints}, moving in \textbf{any coordinates}. 

That quantity was encoded in a function:

\[
L(q, \dot{q}, t) = T - V
\]

where:

\begin{itemize}
  \item \( T \) is the kinetic energy,
  \item \( V \) is the potential energy,
  \item \( q \) are generalized coordinates,
  \item \( \dot{q} \) are their time derivatives.
\end{itemize}

Rather than calculate the forces directly, Lagrange proposed a bold idea:  
write down the difference between kinetic and potential energy as a function \( L \), and find the path that makes the integral of \( L \) over time stationary.

This function, the \textbf{Lagrangian}, became the master key to mechanics.

\begin{tcolorbox}[colback=gray!5!white, colframe=black, title=\textbf{Historical Sidebar: Why Subtract \( V \) from \( T \)?}, fonttitle=\bfseries, arc=1.5mm, boxrule=0.4pt]

The Lagrangian \( L = T - V \) encodes a subtle idea: motion unfolds not by maximizing or minimizing energy itself, but by balancing kinetic and potential energy across time.

The “difference” between motion and constraint—the kinetic striving and the potential resisting—maps out the path a system must take.

In essence: Nature doesn’t seek the shortest path or the fastest path—it seeks the path that makes \( T - V \) “stationary” (neither increasing nor decreasing infinitesimally when varied).
\end{tcolorbox}



\subsection{The Euler--Lagrange Equation in Lagrange’s Notation}

Once the Lagrangian \( L \) is defined as a function of the generalized coordinates \( q_i \), their time derivatives \( \dot{q}_i \), and time \( t \), Lagrange formulated the principle of virtual action:

\[
\delta \int_{t_1}^{t_2} L(q_i, \dot{q}_i, t)\, dt = 0
\]

To derive the equations of motion, Lagrange considered infinitesimal variations \( \delta q_i(t) \) that vanish at the endpoints \( t_1 \) and \( t_2 \). The resulting condition for extremizing the action becomes, for each coordinate \( q_i \):

\[
\frac{d}{dt} \left( \frac{dL}{d\dot{q}_i} \right) - \frac{dL}{dq_i} = 0
\]

Here, the differentials \( dL/d\dot{q}_i \) and \( dL/dq_i \) are total derivatives taken with respect to the variables \( \dot{q}_i \) and \( q_i \), respectively — not partial derivatives in the modern sense.

\medskip

Lagrange expressed the dynamics not as a force equation (\( F = ma \)) but as a variational principle: the actual path taken by a system is that which makes the integral of the difference between kinetic and potential energies stationary.

\medskip

This formulation allowed mechanics to be expressed without reference to geometric diagrams or Newtonian vectors — only analytic expressions. As Lagrange famously wrote:

\begin{quote}
\textit{“No diagrams will be found in this work. The methods I set forth require neither constructions nor geometrical or mechanical reasoning, but only algebraic operations, subject to a regular and uniform procedure.”}
\end{quote}


\subsection{A Mechanics of Possibilities}

Where Newton described the motion that \textit{is}, and Euler described the motion that \textit{resists}, Lagrange described all the motions that \textit{could be}—and selected the real one by a principle of extremality.

Motion, in Lagrange’s hands, became a problem of \textbf{choosing paths from possibility space}, not pushing objects with forces.

\begin{center}
\begin{tabular}{c|c|c}
\textbf{Newton} & \textbf{Euler} & \textbf{Lagrange} \\
\hline
Forces & Rotation/Inertia & Energies \\
\( F = ma \) & \( \tau = I \alpha \) & \( \frac{d}{dt} \frac{\partial L}{\partial \dot{q}} - \frac{\partial L}{\partial q} = 0 \) \\
Trajectories via force balance & Rotations via torque balance & Trajectories via action principle \\
\end{tabular}
\end{center}

\begin{tcolorbox}[colback=blue!5!white, colframe=blue!50!black, 
  title={Historical Sidebar: From Inertia to Extremality}]
  
Lagrange extended Euler’s idea of inertia from a resistance to rotation to a resistance to \textit{any deviation from natural paths}.

The moment of inertia measures how hard it is to twist a body; the Lagrangian measures how hard it is to deviate from a system’s natural trajectory through configuration space.

In both cases, motion resists arbitrary change—but Lagrange generalized this “resistance” into a universal mathematical structure.
\end{tcolorbox}



\subsection{Kepler’s Second Law Reimagined: In the Notation of Lagrange’s Time}

In Newton’s system, Kepler’s Second Law—the law of equal areas swept in equal times—was interpreted as a signature of a central force. Euler related it to angular momentum. But in \textbf{Lagrange’s analytic formulation}, the law emerged not from force diagrams, but from the structure of the equations themselves.

\medskip

Consider a system with one body of mass \( m \) revolving around a fixed center, described in polar coordinates \( (r, \theta) \). The expression for the system’s action depends only on the kinetic and potential energies. In Lagrange’s framework:

\[
L = \frac{1}{2} m \left( \dot{r}^2 + r^2 \dot{\theta}^2 \right) - V(r)
\]

The function \( L \) depends on \( r \), \( \dot{r} \), and \( \dot{\theta} \), but not explicitly on \( \theta \).

\medskip

To find the equations of motion, Lagrange applied the principle of least action:

\[
\delta \int L \, dt = 0
\]

and obtained the analytic equations that now bear his name. For a coordinate \( q \), they are:

\[
\frac{d}{dt} \left( \frac{dL}{d\dot{q}} \right) - \frac{dL}{dq} = 0
\]

Applying this to the coordinate \( \theta \), we observe:

\[
\frac{dL}{d\theta} = 0 \quad \text{(since \( \theta \) does not appear in \( L \))}
\]

Therefore,

\[
\frac{d}{dt} \left( \frac{dL}{d\dot{\theta}} \right) = 0
\]

That is:

\[
\frac{d}{dt} \left( m r^2 \dot{\theta} \right) = 0
\]

which implies:

\[
m r^2 \dot{\theta} = \text{constant}
\]

This is a conserved quantity arising purely from the fact that \( \theta \) is absent from the expression of \( L \). Lagrange recognized this as a \textit{first integral of the motion}.

\medskip

Now recall that the area swept out by the radius vector is:

\[
dA = \frac{1}{2} r^2 \, d\theta \quad \Rightarrow \quad \frac{dA}{dt} = \frac{1}{2} r^2 \dot{\theta}
\]

Hence:

\[
\frac{dA}{dt} = \frac{1}{2} \cdot \frac{\text{constant}}{m}
\]

and so \( \frac{dA}{dt} \) is constant.

\begin{tcolorbox}[colback=gray!5!white, colframe=black, title=\textbf{Historical Sidebar: A Quantity Conserved by Omission}]

Lagrange did not use the term “conjugate momentum.” But he understood something crucial: when a coordinate is absent from the expression of \( L \), its associated time-dependent term remains constant.

He treated this as an analytic fact: a “first integral” of the motion. It meant that something deeper than force was governing the system—a structural regularity built into the formulation of the equations.

\end{tcolorbox}

\subsubsection*{Equal Areas Without Forces}

In Lagrange’s formulation, Kepler’s Second Law is not deduced from central forces or geometry, but from the analytic structure of the motion itself:

\[
\frac{dA}{dt} = \text{constant}
\]

Because the angle \( \theta \) does not appear in the function \( L \), the variation procedure guarantees that its differential coefficient remains unchanged in time. This is enough to recover Kepler’s result, directly from calculus of variations.

\medskip

Where Kepler observed,
and Newton explained by force,
Lagrange revealed that the law followed inevitably from how the system was written.

\begin{quote}
What was once geometry, and then force, became invariance.
\end{quote}
