\section{Joseph-Louis Lagrange: From Inertia to the Lagrangian}

If Euler gave us the mathematics of rotation, then \textbf{Joseph-Louis Lagrange} gave us a new way to think about mechanics entirely: not as forces acting on objects, but as the unfolding of a system’s \textit{configuration over time}, driven by underlying principles of extremality.

Where Newton focused on forces, and Euler on rotational inertia, Lagrange asked:

\begin{quote}
\textit{What if we didn’t think in terms of forces at all? What if we described motion by how a system balances itself along paths of least effort?}
\end{quote}

\subsection{Generalizing Inertia: The Kinetic Energy View}

Euler’s moment of inertia measured a body’s resistance to changes in rotation:

\[
I = \int r^2 \, dm
\]

But Lagrange saw that inertia wasn’t just a property of spinning bodies—it was woven into the \textbf{kinetic energy} of a system:

\[
T = \frac{1}{2} m v^2
\]

For a rotating rigid body:

\[
T = \frac{1}{2} I \omega^2
\]

In other words, moment of inertia was a special case of a broader idea: kinetic energy expressed in generalized coordinates.

Lagrange took this insight and extended it to \textbf{any system}, with \textbf{any constraints}, moving in \textbf{any coordinates}. Whether a bead sliding on a wire, a double pendulum, or a spinning top—every system’s motion could be encoded in a single function:

\[
L = T - V
\]

where:

\begin{itemize}
    \item \( T \) = kinetic energy
    \item \( V \) = potential energy
\end{itemize}

This function, the \textbf{Lagrangian}, became the master key to mechanics.

\begin{tcolorbox}[colback=gray!5!white, colframe=black, title=\textbf{Historical Sidebar: Why Subtract \( V \) from \( T \)?}, fonttitle=\bfseries, arc=1.5mm, boxrule=0.4pt]

The Lagrangian \( L = T - V \) encodes a subtle idea: motion unfolds not by maximizing or minimizing energy itself, but by balancing kinetic and potential energy across time.

The “difference” between motion and constraint—the kinetic striving and the potential resisting—maps out the path a system must take.

In essence: Nature doesn’t seek the shortest path or the fastest path—it seeks the path that makes \( T - V \) “stationary” (neither increasing nor decreasing infinitesimally when varied).
\end{tcolorbox}

\subsection{The Euler–Lagrange Equation}

Once the Lagrangian \( L \) is defined, Lagrange introduced a procedure for extracting the equations of motion:

Suppose a system is described by generalized coordinates \( q_i(t) \). Then the path taken by the system minimizes (or more precisely, makes stationary) the action:

\[
S = \int_{t_1}^{t_2} L(q_i, \dot{q}_i, t) \, dt
\]

Applying the principle of stationary action leads to the \textbf{Euler–Lagrange equation}:

\[
\frac{d}{dt} \left( \frac{\partial L}{\partial \dot{q}_i} \right) - \frac{\partial L}{\partial q_i} = 0
\]

This equation replaced Newton’s \( F = ma \) with something far more general: a rule for deducing motion from energy balances, even in curvilinear coordinates, even with constraints.

\subsection{A Mechanics of Possibilities}

Where Newton described the motion that \textit{is}, and Euler described the motion that \textit{resists}, Lagrange described all the motions that \textit{could be}—and selected the real one by a principle of extremality.

Motion, in Lagrange’s hands, became a problem of \textbf{choosing paths from possibility space}, not pushing objects with forces.

\begin{center}
\begin{tabular}{c|c|c}
\textbf{Newton} & \textbf{Euler} & \textbf{Lagrange} \\
\hline
Forces & Rotation/Inertia & Energies \\
\( F = ma \) & \( \tau = I \alpha \) & \( \frac{d}{dt} \frac{\partial L}{\partial \dot{q}} - \frac{\partial L}{\partial q} = 0 \) \\
Trajectories via force balance & Rotations via torque balance & Trajectories via action principle \\
\end{tabular}
\end{center}

\begin{tcolorbox}[colback=blue!5!white, colframe=blue!50!black, 
  title={Historical Sidebar: From Inertia to Extremality}]
  
Lagrange extended Euler’s idea of inertia from a resistance to rotation to a resistance to \textit{any deviation from natural paths}.

The moment of inertia measures how hard it is to twist a body; the Lagrangian measures how hard it is to deviate from a system’s natural trajectory through configuration space.

In both cases, motion resists arbitrary change—but Lagrange generalized this “resistance” into a universal mathematical structure.
\end{tcolorbox}

\subsection{The Legacy of Lagrange}

Today, the Euler–Lagrange equation underlies classical mechanics, quantum mechanics, field theory, and even machine learning’s variational methods.

Where Newton gave us a story of forces, and Euler gave us a story of torques, Lagrange gave us a story of \textbf{structure}: a universe governed not just by pushes and pulls, but by deeper principles of balance, extremality, and symmetry.

Whenever we write:

\[
\frac{d}{dt} \frac{\partial L}{\partial \dot{q}} - \frac{\partial L}{\partial q} = 0
\]

we’re speaking Lagrange’s language: a mechanics that sees the path not as imposed by force, but as selected by harmony.


\subsection{Kepler’s Second Law Reimagined: The Lagrangian Interpretation in Lagrange’s Time}

In Newton’s system, Kepler’s Second Law—the law of equal areas swept in equal times—was a sign of a central force. Euler reframed it through angular momentum. But in \textbf{Lagrange’s formulation}, it could be derived without appealing directly to forces at all—emerging instead from the structure of the Lagrangian itself.

\medskip

For a planet of mass \( m \) orbiting the Sun under a central force, in polar coordinates \( (r, \theta) \), the Lagrangian is:

\[
L = T - V = \frac{1}{2} m (\dot{r}^2 + r^2 \dot{\theta}^2) - V(r)
\]

where:

\begin{itemize}
    \item \( T \) is the kinetic energy in polar coordinates,
    \item \( V(r) \) is the potential energy, depending only on distance from the Sun.
\end{itemize}

Crucially, **the angle \( \theta \) does not appear explicitly in \( L \)**.

\begin{tcolorbox}[colback=gray!5!white, colframe=black, title=\textbf{Historical Sidebar: A Coordinate “Missing” from the Equation}, fonttitle=\bfseries, arc=1.5mm, boxrule=0.4pt]

Lagrange recognized that when a coordinate does not appear in the Lagrangian, it implies something special: the quantity conjugate to that coordinate (its “momentum” in a generalized sense) remains constant.

Though the formal language of “symmetry” wasn’t fully developed in his time, Lagrange understood this invariance as a structural feature of the equations themselves.

\end{tcolorbox}

Since \( \theta \) is absent from \( L \), we can compute its conjugate momentum:

\[
p_\theta = \frac{\partial L}{\partial \dot{\theta}} = m r^2 \dot{\theta}
\]

And because \( \theta \) does not appear in \( L \), Lagrange’s equations tell us:

\[
\frac{d}{dt} p_\theta = 0
\]

In other words:

\[
m r^2 \dot{\theta} = \text{constant}
\]

This is none other than **angular momentum conservation**.

\medskip

But geometrically:

\[
dA = \frac{1}{2} r^2 d\theta
\quad \Rightarrow \quad
\frac{dA}{dt} = \frac{1}{2} r^2 \dot{\theta}
\]

Substituting:

\[
\frac{dA}{dt} = \frac{1}{2} \frac{p_\theta}{m}
\]

Since \( p_\theta \) is constant, the rate of area swept \( \frac{dA}{dt} \) is constant.

Thus, even without invoking forces, Lagrange could derive Kepler’s Second Law as a natural mathematical consequence of the structure of the Lagrangian:

\[
\frac{dA}{dt} = \text{constant}
\]

\subsubsection{A Structural Law, Not Just a Geometric Rule}

In Lagrange’s system, Kepler’s Second Law isn’t merely an observed regularity, nor only a geometric deduction: it arises from the fact that the Lagrangian depends only on \( r \), not \( \theta \).  
The absence of \( \theta \) in the equation guarantees a constant associated with angular motion.

\begin{quote}
In Newton’s formulation, the law pointed to a central force.  
In Lagrange’s formulation, the law pointed to an invariance in the equations of motion themselves.
\end{quote}

\begin{tcolorbox}[colback=blue!5!white, colframe=blue!50!black, 
  title={Historical Sidebar: What Lagrange Saw in Kepler’s Law}]

To Lagrange, Kepler’s Second Law was not just an observational quirk—it was the signature of a deeper balance within the system’s dynamics.

He saw in it an implicit “first integral” of the motion: a quantity conserved across time, derivable not from force diagrams, but from the analytic structure of the equations themselves.

In Lagrange’s hands, mechanics moved from geometry and force to a universal calculus of trajectories.

\end{tcolorbox}

\subsection{The Mathematical Power of the Lagrangian}

Thus, by writing the system’s kinetic and potential energies in polar coordinates and noting which variables disappeared from the equation, Lagrange could recover Kepler’s Second Law not by geometry, nor by physical intuition, but by the quiet algebraic properties of \( L \) itself.

For Lagrange, this wasn’t just a computational trick—it was a glimpse of a deeper architecture beneath mechanics:

\begin{quote}
Some quantities stay constant not because of external forces, but because of how the problem is written.
\end{quote}

Kepler had seen the sweep of areas.  
Newton had seen the pull of forces.  
Lagrange saw the persistence of structure.

