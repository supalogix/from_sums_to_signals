\section{Lagrange and the Equation of Principles: From Forces to Functions}

If Euler showed that the mechanics of rotation could be written as equations of forces and torques,  
Joseph-Louis Lagrange asked an even more radical question:

\begin{quote}
“What if we could describe motion without forces at all?”
\end{quote}

Where Newton had grounded mechanics in vectors of push and pull, and Euler had extended those ideas to the rotations of solid bodies,  
Lagrange saw a deeper structure hiding beneath the surface:  
a world where motion wasn’t driven by forces acting along coordinates, but by the delicate balance of energies across a system.

\bigskip

\subsection*{From Equations of Motion to Equations of Action}

Lagrange’s insight was to step back from the machinery of vectors and balance sheets of forces.  
Instead, he framed mechanics as a problem of finding paths:  
the trajectory a system follows is the one that makes a certain quantity—the \textbf{action}—stationary.

That quantity was encoded in a function:

\[
L(q, \dot{q}, t) = T - V
\]

where:

\begin{itemize}
  \item \( T \) is the kinetic energy,
  \item \( V \) is the potential energy,
  \item \( q \) are generalized coordinates,
  \item \( \dot{q} \) are their time derivatives.
\end{itemize}

Rather than calculate the forces directly, Lagrange proposed a bold idea:  
write down the difference between kinetic and potential energy as a function \( L \), and find the path that makes the integral of \( L \) over time stationary.

This principle—the \textbf{principle of least action}—transformed mechanics from a theory of forces to a theory of optimization.

And from it emerged a new kind of differential equation:

\[
\frac{d}{dt} \frac{\partial L}{\partial \dot{q}_i} - \frac{\partial L}{\partial q_i} = 0
\]

The \textbf{Euler–Lagrange equation}.

A differential equation not of pushes and pulls, but of energy and variation.

\bigskip

\begin{tcolorbox}[colback=gray!5!white, colframe=black, title=\textbf{Sidebar: The Shift from Euler to Lagrange}, fonttitle=\bfseries, arc=1.5mm, boxrule=0.4pt]

\begin{tabular}{>{\raggedright}p{3.5cm} >{\raggedright}p{5.5cm} >{\raggedright\arraybackslash}p{5.5cm}}
 & \textbf{Euler} & \textbf{Lagrange} \\
\midrule
Equations arise from & Newton’s laws + angular momentum & Principle of least action \\
Governing principle & Balance of forces and torques & Extremization of \( \int L \, dt \) \\
Form of equation & Force = rate of change of momentum & Stationary action → Euler–Lagrange equation
\end{tabular}

\end{tcolorbox}

\bigskip

\subsection*{Bringing Inertia into the Equation of Principles}

In Euler’s mechanics, the moment of inertia \( I \) appeared as a geometric factor in torque equations:

\[
\tau = I \alpha
\]

a parameter in balancing angular accelerations and forces.

But in Lagrange’s hands, \( I \) moved inside the kinetic energy itself:

\[
T = \frac{1}{2} I \omega^2
\]

No longer just a coefficient in an equation of forces, inertia became a building block in the function \( L = T - V \).  
And once inside \( L \), inertia found its way into the variational principle itself.

When the Euler–Lagrange equation is applied to \( L \), the result is not merely a restatement of Euler’s torque laws—it is a derivation of the rotational equations of motion as a consequence of extremizing action.

In this move, Lagrange achieved something profound:

✅ He folded inertia, motion, and forces into a single function.  
✅ He showed that motion follows from a principle, not just from balances.  
✅ He elevated mechanics into an analytical framework, independent of forces along Cartesian axes.

\bigskip

\subsection*{The Historical Leap}

Euler brought moment of inertia into mechanics as a geometric measure;  
he extended Newton’s laws into the rotational world.

But Lagrange saw a deeper order:  
he tied inertia and motion into a universal principle, one that applied equally to rotating tops, swinging pendulums, vibrating strings, and orbiting planets.

Where Euler wrote differential equations from forces,  
Lagrange derived differential equations from an extremal principle.

And in doing so, he turned mechanics from a catalog of motions into a calculus of possibilities.

\bigskip

\begin{quote}
In Euler, we see the world as a balance of pushes and pulls;  
in Lagrange, we see the world as a tapestry woven from a principle.
\end{quote}

The mechanics of forces had become the mechanics of action.

And the road from Lagrange would lead not only to Hamilton, but ultimately to the geometry of spacetime itself.
