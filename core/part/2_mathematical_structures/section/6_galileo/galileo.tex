\section{Galileo’s Experiments: The Renaissance of Motion (1610 - 1638)}

\begin{figure}[H]
\centering
\begin{tikzpicture}[every node/.style={font=\footnotesize}]

% Panel 1 — Galileo explains inertia
\comicpanel{0}{4}
  {Galileo}
  {Aristotle}
  {\textbf{Galileo:} I rolled a ball down a ramp. The flatter the surface, the longer it moved. I think motion continues unless something stops it.}
  {(0,-0.5)}

% Panel 2 — Aristotle looks unimpressed
\comicpanel{6.5}{4}
  {Aristotle}
  {Galileo}
  {\textbf{Aristotle:} So... you’re saying a rock just keeps moving? On its own? That’s absurd. Motion needs a purpose.}
  {(0,-0.5)}

% Panel 3 — Galileo is confused
\comicpanel{0}{0}
  {Galileo}
  {Aristotle}
  {\textbf{Galileo:} It doesn't need a purpose. It just keeps going. That’s inertia.}
  {(0,0.8)}

% Panel 4 — Aristotle lands the punchline
\comicpanel{6.5}{0}
  {Aristotle}
  {Student}
  {\textbf{Aristotle:} Motion without purpose? Might as well say the rock has free will.}
  {(0,0.8)}

\end{tikzpicture}
\caption{Inertia: the radical idea that objects don’t need permission to keep moving.}
\end{figure}

\subimport*{subsection}{1_galileos_big_shot_v2.tex}
\subimport*{subsection}{2_galileo_vs_the_church.tex}