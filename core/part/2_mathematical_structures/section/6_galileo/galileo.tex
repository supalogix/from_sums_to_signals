\section{Galileo’s Experiments: The Renaissance of Motion (1610 - 1638)}


\begin{figure}[H]
  \centering

  % === First row ===
  \begin{subfigure}[t]{0.45\textwidth}
  \centering
  \begin{tikzpicture}
    \comicpanel{0}{0}
      {Galileo}
      {Aristotle}
      {\small I dropped a rock and it kept going. I think motion continues unless something stops it.}
      {(0,-0.5)}
  \end{tikzpicture}
  \caption*{Galileo presents his experiment: motion persists unless interrupted.}
  \end{subfigure}
  \hfill
  \begin{subfigure}[t]{0.45\textwidth}
  \centering
  \begin{tikzpicture}
    \comicpanel{0}{0}
      {Galileo}
      {Aristotle}
      {A rock just keeps moving? On its own? That’s absurd. Motion needs a purpose.}
      {(0,-0.5)}
  \end{tikzpicture}
  \caption*{Aristotle remains unconvinced: motion must have a purpose.}
  \end{subfigure}

  \vspace{1em}

  % === Second row ===
  \begin{subfigure}[t]{0.45\textwidth}
  \centering
  \begin{tikzpicture}
    \comicpanel{0}{0}
      {Galileo}
      {Aristotle}
      {It doesn't need a purpose. It just keeps going. That’s inertia.}
      {(0,0.8)}
  \end{tikzpicture}
  \caption*{Galileo explains inertia: motion without need for intent.}
  \end{subfigure}
  \hfill
  \begin{subfigure}[t]{0.45\textwidth}
  \centering
  \begin{tikzpicture}
    \comicpanel{0}{0}
      {Aristotle}
      {Student}
      {Motion without purpose? You might as well say the rock has free will.}
      {(0,0.8)}
  \end{tikzpicture}
  \caption*{Aristotle delivers the punchline: destiny for rocks.}
  \end{subfigure}

  \caption*{Inertia: the radical idea that objects don’t need permission to keep moving.}
\end{figure}


\subimport*{subsection}{1_galileos_big_shot_v2.tex}
\subimport*{subsection}{2_galileo_vs_the_church.tex}