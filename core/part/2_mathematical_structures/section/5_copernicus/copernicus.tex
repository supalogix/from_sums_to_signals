\section{Copernicus: Geometry Unchanged, Cosmos Recentered}

\subsection{Rearranging the Heavens: Copernicus’s Geometric Revolution}

Copernicus didn’t invent new mathematics. What he did was more subtle — and more radical. He took the geometrical machinery of Ptolemy’s astronomy and asked a simple, disorienting question:

\begin{quote}
\textbf{What if the Sun, not the Earth, was at the center of it all?}
\end{quote}

This was not an algebraic revolution. Copernicus worked with the same tools as Ptolemy: chords, angles, triangles, and intersecting circles. But he used them to build an entirely new cosmological structure.

\medskip

\begin{HistoricalSidebar}{Feyerabend, Helocentrism, and the Accidental Truth of Error}

  \textbf{Paul Feyerabend} (1924–1994) was one of the most controversial philosophers of science in the 20th century. 
  Best known for his book \textit{Against Method}, Feyerabend challenged the very idea that science advances through 
  a strict, rational method. He argued instead for \textbf{epistemological anarchism}: the view that “anything goes” 
  when it comes to the history of scientific discovery.
  
  \medskip
  
  One of Feyerabend’s core ideas was \textbf{incommensurability}: the claim that competing scientific theories often 
  operate in different conceptual worlds, making them impossible to directly compare by neutral standards. According 
  to him, the transition from one theory to another (say, from geocentric to helocentric) isn’t always driven by logic 
  and evidence, but by cultural shifts, aesthetic preferences, and even historical accidents.
  
  \medskip

  Enter Helocentrism.

  \medskip

  \textbf{Before the Sun was a ball of gas, it was a hearth.}

  \medskip
  
  The \textbf{Pythagoreans} envisioned the cosmos not as Earth-centered, but organized around a \textit{Central Fire}.
  You can think of it as an unseen source of cosmic harmony that they called \textbf{Hestia}, the divine hearth. This 
  was no physical sun, but a metaphysical flame around which all things revolved in balance.

  \medskip
  
  One of their followers, \textbf{Philolaus}, proposed that Earth and other celestial bodies circled this fire
  not for physical reasons, but for philosophical ones: \textit{symmetry, balance, and divine proportion}.

  \medskip
  
  \textbf{Copernicus}, a thousand years later, rediscovered this idea through ancient sources. Though he replaced the 
  Central Fire with the \textbf{Sun}, he kept the symbolic core: the cosmos was not a chaotic swirl but a harmonious 
  machine, mathematically tuned, with a radiant center.

  \medskip
  
  To him, the Sun was not just luminous—it was \textit{luminous with meaning}.

  \medskip

  \begin{quote}
  \textit{In the middle of all sits the Sun. Who could place this lamp in another or better place?} \\
  — \textbf{Nicolaus Copernicus}
  \end{quote}
  
  \medskip
  
  \textbf{Feyerabend’s moral?} Science progresses not because it always gets things “right,” but because even wrong 
  ideas can open the door to profound truths. The caloric theory didn’t just fail—it failed \textit{productively}.

  \medskip
  
  \begin{quote}
  \textit{“The only principle that does not inhibit progress is: anything goes.”} —Paul Feyerabend, \textit{Against Method}
  \end{quote}
  
\end{HistoricalSidebar}

\subsection{What He Actually Did in \textit{De revolutionibus} (1543)}

Copernicus took the Ptolemaic system — circles on circles — and simply moved the center. This switch to a heliocentric model naturally explained:

\begin{itemize}
  \item \textbf{Retrograde motion} as a visual illusion from a moving Earth,
  \item \textbf{Planetary brightness} as a function of varying distance from Earth,
  \item \textbf{Planetary order} where inner vs. outer planets now had geometric meaning.
\end{itemize}

Yes, Copernicus still used epicycles — about 40 in total (fewer than Ptolemy, but not drastically). Each planet moved on a small circle (epicycle) whose center moved on a larger circle (deferent), but now around the Sun.

\begin{itemize}
  \item Earth itself moved around the Sun.
  \item Planetary positions were computed \textbf{relative to a moving Earth}.
  \item This required the same methods of geometric computation: central angles, chords, and proportional diagrams.
\end{itemize}

Like Ptolemy, Copernicus used the function:

\[
\text{chord}(\theta) = 2R \cdot \sin\left(\frac{\theta}{2}\right)
\]

He applied it to compute angular separations — such as elongation, quadrature, and conjunction — by measuring the chord between two points on the orbit circle.

\subsection{A Typical Copernican Calculation: Elongation of Venus}

\begin{enumerate}
  \item Look up Earth’s heliocentric position on a given date.
  \item Look up Venus’s heliocentric longitude from a table.
  \item Subtract to find angular separation as seen from the Sun.
  \item Use chord tables and geometry to compute the angle as seen from Earth — accounting for Earth's own motion.
\end{enumerate}

\begin{center}
\begin{tcolorbox}[colback=gray!5!white, colframe=black, boxrule=0.3pt, arc=1mm, width=0.9\linewidth]
\textbf{Diagram structure:}  
\begin{itemize}
  \item Sun at the center  
  \item Earth orbiting on one circle  
  \item Venus on a smaller epicycle around its deferent  
  \item Lines drawn from Sun to Venus and Sun to Earth  
  \item Angle between those lines = elongation  
\end{itemize}
Then: use \textbf{chord tables} to compute visual angle or distance.
\end{tcolorbox}
\end{center}


\begin{quote}
Copernicus didn’t change the tools — he changed the center.
\end{quote}

He inherited Ptolemy’s machinery but pointed it in a new direction. What had once been mathematical scaffolding for an Earth-centered universe became, in Copernicus’s hands, the geometry of a solar system.

\begin{HistoricalSidebar}{Augustine’s Firmament: Between Scripture and Cosmos}

  Long before Copernicus re-centered the cosmos, \textbf{Augustine of Hippo} wrestled with an older cosmological puzzle: the \textbf{firmament} of Genesis.

  \medskip
  
  Genesis 1 described God placing a firmament—Latin \emph{firmamentum}, Greek \emph{stereōma}—to separate the “waters above” from the “waters below.” Some early interpreters imagined a solid dome; others saw crystalline spheres.

  \medskip
  
  Augustine wasn’t satisfied with either.

  \medskip
  
  In his \emph{De Genesi ad Litteram}, Augustine explored multiple interpretations:

  \medskip
  
  \begin{itemize}
    \item Was the firmament a literal structure in the heavens?
    \item Or a metaphor for invisible, spiritual realities?
    \item Or a poetic way of describing cosmic order?
  \end{itemize}

  \medskip
  
  He proposed that the “waters above” might refer to angelic or immaterial beings—or simply to mysteries beyond human knowledge. Above all, he warned against reading cosmological descriptions as literal physics.
  
  \medskip
  
  \textbf{For Augustine, the cosmos was readable—but not like a geometry problem.} Scripture spoke in layered ways: literal, allegorical, moral, anagogical. To understand the universe rightly was to interpret it rightly.
  
  \medskip
  
  \begin{center}
  \emph{If Copernicus shifted the cosmos physically, Augustine reminded that meaning was never just in the map.}
  \end{center}
  
\end{HistoricalSidebar}