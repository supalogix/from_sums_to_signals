
\section{Ptolemy and The Celestial Circles: When Geometry Became a Calculator}

\subsection{From Harmony to Heuristics: When Math Got Down to Earth}

The Pythagoreans dreamed of a cosmos that resonated like a lyre — where planets sang and proportions whispered secrets of divine order. But while they listened for music in the stars, later thinkers began asking a more pragmatic question: \textit{Where exactly is Mars going to be next Tuesday?}

Enter Claudius Ptolemy.

Where the Pythagoreans sought meaning, Ptolemy sought measurement. He kept the circular perfection, the geocentric certainty, and the divine symmetry — but turned these ideals into instruments of calculation.

Instead of chords on strings, he used chords in circles. Instead of mystical harmony, he gave us geometric astronomy.

And so began the age of epicycles.

Ptolemy’s model wasn’t just about celestial prediction; it was about \emph{preserving a metaphysical worldview}. His system endured for over a thousand years, not because it was the most accurate, but because it was philosophically elegant. It kept the cosmos ordered, the Earth central, and the heavens divine.

But when you looked at the planets, they defied those expectations. They sped up, slowed down, and sometimes even moved backwards (retrograde motion). Ptolemy reconciled this by constructing a system of \textbf{epicycles and deferents} — circles moving on other circles — all governed by geometric rules.

To compute planetary positions, he used:

\begin{itemize}
    \item Chords of central angles in a circle,
    \item Geometric transformations of rotating systems,
    \item Interpolation from chord tables — not modern trigonometric functions.
\end{itemize}

\subsection{The Chord Table: A Trigonometric System Without Trig}

Ptolemy used the function:

\[
\text{chord}(\theta) = 2R \cdot \sin\left(\frac{\theta}{2}\right)
\]

with the Babylonian \( R = 60 \). He tabulated chords for angles from \( 0^\circ \) to \( 180^\circ \) in half-degree steps.

Ptolemy, following the Greek tradition and using Babylonian sexagesimal conventions, defined the \textbf{chord} of an angle \( \theta \) as the length of the straight line connecting the endpoints of an arc of a circle of radius \( R = 60 \). This definition predates the modern sine function.

\begin{table}[H]
\centering
\renewcommand{\arraystretch}{1.4}
\begin{tabular}{|c|c|}
\hline
\textbf{Angle \( \theta \)} & \textbf{Chord Length (in terms of \( R \))} \\
\hline
\( 0^\circ \) & \( 0 \) \\
\( 30^\circ \) & Geometrically constructed value \\
\( 60^\circ \) & \( R \) \\
\( 90^\circ \) & \( R \cdot \sqrt{2} \) \\
\( 120^\circ \) & \( R \cdot \sqrt{3} \) \\
\( 180^\circ \) & \( 2R \) \\
\hline
\end{tabular}
\caption{Sample chord values from Ptolemy’s table using radius \( R = 60 \). Expressions are shown in geometric terms, avoiding modern trigonometric notation.}
\end{table}


\subsection{Tracing Divinity in the Firmament: A Geometric Identity for the Sky}

Ptolemy also derived the identity:

\[
\text{chord}(A + B) \cdot \text{chord}(A - B) = \text{chord}^2(A) - \text{chord}^2(B)
\]

A precursor to modern sine addition formulas, this was used to derive compound angles geometrically which allowed Ptolemy to compute values he hadn't explicitly tabulated.

For Ptolemy, geometry \textit{was reality}. Circles, chords, and ratios didn’t just describe planetary motion — they \textit{were} planetary motion.


\begin{figure}[H]
    \centering
    \begin{tikzpicture}[scale=2.5, every node/.style={font=\small}]
    
    % Earth at center
    \filldraw[black] (0,0) circle (0.02) node[below left] {Earth};
    
    % Deferent (large circle)
    \draw[thick] (0,0) circle (1);
    
    % Epicycle center (on the deferent)
    \coordinate (C) at (60:1);
    \filldraw[gray!50] (C) circle (0.015);
    \node[above right] at (C) {Epicycle center};
    
    % Epicycle (small circle on deferent)
    \draw[dashed] (C) circle (0.25);
    
    % Planet position on epicycle
    \coordinate (P) at ($(C) + (150:0.25)$);
    \filldraw[blue] (P) circle (0.02);
    \node[above left] at (P) {Planet};
    
    % Chord from Earth to planet
    \draw[red, thick] (0,0) -- (P) node[midway, above right] {Chord};
    
    % Label arc angle (planet's apparent motion)
    \draw[->, thick] (0.4,0) arc[start angle=0, end angle=60, radius=0.4];
    \node at (30:0.55) {$\theta$};
    
    % Optional: radii to epicycle center and to planet
    \draw[dotted] (0,0) -- (C);
    \draw[dotted] (C) -- (P);
    
    \node at (30:1.05) {Deferent};
    \node at ($(C)!0.5!(P) + (-0.05, 0.08)$) {Epicycle};
    
    \end{tikzpicture}
    \caption{Ptolemy’s epicycle system: A small circle (epicycle) rides a larger one (deferent) centered on Earth. The chord from Earth to the planet models its apparent position.}
\end{figure}
