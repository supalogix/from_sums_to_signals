\section{Al-Battānī and the Rise of Trigonometric Functions: From Geometry to Algebra}

\subsection{From Chords to Sines: A New Language for the Sky}

By the 9\textsuperscript{th} century, Islamic astronomers had inherited the full technical apparatus of Greek astronomy: Ptolemy’s chord tables, geometric epicycles, and the firmament's nested spheres. But they didn’t just preserve it — they transformed it.

One of the most important of these reformers was \textbf{Al-Battānī} (c.~858–929), a Syrian astronomer whose work bridged classical Greek astronomy and the new Islamic mathematical tradition. In his magnum opus, the \textit{Kitāb az-Zīj}, he did something subtle but revolutionary:

\begin{center}
\textbf{He replaced chords with sines.}
\end{center}

\medskip

Where Ptolemy had computed the straight-line segment between arc endpoints, Al-Battānī instead used the vertical projection from a point on the arc down to the circle’s radius — what we now call the \textbf{sine function}.

\medskip

This was not just notational convenience. It was a conceptual shift:

\begin{itemize}
    \item \textbf{Chords} are geometric: defined by line segments across circles.
    \item \textbf{Sines} are algebraic: functions of angles, definable and tabulatable.
\end{itemize}

\subsection{The New Trigonometric Arsenal}

Al-Battānī introduced and systematically used the sine (\( \sin \)), cosine (\( \cos \)), and tangent (\( \tan \)) functions in computations involving:

\begin{itemize}
    \item Shadow lengths and solar altitudes,
    \item Angles of elevation for celestial bodies,
    \item Spherical triangles for planetary motion.
\end{itemize}

He also derived several trigonometric identities, including:
\[
\cos(x) = \sin(90^\circ - x)
\]
\[
\tan(x) = \frac{\sin(x)}{\cos(x)}
\]

These were the building blocks of what would eventually become modern trigonometry.

\subsection{Legacy}

Al-Battānī's tables of sine values (using a radius \( R = 60 \), consistent with Babylonian conventions) remained influential for centuries. They were cited by European scholars including Regiomontanus and Copernicus, and helped usher in the transition from geometric astronomy to analytic trigonometry.

By abandoning the chord in favor of the sine, Al-Battānī set the stage for a new era — where geometry bent to algebra, and the heavens were no longer just drawn, but calculated.
