\section{Riemann Bends the World: Motion Finds a New Geometry (1854)}

Hamilton taught us to see motion as structure. He turned trajectories into flows and mechanics into geometry. The key player? The \textbf{dot product}—a quiet mathematical tool that suddenly ran the show. It told us how aligned a force was with motion, how much energy moved in which direction. Motion became a matter of angles and alignment. Clean. Elegant. Flat.

But Hamilton’s world was still smooth and Euclidean. Space was a fixed, invisible stage. The dot product was always the same, no matter where you were. 

\textbf{Then Riemann bent the world.}

In his 1854 habilitation lecture—easily one of the most quietly explosive talks in the history of mathematics—Bernhard Riemann asked a question that detonated the foundations of geometry:

\begin{quote}
\textit{What if the dot product itself could change from place to place?}
\end{quote}

In flat space, the infinitesimal distance between two points is given by:
\[
ds^2 = dx^2 + dy^2 + dz^2
\]
Which is just the square of a standard dot product:
\[
ds^2 = \vec{dx} \cdot \vec{dx}
\]

But Riemann didn’t want “standard.” He wanted space to be free to twist, stretch, and curve. So he replaced the dot product with something far more flexible:

\[
ds^2 = \sum_{i,j=1}^n g_{ij}(x) \, dx^i dx^j
\]

The functions \( g_{ij}(x) \) form the \textbf{metric tensor}—a position-dependent field of dot products. At every point in space, they tell you how to measure length, angle, and curvature. The geometry of motion wasn’t fixed anymore. It varied. Locally. Smoothly. Dynamically.

\textbf{Hamilton gave us the form of motion. Riemann gave us its texture.}

In Riemann’s world:
\begin{itemize}
  \item Direction is no longer universal—it depends on where you are.
  \item Distance is no longer absolute—it bends and twists with the shape of space.
  \item The dot product becomes a question, not a fact: \textbf{“How does motion behave \emph{here}?”}
\end{itemize}

The real magic? This wasn’t just geometry. This was a way to guide motion.

The metric tells us how steep a slope is. How far a step takes us. It defines the very \emph{feel} of change. And that’s exactly what powers one of the core tools in modern machine learning:

\textbf{Gradient descent.}

To move downhill, you need to know which direction is steepest. But on a curved space, “steepest” depends on the local metric. The gradient isn’t just a list of partial derivatives anymore. It becomes the vector that \emph{maximizes increase under the local dot product}:

\[
\text{grad}_g f \quad \text{is the unique vector such that} \quad g(\text{grad}_g f, v) = df(v) \quad \forall v
\]

This is Riemann’s twist on Hamilton’s idea: use a dot product to measure how motion aligns with change—but let the dot product change, too.

\textbf{In Riemann’s world, motion doesn’t just follow rules. It follows the shape of space itself.}


\begin{tcolorbox}[colback=blue!5!white, colframe=blue!50!black,
title={Sidebar: Hamilton’s Dot Product Grows Up}]
Hamilton introduced the dot product as a tool for measuring projection, alignment, and energy.

Riemann turned that dot product into a \textbf{field}—a geometry where every point has its own rulebook for measurement.

Together, they gave us the key to modern optimization:
\begin{itemize}
  \item Measure alignment (Hamilton)
  \item Let that measurement depend on context (Riemann)
  \item Use it to move efficiently through space (Gradient Descent)
\end{itemize}
\end{tcolorbox}

\subsection{Descent Finds a Geometry: When Optimization Started Walking the Curve}

By bending the rules of space, Riemann didn’t just rewrite geometry—he reshaped how motion \textit{happens}.

With a local dot product at every point, you can now define what it means to move downhill—not just in flat terrain, but across landscapes of curvature. Whether you're walking across a hilly manifold, navigating a space of probability distributions, or optimizing a neural network, one truth remains:

\textbf{Motion listens to geometry.}

\begin{quote}
If Hamilton gave us the music of motion,  
Riemann handed us the score—drawn not on a grid,  
but on a surface that bends and curves beneath each step.
\end{quote}

This changed everything.

Gradient descent was no longer just about partial derivatives—it became a choreography. A direction, weighted and warped by the local metric. You weren’t just chasing the steepest slope. You were following the slope as measured by the ground you stood on.

\textbf{Descent became a dialogue between direction and space.}

And when we get to optimization in machine learning—on curved parameter spaces, in high-dimensional loss surfaces—Riemann’s influence will echo again:

\begin{quote}
To find the optimal path, you don’t just fall downhill.  
You fall \textit{with respect to the space you’re in}.
\end{quote}



\subsection{Kepler Reimagined: When Planets Followed Gradients}

Let’s return to a law that once looked merely poetic:

\begin{quote}
\textit{A planet sweeps out equal areas in equal times as it orbits the Sun.}
\end{quote}

To Kepler, this was celestial harmony. To Newton, a byproduct of force.  
But to us—living in a post-Riemann, post-Hamilton world—it reveals something deeper:

\textbf{Motion that follows geometry. Flow that preserves structure.}

Kepler’s Second Law isn’t just about ellipses—it’s about conservation. In Hamiltonian mechanics, this flow arises from the conservation of angular momentum, encoded geometrically in the preservation of a symplectic form:
\[
\omega = dq \wedge dp
\]

This tells us the motion preserves area—not just on paper, but in phase space itself. The orbit is a kind of sacred choreography: no volume lost, no symmetry broken.

But let’s zoom in again—beyond phase space, into the curved fabric of configuration space. What’s guiding this flow?

\begin{itemize}
  \item The gravitational potential defines a scalar field \( f \).
  \item The geometry of space, defined by a Riemannian metric \( g \), tells us how to measure direction and steepness.
  \item Together, they shape how motion happens—not arbitrarily, but as a response to structure.
\end{itemize}

On this curved space, “falling toward the Sun” doesn’t mean charging straight at it. It means moving in the direction that, relative to the metric, most steeply reduces potential energy.

That direction is called the \textbf{Riemannian gradient}:
\[
g(\text{grad}_g f, v) = df(v) \quad \forall v
\]

This is where Riemann completes the story Newton began:  
\textbf{Motion is descent—structured by geometry, aligned with symmetry.}

So what is Kepler’s Second Law now?

It’s not just a description. It’s a constraint.

\begin{quote}
The planet descends through a gravitational potential,  
but only along paths that preserve angular momentum—  
a constraint enforced by the curvature of space itself.
\end{quote}

\textbf{The orbit becomes a solution to a geometric optimization problem:}

\begin{itemize}
  \item Minimize energy,  
  \item Subject to conserved angular momentum,  
  \item Guided by a metric that shapes what “steepest” means at every point.
\end{itemize}

\begin{tcolorbox}[colback=blue!5!white, colframe=blue!50!black,
title={Kepler’s Law as Gradient-Constrained Motion}]
Kepler’s Law encodes a system where motion obeys two masters:

\begin{itemize}
  \item A gradient that pulls,
  \item And a geometry that resists—preserving structure along the way.
\end{itemize}

From this tension, the orbit emerges.  
What looks like free fall is actually structured descent.
\end{tcolorbox}

In the end, Kepler wasn’t just watching planets.  
He was sketching the blueprint for modern optimization:

\textbf{Descent, constrained by symmetry, unfolding on curved space.}

And that blueprint still guides us—whether we’re modeling Mars, or training a neural network.
