\subsection{The Philosophy of Trajectories: An Imagined Soviet Debate}

When I picture the development of optimal control theory in the Soviet Union, I don’t just imagine engineers hunched over missile paths. I imagine something more ideological—a scene in which the foundations of classical physics itself are called into question. After all, Newtonian mechanics wasn’t just math; it was metaphysics. It treated the universe like a passive machine, unfolding from divine initial conditions. That might fly in Cambridge, but in Moscow? In the middle of the 20\textsuperscript{th} century? Not without a dialectical audit.

So here’s how I imagine things going down, somewhere inside a smoky academic office at the Academy of Sciences, circa 1957:

{
\ttfamily
\textbf{Party Official:} Comrade Pontryagin, I’ve been reviewing the foundations of your optimal control theory.  

\textbf{Pontryagin:} Yes, it’s all based on the Hamiltonian formalism. Very effective for guiding dynamic systems toward—  

\textbf{Party Official:} Hamiltonian, yes. Newtonian, Lagrangian, all very... tsarist-sounding.  

\textbf{Pontryagin:} I assure you, there’s nothing religious in the equations.  

\textbf{Party Official:} Are you aware that Newton said, “The most beautiful system of the sun, planets, and comets could only proceed from the counsel and dominion of an intelligent and powerful being”?  

\textbf{Pontryagin:} Perhaps... but it models ballistic motion quite reliably.  

\textbf{Party Official:} And what of Lagrange, who wrote that the goal of mechanics is to “glorify the works of the Creator through the elegance of mathematical expression”? Or Hamilton, who referred to dynamics as “a science of the divine harmonies of motion”?  

\textbf{Pontryagin:} Yes, well... I mostly skipped the theological prefaces.  

\textbf{Party Official:} It’s not about prefaces. It’s about foundations. These men built mechanics on top their religious beliefs. We need a framework grounded in dialectical materialism... not theology.  

\textbf{Pontryagin:} You want a dialectical Hamiltonian?  

\textbf{Party Official:} I want mathematics rooted in material reality, not metaphysical abstraction. Marx didn’t optimize over cotangent bundles. He studied systems in contradiction.  

\textbf{Pontryagin:} Then let me show you the adjoint equation. It doesn’t just observe: it responds. It evolves with feedback. That’s the costate: a shadow that adapts to the system’s goals.  

\textbf{Party Official:} Now you're talking. Control, feedback, transformation—this is Marxist calculus.  

\textbf{Pontryagin:} So... we keep the equations, just add historical agency?  

\textbf{Party Official:} Precisely. Keep your variational principles. Just make sure they steer us toward socialism. 
}


\begin{tcolorbox}[colback=gray!5!white, colframe=black, title=\textbf{Sidebar: The Origins of “Politically Correct”}, fonttitle=\bfseries, arc=1.5mm, boxrule=0.4pt]

    The phrase \textbf{“politically correct”} didn’t begin as a punchline about college campuses or cancel culture. Its roots trace back to early 20\textsuperscript{th}-century Marxist and communist circles, where it had a far more literal—and ominous—meaning.
    
    One of the most circulated origin stories involves an American visitor to the Soviet Union who pointed out a factual inconsistency in a Party official’s statement. The official reportedly replied:  
    \begin{quote}
    \textit{“Yes, you are technically correct... but you are not politically correct.”}
    \end{quote}
    
    In that moment, “politically correct” didn’t mean polite or inclusive—it meant \emph{aligned with the Party line}. Truth was subordinate to ideology. Facts were negotiable. What mattered was whether your statements advanced the revolutionary narrative.
    
    The term later resurfaced in Western leftist circles, often as an ironic self-critique. By the 1980s and 90s, it had migrated into American culture, losing its Soviet flavor and gaining new meaning as a label for progressive language policing.
    
    So next time someone accuses you of being politically incorrect, just smile and say: “Yes, but I’m historically grounded.”
    
\end{tcolorbox}

Here’s a darker, more cynical version of the sidebar—emphasizing ideological control, truth distortion, and the unnerving flexibility of language under authoritarian regimes:

\begin{tcolorbox}[title=Sidebar: “Politically Correct” — When Truth Requires a Permit, colback=black!5!white, colframe=black, fonttitle=\bfseries, coltitle=black]

The phrase \textbf{“politically correct”} wasn’t born in a university seminar on gender-inclusive language. It was born in the smoke-choked offices of totalitarian bureaucracy, where truth had to file paperwork.

According to one apocryphal but telling account, an American intellectual visiting the Soviet Union pointed out a factual error in a Party statement. The official replied, unflinching:
\begin{quote}
\textit{“Yes, you are technically correct... but you are not politically correct.”}
\end{quote}

The implication? In Stalinist logic, being right wasn’t enough. You had to be \emph{right in the right way}.  
Alignment with objective reality was secondary to alignment with ideological necessity.

To be “politically correct” meant saying what history—according to the Party—\emph{ought} to be true. If the facts contradicted the narrative, it was the facts that needed re-education.

In this light, “political correctness” wasn’t about avoiding offense—it was about controlling reality through language. Not a courtesy. A commandment.

\textit{Technically correct? That’s bourgeois logic.}

\end{tcolorbox}

\begin{tcolorbox}[title=Sidebar: “Politically Correct” — Where Truth Gets Redacted, colback=black!5!white, colframe=black, fonttitle=\bfseries, coltitle=black]

    The phrase \textbf{“politically correct”} didn’t start as campus jargon. It started as a warning.
    
    In the Soviet Union, being “technically correct” was never the point. One oft-repeated story tells of an American who pointed out a factual error during a Party meeting. A Soviet official responded, coldly:
    \begin{quote}
    \textit{“Yes, you are technically correct... but you are not politically correct.”}
    \end{quote}
    
    That wasn’t a joke. It was a verdict.
    
    To be politically correct meant your facts served the revolution. Your equations bent with historical necessity. Your speech marched in formation. Accuracy was useful only if it was loyal. Otherwise, it was sabotage.
    
    This was not epistemology. It was survival.
    
    In the USSR, truth was not discovered—it was issued, stamped, and occasionally shot.
    
    \textit{And if you found yourself correct but not aligned, the next correction was yours.}
    
\end{tcolorbox}


\begin{tcolorbox}[title=Sidebar: “Politically Correct” — A Truth Too Convenient?, colback=black!5!white, colframe=black, fonttitle=\bfseries\scshape, coltitle=black]

    The phrase \textbf{“politically correct”} didn’t originate in the seminar rooms of liberal arts colleges—it emerged from the ideological trenches of Marxist-Leninist orthodoxy.  

    \medskip
    
    In the early Soviet Union, to be \textit{politically correct} meant your ideas aligned with the Party line—even if they clashed with observable reality. A mathematically precise result could still be ideologically suspect. Truth was negotiable; loyalty was not.
    \medskip
    
    One story—often repeated but never quite sourced—tells of an American intellectual visiting the USSR. After pointing out a factual error during a Party meeting, a Soviet official allegedly responded:  

    \begin{quote}
    \textit{“Yes, you are technically correct... but you are not politically correct.”}
    \end{quote}
    
    Is the story true? Possibly not. But its survival says something deeper: in totalitarian regimes, correctness is not a matter of logic—it is a matter of alignment.  

    \medskip
    
    As historian Tony Judt noted, Communist discourse frequently subordinated facts to revolutionary narrative. What mattered wasn’t whether a statement was true in the empirical sense, but whether it advanced historical inevitability as defined by the Party.

    \medskip
    
    By the time the phrase resurfaced in the West in the 1980s—this time with a sardonic edge—it had already mutated from an instrument of ideological control into a critique of that control.

    \medskip
    
    \textit{In the USSR, being correct wasn’t enough. You had to be correct in the way history demanded.}

    \medskip
    
    \textbf{Technically correct? That’s counterrevolutionary nuance.}
    
\end{tcolorbox}

