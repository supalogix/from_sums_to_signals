\section{Blaise Pascal: The First Mechanical Calculator for Arithmetic}

After centuries of mechanical devices that assisted measurement and sequencing, \textbf{Blaise Pascal} (1623--1662) initiated a new phase in the history of mechanical computation: the automation of numerical arithmetic itself.

\subsection{The Pascaline: Mechanical Addition and Subtraction}

In 1642, at the age of just 19, Pascal designed and built the \textbf{Pascaline}---the first known mechanical calculator capable of performing direct addition and subtraction of numbers.

Key features of the Pascaline included:

\begin{itemize}
    \item \textbf{Digit Wheels}: Each wheel represented a decimal digit (0--9). Turning a wheel advanced the corresponding place value mechanically.
    \item \textbf{Carry Mechanism}: When a wheel completed a full turn from 9 to 0, a mechanical linkage automatically advanced the next higher wheel by one unit---an essential innovation for multi-digit addition.
    \item \textbf{Subtraction Mechanism}: Through complement techniques and reverse gearing, the machine could also perform subtraction operations.
\end{itemize}

Unlike previous instruments like astrolabes or water clocks, the Pascaline directly computed discrete numerical results. It was not a mere measuring device or sequencer---it was a true mechanical embodiment of arithmetic operations.

\subsection{Legacy}

Pascal's invention marked a critical transition: from machines that modeled continuous natural processes to devices that could perform abstract, symbolic computation. Though the Pascaline was expensive, limited in scalability, and difficult to produce, it demonstrated for the first time that mechanical systems could handle arithmetic systematically---paving the way for later developments in mechanical computing and, eventually, programmable computers.

