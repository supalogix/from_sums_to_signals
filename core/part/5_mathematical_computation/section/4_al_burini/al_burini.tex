\section{Al-Biruni: Precision Instruments and Manual Computation}

While Al-Jazari developed programmable mechanical sequences, \textbf{Al-Biruni} (973--1050 AD) advanced the art of mechanical computation in a different direction---through the design of precision instruments that transformed manual measurement into rigorous calculation.

\subsection{Celestial Measurement and Geometric Computation}

Al-Biruni's contributions centered on creating highly accurate tools for astronomy, geography, and surveying. His achievements included:

\begin{itemize}
    \item \textbf{Astrolabes}: Instruments capable of calculating the altitude of stars, determining time from celestial observations, and solving complex spherical astronomy problems.
    \item \textbf{Sundials and Quadrants}: Devices that allowed for the measurement of solar positions and the determination of latitude based on the Sun’s angle.
    \item \textbf{Earth's Radius}: Using ingenious methods involving mountain height and horizon distance, Al-Biruni produced one of the most accurate measurements of the Earth's radius in antiquity.
\end{itemize}

These devices served as \textit{manual-mechanical computation aids}, translating geometric relationships into observable, measurable quantities. Though lacking formal mechanical gears or automated motion, they effectively mechanized aspects of mathematical astronomy and geographical calculation.

\subsection{Legacy}

Al-Biruni demonstrated that careful instrument design could bridge the gap between theoretical mathematics and practical computation. His work optimized the mechanical aspects of surveying and astronomical analysis, laying a foundation for later developments in navigational science and observational astronomy, even though he did not construct autonomous mechanical computers.

