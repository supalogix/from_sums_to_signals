\subsection{From Geometry to Calculation: The Problem Euclid Ignored}

Euclid built a mathematical world where space was infinitely divisible, but he never had to confront the consequences of that assumption. He described shapes and lines as if they were perfectly continuous, never stopping to ask what happens when you actually try to measure something in this infinite world.  

But Archimedes did.  

Unlike Euclid, who worked with static perfection, Archimedes wanted to calculate things: areas, volumes, curves. And in doing so, he ran straight into the problem that Euclid’s clean geometric world had left unresolved:  

\begin{quote}
    \textbf{How do you measure something that is made of infinitely many pieces?}
\end{quote}

If space is truly continuous, then measuring anything—like the area of a curved shape—should require adding up an infinite number of tiny parts. And that sounds... impossible.  

But Archimedes, being Archimedes, found a way.




\begin{figure}[H]
\centering
\begin{tikzpicture}[every node/.style={font=\footnotesize}, scale=1]

% Panel 1 — Euclid drawing a perfect triangle
\comicpanel{0}{4}
  {Euclid}
  {Student}
  {Behold: a perfect triangle made of infinitely divisible lines.}
  {(-0.4,-0.6)}

% Panel 2 — Student looks confused
\comicpanel{6.5}{4}
  {Euclid}
  {Student}
  {Cool, but like... how long is it?}
  {(0,-0.5)}

% Panel 3 — Archimedes enters
\comicpanel{0}{0}
  {Archimedes}
  {Euclid}
  {If I slice it into infinite slivers and add them all up, I can tell you.}
  {(0.3,-0.7)}

% Panel 4 — Euclid is slightly horrified
\comicpanel{6.5}{0}
  {Archimedes}
  {Euclid}
  {Also, I invented calculus. But don’t tell anyone—it won’t be cool for another 2000 years.}
  {(0.4,-0.5)}

\end{tikzpicture}
\caption{Euclid drew perfection. Archimedes did the math.}
\end{figure}
