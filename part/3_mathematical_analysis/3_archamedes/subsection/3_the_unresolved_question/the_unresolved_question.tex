\subsection{The Unresolved Question: What Does "Closer and Closer" Really Mean?}  

Archimedes’ method was powerful, but it had an unspoken flaw: it relied on the assumption that when you keep refining an approximation, you will eventually get as close as you want to the real answer.  

That might seem obvious, but it raises a deeper question:  

\begin{quote}
    \textbf{What does it actually mean for one value to "approach" another?}
\end{quote}

Archimedes could prove that the gap between his approximation and the true value would keep shrinking, but he never had a way to define what happens at the end of the process.  

He worked around the problem by assuming that, eventually, the difference between his approximations and the real value would become so small that it might as well be zero. But he never actually proved that an infinite process had a well-defined result; because at the time, nobody even knew how to formalize such a thing.  

For centuries, mathematicians would continue using the method of exhaustion, believing that their approximations were getting arbitrarily close to the true answer, without ever questioning what that really meant.  

But eventually, someone would.  

\begin{figure}[H]
\centering
\begin{tikzpicture}[every node/.style={font=\footnotesize}, scale=1]

% Panel 1 — Archimedes with polygons and a circle
\comicpanel{0}{4}
  {Archimedes}
  {Student}
  {Each polygon gets us closer and closer to the true area of the circle!}
  {(-0.3,-0.6)}

% Panel 2 — Student thinking deeply
\comicpanel{6.5}{4}
  {Archimedes}
  {Student}
  {Okay, but like… how close is “close”? When do we actually get there?}
  {(0,-0.6)}

% Panel 3 — Archimedes stalling
\comicpanel{0}{0}
  {Archimedes}
  {Student}
  {Well... the difference becomes smaller than any quantity you can name.}
  {(-0.1,-0.5)}

% Panel 4 — Student delivers the existential blow
\comicpanel{6.5}{0}
  {Archimedes}
  {Student}
  {So basically, we just… believe in the end of a process that never ends?}
  {(0.5,-0.5)}

\end{tikzpicture}
\caption{Approaching the truth is easy. Defining "approach" is the hard part.}
\end{figure}



