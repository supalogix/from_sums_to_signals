\section{Cavalieri: Geometry by Indivisibles (1635)}

Bonaventura Cavalieri, a student of Galileo, introduced a method based on the idea that a geometric figure could be thought of as composed of an infinite number of indivisible elements.

\vspace{0.5em}
For a planar region, Cavalieri reasoned:

\begin{quote}
\textit{“A plane figure is made up of an infinite number of lines (indivisibles) parallel to the base.”}
\end{quote}

The area of such a figure was not measured using coordinates or functions, but rather:

\begin{quote}
\textit{“As the sum of all its lines multiplied by the common interval between them.”}
\end{quote}

There was no formal algebraic notation. Instead, figures were compared by imagining that corresponding indivisibles in two figures could be matched in proportion. Cavalieri would assert, for example, that the area beneath a curve increased “as the sum of its ordinate lines.”

\vspace{1em}


\begin{tcolorbox}[colback=gray!5!white, colframe=black, title=\textbf{Historical Sidebar: Cavalieri Before the Integral}, fonttitle=\bfseries, arc=1.5mm, boxrule=0.4pt]

    Bonaventura Cavalieri, working in the early 17\textsuperscript{th} century, did not use algebraic symbols like \( \int \) or even variables such as \( x \). Instead, his reasoning was entirely geometric, grounded in visual proportion and spatial intuition.
    
    He described the area of a shape as being composed of an infinite number of “indivisible” elements. For a plane figure, these were conceived as lines—each parallel to the base of the figure—stacked together to form the whole.
    
    \vspace{0.5em}
    \textit{“The area of a figure is composed of an infinite number of lines (indivisibles) parallel to the base.”}
    
    \vspace{0.5em}
    This method treated area not as a numerical quantity derived from coordinates, but as a geometric aggregation:
    \begin{center}
    \textbf{Area = Sum of all lines (ordinates) × base interval}
    \end{center}
    
    While Cavalieri never wrote an equation in the modern sense, his intuition foreshadowed the integral. What we now express algebraically as the area under a curve—such as the region beneath \( y = x \) from \( 0 \) to \( a \)—was, for Cavalieri, a visual stacking of infinitely thin lines, with no formal limit process.
    
    \vspace{0.3em}
    \textit{Cavalieri did not compute integrals—he composed figures.}
\end{tcolorbox}