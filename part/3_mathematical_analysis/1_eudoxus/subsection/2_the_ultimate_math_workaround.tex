\subsection{The Ultimate Math Workaround: Math Without Numbers (But Still Technically Math)}

When ancient Greek mathematicians stumbled upon irrational numbers—like \( \sqrt{2} \)—they faced a philosophical crisis. These were quantities that couldn’t be written as neat fractions or measured exactly with a ruler. For a culture that saw number and geometry as the architecture of the cosmos, this was more than inconvenient—it was destabilizing.

\textbf{Enter Eudoxus}, the quiet genius of mathematical rescue missions.

Instead of trying to define irrational magnitudes directly, he did something profound:

\textbf{He stopped measuring—and started comparing.}

\medskip

Rather than ask “how much,” Eudoxus asked “how does it compare?”

Imagine trying to understand someone’s wealth without knowing exact amounts. You don’t know how much gold Alice or Bob has—but you do know things like:

\begin{itemize}
    \item \textit{If I multiply Alice’s gold by 5, it’s still less than 4 times Bob’s.}
    \item \textit{If I multiply Alice’s gold by 7, it exactly matches 6 times Bob’s.}
\end{itemize}

Eudoxus realized that this sort of relational thinking could be generalized—and in doing so,  
\textbf{he effectively invented the modern theory of proportions.}

Instead of defining numbers as specific values, he defined their \textit{behavior}: how one quantity scales in relation to another. Two ratios were equal if, no matter how you scaled them up (using whole numbers), their comparative behavior always stayed the same—greater than, equal to, or less than.

\medskip

This was more than clever—it was revolutionary.

\begin{itemize}
    \item It let you compare irrational quantities without ever having to name them.
    \item It replaced arithmetic with logical comparison.
    \item It saved Greek math from irrational despair—and quietly introduced the idea that mathematics could be rigorous without being numeric.
\end{itemize}

\textit{It was like doing math with shadows instead of objects—but still knowing exactly what shape you were dealing with.}

Rather than calculate something like \( \sqrt{2} \), you could "trap" it between known proportions. If one ratio (like \( \frac{7}{5} \)) is always too small, and another (like \( \frac{10}{7} \)) is always too big, then the true value must live somewhere in between—even if you can never write it down exactly.

\medskip

So while Eudoxus didn’t invent irrational numbers, he gave us something just as powerful:  
\textbf{A way to understand the inexpressible—through proportion.}

\begin{quote}
\textit{Not everything needs to be counted.  Sometimes, it’s enough to compare.}
\end{quote}



\subsection{Eudoxus: Proportions Without Numbers}

Eudoxus didn’t approximate irrational numbers the way we might today—by calculating digits, halving intervals, or refining decimal approximations. His method was something far more foundational:  

\begin{itemize}
    \item \textbf{Qualitative, not computational.}
    \item \textbf{Based on proportions, not measurements.}
    \item \textbf{Built on comparisons, not calculations.}
\end{itemize}

Instead of trying to “find” the square root of 2 with better and better guesses, Eudoxus asked:

\begin{quote}
\textit{How can I know two ratios are equal, even if I don’t know what the values are?}
\end{quote}

\noindent His answer was subtle—and brilliant. He said:  
Two ratios \( \frac{A}{B} \) and \( \frac{C}{D} \) are equal if they behave the same way \textit{under all integer scaling}. That is:

\[
\text{If } mA > nB \quad \text{then} \quad mC > nD,
\]
\[
\text{If } mA = nB \quad \text{then} \quad mC = nD,
\]
\[
\text{If } mA < nB \quad \text{then} \quad mC < nD.
\]

\medskip

\noindent This approach is now known as the \textbf{Eudoxian definition of ratio}, and it sidesteps the need to ever write down or calculate irrational numbers directly.

\subsubsection*{Example: Comparing \( \sqrt{2} \) Without Measuring It}

Imagine you're comparing the diagonal of a square (which is \( \sqrt{2} \) times the side length) to a rational guess like \( \frac{7}{5} \). You don't measure it—you just test whether multiplying the side and diagonal by various whole numbers changes the comparison.

\begin{itemize}
    \item Try \( m = 7 \), \( n = 5 \). Then check: is \( 7 \cdot \text{side} < 5 \cdot \text{diagonal} \)? If so, then \( \frac{7}{5} \) is too small.
    \item Try \( m = 10 \), \( n = 7 \). Then check: is \( 10 \cdot \text{side} > 7 \cdot \text{diagonal} \)? If so, then \( \frac{10}{7} \) is too big.
\end{itemize}

\noindent You’re not calculating anything—you’re just trapping the truth between patterns of inequality.

\medskip

\noindent This technique was the precursor to the \textbf{method of exhaustion}, a geometric way of approximating areas and magnitudes by surrounding them with ever-tighter bounds—without ever invoking decimals or direct measurements.

\begin{quote}
\textit{It was math without numbers. Just pure logic, ratio, and rigor.}
\end{quote}

\noindent Eudoxus’s framework would later influence Euclid’s \textit{Elements}, form the basis of rigorous geometry, and—centuries down the line—inspire the logic behind Cauchy sequences, Dedekind cuts, and the modern real number line.

\begin{tcolorbox}[title=Historical Sidebar: Why the Greeks Trusted Geometry, colback=gray!5, colframe=black, fonttitle=\bfseries]

    To modern eyes, it's strange that Greek mathematics was so thoroughly geometric. Why not just use numbers?

    \medskip
    
    The answer is philosophical: to the Greeks, geometry was about things you could \textit{see}, reason about, and prove with certainty. Arithmetic — especially once irrational numbers entered the picture — felt unstable and paradoxical.

    \medskip
    
    After the discovery that some magnitudes (like the diagonal of a square) couldn't be expressed as a ratio of whole numbers, the entire project of ``All is number'' cracked. If numbers couldn't describe reality, then perhaps they were not the foundation of mathematics after all.

    \medskip
    
    So the Greeks pivoted. Rather than abandoning math, they doubled down on \textbf{geometry and comparison}. Eudoxus’s brilliant workaround was to define ratios not in terms of computation, but in terms of how magnitudes could be \emph{compared} to one another — visually, relationally, and (most importantly) using pure logic.

    \medskip
    
    This allowed rigorous reasoning to continue even when numbers failed.

    \medskip
    
    In a way, geometry was their insurance policy: a stable framework for truth when arithmetic proved treacherous.
    
    \end{tcolorbox}
    