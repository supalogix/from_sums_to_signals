\section{Copernicus: Geometry Unchanged, Cosmos Recentered}

Copernicus didn’t invent new mathematics. What he did was more subtle — and more radical. He took the geometrical machinery of Ptolemy’s astronomy and asked a simple, disorienting question:

\begin{quote}
\textbf{What if the Sun, not the Earth, was at the center of it all?}
\end{quote}

This was not an algebraic revolution. Copernicus worked with the same tools as Ptolemy: chords, angles, triangles, and intersecting circles. But he used them to build an entirely new cosmological structure.

\subsubsection*{A Familiar Toolkit in a New Coordinate System}

Copernicus’s mathematical practice was strikingly traditional:

\begin{center}
\renewcommand{\arraystretch}{1.4}
\begin{tabular}{|c|p{10cm}|}
\hline
\textbf{Tool} & \textbf{Description} \\
\hline
\checkmark\hspace{0.5em}Euclidean Geometry & Like Ptolemy, he used classical constructions with chords, angles, triangles, and proportions. \\
\checkmark\hspace{0.5em}Chord-Based Trigonometry & He relied on geometric chord tables (not sine or cosine yet) to relate angular separation to distances. \\
\checkmark\hspace{0.5em}Arithmetic + Tables & Computations were done numerically with lookup tables — no symbolic algebra. \\
\ding{55}\hspace{0.5em}Algebra & No \( x = r \cos \theta \); symbolic algebra as we know it hadn’t yet developed. \\
\ding{55}\hspace{0.5em}Calculus & Still two centuries away. \\
\hline
\end{tabular}
\end{center}

\subsection{What He Actually Did in \textit{De revolutionibus} (1543)}

\subsubsection{Re-centered the Universe}

Copernicus took the Ptolemaic system — circles on circles — and simply moved the center. This switch to a heliocentric model naturally explained:

\begin{itemize}
  \item \textbf{Retrograde motion} — as a visual illusion from a moving Earth,
  \item \textbf{Planetary brightness} — as a function of varying distance from Earth,
  \item \textbf{Planetary order} — inner vs. outer planets now had geometric meaning.
\end{itemize}

\subsubsection{Still Used Epicycles}

Yes, Copernicus still used epicycles — about 40 in total (fewer than Ptolemy, but not drastically). Each planet moved on a small circle (epicycle) whose center moved on a larger circle (deferent) — but now around the Sun.

\begin{itemize}
  \item Earth itself moved around the Sun.
  \item Planetary positions were computed \textbf{relative to a moving Earth}.
  \item This required the same methods of geometric computation: central angles, chords, and proportional diagrams.
\end{itemize}

\subsubsection{Used Chord Tables, Not Sine Functions}

Like Ptolemy, Copernicus used the function:

\[
\text{chord}(\theta) = 2R \cdot \sin\left(\frac{\theta}{2}\right)
\]

He applied it to compute angular separations — such as elongation, quadrature, and conjunction — by measuring the chord between two points on the orbit circle.

\subsection{A Typical Copernican Calculation: Elongation of Venus}

\begin{enumerate}
  \item Look up Earth’s heliocentric position on a given date.
  \item Look up Venus’s heliocentric longitude from a table.
  \item Subtract to find angular separation as seen from the Sun.
  \item Use chord tables and geometry to compute the angle as seen from Earth — accounting for Earth's own motion.
\end{enumerate}

\begin{center}
\begin{tcolorbox}[colback=gray!5!white, colframe=black, boxrule=0.3pt, arc=1mm, width=0.9\linewidth]
\textbf{Diagram structure:}  
\begin{itemize}
  \item Sun at the center  
  \item Earth orbiting on one circle  
  \item Venus on a smaller epicycle around its deferent  
  \item Lines drawn from Sun to Venus and Sun to Earth  
  \item Angle between those lines = elongation  
\end{itemize}
Then: use \textbf{chord tables} to compute visual angle or distance.
\end{tcolorbox}
\end{center}


\begin{quote}
Copernicus didn’t change the tools — he changed the center.
\end{quote}

He inherited Ptolemy’s machinery but pointed it in a new direction. What had once been mathematical scaffolding for an Earth-centered universe became, in Copernicus’s hands, the geometry of a solar system.

