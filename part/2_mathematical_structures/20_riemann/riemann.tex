\section{Riemann and the Reinvention of Integration (1854)}

\subsection{Riemannian Metric: The Dot Product Evolves}

Hamilton gave us a way to think of motion as geometry. His reformulation of mechanics turned trajectories into flows and forces into structure. In doing so, he made the \textbf{dot product}—a way of comparing direction and change—the central actor in understanding dynamics.

But in Hamilton’s world, space was still flat. The dot product was fixed. Motion unfolded on a smooth, Euclidean stage.

\medskip

\textbf{Then came Riemann.}

In his revolutionary 1854 habilitation lecture, \textit{“On the Hypotheses Which Lie at the Foundations of Geometry,”} Bernhard Riemann proposed a stunning generalization:  
\textbf{What if the dot product itself could vary from point to point?}

In flat Euclidean space, the square of the infinitesimal distance between two points is given by:
\[
ds^2 = dx^2 + dy^2 + dz^2
\]
Which is really just a 3D dot product:
\[
ds^2 = \vec{dx} \cdot \vec{dx}
\]

Riemann generalized this idea by replacing the standard dot product with a position-dependent version. In coordinates \( x^1, x^2, \dots, x^n \), he defined:
\[
ds^2 = \sum_{i,j=1}^n g_{ij}(x) \, dx^i dx^j
\]

The functions \( g_{ij}(x) \) define the \textbf{metric tensor}—a generalized, localized dot product that varies across the manifold. At each point, it tells you how to compare directions: how to measure length, angle, and curvature.

This was no longer a single universal dot product—it was a field of dot products. Geometry had gone dynamic.

What Hamilton treated as a static structure—momentum, energy, alignment—Riemann treated as a fluid geometry:

\begin{itemize}
  \item Instead of one way to measure direction and distance, there were infinitely many—one at each point.
  \item Instead of a single coordinate grid, space itself could stretch, shear, and curve.
  \item Instead of treating the dot product as fixed, Riemann made it contextual—a local rulebook for each patch of the manifold.
\end{itemize}

\textbf{Every dot product was now a question: “How does space behave \textit{here}?”}

The power of Riemann’s metric was not just in describing geometry—it was in guiding motion.  The metric tells us how steep a slope is. How far a step moves.  In essence: \textbf{how change feels locally}.

This is the exact intuition behind \textbf{gradient descent}:

\begin{quote}
To move “downhill” in a curved space, you need to know which direction is steepest—and that depends on the local dot product.
\end{quote}

On a Riemannian manifold, the gradient of a function \( f \) is not just the vector of partial derivatives. It is the vector that, under the metric \( g \), points in the direction that maximally increases \( f \):
\[
\text{grad}_g f \quad \text{is the unique vector such that} \quad g(\text{grad}_g f, v) = df(v) \quad \forall v
\]

This equation is a dot product—but now one defined by the metric tensor \( g \). It is the Riemannian generalization of Hamilton’s core idea: use a directional product to measure how change aligns with motion.

\begin{tcolorbox}[colback=blue!5!white, colframe=blue!50!black,
title={Sidebar: Hamilton’s Dot Product Grows Up}]
Hamilton introduced the dot product as a tool for measuring projection, alignment, and energy.

Riemann turned that dot product into a \textbf{field}—a geometry where every point has its own rulebook for measurement.

Together, they gave us the key to modern optimization:
\begin{itemize}
  \item Measure alignment (Hamilton)
  \item Let that measurement depend on context (Riemann)
  \item Use it to move efficiently through space (Gradient Descent)
\end{itemize}
\end{tcolorbox}

\subsubsection*{A Geometry of Descent}

By unifying dot products with curvature, Riemann made it possible to define gradients and descent directions on curved spaces—on manifolds of shapes, probabilities, or quantum states.

\begin{quote}
Just as Hamilton revealed the symphony behind motion,  
Riemann gave us the score paper: a flexible sheet that bends, stretches, and curves—guiding every note of change.
\end{quote}

Later, when we study gradient descent and machine learning on manifolds, we’ll see Riemann’s influence again:  
\textbf{To find the optimal path, you don’t just follow the slope—you follow the slope that’s defined by the geometry beneath your feet.}


\subsubsection*{Kepler Reimagined: Gradients and the Geometry of Motion}

Let us return to Kepler’s Second Law:

\begin{quote}
\textit{A planet sweeps out equal areas in equal times as it orbits the Sun.}
\end{quote}

At first glance, this seems like a geometric curiosity—an aesthetic feature of ellipses.  
But from a modern, Riemannian perspective, it encodes something deeper:  
\textbf{A conserved, geometry-driven flow.}

In Hamiltonian mechanics, Kepler’s Second Law emerges from the conservation of angular momentum. The orbiting body traces a path through phase space that preserves a particular 2-form—the symplectic structure:
\[
\omega = dq \wedge dp
\]

This preservation means the flow is volume-preserving in the sense of differential geometry. But what if we zoom in further—into the geometry that defines how motion responds to the shape of space?

Here’s the insight:

\begin{itemize}
  \item The gravitational potential defines a scalar function \( f \) over configuration space.
  \item The geometry of space—captured by a Riemannian metric \( g \)—tells us how that potential changes in different directions.
  \item The motion of the planet follows curves that \textbf{respect both}: the shape of the space, and the shape of the potential.
\end{itemize}

On a Riemannian manifold, the direction of steepest descent (e.g., toward the Sun) is not just \( -\nabla f \), but \( -\text{grad}_g f \), where the gradient is defined via:
\[
g(\text{grad}_g f, v) = df(v) \quad \forall v
\]

This gradient dictates how the planet “feels” the slope of gravity—not in abstract space, but within the actual geometry of the orbiting system.

Now consider Kepler’s Second Law again:

\begin{quote}
The planet doesn’t just fall toward the Sun—it flows through a geometry that guides its motion to preserve area.
\end{quote}

This flow can be interpreted as a balance between the local gravitational gradient and the geometry of the orbit. The planet’s trajectory is a solution to a constrained optimization problem:
\begin{itemize}
  \item Minimize energy (potential and kinetic),
  \item Constrained to conserve angular momentum,
  \item Subject to a local notion of “steepest descent” defined by the curved geometry of space.
\end{itemize}

\begin{tcolorbox}[colback=blue!5!white, colframe=blue!50!black,
title={Kepler’s Second Law as Gradient-Constrained Motion}]
Kepler’s Law describes more than equal areas.  
It encodes how a system flows through space while preserving symmetries.

In modern terms:
\begin{itemize}
  \item Motion follows a gradient,
  \item The gradient is defined by the Riemannian metric,
  \item The flow conserves a geometric structure (area or angular momentum).
\end{itemize}

It is not just celestial—it is algorithmic.
\end{tcolorbox}

This reimagining prepares us for gradient descent in curved spaces. The intuition is already present in Kepler’s planetary motion:  
\textbf{descent guided by geometry, structure preserved by flow.}
