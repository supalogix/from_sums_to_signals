\section{Ptolemy and The Celestial Circles: When Geometry Became a Calculator}

\subsection{Modeling the Heavens with Circles and Chords}

If Archimedes’ method of exhaustion showed how geometry could slice the infinite into manageable parts, then **Ptolemy** showed how geometry could **move the planets** — or at least explain how they appeared to move.

Where Archimedes was concerned with approximation, Ptolemy applied geometric precision to a different kind of problem: predicting where Mars would appear in the sky, or how the Moon would arc through the night. And remarkably, he did all of it without algebra, calculus, or even sine and cosine.

Ptolemy didn’t just reference Euclid — he built his entire model of the cosmos on it. His *Almagest* is perhaps the most ambitious application of **pure geometry** in ancient science. Let’s break down how that worked.

Ptolemy, like most astronomers of antiquity, didn’t just \emph{prefer} the idea that celestial bodies moved in uniform circles around the Earth—he believed it was a \textbf{necessary truth}. These weren’t merely observational claims; they reflected deep philosophical commitments rooted in \textbf{Platonism} and \textbf{Aristotelian physics}.

\begin{itemize}

    \item Geocentric: The Earth wasn’t simply located at the center... it was \emph{anchored} there. The Earth represented the realm of change, decay, and imperfection. In contrast, the heavens were considered eternal, divine, and unchanging. Thus, it was natural to imagine the Earth as the fixed center of a perfectly ordered cosmos. Ptolemy reinforced this worldview in his astronomical treatise, the \emph{Almagest}, and designed a model in which everything (i.e. the Sun, Moon, planets, and stars) revolved around a stationary Earth.

    \item Circular: Why circles? Because the circle was regarded as the most \textbf{perfect} geometric shape. With no beginning or end, no sharp turns or asymmetry, it symbolized eternity and divine order. If the heavens were perfect, their motions had to be as well, and that meant circles.  Of course, the actual motions of the planets were far from perfectly circular. To account for this, Ptolemy introduced \emph{epicycles}: smaller circular motions superimposed on larger circular orbits to preserve circularity while better matching the planets' observed paths. It was a compromise designed to protect the philosophical ideal, not necessarily to embrace empirical messiness.

    \item Uniform: Lastly, celestial motion had to be \textbf{uniform} (i.e. meaning constant in speed). Any variation implied forces, causes, or external influences: phenomena associated with the corruptible Earth, not the eternal heavens. Uniform motion was part of the divine harmony presumed to govern the cosmos.  To explain the observed irregularities (i.e. retrograde motion, where planets appear to move backward in the sky) Ptolemy added mathematical devices such as the \emph{equant}, a point offset from the center of the deferent circle, to ensure that motion appeared uniform from a particular vantage.

\end{itemize}

Ptolemy’s model wasn’t just about celestial prediction; it was about \emph{preserving a metaphysical worldview}. His system endured for over a thousand years, not because it was the most accurate, but because it was philosophically elegant. It kept the cosmos ordered, the Earth central, and the heavens divine.

But when you looked at the planets, they defied those expectations. They sped up, slowed down, and sometimes even moved backwards (retrograde motion). Ptolemy reconciled this by constructing a system of \textbf{epicycles and deferents} — circles moving on other circles — all governed by geometric rules.

To compute planetary positions, he used:

\begin{itemize}
    \item Chords of central angles in a circle,
    \item Geometric transformations of rotating systems,
    \item Interpolation from chord tables — not modern trigonometric functions.
\end{itemize}

\subsection{Geometry Was the Calculator}

There were no algebraic formulas in the Almagest. No \( x \), no \( y \), no \( \sin \theta \) or \( \cos \theta \). Just:

\begin{itemize}
    \item Lines,
    \item Angles,
    \item Proportions,
    \item Tables of chords.
\end{itemize}

To find the position of Mars on a given day, Ptolemy would:

\begin{enumerate}
    \item Construct a geometric diagram,
    \item Place the epicycle on its deferent,
    \item Use known angles to determine chords,
    \item Interpolate values from his trigonometric chord table.
\end{enumerate}

\subsection{Solving Problems with Geometry Alone}

To determine the angle between Earth, a planet, and the center of the epicycle, Ptolemy:

\begin{itemize}
    \item Constructed a triangle using radii and chords,
    \item Applied proportions from \textit{Euclid VI},
    \item Used geometric reasoning.
\end{itemize}

\subsection{The Chord Table: A Trigonometric System Without Trig}

Ptolemy used the function:

\[
\text{chord}(\theta) = 2R \cdot \sin\left(\frac{\theta}{2}\right)
\]

with the Babylonian \( R = 60 \). He tabulated chords for angles from \( 0^\circ \) to \( 180^\circ \) in half-degree steps.

Ptolemy, following the Greek tradition and using Babylonian sexagesimal conventions, defined the \textbf{chord} of an angle \( \theta \) as the length of the straight line connecting the endpoints of an arc of a circle of radius \( R = 60 \). This definition predates the modern sine function.

\begin{table}[H]
\centering
\renewcommand{\arraystretch}{1.4}
\begin{tabular}{|c|c|}
\hline
\textbf{Angle \( \theta \)} & \textbf{Chord Length (in terms of \( R \))} \\
\hline
\( 0^\circ \) & \( 0 \) \\
\( 30^\circ \) & Geometrically constructed value \\
\( 60^\circ \) & \( R \) \\
\( 90^\circ \) & \( R \cdot \sqrt{2} \) \\
\( 120^\circ \) & \( R \cdot \sqrt{3} \) \\
\( 180^\circ \) & \( 2R \) \\
\hline
\end{tabular}
\caption{Sample chord values from Ptolemy’s table using radius \( R = 60 \). Expressions are shown in geometric terms, avoiding modern trigonometric notation.}
\end{table}


\subsection{A Geometric Identity for the Sky}

Ptolemy also derived the identity:

\[
\text{chord}(A + B) \cdot \text{chord}(A - B) = \text{chord}^2(A) - \text{chord}^2(B)
\]

A precursor to modern sine addition formulas, this was used to derive compound angles geometrically which allowed Ptolemy to compute values he hadn't explicitly tabulated.

For Ptolemy, geometry \textit{was reality}. Circles, chords, and ratios didn’t just describe planetary motion — they \textit{were} planetary motion.


\begin{figure}[H]
    \centering
    \begin{tikzpicture}[scale=2.5, every node/.style={font=\small}]
    
    % Earth at center
    \filldraw[black] (0,0) circle (0.02) node[below left] {Earth};
    
    % Deferent (large circle)
    \draw[thick] (0,0) circle (1);
    
    % Epicycle center (on the deferent)
    \coordinate (C) at (60:1);
    \filldraw[gray!50] (C) circle (0.015);
    \node[above right] at (C) {Epicycle center};
    
    % Epicycle (small circle on deferent)
    \draw[dashed] (C) circle (0.25);
    
    % Planet position on epicycle
    \coordinate (P) at ($(C) + (150:0.25)$);
    \filldraw[blue] (P) circle (0.02);
    \node[above left] at (P) {Planet};
    
    % Chord from Earth to planet
    \draw[red, thick] (0,0) -- (P) node[midway, above right] {Chord};
    
    % Label arc angle (planet's apparent motion)
    \draw[->, thick] (0.4,0) arc[start angle=0, end angle=60, radius=0.4];
    \node at (30:0.55) {$\theta$};
    
    % Optional: radii to epicycle center and to planet
    \draw[dotted] (0,0) -- (C);
    \draw[dotted] (C) -- (P);
    
    \node at (30:1.05) {Deferent};
    \node at ($(C)!0.5!(P) + (-0.05, 0.08)$) {Epicycle};
    
    \end{tikzpicture}
    \caption{Ptolemy’s epicycle system: A small circle (epicycle) rides a larger one (deferent) centered on Earth. The chord from Earth to the planet models its apparent position.}
\end{figure}
