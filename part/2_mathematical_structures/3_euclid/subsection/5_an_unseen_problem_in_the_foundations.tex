\subsection{An Unseen Problem in the Foundations}

But assumptions, no matter how useful, don’t always hold up forever.

Euclid’s geometry gave mathematicians an extraordinarily powerful way to describe space, but it also left a question completely unanswered:  

\begin{quote}
\textbf{How do we know that space is actually continuous?}
\end{quote}

If a line is infinitely divisible, does that mean it’s actually made up of infinitely many points, packed so closely together that no gaps remain? Or is there some deeper structure beneath it, something more fundamental that has been overlooked?

For Euclid, these weren’t real concerns. A line was a line, a plane was a plane, and space worked exactly the way it was supposed to. There was no need to go digging for deeper truths.

But the universe doesn’t care about what’s supposed to be true. It only cares about what actually is. And centuries later, when mathematicians finally started scrutinizing these assumptions, they would discover something lurking beneath Euclid’s smooth, perfect world; something that no one in ancient Greece ever saw coming.



\begin{figure}[H]
\centering
\begin{tikzpicture}[every node/.style={font=\footnotesize}]

% Panel 1 — Euclid explains infinite divisibility
\comicpanel{0}{4}
  {Euclid}
  {Student}
  {A line is infinitely divisible. Between any two points, there are more points.}
  {(-0.6,-0.5)}

% Panel 2 — Student questions composition
\comicpanel{6.5}{4}
  {Euclid}
  {Student}
  {So what’s a line actually made of? Just... points? Nothing else?}
  {(0.3,-0.5)}

% Panel 3 — Euclid confidently answers
\comicpanel{0}{0}
  {Euclid}
  {Student}
  {No. There are infinitely perfectly arranged many points on a line, but the line exists independently of the points.}
  {(-0.6,-0.6)}

% Panel 4 — Student delivers the Chekhov’s gun
\comicpanel{6.5}{0}
  {Euclid}
  {Student}
  {What if half the points on a line disagree with the other half about where they belong?}
  {(0.2,-0.6)}

\end{tikzpicture}
\caption{When you build geometry from infinitely many points, eventually some of them start acting weird.}
\end{figure}