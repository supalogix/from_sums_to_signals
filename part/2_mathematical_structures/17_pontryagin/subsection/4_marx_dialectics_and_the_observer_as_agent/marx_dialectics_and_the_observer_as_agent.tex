\subsection{Marx, Dialectics, and the Observer as Agent}

Enter the dialectical method of Karl Marx—a framework that explicitly rejects the passivity of classical observation. Building on Hegelian dialectics, Marx argued that systems (especially historical and social ones) evolve not through static laws but through **contradictions**, **conflicts**, and **transformative agency**.

Marx's dialectic is dynamic, recursive, and *active*. It centers not on the detached observer but on the embedded participant—on labor, on struggle, on the capacity to intervene in the world. Unlike classical mechanics, which seeks equilibrium, dialectics seeks transformation. Reality is not fixed, but in flux; not neutral, but shaped by forces with interests, agency, and intention.

This mode of thinking deeply informed Soviet ideology—particularly its vision of science and engineering. In the USSR, control theory was not just a tool for optimizing factories or rockets; it became a symbol of human mastery over chaotic systems. The ideal was not to passively model the world, but to **shape it**, to harness feedback, to drive systems toward desired futures. Mathematics became a *weapon of planning*, not just of prediction.

\begin{quote}
\emph{What if the observer isn’t passive? What if the system can be steered? What if mathematics isn't just describing what is, but prescribing what ought to be?}
\end{quote}