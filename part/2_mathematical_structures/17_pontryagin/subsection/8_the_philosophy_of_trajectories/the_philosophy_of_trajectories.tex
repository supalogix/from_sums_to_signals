\subsection{The Philosophy of Trajectories: An Imagined Soviet Debate}

When I picture the development of optimal control theory in the Soviet Union, I don’t just imagine engineers hunched over missile paths. I imagine something more ideological—a scene in which the foundations of classical physics itself are called into question. After all, Newtonian mechanics wasn’t just math; it was metaphysics. It treated the universe like a passive machine, unfolding from divine initial conditions. That might fly in Cambridge, but in Moscow? In the middle of the 20\textsuperscript{th} century? Not without a dialectical audit.

So here’s how I imagine things going down, somewhere inside a smoky academic office at the Academy of Sciences, circa 1957:

{
\ttfamily
\textbf{Party Official:} Comrade Pontryagin, I’ve been reviewing the foundations of your optimal control theory.  

\textbf{Pontryagin:} Yes, it’s all based on the Hamiltonian formalism. Very effective for guiding dynamic systems toward—  

\textbf{Party Official:} Hamiltonian, yes. Newtonian, Lagrangian, all very... tsarist-sounding.  

\textbf{Pontryagin:} I assure you, there’s nothing religious in the equations.  

\textbf{Party Official:} Are you aware that Newton said, “The most beautiful system of the sun, planets, and comets could only proceed from the counsel and dominion of an intelligent and powerful being”?  

\textbf{Pontryagin:} Perhaps... but it models ballistic motion quite reliably.  

\textbf{Party Official:} And what of Lagrange, who wrote that the goal of mechanics is to “glorify the works of the Creator through the elegance of mathematical expression”? Or Hamilton, who referred to dynamics as “a science of the divine harmonies of motion”?  

\textbf{Pontryagin:} Yes, well... I mostly skipped the theological prefaces.  

\textbf{Party Official:} It’s not about prefaces. It’s about foundations. These men built mechanics on top their religious beliefs. We need a framework grounded in dialectical materialism... not theology.  

\textbf{Pontryagin:} You want a dialectical Hamiltonian?  

\textbf{Party Official:} I want mathematics rooted in material reality, not metaphysical abstraction. Marx didn’t optimize over cotangent bundles. He studied systems in contradiction.  

\textbf{Pontryagin:} Then let me show you the adjoint equation. It doesn’t just observe: it responds. It evolves with feedback. That’s the costate: a shadow that adapts to the system’s goals.  

\textbf{Party Official:} Now you're talking. Control, feedback, transformation—this is Marxist calculus.  

\textbf{Pontryagin:} So... we keep the equations, just add historical agency?  

\textbf{Party Official:} Precisely. Keep your variational principles. Just make sure they steer us toward socialism. 
}
