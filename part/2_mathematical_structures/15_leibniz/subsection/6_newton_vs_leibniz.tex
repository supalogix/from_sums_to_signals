
\subsection{Newton vs. Leibniz: The Calculus Wars (or, How to Win at Science Without Actually Doing Science)}

Newton and Leibniz arrived at the same results but through completely different perspectives. Newton saw calculus as a tool for motion and physics, while Leibniz developed a symbolic algebraic system that was more general. 

\textbf{Unfortunately, their works were so similar that it started an international feud.} 

See, the \textbf{Academy of Sciences} had just become a thing, and in the 17th century, academics were basically the \textbf{rock stars of their day}. And much like modern academics looking back at the Enlightenment as some kind of mathematical golden age, they were all incredibly full of themselves. Imagine if physicists today had the same cultural status as pop stars—except instead of drama about who stole whose song, they were accusing each other of \textit{stealing calculus.} 

\textbf{And that’s exactly what happened.} 

Newton’s people claimed he had invented calculus first, but Leibniz (who had published his notation first) insisted he had come up with it independently. A full-on academic war broke out, with England and Germany turning their mathematicians into ideological weapons. \textbf{National pride was at stake.} This was like the Cold War, except instead of nuclear bombs, the main weapon was \textit{passive-aggressive letters and obscure Latin insults.} 

And if you’re wondering who won? \textbf{Well, that depends on how you define "winning."} 

Newton didn’t just win because he was better at math. \textbf{He won because he was better at politics.} 

Which is kind of ironic, because by all accounts, Newton was an even bigger \textit{asshole} than Galileo. But unlike Galileo, Newton had \textbf{connections.} He wasn’t just some rogue scientist poking holes in church doctrine. No, he was:  

\begin{itemize}
    \item President of the Royal Society (the most powerful scientific institution in England).  
    \item Master of the Mint (yes, the guy in charge of the country’s money).
    \item The Crown’s favorite intellectual mascot (basically England’s Einstein before Einstein was a thing).
    \item Deeply tied to the Church, which, as Galileo learned, is always a useful political move.
\end{itemize}

Oh, and did I mention that Newton wrote more about theology than he did about math or science? Seriously. The man spent more time trying to decode biblical prophecy than working on physics. But he played the political game like a pro. And so, when England needed to pick a side in the Calculus Wars, Newton’s side mysteriously ended up being the ``official'' version. 

\medskip

\begin{tcolorbox}[
  title={Historical Sidebar: Newton’s Other Code — The Book of Revelation},
  colback=gray!5!white,
  colframe=black,
  fonttitle=\bfseries,
  width=\textwidth,
  boxsep=5pt,
  arc=2mm,
  before skip=10pt,
  after skip=10pt
]
\textbf{Isaac Newton} is best remembered for inventing calculus, defining the laws of motion, and getting hit on the head by gravity. But late into the night, he was often buried not in mathematics—but in prophecy.

Newton believed that the Bible, like nature, was a divine code. In his view, God’s laws governed both the motion of planets and the unfolding of time. This wasn’t metaphor. Newton studied the books of \textit{Daniel} and \textit{Revelation} with the same rigor he applied to optics and mechanics.

He wrote:

\begin{quote}
\small
“About the time of the end, a body of men will be raised up who will turn their attention to the prophecies, and insist upon their literal interpretation, in the midst of much clamor and opposition.”
\end{quote}

Though he criticized others for trying to predict the end times, he couldn’t resist doing it himself—calculating, from the Book of Daniel, that the world would not end before the year \textbf{2060}. In his own words:

\begin{quote}
\small
“It may end later, but I see no reason for its ending sooner.”
\end{quote}

Newton kept most of this work secret, fearing backlash. But after his death, a massive trove of handwritten theology and eschatology surfaced—proof that the father of modern science had been moonlighting as a biblical codebreaker.

\end{tcolorbox}

\medskip

\textbf{Leibniz, on the other hand, was not built for this nonsense.} By all accounts, he was actually more of a scientist than Newton and contributed more to math and science overall. But unlike Newton, he didn’t enjoy playing the political game because he considered it beneath him. And as history has shown, refusing to play the game is the fastest way to get erased from it. 

Which is why today, when people think of calculus, they think of Newton. And what do they remember Leibniz for? 

\textbf{Monads.}  

Yes. Instead of being universally credited with co-inventing calculus, the man is mostly remembered for \textit{Monads} --- his theory that reality is made up of little conscious atoms. If that sounds vaguely familiar, that’s because it was basically \textbf{midichlorians before George Lucas invented the Force}. 

\textbf{So, yeah. Newton may have been a terrible person, but at least he wasn’t remembered for inventing magic bacteria.}





\begin{figure}[H]
\centering
\begin{tikzpicture}[every node/.style={font=\footnotesize}]

% Panel 1
\comicpanel{0}{4}
  {Galileo}
  {Newton}
  {\textbf{Galileo:} So, you finished what I started with motion?}
  {(0,-0.5)}

% Panel 2
\comicpanel{6.5}{4}
  {Newton}
  {Galileo}
  {\textbf{Newton:} Yes. Invented calculus to do it. Not that the Germans would admit it.}
  {(0,-0.5)}

% Panel 3
\comicpanel{0}{0}
  {Galileo}
  {Newton}
  {\textbf{Galileo:} Whoa, this seems personal.}
  {(0,0.8)}

% Panel 4
\comicpanel{6.5}{0}
  {Newton}
  {Galileo}
  {\textbf{Newton:} He called my method “barbaric” in Latin. I took that very personally.}
  {(0,0.8)}

\end{tikzpicture}
\caption{Inventing Calculus Was Easy. The Real Challenge Was Surviving the Petty Academic Feuds.}
\end{figure}












