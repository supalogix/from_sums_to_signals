\section{From Theos to Logos: The Enlightenment and The Secularization of Reason (1600–present)}

\subsection{Mathematics Without Revelation}

The Scholastic thinkers had tried to harmonize divine revelation with human reason. But by the 17\textsuperscript{th} and 18\textsuperscript{th} centuries, a new ambition emerged: \textit{Could reason stand alone?}

This was the project of the Enlightenment.

No longer content to derive truth from theology, Enlightenment philosophers turned toward the autonomy of human reason. Thinkers like \textbf{René Descartes} insisted that certainty could be grounded not in God’s word, but in the clarity of thought itself. \textbf{Cogito, ergo sum} became the new axiom.

\medskip

In mathematics, this shift played out in two intertwined revolutions:

\begin{itemize}
    \item \textbf{The Rationalist Project}: Led by Descartes and Leibniz, this view held that the universe was a vast logical machine, and mathematics was the key to unlocking its gears.
    \item \textbf{The Empiricist Turn}: Figures like Newton and Locke grounded truth in experience and observation, with mathematics as a language to describe the natural world—not because it was divinely revealed, but because it worked.
\end{itemize}

\subsection{Kant and the Moral Geometry of Thought}

Perhaps no one synthesized this moment better than \textbf{Immanuel Kant}. Trained in the rigor of Newtonian mechanics and inspired by the clarity of Euclidean geometry, Kant argued that certain mathematical truths were not learned from experience, but structured into our minds as conditions for having experience at all.

Geometry, to Kant, was a kind of inner architecture—what he called “\textit{synthetic a priori}” knowledge. Space, time, and causality were not just features of the world; they were part of the mind’s framework for understanding it.

\medskip

This idea had profound implications:

\begin{itemize}
    \item It broke the final tie between God and mathematics.
    \item It suggested that certainty could be philosophical—even metaphysical—without needing revelation.
    \item It cleared the way for a secular, self-sufficient model of knowledge: one in which mathematics was foundational not because it was divine, but because it was structurally necessary.
\end{itemize}

\subsection{Toward the Age of Formalism}

By the 19\textsuperscript{th} century, thinkers like \textbf{Frege}, \textbf{Boole}, and \textbf{Peano} began to systematize this secular confidence into symbolic logic. They weren't just doing math—they were redefining what it meant to \textit{reason} at all.

And when Bertrand Russell and David Hilbert inherited this tradition, they carried forward a vision forged in Enlightenment fire: that reason could be purified, formalized, and mechanized.

But even as they built this new fortress of logic, they carried the ghosts of theology with them. The need for certainty. The hope for a unified system. The longing to anchor truth itself.

\begin{quote}
    From Aquinas’s First Mover to Hilbert’s Formal Axioms, the dream remained the same: that knowledge could rest on something ultimate, fixed, and unshakable.
\end{quote}

\subsection{From Atomism to Axioms: Russell, Hilbert, and the Dream of Formalization}

By the early 20th century, a group of mathematicians known as the \textbf{Formalists}---led by \textbf{David Hilbert}---dreamed of turning mathematics into a kind of perfect machine. The goal? To formalize all of math into a neat list of axioms and rules, so that every mathematical truth could be derived mechanically, like running a program. No ambiguity. No mystery. Just logic, grinding forward with total reliability.

Hilbert famously declared: \textit{``We must know. We will know.''} Mathematics, in his vision, would become a closed system: all truths could be proven, and all proofs could be checked by following the rules.

This belief wasn’t limited to mathematicians. It resonated with a broader intellectual movement known as \textbf{logical positivism}, led by philosophers like \textbf{Rudolf Carnap} and the \textbf{Vienna Circle}. Logical positivists believed that all meaningful statements could either be verified empirically (through observation) or proven logically (through deduction). Mathematics, to them, was the purest expression of this ideal: a universe of perfect clarity, built from self-evident axioms and strict logical steps. If we could formalize math, they argued, we could anchor all of science and reasoning on unshakable foundations.

But there was a deeper question lurking behind all of this: \textit{How do we know what’s true—or even what’s right or wrong?} People often assume we always know. Not so. And in the early 20th century, this uncertainty reached even into mathematics.

Enter \textbf{Bertrand Russell}.

Russell proposed an ambitious idea called \textbf{logical atomism}. The goal was to break down all mathematical concepts and arguments into tiny, irreducible logical units---``atoms'' of reason---and then show that they could all be derived from pure logic. It was a philosophical chain reaction waiting to happen.

Unfortunately, it didn’t quite work out that way.

After years of painstaking effort, Russell---alongside Alfred North Whitehead---produced the monumental \textit{Principia Mathematica}. But instead of reducing mathematics to pure logic, it revealed how staggeringly difficult that task really was. Still, it was a landmark. \textit{Principia} didn’t solve the problem, but it deeply influence \textbf{Hilbert}, and changed the course of mathematical logic.

It was, in a way, the mathematical equivalent of what physicists did when they split the atom: an attempt to reduce the complex to its fundamental components. In physics, splitting the atom led to a new kind of energy and a new kind of danger. In mathematics, analyzing these logical ``atoms'' led to a new kind of mathematics altogether.

Hilbert approached the problem from a completely different angle. Instead of trying to reduce mathematics to logic, he asked: \textit{What must a foundational system for mathematics be like if we want it to work at all?}

His answer came in the form of three key requirements:

\begin{enumerate}
    \item \textbf{Consistency} – The system must not produce contradictions.
    \item \textbf{Completeness} – Every true mathematical statement must be provable within the system.
    \item \textbf{Decidability} – There must be a definite procedure, a mechanical method, to determine whether any given statement is provable.
\end{enumerate}

Hilbert believed this vision was not only reasonable, but inevitable. Mathematics, he hoped, could still become what Russell tried and failed to construct: a fortress of certainty, guarded by logic, powered by method.

Then came Kurt Gödel.