
\section{Avoiding Spurious Correlation: Using Inclusion Maps}

Previously, we showed how mutual information and vector clocks help reconstruct the causal structure of high-frequency trading. However, even with these tools, one question remains: 

\textbf{Are these relationships truly causal, or are they just statistical coincidences?}  

\textbf{But} in the chaotic environment of high-frequency trading, simply tracking mutual information and causal timing isn't enough. If multiple machines react to an external macroeconomic event, their trades may appear causally linked—even when they are not.

\textbf{Therefore}, we introduce inclusion maps, a measure-theoretic tool that rigorously separates true causation from spurious correlation.

\paragraph{Step 1: Defining the Financial Spaces with Trading Machines}

Let’s apply this to our previous example, where high-frequency trading machines execute trades across different exchanges.

\begin{itemize}
    \item \( (\Omega, \mathcal{F}, \mu) \) represents the entire financial market state.
    \item \( S \subset \Omega \) is the set of trades executed by one machine.
    \item \( T \subset \Omega \) is the observed price changes in the market.
    \item \( C \subset \Omega \) is the true causal set—macro events like liquidity shocks, economic news, or institutional trades that influence both trades and price movements.
\end{itemize}

If trades genuinely cause price changes, we expect to find an inclusion map:

\[
\phi: S \to T
\]

which preserves the measure structure, meaning the probability measure on trade volume naturally maps to the measure on price changes.

However, if trade volume and price changes only appear correlated because of a hidden external factor \( C \), the correct mapping is:

\[
\phi: S \to \Omega \setminus C.
\]

This means the trades do \textbf{not} directly cause price movements; rather, both are influenced by a shared macroeconomic force. 

\paragraph{Step 2: Applying Mutual Information to Identify True Causal Influence}

To quantify the strength of the relationship, we compute the mutual information:

\[
I(S; T) = H(S) + H(T) - H(S, T),
\]

which captures how much knowing \( S \) (trade volume) reduces uncertainty in \( T \) (price changes). 

\textbf{Now, let’s introduce a test:}

If \( S \) truly causes \( T \), then conditioning on \( C \) (the true causal set) should not reduce the mutual information:

\[
I(S; T \mid C) \approx I(S; T).
\]

\textbf{But} if the relationship is spurious, conditioning on \( C \) removes the dependency:

\[
I(S; T \mid C) \approx 0.
\]

This tells us that once we account for the real macroeconomic forces, the apparent link between machine-generated trades and price fluctuations disappears—indicating a spurious correlation.

\paragraph{Step 3: Economic Impact of Misinterpreting Correlation}

\textbf{What happens if we fail to distinguish true causation from coincidence?}

Consider the same high-frequency trading system from before, where thousands of machines execute independent trades based on observed market fluctuations. If the model misidentifies correlation as causation, it will trade aggressively based on false signals. 

\paragraph{Case 1: Machines Trade on Spurious Correlations}
\begin{itemize}
    \item Each machine executes 85,000 trades per second.
    \item The average loss per trade from reacting to false correlations is \$0.0003.
    \item With 1,000 machines running, the total loss per second is:

    \[
    85,000 \times 0.0003 \times 1,000 = \text{\$25,500 per second}.
    \]

\end{itemize}

\paragraph{Case 2: Machines Filter Out False Signals with Inclusion Maps}
\begin{itemize}
    \item The number of trades per second decreases to 60,000, reducing unnecessary trades.
    \item The average profit per trade increases to \$0.0011 because trades are based on true causal relationships.
    \item With the same 1,000 machines, the total revenue per second becomes:

    \[
    60,000 \times 0.0011 \times 1,000 = \text{\$66,000 per second}.
    \]

\end{itemize}

\textbf{Result:} By filtering out spurious correlations using inclusion maps, the system shifts from losing \$25,500 per second to earning \$66,000 per second, a net gain of \$91,500 per second.

\paragraph{Step 4: Interpreting Inclusion Maps in Measure Theory}

If the inclusion map \( \phi: S \to T \) exists and preserves the measure structure, we conclude that trades are measurably influencing price changes.

\textbf{However}, if the inclusion map instead leads to \( \Omega \setminus C \), the measure space of trades does not fit within the true causal space. This confirms that the observed relationship is a statistical coincidence, not a true causal mechanism.

\subsubsection*{Final Takeaway: Measure Theory as a Financial Defense Mechanism}

\textbf{But why does this matter?} Without measure theory:
\begin{itemize}
    \item We couldn’t rigorously test for spurious correlations.
    \item Trading models would continue making costly, incorrect decisions.
    \item The financial system would be flooded with unnecessary noise, increasing volatility and inefficiency.
\end{itemize}

\textbf{Therefore}, inclusion maps and Lebesgue integration provide a financial defense mechanism—a way to mathematically shield trading models from statistical illusions and ensure they only act on true causal signals.

\begin{quote}
\textbf{Mathematics isn’t just about proving theorems—it’s about making (or saving) millions of dollars per second.}
\end{quote}

\subsection{Category Theory: When Finance Gets Too Complicated}

At some point, when financial models become overwhelmingly complex, mathematicians turn to category theory—a framework designed to unify and simplify abstract structures. Instead of getting lost in thousands of equations, category theory lets us step back and see the relationships between them in a commutative diagram.

In the context of high-frequency trading (HFT), category theory helps us analyze how different trading machines interact with the market. A well-optimized system doesn’t just react to individual trades—it understands the broader market structure and filters out irrelevant noise.

\begin{center}
    \textbf{Commutative Diagram for Detecting Spurious Correlation in HFT}
\end{center}

\begin{tikzpicture}[>=latex, scale=1.1, every node/.style={scale=1.2}]
  
  % Define matrix for the true causal relationships
  \matrix (m) [matrix of math nodes, row sep=3.5em, column sep=7em] {
      S \\  % Trade Events
      C \\  % True Causes
      T \\  % Price Changes
  };

  % Define the spurious correlation model to the right
  \matrix (q) [matrix of math nodes, row sep=3.5em, column sep=7em, right of=m, node distance=8cm] {
      S \\  % Trade Events
      \Omega \setminus C \\  % Spurious Space
      T \\  % Price Changes
  };

  % True causal structure (Left Column)
  \path[->] (m-1-1) edge node[left] {$\phi_C: S \to C$} (m-2-1);
  \path[->] (m-2-1) edge node[left] {$\phi_T: C \to T$} (m-3-1);

  % Spurious Correlation Structure (Right Column)
  \path[->] (q-1-1) edge node[right] {$\phi': S \to \Omega \setminus C$} (q-2-1);
  \path[->] (q-2-1) edge node[right] {$\phi_T': \Omega \setminus C \to T$} (q-3-1);

  % KL Divergence Regularization (Dashed Arrows)
  \path[->, dashed] (m-1-1.east) edge node[above] {$D_{KL}$} (q-1-1.west);
  \path[->, dashed] (m-2-1.east) edge node[above] {$D_{KL}$} (q-2-1.west);
  \path[->, dashed] (m-3-1.east) edge node[above] {$D_{KL}(P(P | V) || Q(P | V))$} (q-3-1.west);
  
\end{tikzpicture}

\subsection{Understanding the Diagram: Filtering Causal Signals in Algorithmic Trading}

Previously, we saw how mutual information and vector clocks help reconstruct the causal structure of market trades. However, that still doesn’t tell us whether the relationships are truly causal or just statistical illusions.

\textbf{But} in high-frequency trading, a model that confuses correlation with causation will waste millions in unnecessary trades.

\textbf{Therefore}, we turn to inclusion maps and category theory to formally separate genuine price influence from meaningless coincidences.

\subsubsection*{Components of the Diagram}

\begin{itemize}
    \item \textbf{\( S \) (Trade Events)}: Represents all executed trades by a machine.
    \item \textbf{\( C \) (True Causes)}: Represents actual macroeconomic forces or institutional trades affecting both \( S \) and \( T \).
    \item \textbf{\( T \) (Price Changes)}: Represents the observed market price fluctuations.
    \item \textbf{\( \Omega \setminus C \) (Spurious Correlations)}: Represents external market noise that makes trades and price movements look correlated even when they are not.
\end{itemize}

\subsubsection*{Interpreting the Arrows}

\paragraph{True Causal Path (Left-Side Arrows)}
\begin{itemize}
    \item \( \phi_C: S \to C \) suggests that machine-generated trades influence market price via a real economic force.
    \item \( \phi_T: C \to T \) ensures that market prices are responding to this true economic signal.
\end{itemize}

\paragraph{Spurious Correlation Path (Right-Side Arrows)}
\begin{itemize}
    \item \( \phi': S \to \Omega \setminus C \) means trades appear correlated with price movements, but are actually responding to some external factor unrelated to trading activity.
    \item \( \phi_T': \Omega \setminus C \to T \) implies that market movements are also reacting to this unrelated factor, making it look like trades at \( S \) are influencing price when they aren’t.
\end{itemize}

\subsubsection*{The Role of KL Divergence: Filtering Spurious Correlations}

\begin{itemize}
    \item The \textbf{dashed arrows} represent KL divergence regularization, which tells us whether a trading model is relying on real economic signals or overfitting to noise.
    \item If KL divergence \( D_{KL}(P || Q) \) is small, then removing certain features does not change the predictive power, indicating a spurious correlation.
    \item If KL divergence is large, the feature contains real economic significance and should not be discarded.
\end{itemize}

\subsubsection*{Economic Analysis: The Cost of Spurious Correlations}

\textbf{What happens if a machine trades based on spurious correlations?}

\paragraph{Scenario 1: The Algorithm Trades on False Signals}
\begin{itemize}
    \item A single machine executes 85,000 trades per second.
    \item The false signal leads to unnecessary transactions with an average loss per trade of \$0.00025.
    \item If a firm operates 1,000 machines, this results in:

    \[
    85,000 \times 0.00025 \times 1,000 = \text{\$21,250 lost per second}.
    \]

\end{itemize}

\paragraph{Scenario 2: The Algorithm Filters Out Spurious Correlations}
\begin{itemize}
    \item The number of trades per second is reduced to 60,000, focusing only on true causal signals.
    \item The average profit per trade increases to \$0.0012.
    \item With the same 1,000 machines, the total revenue per second is:

    \[
    60,000 \times 0.0012 \times 1,000 = \text{\$72,000 per second}.
    \]

\end{itemize}

\textbf{Conclusion:} Using inclusion maps and KL divergence, the model shifts from losing \$21,250 per second to earning \$72,000 per second, a net gain of \$93,250 per second.

\subsection{Final Takeaway: Category Theory as a Financial Shield}

Category theory and measure theory aren’t just abstract tools—they serve as financial shields, protecting trading systems from making costly mistakes. 

\textbf{But why stop here?} As trading algorithms evolve, so do the mathematical tools we need to understand them. The next revolution in finance might come from the next mathematical breakthrough—and if history tells us anything, it will probably involve measure theory again.

\begin{quote}
\textbf{Mathematics: where the difference between correlation and causation is the difference between losing and making a million dollars per second.}
\end{quote}
