\subsection{The Shadows of Numbers (or: Zeno Ruins Everything)}

Eudoxus thought he’d patched the system. Ratios still worked. Magnitudes behaved. Irrational numbers weren’t a crisis—they were just a new kind of quantity to fold into the mix.

But beneath all that elegance lurked an older, stranger crack in the foundation. A paradox so unsettling that even Plato decided not to fix it—he just built a new metaphysics around it.

\textbf{Enter Zeno.}

Zeno’s paradoxes weren’t just brain teasers; they were landmines buried in the concept of continuity itself. If motion was real, Zeno argued, then so was an infinite number of steps in any journey. But if that were true—how could anything ever move?

\textit{(An arrow can’t fly. Achilles can’t outrun a tortoise. Reality can’t actually… happen.)}

\vspace{0.5em}
To the Greeks, motion felt obvious. But Zeno showed that it wasn’t logically obvious. Not if you assumed space and time were made of smooth, uninterrupted quantities.  
Not if you assumed the number line had no gaps.  

\textbf{And yet—everyone kept assuming.}  

Plato took the escape hatch: motion belonged to the broken sensory world, not to the timeless realm of Forms. Eudoxus focused on saving geometry. But nobody really addressed the underlying issue:  

\textit{What does it even mean for space and time to be continuous?}  

Zeno wasn’t just trolling. He was pointing at a flaw the ancients couldn’t yet define—and modern mathematics would spend centuries trying to fix.

\begin{figure}[H]
\centering
\begin{tikzpicture}[every node/.style={font=\footnotesize}]

% Panel 1 — Zeno explaining the paradox
\comicpanel{0}{4}
  {Student}
  {Zeno}
  {\textbf{Zeno:} To move from A to B, you must first go halfway.  
Then half of that. Then half of that...  
Infinite steps. So, you never arrive.}
  {(0,-0.5)}

% Panel 2 — Student thinking
\comicpanel{6.5}{4}
  {Student}
  {Zeno}
  {\textbf{Student:} Wait—are you saying walking is logically impossible?}
  {(0,-0.5)}

% Panel 3 — Plato enters
\comicpanel{0}{0}
  {Zeno}
  {Plato}
  {\textbf{Plato:} Don’t worry. Movement only happens in the imperfect world.  
The true Forms remain still.}
  {(0,0.8)}

% Panel 4 — Student, more confused
\comicpanel{6.5}{0}
  {Student}
  {Plato}
  {\textbf{Student:} So I can’t walk, but it’s okay,  
because the perfect version of me isn’t moving anyway?}
  {(0,0.8)}

\end{tikzpicture}
\caption{Zeno exposed the paradox of motion. Plato responded with metaphysics. The rest of the world just kept walking.}
\end{figure}
