\subsection{The Greeks: Taking Math Too Seriously Since 500 BCE}

Unlike the Egyptians, who used math to solve real problems, the Greeks used it to ask deep, unsettling questions that ruined everything.

\textbf{Enter Pythagoras}, a man who looked at the concept of numbers and asked: What if numbers were the secret code of the universe?

Now, at this point, numbers were doing just fine. They were helping farmers measure land, helping merchants keep track of debts, and generally not causing existential crises.  

But Pythagoras wasn’t satisfied. He and his followers (the Pythagoreans) believed that numbers weren’t just practical. They were sacred.  

\begin{quote}
To them, everything could be explained by whole numbers and their ratios.  Which sounds great. Except...  \textbf{Some numbers can’t be written as a fraction.}
\end{quote}

Cue immediate panic.  

The problem started with the \textbf{diagonal of a square}. If you take a square with a side length of 1, then by the Pythagorean Theorem:

\[
1^2 + 1^2 = 2
\]

which means the diagonal is:

\[
\sqrt{2}
\]

No big deal, right? Just write it as a fraction. That’s what the Pythagoreans tried to do:

\[
\sqrt{2} = \frac{p}{q}
\]

where \( p \) and \( q \) are whole numbers with no common factors (this is called an irreducible fraction). But when they tried to prove it, everything fell apart.

\begin{tcolorbox}[title=Historical Sidebar: The Proof That Broke Their Brains, colback=gray!5, colframe=black, fonttitle=\bfseries]

  The discovery that \( \sqrt{2} \) could not be written as a ratio of whole numbers shook the foundations of early Greek mathematics. It wasn't just a technical glitch — it was a philosophical crisis. Here's the proof that did it:
  
  \begin{enumerate}
  \item Suppose that \( \sqrt{2} \) can be written as a fraction:
  
     \[
     \sqrt{2} = \frac{p}{q}
     \]
  
     where \( p \) and \( q \) have \textbf{no common factors}.
  
  \item Squaring both sides:
  
     \[
     2 = \frac{p^2}{q^2}
     \]
  
  \item Multiplying both sides by \( q^2 \):
  
     \[
     2q^2 = p^2
     \]
  
     So \( p^2 \) is \textbf{even}, which means \( p \) must also be even. Let \( p = 2k \) for some integer \( k \).
  
  \item Substituting back in:
  
     \[
     2q^2 = (2k)^2 = 4k^2
     \]
  
     Dividing both sides by 2:
  
     \[
     q^2 = 2k^2
     \]
  
     So \( q^2 \) is also even, meaning \( q \) is even too.
  
  \item But if both \( p \) and \( q \) are even, then they share a common factor of 2 --- contradicting our original assumption that they had no common factors.
  
  \item \textbf{Conclusion:} Our assumption must be false. Therefore:
  
     \[
     \sqrt{2} \text{ is NOT a fraction.}
     \]
  
  \end{enumerate}
  
  \medskip
  
  This was the first formal proof that a number could exist but \emph{not be a ratio of integers}. It marked the birth of \textbf{irrational numbers} — and the beginning of ancient mathematicians realizing that not everything could be tamed by whole numbers and geometry.
  
\end{tcolorbox}
