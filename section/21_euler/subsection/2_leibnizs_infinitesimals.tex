\subsection{Leibniz’s Infinitesimals: Algebraic, But Still a Mess}  

Leibniz, at least, tried to make calculus \textbf{more symbolic}. His notation,  

\[
dy = \frac{dx}{dx} dy
\]

had some major advantages:

\begin{itemize}
    \item It was \textbf{better for algebra}—you could manipulate differentials like fractions.
    \item It made calculus more \textbf{generalizable}, helping it evolve beyond just motion problems.
\end{itemize}

But let’s not give Leibniz too much credit. His notation still got \textbf{clunky} when dealing with complex equations, and \textbf{nobody really knew what infinitesimals actually were.} Were they numbers? Were they just really, really small? Were they some kind of mathematical hallucination? The debate went unresolved for centuries.
