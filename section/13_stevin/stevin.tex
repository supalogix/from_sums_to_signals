\section{The History of How Simon Stevin Turned Fractions into Decimals}

\subsection{Background: Before Stevin}

Decimal fractions existed long before Simon Stevin. They appear in the works of Islamic mathematicians like Al-Kashi (15th century) and were known to European astronomers such as Regiomontanus and Copernicus. But these decimals were usually limited to specialized contexts — like astronomical tables — and often written in sexagesimal (base-60) notation.

\medskip

In everyday arithmetic, people still used common fractions like \( \frac{3}{4} \), which made calculations with money, weights, and measures cumbersome and prone to error.

\subsection*{1585: Stevin Publishes \textit{De Thiende} ("The Tenth")}

In 1585, Stevin published a short but revolutionary book called \textit{De Thiende}, or "The Tenth." In it, he proposed that decimal fractions weren't just useful — they should become the \textit{default} method for doing arithmetic.

\medskip

\textbf{What he proposed:}
\begin{itemize}
    \item Any number, no matter how awkward, could be written as a sum of tenths, hundredths, thousandths, and so on.
    \item Instead of \( \frac{1}{2} \), write \( 0.5 \). Instead of \( \frac{3}{4} \), write \( 0.75 \).
    \item Eliminate the need for common denominators, mental gymnastics, or awkward conversions.
\end{itemize}

\medskip

\textbf{Why it mattered:}
\begin{itemize}
    \item It simplified complex calculations in trade, taxation, land surveying, engineering, and navigation.
    \item It made arithmetic more algorithmic — easier to teach, easier to standardize, and ready for mechanical use.
\end{itemize}

\subsection{Stevin's Notation (Clunky but Conceptual)}

Stevin didn’t use a decimal point — that refinement came later, from John Napier and others. Instead, he used \textbf{superscripts} to indicate place value.

\[
2^{0}1^{1}4^{2} \quad \text{for what we now write as} \quad 2.14
\]

Where:
\begin{itemize}
    \item The superscript \( ^{0} \) indicates the units place,
    \item \( ^{1} \) the tenths place,
    \item \( ^{2} \) the hundredths place, and so on.
\end{itemize}

His notation looked strange, but the philosophy behind it was brilliant: \textbf{every number can be decomposed into sums of powers of ten}.

\subsection*{Stevin's Deeper Insight: Decimals Are Universal}

In \textit{De Thiende}, Stevin wrote:

\begin{quote}
"There is no number which cannot be written by using only the unit, ten, and its powers."
\end{quote}

He argued that:
\begin{itemize}
    \item All arithmetic could — and should — be conducted with decimals.
    \item Common fractions were arbitrary and outdated.
    \item Even irrational numbers could be approximated through infinite decimal expansions.
\end{itemize}

\subsection{Influence and Legacy}

\textbf{Short-term:}
\begin{itemize}
    \item Stevin’s notation didn’t catch on, but his core idea — using decimals — quickly gained traction.
    \item Within a few decades, mathematicians like John Napier adopted the decimal point, making Stevin’s vision more readable and widespread.
\end{itemize}

\textbf{Long-term:}
\begin{itemize}
    \item Decimal arithmetic laid the foundation for:
    \begin{itemize}
        \item The metric system,
        \item Decimal currency systems,
        \item Scientific notation,
        \item Modern calculator logic.
    \end{itemize}
    \item Today, decimal notation is the default format for real numbers across science, engineering, and finance.
\end{itemize}

\subsection{Summary: How Stevin Turned Fractions into Decimals}

\begin{center}
\renewcommand{\arraystretch}{1.4}
\begin{tabular}{|l|p{10cm}|}
\hline
\textbf{Step} & \textbf{Description} \\
\hline
Inspiration & Recognized that fractions were awkward and inefficient for everyday calculation. \\
\hline
Idea & Proposed that all numbers could be written using tenths, hundredths, and so on. \\
\hline
Notation & Used superscripts (e.g. \( 2^{0}1^{1}4^{2} \)) to indicate place values before decimal points were in use. \\
\hline
Publication & \textit{De Thiende} (1585), which promoted decimal arithmetic for all fields. \\
\hline
Impact & Paved the way for modern decimal notation, the metric system, and numerical computing. \\
\hline
\end{tabular}
\end{center}


\begin{tcolorbox}[colback=gray!5!white, colframe=black!80!white, title={Historical Sidenote: Stevin’s Theology of Decimal Math}, fonttitle=\bfseries, arc=1.5mm, boxrule=0.4pt]

    Simon Stevin didn’t just believe that decimals were useful — he believed they were \textit{right}. Behind his mathematical reforms was a deeply Calvinist conviction: that knowledge should be accessible to all, not hoarded by elites. Just as Reformers like Luther argued that Scripture should be available in the language of the people, Stevin insisted that mathematics should be written in the vernacular — and so he published all his work in Dutch.
    
    This wasn’t just a pedagogical choice. It was theological. Stevin believed that mathematics was a gift from God meant for everyone — not just scholars, clergy, or merchants. In his eyes, decimals weren’t simply more efficient than fractions; they reflected a divine clarity embedded in creation itself.
    
    His motto, \textit{“Wiskunde voor de Gemeene Man”} (“Mathematics for the Common Man”), wasn’t a slogan. It was a mission. Decimal notation, written in Dutch, was part of a broader vision: that truth — mathematical and spiritual — belonged to the people.
    
\end{tcolorbox}

