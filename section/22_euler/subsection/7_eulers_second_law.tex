\subsection{Kepler’s Second Law: Who Did What, and Why Euler Wins Anyway}

Now that the calculus wars have cooled and the concept of rotational inertia is on the table, we can revisit the idea that started all of this: \textbf{Kepler’s Second Law} — the law that says a planet sweeps out equal areas in equal times.

This elegant geometric principle became a lens through which Newton, Leibniz, and Euler each interpreted motion — each in radically different ways. And the way they did it reflects how they saw the universe.

\subsubsection{Newton: Geometry in the Service of Force}

In the \textit{Principia}, Newton didn’t use algebra or calculus to explain Kepler’s Second Law. He went full Euclid.

He imagined a planet moving in discrete steps — tracing a piecewise linear orbit. At each instant, it moved in a straight line, then was pulled slightly toward the sun. These repeated deflections formed triangular wedges. He showed that:

\begin{quote}
If a planet sweeps out equal areas in equal times, then the force acting on it must always point toward the center.
\end{quote}

And he proved the converse: if the force is always directed toward a center (a central force), then the planet must sweep out equal areas in equal times.

\textbf{Newton’s notation:}
\begin{itemize}
    \item No \( r \) or \( \theta \)
    \item No \( \frac{dA}{dt} \)
    \item Just geometry, lines, triangles, proportions, and ratios.
\end{itemize}

His insight was physical, but his method was classical.

\textbf{Key Contribution:}
\begin{itemize}
    \item \textbf{Equal areas} \(\Rightarrow\) \textbf{Central force}
    \item \textbf{Central force} \(\Rightarrow\) \textbf{Equal areas}
\end{itemize}

\subsubsection{Leibniz: Algebra in the Service of Accumulation}

Leibniz never directly addressed Kepler’s laws, but he invented something that would become essential: a language for infinitesimal change. Where Newton used triangles, Leibniz used differentials.

For Leibniz, motion was not geometric but algebraic — not drawn, but summed. He introduced symbols like:

\[
dx, \quad dy, \quad \int
\]

Even though he never wrote down an equation for area swept over time, the concept of summing infinitely small pieces — infinitesimal arcs, infinitesimal times — was baked into his calculus.

He might have imagined area as accumulating via:

\[
dA = \text{(something)} \cdot dt
\]

Though the modern form using radius and angle came later, Leibniz laid the foundation. He thought not in terms of curves on a page, but quantities in symbolic motion — each differential a whisper of change.

\textbf{Key Contribution:}
\begin{itemize}
    \item Provided a symbolic system for expressing infinitesimal change
    \item Created the tools needed to reinterpret Kepler’s geometry as algebra
\end{itemize}

\subsubsection{Euler: The Synthesis — Motion, Resistance, and Invariance}

Euler entered the scene armed with Newton’s physics, Leibniz’s calculus, and his own uncanny ability to see structure in motion. Where Newton had relied on geometry and Leibniz had crafted a symbolic language of differentials, Euler fused the two into a coherent analytical mechanics.

He did not use modern vector notation or partial derivatives. Instead, he worked with differentials and algebraic expressions of motion, writing things like:

\[
dz = M\,dx + N\,dy
\]

and thinking of \( M \) and \( N \) as expressions of how a quantity changed with respect to \( x \) and \( y \) independently — though he wouldn’t call them “partials” as we do today.

For Kepler’s Second Law, Euler understood that the sweeping of area over time implied a relationship between a planet’s distance from the sun and its rotational velocity. He recognized that when the radius to the sun was shorter, the velocity had to be greater to sweep the same area — and vice versa.

Euler interpreted this balance as a kind of conserved quantity, though not yet named “angular momentum.” In his language, this meant:

- The product of distance and rotational speed must remain constant
- The accumulation of small sectors (areas) over time must proceed at a uniform rate

He may have written:

\[
\text{Let } dA \text{ be the differential area swept, and } t \text{ the time, then } \frac{dA}{dt} = \text{constant}
\]

But Euler would not have broken this into \( r^2 \frac{d\theta}{dt} \) — polar coordinates were not fully formalized yet.

\textbf{What Euler contributed:}
\begin{itemize}
    \item Treated area sweep as a continuous accumulation of differentials in time
    \item Identified the conservation principle implicit in Kepler’s law
    \item Created a language to analyze motion via functions of multiple changing quantities (time, distance, angle), without modern coordinate systems
\end{itemize}

Euler's genius was not in introducing new physical laws — but in giving existing ones a form that could be used, manipulated, and generalized. His work didn’t just preserve Kepler’s Second Law — it made it calculable.

\subsubsection{From Geometry to Polar Coordinates: The Path to Modern Notation}

The expressions we now associate with Kepler’s Second Law — like:

\[
\frac{dA}{dt} = \frac{1}{2} r^2 \frac{d\theta}{dt}
\]

— were not written by Kepler, Newton, Leibniz, or even Euler. They emerged in the later 18th and early 19th centuries, as the **polar coordinate system** was formalized by mathematicians such as **Gregory**, **Bernoulli**, and **Euler himself** in related contexts.

In polar coordinates, a point’s position is defined by:

\[
(r, \theta)
\]

Where:
- \( r \) is the distance from a central origin (e.g., the sun)
- \( \theta \) is the angle from a fixed axis

With this system, the area swept out by a moving object over time could be written as:

\[
\frac{dA}{dt} = \frac{1}{2} r^2 \frac{d\theta}{dt}
\]

This formula unifies:
- Geometry (via area)
- Motion (via \( \frac{d\theta}{dt} \))
- Position (via \( r \))

And ultimately leads to a modern interpretation of Kepler’s Second Law as the conservation of angular momentum:

\[
L = m r^2 \frac{d\theta}{dt}
\]

But this leap required the mathematical infrastructure that Euler helped lay, even if he did not yet write with these symbols.

\textbf{Why Polar Coordinates Mattered:}
\begin{itemize}
    \item Translated classical geometric motion into symbolic calculus
    \item Allowed explicit computation of area sweep rates
    \item Enabled the unification of Kepler’s laws with Newtonian mechanics and conservation principles
\end{itemize}

In short, polar coordinates turned Euler’s insights into tools — and gave future physicists the power to simulate the sky with a pen.



\subsubsection{Final Scorecard: Who Did What?}

\begin{center}
\renewcommand{\arraystretch}{1.6}
\begin{tabular}{|c|p{4.5cm}|p{6.5cm}|}
\hline
\textbf{Mathematician} & \textbf{What They Saw} & \textbf{What They Contributed} \\ \hline
\textbf{Kepler} & The sky & Planets sweep equal areas in equal time \\ \hline
\textbf{Newton} & Geometry in motion & Equal areas imply central force; proved it with classical constructions \\ \hline
\textbf{Leibniz} & Infinitesimal change & Introduced calculus as a symbolic method for accumulation \\ \hline
\textbf{Euler} & Invariance in motion & Interpreted Kepler's law as angular momentum conservation; bridged geometry and dynamics \\ \hline
\end{tabular}
\end{center}

\begin{quote}
\textit{Newton showed the force. Leibniz wrote the language. Euler unified the laws.}
\end{quote}

\subsubsection{Diagram: One Law, Three Interpretations}

\begin{figure}[H]
\centering
\begin{tikzpicture}[node distance=1.4cm and 3.2cm, every node/.style={align=center}]
\node (kepler) {\textbf{Kepler} \\ Area swept = constant};
\node (newton) [below left=of kepler] {\textbf{Newton} \\ Geometry and force};
\node (leibniz) [below right=of kepler] {\textbf{Leibniz} \\ Symbolic accumulation};
\node (euler) [below=of kepler] {\textbf{Euler} \\ Angular momentum conserved};

\draw[->] (kepler) -- (newton);
\draw[->] (kepler) -- (leibniz);
\draw[->] (newton) -- (euler);
\draw[->] (leibniz) -- (euler);
\end{tikzpicture}
\caption{Kepler saw the pattern. Newton proved the force. Leibniz wrote the symbols. Euler explained the physics.}
\end{figure}
