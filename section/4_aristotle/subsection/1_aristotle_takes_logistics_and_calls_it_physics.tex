\subsection{Aristotle Takes Logistics and Calls It Physics}

Plato tried to sidestep the problem of motion by treating it as a flaw—a clumsy imitation of mathematical perfection. \textbf{But} Aristotle wasn’t having it. He looked at the world and saw something obvious: \textbf{things move because that’s what they do.}  

Forget divine craftsmen or shadowy ideal Forms—\textbf{motion wasn’t an illusion, it was part of an object’s very being.} Instead of treating motion as something imposed from the outside, Aristotle argued that motion was \textbf{internal to the nature of things}. Objects didn’t move because of external forces or hidden Forms—they moved because they were \textbf{fulfilling their purpose} (\textit{Physics}, II.1, 192b8-15).  

And if that sounds like someone who spent his childhood surrounded by logistics, administration, and political order—it’s because he did.  

Aristotle wasn’t just any philosopher; he was raised in the inner circle of Macedonian power. His father, Nicomachus, was the personal physician to \textbf{King Philip II}, which meant Aristotle grew up in the palace, surrounded by generals, bureaucrats, and aristocrats managing everything from military campaigns to grain supplies.  

To him, \textbf{the universe was like a well-run kingdom}. Everything had its proper role, its assigned place, and its function. Motion wasn’t chaos—it was the orderly unfolding of things reaching their intended destinations. His physics wasn’t just about movement; it was a \textbf{cosmic inventory system} that classified how and why everything moved.  


\begin{figure}[H]
\centering
\begin{tikzpicture}[every node/.style={font=\footnotesize}]

% Panel 1 — Aristotle explains his idea of motion
\comicpanel{0}{4}
  {Aristotle}
  {Plato}
  {\textbf{Aristotle:} Motion isn’t a flaw. It’s everything striving to fulfill its purpose.}
  {(0,-0.5)}

% Panel 2 — Plato skeptical
\comicpanel{6.5}{4}
  {Plato}
  {Aristotle}
  {\textbf{Plato:} So you're telling me rocks have ambition now?}
  {(0,-0.5)}

% Panel 3 — Aristotle clarifies
\comicpanel{0}{0}
  {Aristotle}
  {Plato}
  {\textbf{Aristotle:} Not ambition. Just... a very strong sense of duty.}
  {(0,0.8)}

% Panel 4 — Plato sighs
\comicpanel{6.5}{0}
  {Plato}
  {Aristotle}
  {\textbf{Plato:} You grew up in a palace, didn’t you?}
  {(0,0.8)}

\end{tikzpicture}
\caption{Aristotle explains motion as an orderly drive toward purpose. Plato remains skeptical.}
\end{figure}

