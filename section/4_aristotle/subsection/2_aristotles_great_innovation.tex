\subsection{Aristotle’s Great Innovation: Motion is Intrinsic}

For Aristotle, motion wasn’t just an accident of reality—it was \textbf{deeply tied to what an object is}. Every object had a built-in drive to behave in a certain way, based on its nature. This led him to classify motion into two types (\textit{Physics}, VIII.4, 255b30–256a4):

\begin{itemize}
    \item \textbf{Natural Motion}: Motion that occurs \textbf{because of an object’s intrinsic nature}.
    \begin{itemize}
        \item Rocks \textbf{fall} because their nature is to move toward the center of the universe (Earth) (\textit{On the Heavens}, I.2, 268b1–10).
        \item Fire \textbf{rises} because it seeks its proper place in the heavens.
        \item Planets move in \textbf{circular orbits} because \textbf{circular motion is the most perfect and eternal form of movement} (\textit{Metaphysics}, XII.8, 1073a25–1074a32).
    \end{itemize}
    \item \textbf{Violent Motion}: Motion that happens only when \textbf{something forces an object away from its natural behavior}.
    \begin{itemize}
        \item A cart moves when \textbf{pushed}, but it will stop as soon as the pushing stops (\textit{Physics}, IV.8, 215a20–30).
        \item A rock thrown upward is moving \textbf{against its nature}, so it eventually gives up and falls.
    \end{itemize}
\end{itemize}

Unlike Plato, who explained motion using abstract mathematics, Aristotle’s motion had \textbf{a cause rooted in the object itself}. If an object was in motion, it was because \textbf{that was what it was meant to do}.

But this wasn’t just physics in the modern sense—it was part of something much broader.


\begin{tcolorbox}[title=Historical Sidebar: Parmenides and the War on Motion, colback=gray!5, colframe=black, fonttitle=\bfseries]

  Long before Zeno was writing paradoxes, his teacher \textbf{Parmenides} was declaring war on motion itself. His core idea? \textbf{Change is impossible.} And not just difficult or unlikely — logically impossible.

  \medskip
  
  In his poem \textit{On Nature}, Parmenides argues that reality is one, unchanging, eternal, and indivisible. What \emph{is}, he says, simply \textbf{is}. And what \emph{is not} cannot be — not even as a transitional phase.

  \medskip
  
  So if something changes — if it “comes into being” or “ceases to be” — then it must have passed through a state of not-being. But according to Parmenides, \textbf{non-being is nothing}, and \textit{nothing} can’t exist.
  
  \medskip
  
  \textbf{Therefore:} \textit{Change is incoherent. Becoming is an illusion. Motion is a lie.}

  \medskip
  
  This view wasn’t just radical — it was terrifying. If Parmenides was right, then all sensory experience was fundamentally wrong. Movement, time, growth — all deception. The world we see? A painted veil.
  
  \medskip
  
  Later thinkers were forced to respond. Zeno doubled down, turning Parmenides’ ideas into paradoxes that would haunt mathematicians for centuries. Plato tried to compartmentalize the problem by separating the eternal world of Forms from the unstable world of appearances. Aristotle would later reject the premise entirely and build a framework where motion could be explained logically and coherently.

  \medskip
  
  But it all began with Parmenides — the philosopher who looked at the world and said: \textbf{“None of this should be happening.”}
  
\end{tcolorbox}
  