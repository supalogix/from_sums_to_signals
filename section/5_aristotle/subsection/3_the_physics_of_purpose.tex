\subsection{The Physics of Purpose: Teleology in Motion}

Before science became a separate discipline, it was part of a larger project known as \textbf{natural philosophy}—the attempt to understand nature not just through measurement, but through meaning. For Aristotle, motion wasn’t just about where something was going; it was about \textbf{why} it was going there.

He believed that \textbf{everything in nature strives toward its natural place and function}. This is the foundation of \textbf{teleology}—the idea that motion and change happen \textit{for the sake of something} (\textit{Physics}, II.8, 199a5–15).

\begin{itemize}
    \item The Earth is at the center of the universe because \textbf{that is where it belongs} (\textit{On the Heavens}, II.14, 296a1–20).
    \item Heavy objects fall because they are \textbf{trying to return to their rightful place}.
    \item The stars and planets move in circles because \textbf{circular motion is eternal and divine} (\textit{Metaphysics}, XII.7, 1072b30–1073a10).
\end{itemize}

Motion, then, wasn’t merely caused by pushes and pulls. It was a sign of an object \textbf{fulfilling its essence}. Every stone that falls, every flame that rises, every celestial body that spins is trying to complete its nature—to reach its telos, its end, its purpose.

This was Aristotle’s physics of becoming: a world animated not just by forces, but by destiny.

\begin{tcolorbox}[title=Historical Sidebar: Before Science Had Equations It Had Natural Philosophy, colback=gray!5, colframe=black, fonttitle=\bfseries]

  Before physics became a lab-based science, it was part of something older and more holistic: \textbf{natural philosophy}. This wasn’t just about forces and particles — it was an attempt to understand nature in its fullness: what it is, why it changes, and what it might be for.

  \medskip
  
  Natural philosophy asked big questions:

  \medskip

  \begin{itemize}
      \item What is the world made of?
      \item Why do things move?
      \item What is change? What is purpose?
  \end{itemize}

  \medskip
  
  And these questions weren’t invented by Aristotle. He inherited them.

  \medskip
  
  Long before him, the \textbf{Pre-Socratic philosophers} were already laying the groundwork:

  \medskip
  
  \begin{itemize}
      \item \textbf{Thales} (6th c. BCE) proposed that everything came from water — not just as a substance, but as a principle of unification.
      \item \textbf{Anaximander} introduced the idea of the \textit{apeiron}, an indefinite substance behind all things.
      \item \textbf{Heraclitus} argued that everything is in flux and that fire (and \textit{logos}) is the guiding principle of change.
      \item \textbf{Empedocles} described nature as a dynamic balance of four elements driven by love and strife.
  \end{itemize}

  \medskip
  
  What they lacked in testable hypotheses, they made up for in ambition: they were trying to understand \textbf{why nature behaves the way it does} — not just how.
  
  \medskip
  
  Aristotle systematized this tradition. He took their questions and gave them structure, categories, definitions. He didn’t invent natural philosophy — he codified it.

  \medskip
  
  And for over a thousand years, to study nature was to ask what everything was \textbf{made of}, \textbf{made for}, and \textbf{moving toward}.
  
\end{tcolorbox}

