\documentclass{article}
\usepackage{amssymb, amsmath}
\usepackage{amsthm}
\usepackage[T1]{fontenc}  % Proper encoding for text
\usepackage[utf8]{inputenc} % Handles UTF-8 encoding (for most special characters)
\usepackage{lmodern} % Modern font to avoid encoding issues
\usepackage{parskip} % Removes indentation and adds space between paragraphs
\usepackage{amsmath}
\usepackage{tikz}
\usetikzlibrary{matrix,arrows}
\usepackage{pgfplots}
\pgfplotsset{compat=1.18}  % Ensures compatibility with the latest PGFPlots version
\usepackage{booktabs}
\usepackage{array}
\usepackage{titlesec}
\usepackage{makecell}
\usetikzlibrary{positioning}
\usepackage{dot2texi} % Enable Graphviz inside LaTeX
\usetikzlibrary{shapes,arrows}
\usepackage{fancyhdr}
\pagestyle{fancy}  % Set page style to fancy
\fancyhf{}         % Clear default header/footer
\usetikzlibrary{trees, arrows.meta, matrix, decorations.pathreplacing}
\usetikzlibrary{trees, arrows.meta}
\usetikzlibrary{positioning, shapes.callouts}
\usepackage{caption}
\usepackage{subcaption}
\usepackage{float}
\usepackage{bussproofs}
\usepackage{lmodern}
\usepackage{geometry}
\usepackage{xcolor}
\usepackage[most]{tcolorbox}
\usepackage{tikzpeople}
\usepackage{ifthen} % Required for \ifthenelse
\usetikzlibrary{calc, arrows.meta, positioning}
\usetikzlibrary{matrix}
\usepackage{setspace}
\usepackage{tikz-cd}
\usetikzlibrary{fit, positioning, shapes}
\usepackage{import}
\usepackage{pgfgantt}
\usepackage{tabularx}  % Add this to your preamble
\usepackage[most]{tcolorbox} % add to your preamble
\tcbuselibrary{listings}
\usepackage{luamplib}
\usepackage{fontspec}
%\setmainfont{TeX Gyre Termes}

\usepackage{pifont}

%\usepackage{chngcntr}
%\counterwithin*{chapter}{part}




\usepackage{polyglossia}
\setmainlanguage{english}
\setotherlanguage{greek}
\newfontfamily\greekfont{Times New Roman} % or any font that supports Greek








\usepackage{listings}
\usepackage{color}
\usepackage{caption}
\usepackage{amsmath}
\usepackage{amssymb}
\usepackage[utf8]{inputenc}
\usepackage{caption}






\usepackage{etoolbox} % Needed for patching commands

% Insert new page before \section
\pretocmd{\section}{\clearpage}{}{}

% Insert new page before \subsection
\pretocmd{\subsection}{\clearpage}{}{}


\usepackage{etoolbox}
\usepackage{xcolor}
\usepackage{lipsum} % Optional, for testing
\usepackage{graphicx}



% Define a horizontal rule that's 80% of the linewidth
%\newcommand{\figureseparator}{
 % \par\vspace{0.5em}
 % \noindent\makebox[\linewidth]{\rule{0.8\linewidth}{0.4pt}}%
 % \par\vspace{0.5em}
%}


\usepackage{needspace}

\newcommand{\figureseparator}{
  \needspace{3\baselineskip} % Reserve space to prevent page break
  \par\vspace{0.5em}
  \noindent\makebox[\linewidth]{\rule{0.8\linewidth}{0.4pt}}%
  \par
  \vspace{0.5em}
}

% Automatically add separators before and after every figure
%\AtBeginEnvironment{figure}{\figureseparator}
%\AtEndEnvironment{figure}{\figureseparator}












\usepackage[most]{tcolorbox}
\tcbuselibrary{skins, breakable}

% Define a custom figure box style
\tcolorboxenvironment{figure}{
  boxrule=0.4pt,
  colframe=black!40,
  colback=white,
  sharp corners=south,
  drop shadow southeast,
  enhanced,
  breakable,
  before skip=10pt,
  after skip=10pt,
  width=\linewidth
}


\newtcolorbox{calloutfigure}[1][]{%
  breakable,
  colback=blue!5,
  colframe=blue!40,
  title=#1,
  fonttitle=\bfseries,
  sharp corners=south,
  enhanced,
  drop shadow southeast,
  before skip=10pt,
  after skip=10pt,
}



\tcbset{
  sidebarstyle/.style={
    colback=gray!10!white,
    colframe=black,
    fonttitle=\bfseries,
    title={Historical Sidebar},
    boxrule=0.5pt,
    arc=2mm,
    left=1em,
    right=1em,
    top=0.5em,
    bottom=0.5em,
    enhanced,
    sharp corners=south,
    before skip=1em,
    after skip=1em
  }
}








\newcommand{\comicpanel}[6]{
  % === PARAMETER EXPLANATION ===
  % #1: x-shift — Horizontal panel position
  % #2: y-shift — Vertical panel position
  % #3: left character label (always shown below left stick figure)
  % #4: right character label (always shown below right stick figure)
  % #5: callout text
  % #6: callout pointer (relative to callout box)

  % === PANEL BOX ===
  \draw[thick] (#1,#2) rectangle ++(6,4);

  % === LEFT STICK FIGURE ===
  \begin{scope}[shift={(#1+1.2,#2+1.0)}]
    \draw (0,1) circle (0.2);
    \draw (0,0.8) -- (0,0);
    \draw (0,0.6) -- (-0.3,0.3);
    \draw (0,0.6) -- (0.3,0.3);
    \draw (0,0) -- (-0.2,-0.5);
    \draw (0,0) -- (0.2,-0.5);
    \node[below=2pt] at (0,-0.5) {\scriptsize \textbf{#3}};
  \end{scope}

  % === RIGHT STICK FIGURE ===
  \begin{scope}[shift={(#1+4.7,#2+1.0)}]
    \draw (0,1) circle (0.2);
    \draw (0,0.8) -- (0,0);
    \draw (0,0.6) -- (-0.3,0.3);
    \draw (0,0.6) -- (0.3,0.3);
    \draw (0,0) -- (-0.2,-0.5);
    \draw (0,0) -- (0.2,-0.5);
    \node[below=2pt] at (0,-0.5) {\scriptsize \textbf{#4}};
  \end{scope}

  % === SPEECH BUBBLE ===
  \node[draw, rectangle callout, rounded corners=2pt,
        callout relative pointer={#6},
        text width=4.5cm, align=left] 
        at (#1+3,#2+3.2) {#5};
}



\usepackage[most]{tcolorbox}

\tcbset{
  examplebox/.style={
    colback=gray!5!white,
    colframe=gray!75!black,
    fonttitle=\bfseries,
    coltitle=black,
    title=Example,
    sharp corners,
    boxrule=0.5pt,
    leftrule=0pt,
    rightrule=0pt,
    bottomrule=0pt,
    toptitle=1mm,
    bottomtitle=1mm,
  }
}

\newtcolorbox{example}[1][]{examplebox, #1}


\usepackage{setspace}  % for line spacing



% In your preamble
\usepackage{tikz}
\usepackage{setspace}
\usepackage{xcolor}

\renewenvironment{quote}
  {
    \par\vspace{2em}
    \noindent
    \begin{tikzpicture}[remember picture, overlay]
      % Giant left quote mark (top-left)
      \node[anchor=north west, font=\fontsize{60}{60}\selectfont\color{gray!40}]
        at ([xshift=-1em,yshift=1.5em]0,0) {``};
    \end{tikzpicture}

    \begin{list}{}{\leftmargin=3em \rightmargin=3em}
    \item\relax
    \color{black!70}
    \bfseries\itshape
    \begin{spacing}{1.3}
    \fontsize{12pt}{16pt}\selectfont % ← placed correctly here
  }
  {
    \end{spacing}
    \end{list}

    % Giant right quote mark (bottom-right)
    \begin{tikzpicture}[remember picture, overlay]
      \node[anchor=south east, font=\fontsize{60}{60}\selectfont\color{gray!40}]
        at ([xshift=1em,yshift=-1.2em]\textwidth,0) {''};
    \end{tikzpicture}
    \par\vspace{0.25em}
  }















\renewcommand{\headrulewidth}{0pt}
% Define copyright notice in the footer
\fancyhead[R]{\thepage} % Centered page number
\fancyfoot[L]{\small \textcopyright~2025 Jonathan Nacionales}
\fancyfoot[R]{\small jnaciona@gmail.com | 310.227.5510}



\titleformat{\section}
  {\normalfont\Large\bfseries}{\thesection}{1em}{}[{\titlerule[0.8pt]}]
\newcommand{\sectionbreak}{\clearpage}







\tikzset{
  dagnode/.style={
    draw,
    rounded corners,
    minimum height=0.8cm,
    minimum width=2.2cm,
    font=\scriptsize,
    text centered,
    fill=blue!5
  },
  arrow/.style={
    ->,
    thick
  }
}


\begin{document}
\onehalfspacing
\setlength{\parskip}{1.5em} 


\begin{titlepage}
  \centering

  {\LARGE\bfseries From Sums to Signals: A Mathematical Guide to Deep Learning \par}
  \vspace{1em}
  {\large Featuring a Special Appearance by High-Frequency Trading \par}
  \author{Jonathan Nacionales}
\date{\today}

  \vfill

 \begin{figure}[H]
\centering
\begin{tikzpicture}[every node/.style={font=\footnotesize}]

% Panel 1 — Engineer 1 with buzzwords
\comicpanel{0}{4}
  {Engineer 1}
  {Engineer 2}
  {We amplify latent embeddings with dynamic gradient fusion layers and recursive self-attention.}
  {(-0.6,-0.5)}

% Panel 2 — Engineer 2 simplifies
\comicpanel{6.5}{4}
  {Engineer 1}
  {Engineer 2}
  {So... you're summing weighted functions of inputs again?}
  {(.5,-0.5)}

% Panel 3 — Engineer 1 gets even worse
\comicpanel{0}{0}
  {Engineer 1}
  {Engineer 2}
  {With Fourier-regularized signal priors injected through a learnable hyperkernel manifold.}
  {(-0.7,-0.7)}

% Panel 4 — Engineer 2 drops the truth
\comicpanel{6.5}{0}
  {Engineer 1}
  {Engineer 2}
  {Talking like that is why execs hire consultants to explain things to them. Please stop.}
  {(0.7,-0.6)}

\end{tikzpicture}
\caption{Machine learning: just math with a hype wrapper.}
\end{figure}
\end{titlepage}

%\frontmatter
\tableofcontents

%\mainmatter % <--- This starts chapter numbering at 1

\newpage

%\part{Introduction}
%\subimport*{section/1_bridging_the_gap}{bridging_the_gap.tex}

%\part{Mathematical Structures}
%\subimport*{part/2_mathematical_structures/1_pythagorous}{pythagorous.tex}
%\subimport*{part/2_mathematical_structures/2_eudoxus}{eudoxus.tex}
%\subimport*{part/2_mathematical_structures/3_euclid}{euclid.tex}
%\subimport*{part/2_mathematical_structures/4_archamedes}{archamedes.tex}
%\subimport*{part/2_mathematical_structures/5_potolemy}{potolemy.tex}
%\subimport*{part/2_mathematical_structures/6_al_battani}{al_battani.tex}
%\subimport*{part/2_mathematical_structures/7_stevin}{stevin.tex}
%\subimport*{part/2_mathematical_structures/8_copernicus}{copernicus.tex}
%\subimport*{part/2_mathematical_structures/9_galileo}{galileo.tex}
%\subimport*{part/2_mathematical_structures/10_kepler}{kepler.tex}
%\subimport*{part/2_mathematical_structures/11_cavalieri}{cavalieri.tex}
%\subimport*{part/2_mathematical_structures/12_wallis}{wallis.tex}
%\subimport*{part/2_mathematical_structures/13_barrow}{barrow.tex}
%\subimport*{part/2_mathematical_structures/14_newton}{newton.tex}
%\subimport*{part/2_mathematical_structures/15_leibniz}{leibniz.tex}
%\subimport*{part/2_mathematical_structures/16_euler}{euler.tex}
%\subimport*{part/2_mathematical_structures/17_lagrange}{lagrange.tex}
%\subimport*{part/2_mathematical_structures/18_laplace}{laplace.tex}
%\subimport*{part/2_mathematical_structures/19_hamilton}{hamilton.tex}
%\subimport*{part/2_mathematical_structures/20_riemann}{riemann.tex}
%\subimport*{part/2_mathematical_structures/21_heavyside}{heavyside.tex}
%\subimport*{part/2_mathematical_structures/22_gibbs}{gibbs.tex}
%\subimport*{part/2_mathematical_structures/23_hilbert}{hilbert.tex}

%\part{Mathematical Analysis}
%\subimport*{part/3_mathematical_analysis/1_cauchy}{cauchy.tex}
%\subimport*{part/3_mathematical_analysis/2_fourier}{fourier.tex}
%\subimport*{part/3_mathematical_analysis/3_riemann}{riemann.tex}
%\subimport*{part/3_mathematical_analysis/4_weierstrass}{weierstrass.tex}
%\subimport*{part/3_mathematical_analysis/5_dedekind}{dedekind.tex}
%\subimport*{part/3_mathematical_analysis/6_cantor}{cantor.tex}
%\subimport*{part/3_mathematical_analysis/7_borel}{borel.tex}
%\subimport*{part/3_mathematical_analysis/8_lebesgue}{lebesgue.tex}

%\part{Mathematical Uncertainty}
%\subimport*{part/4_mathematical_uncertainty/1_kolmogorov}{kolmogorov.tex}
%\subimport*{part/4_mathematical_uncertainty/2_fisher}{fisher.tex}
%\subimport*{part/4_mathematical_uncertainty/3_shannon}{shannon.tex}
%\subimport*{part/4_mathematical_uncertainty/4_kullback_and_leibler}{kullback_and_leibler.tex}
%\subimport*{part/4_mathematical_uncertainty/5_jeffries}{jeffries.tex}
%\subimport*{part/4_mathematical_uncertainty/6_jordan}{jordan.tex}
%\subimport*{part/4_mathematical_uncertainty/7_amari}{amari.tex}

%\part{Mathematical Foundations}
%\subimport*{part/5_mathematical_foundations/1_plato}{plato.tex}
%\subimport*{part/5_mathematical_foundations/2_aristotle}{aristotle.tex}
%\subimport*{part/5_mathematical_foundations/3_plotinus}{plotinus.tex}
%\subimport*{part/5_mathematical_foundations/4_augustine}{augustine.tex}
%\subimport*{part/5_mathematical_foundations/5_aquinus}{aquinus.tex}
%\subimport*{part/5_mathematical_foundations/6_godel}{godel.tex}
%\subimport*{part/5_mathematical_foundations/7_mckay}{mckay.tex}
%\subimport*{part/5_mathematical_foundations/8_hoeffding}{hoeffding.tex}
%\subimport*{part/5_mathematical_foundations/9_wittgenstein}{wittgenstein.tex}
%\subimport*{part/5_mathematical_foundations/10_chaitin}{chaitin.tex}

%\part{Mathematical Computation}
%\subimport*{part/6_mathematical_computation/1_turing}{turing.tex}
%\subimport*{part/6_mathematical_computation/2_shannon}{shannon.tex}
%\subimport*{part/6_mathematical_computation/3_von_neumann}{von_neumann.tex}
%\subimport*{part/6_mathematical_computation/4_ulmann}{ulmann.tex}
%\subimport*{part/6_mathematical_computation/5_nash}{nash.tex}
%\subimport*{part/6_mathematical_computation/6_pontryagin}{pontryagin.tex}
%\subimport*{part/6_mathematical_computation/7_rossenblatt}{rossenblatt.tex}
%\subimport*{part/6_mathematical_computation/8_minskey}{minskey.tex}
%\subimport*{part/6_mathematical_computation/9_hinton}{hinton.tex}
%\subimport*{part/6_mathematical_computation/10_hopfield}{hopfield.tex}
%\subimport*{part/6_mathematical_computation/11_sejnowski}{sejnowski.tex}
%\subimport*{part/6_mathematical_computation/12_yann_lecun}{yann_lecun.tex}
%\subimport*{part/6_mathematical_computation/13_hinton}{hinton.tex}
%\subimport*{part/6_mathematical_computation/14_hochreiter}{hochreiter.tex}
%\subimport*{part/6_mathematical_computation/15_alexnet}{alexnet.tex}
%\subimport*{part/6_mathematical_computation/16_harris}{harris.tex}
%\subimport*{part/6_mathematical_computation/17_vaswani}{vaswani.tex}
%\subimport*{part/6_mathematical_computation/18_yann_lecun}{yann_lecun.tex}
%\subimport*{part/6_mathematical_computation/19_shwartz}{shwartz.tex}

%\part{A Practical Use Case in High Frequency Trading}
%\section{Gödel, Markets, and the Cost of Ignoring the Right Mathematics}

\subsection{The Market Is Not Complete (And Neither Is Your Math)}

Pontryagin tried to control the future. Gödel warned we could never fully predict it.

Between them lies the paradox of modern finance: a system obsessed with optimization, built atop models that are fundamentally incomplete.

Control theory tells us how to act under constraints. Gödel reminds us those constraints will never close fully. Together, they whisper a hard truth: no matter how advanced our math gets, the system will always have blind spots—especially when it starts trading against itself.

\medskip
\noindent
Mathematicians still argue about infinity, physicists about discreteness. But finance? Finance just assumes it can take derivatives faster than reality can respond. And when it’s wrong, it crashes.

\medskip
\noindent
The market may not care about Gödel’s incompleteness theorem explicitly—but it lives it daily. Every trading strategy contains a proof it hopes will hold longer than its competitors'. Every hedge is a patch on a model that doesn’t quite close. And when someone finds a missing piece—or a better approximation—they arbitrage the rest of us.

\medskip
\noindent
This is why we need tools like \textbf{measure theory} and \textbf{Lebesgue integration}—not as ivory tower abstractions, but as armor for survival. In a world where trades fire faster than causality should allow, and where the boundary between model and market is fluid, we need ways to quantify uncertainty without assuming too much structure.

\begin{quote}
\emph{Gödel taught us that no system is self-complete. Pontryagin gave us tools to steer within those gaps. Finance monetized both.}
\end{quote}

So no, your model isn’t airtight. It never was. But with the right mathematics, it can be adaptive. And in this economy, that's the only kind of truth worth betting on.

\begin{figure}[H]
    \centering
  
    % === First row ===
    \begin{subfigure}[t]{0.45\textwidth}
    \centering
    \begin{tikzpicture}
      \comicpanel{0}{0}
        {Gödel}
        {}
        {\footnotesize I believe in universal truth, even if unreachable.}
        {(0,-0.6)}
    \end{tikzpicture}
    \caption*{Gödel puts his faith in incompleteness.}
    \end{subfigure}
    \hfill
    \begin{subfigure}[t]{0.45\textwidth}
    \centering
    \begin{tikzpicture}
      \comicpanel{0}{0}
        {Pontryagin}
        {}
        {\footnotesize I believe in differential equations. If I can’t prove it, I can still control it.}
        {(0,-0.6)}
    \end{tikzpicture}
    \caption*{Pontryagin puts his faith in dynamics.}
    \end{subfigure}
  
    \vspace{1em}
  
    % === Second row ===
    \begin{subfigure}[t]{0.45\textwidth}
    \centering
    \begin{tikzpicture}
      \comicpanel{0}{0}
        {Gödel}
        {}
        {\footnotesize Your models will never see the whole. There will always be a blind spot.}
        {(0,-0.6)}
    \end{tikzpicture}
    \caption*{Godel believed in a higher order -- not because math proved it -- but because math couldn’t.}
    \end{subfigure}
    \hfill
    \begin{subfigure}[t]{0.45\textwidth}
    \centering
    \begin{tikzpicture}
      \comicpanel{0}{0}
        {Pontryagin}
        {}
        {\footnotesize Blind spots? That’s not a bug. That’s where the future hides.}
        {(0,-0.6)}
    \end{tikzpicture}
    \caption*{Pontryagin probably believed Marx's end of history is just a local minimum.}
    \end{subfigure}
  
    \caption{Gödel believed truth required faith. Pontryagin believed truth required calculus. Finance believed both... and hedged accordingly.}
  \end{figure}
  


\subsection{From Measure Theory to Market Moves; And, Why You Should Call Me}

Now, here’s the part where things come full circle. I’m going to show you exactly how we use everything we've learned to solve problems in high-frequency trading. And before you think this is just an abstract exercise in mathematical rigor: this is no coincidence.

\begin{quote}
This white paper? It’s just \textbf{fancy marketing material}. 
\end{quote}

I’m doing what DeepSeek did: reading the papers, learning the math, and repurposing it—not to write proofs, but to write code. To build systems. To make decisions in environments where every assumption leaks. Because in this economy, theory is only as good as what it lets you optimize under fire.

So \textbf{if you need some consultation for your machine learning project, come holla at me.} Whether it’s financial models, sales optimization, marketing analytics, or scientific research, I’ve got you covered. The same principles that power high-frequency trading also apply to understanding consumer behavior, optimizing business strategies, and solving complex scientific problems.

\begin{quote}
Because in this game, \textbf{the edge doesn’t go to the ones who just read the papers: it goes to the ones who actually apply them.}
\end{quote}
%\section{High-Frequency Trading: Where Math Moves Faster Than Money}

\subsection{Beyond Timestamps: Causal Inference in the Relativistic World of HFT}

High-frequency trading (HFT) is a battlefield of speed, precision, and mathematical sophistication. A single trading machine can execute thousands of trades per second, but an HFT firm operates at an even greater scale—deploying \textbf{thousands or even tens of thousands of machines}, all responding to market changes in milliseconds. 

At this scale, fundamental problems emerge: 

\begin{itemize}
    \item \textbf{How do we determine which trades influenced which?}
    \item \textbf{How do we filter meaningful signals from noise?}
    \item  \textbf{And how do we avoid trading on illusions rather than real market structure?} 
\end{itemize}


Traditional financial models rely on timestamps to reconstruct trade sequences, but this approach fails under HFT conditions for three major reasons:

\begin{itemize}
    \item There is no single global clock—trades occur asynchronously.
    \item Network latency distorts timestamps, making event ordering unreliable.
    \item Market activity is distributed across multiple exchanges and thousands of independent trading machines.
\end{itemize}

If we naively assume that timestamps determine order, we run into paradoxes where a trade appears to happen "before" its own cause—simply due to differences in latency.

To solve this, HFT systems use \textbf{vector clocks}—a tool from distributed computing that, much like \textbf{Einstein’s worldlines in spacetime}, tracks causality rather than absolute time. Instead of treating trades as stationary points in time, we view them as moving along their own trajectories, each with its own velocity in the "market spacetime."

\subsection{Vector Clocks: Tracking the Flow of Market Influence in Spacetime}

In relativity, objects do not merely exist at fixed positions in space—they follow \textbf{worldlines} through spacetime. Their movement through time is governed by the \textbf{four-velocity}, which describes their motion relative to an observer. The four-velocity of an object with proper time \( \tau \) is given by:

\[
U^\mu = \frac{dx^\mu}{d\tau}
\]

where \( x^\mu \) is the four-position in spacetime, and \( \tau \) is the proper time of the object.

Much like moving objects in spacetime, trades in an HFT system are not stationary—they move through market time, affected by delays, dependencies, and network latencies. Two trades that appear simultaneous in one exchange may not be so in another. Just as different observers in relativity disagree on the timing of events but still agree on causal structure, vector clocks ensure that market events respect causality.

\subsection{Vector Clocks as Market Four-Velocities}

If a trade \( A \) influences trade \( B \), then \( A \) should appear before \( B \) in our system, even if the raw timestamps suggest otherwise. The vector clock ensures this by recording a \textbf{dependency structure}, rather than relying on a single flawed timeline.

Mathematically, for each trade \( T_i \), we maintain a vector \( V(T_i) \) where each element counts how many events have occurred in a given market stream. When one trade influences another, its vector is updated accordingly:

\[
V(T_j) = \max(V(T_i), V(T_j)) + 1
\]

This is analogous to how an object’s four-velocity changes as it moves through spacetime—it accounts for \textbf{relative movement} rather than absolute positioning. Just as trains moving at different speeds separate over time in different frames of reference, trades move apart in "market spacetime" due to latency and execution time differences.

\subsection{Financial Relativity: The Moving Trades}

In relativistic mechanics, simultaneity is relative. If two trains are moving at different speeds, an observer on one train sees events in a different order than an observer on the other. Likewise, in HFT, each trading venue has its own clock, and no single timeline can capture the true order of trades.

Vector clocks, like four-velocity, track the \textbf{relative motion} of trades, ensuring that causality is preserved even when timestamps seem contradictory. Instead of assuming a fixed "Newtonian" market clock, we recognize that every trade has its own motion through time—just as every object in relativity has its own frame of reference.

\newpage


\begin{center}
\begin{tikzpicture}[scale=1.2]

    % Define x-coordinates for worldlines
    \newcommand{\Xone}{0}
    \newcommand{\Xtwo}{3}
    \newcommand{\Xthree}{6}
    \newcommand{\Xexchange}{9}

    % Define total height
    \newcommand{\Height}{10}

    % Machine labels at the top
    \node[above] at (\Xone,\Height) {\small Machine 1};
    \node[above] at (\Xtwo,\Height) {\small Machine 2};
    \node[above] at (\Xthree,\Height) {\small Machine 3)};
    \node[above] at (\Xexchange,\Height) {\small Exchange A};

    % Draw vertical worldlines
    \draw[dotted, ->] (\Xone,\Height) -- (\Xone,0);
    \draw[dotted, ->] (\Xtwo,\Height) -- (\Xtwo,0);
    \draw[dotted, ->] (\Xthree,\Height) -- (\Xthree,0);
    \draw[dotted, ->] (\Xexchange,\Height) -- (\Xexchange,0);

    % Define tick mark spacing manually
    \newcommand{\TickSpacing}{0.5}

    % Define 20 tick marks starting from the **top** and going down
    \newcommand{\TickZero}{\Height}  % Top tick
    \newcommand{\TickOne}{\Height - 0.5}
    \newcommand{\TickTwo}{\Height - 1.0}
    \newcommand{\TickThree}{\Height - 1.5}
    \newcommand{\TickFour}{\Height - 2.0}
    \newcommand{\TickFive}{\Height - 2.5}
    \newcommand{\TickSix}{\Height - 3.0}
    \newcommand{\TickSeven}{\Height - 3.5}
    \newcommand{\TickEight}{\Height - 4.0}
    \newcommand{\TickNine}{\Height - 4.5}
    \newcommand{\TickTen}{\Height - 5.0}
    \newcommand{\TickEleven}{\Height - 5.5}
    \newcommand{\TickTwelve}{\Height - 6.0}
    \newcommand{\TickThirteen}{\Height - 6.5}
    \newcommand{\TickFourteen}{\Height - 7.0}
    \newcommand{\TickFifteen}{\Height - 7.5}
    \newcommand{\TickSixteen}{\Height - 8.0}
    \newcommand{\TickSeventeen}{\Height - 8.5}
    \newcommand{\TickEighteen}{\Height - 9.0}
    \newcommand{\TickNineteen}{\Height - 9.5}

    % Add tick marks with named variables
    \foreach \x in {\Xone,\Xtwo,\Xthree,\Xexchange} {
        \foreach \tick in {\TickZero,\TickOne,\TickTwo,\TickThree,\TickFour,\TickFive,
                           \TickSix,\TickSeven,\TickEight,\TickNine,\TickTen,\TickEleven,
                           \TickTwelve,\TickThirteen,\TickFourteen,\TickFifteen,
                           \TickSixteen,\TickSeventeen,\TickEighteen,\TickNineteen} {
            \draw[dotted] (\x-0.2,\tick) -- (\x+0.2,\tick);
        }
    }

    % Define trade event positions using named tick variables
    \newcommand{\YA}{\TickTwo}
    \newcommand{\YB}{\TickSix}
    \newcommand{\YC}{\TickTen}
    \newcommand{\YD}{\TickFourteen}
    \newcommand{\YE}{\TickEighteen}
    
    

    % Trade A
    \filldraw[black] (\Xthree,\TickOne) circle (2pt) node[left=4pt, fill=white, inner sep=5pt] {\parbox{2cm}{\raggedleft \tiny Buy REQ\\ \small $[0,0,1]$}};
    \draw[->, >=triangle 45, shorten >=15pt, shorten <=15pt] (\Xthree,\TickOne) -- (\Xexchange,\TickOne); 
   
    \filldraw[black] (\Xexchange,\TickOne) circle (2pt) node[right=4pt, fill=white, inner sep=5pt] {\parbox{2cm}{\small Trade A }}; 
   
    \draw[->, >=triangle 45, shorten >=15pt, shorten <=15pt] (\Xexchange,\TickOne) -- (\Xthree,\TickTen); 
    \filldraw[black] (\Xthree,\TickTen) circle (2pt) node[left=4pt, fill=white, inner sep=5pt] {\parbox{2cm}{\raggedleft \tiny Buy AWK \\ \small $[0,3,1]$}};
    
    \filldraw[black] (\Xthree,\TickEleven) circle (2pt) node[right=4pt, fill=white, inner sep=5pt] {\parbox{2cm}{\tiny Broadcast \\ \small $[0,3,2]$}}; 
    
    \filldraw[black] (\Xone,\TickFourteen) circle (2pt) node[left=4pt, fill=white, inner sep=5pt] {\parbox{2cm}{\raggedleft \tiny Receive\\ \small $[1,3,3]$}};
    \draw[dashed, ->, >=triangle 45, shorten >=15pt, shorten <=15pt] (\Xthree,\TickEleven) -- (\Xone,\TickFourteen); 
    
    \filldraw[black] (\Xtwo,\TickSixteen) circle (2pt) node[right=4pt, fill=white, inner sep=5pt] {\parbox{2cm}{\tiny Receive\\ \small $[0,1,3,0]$}};
    \draw[dashed, ->, >=triangle 45, shorten >=15pt, shorten <=15pt] (\Xthree,\TickEleven) -- (\Xtwo,\TickSixteen); 
   
    
    
    % Trade B
    \filldraw[black] (\Xtwo,\TickOne) circle (2pt) node[left=4pt, fill=white, inner sep=5pt] {\parbox{2cm}{\raggedleft \tiny Buy REQ\\ \small $[0,1,0]$}};
    \draw[->, >=triangle 45, shorten >=15pt, shorten <=15pt] (\Xtwo,\TickOne) -- (\Xexchange,\TickThree); 
   
    \filldraw[black] (\Xexchange,\TickThree) circle (2pt) node[right=4pt, fill=white, inner sep=5pt] {\parbox{2cm}{\small Trade B }}; 
    
    \draw[->, >=triangle 45, shorten >=15pt, shorten <=15pt] (\Xexchange,\TickThree) -- (\Xtwo,\TickFour); 
    \filldraw[black] (\Xtwo,\TickFour) circle (2pt) node[left=4pt, fill=white, inner sep=5pt] {\parbox{2cm}{\raggedleft \tiny Buy AWK \\ \small $[0,2,0]$}};

    
     \filldraw[black] (\Xtwo,\TickFive) circle (2pt) node[anchor=north west,  fill=white, inner sep=5pt, xshift=6pt, yshift=-6pt] {\parbox{2cm}{\tiny Broadcast \\ \small $[0,3,0]$}}; 
    
    \filldraw[black] (\Xone,\TickNine) circle (2pt) node[left=4pt, fill=white, inner sep=5pt] {\parbox{2cm}{\raggedleft \tiny Receive\\ \small $[2,3,0]$}};
    \draw[dashed, ->, >=triangle 45, shorten >=15pt, shorten <=15pt] (\Xtwo,\TickFive) -- (\Xone,\TickNine); 
    
    \filldraw[black] (\Xthree,\TickFive) circle (2pt) node[right=4pt, fill=white, inner sep=5pt] {\parbox{1cm}{\tiny Receive\\ \small $[0,1,3,0]$}};
    \draw[dashed, ->, >=triangle 45, shorten >=15pt, shorten <=15pt] (\Xtwo,\TickFive) -- (\Xthree,\TickFive); 
    
    
    
    % Trade C
     \filldraw[black] (\Xone,\TickOne) circle (2pt) node[left=4pt, fill=white, inner sep=5pt] {\parbox{2cm}{\raggedleft \tiny Buy REQ\\ \small $[1,0,0]$}};
    \draw[->, >=triangle 45, shorten >=15pt, shorten <=15pt] (\Xone,\TickOne) -- (\Xexchange,\TickFive); 
   
    \filldraw[black] (\Xexchange,\TickFive) circle (2pt) node[right=4pt, fill=white, inner sep=5pt] {\parbox{2cm}{\small Trade C}}; 
    
    \draw[->, >=triangle 45, shorten >=15pt, shorten <=15pt] (\Xexchange,\TickFive) -- (\Xone,\TickEleven); 
    \filldraw[black] (\Xone,\TickEleven) circle (2pt) node[left=4pt, fill=white, inner sep=5pt] {\parbox{2cm}{\raggedleft \tiny Buy AWK \\ \small $[0,0,2,0]$}};
    
     \filldraw[black] (\Xone,\TickTwelve) circle (2pt) node[right=4pt, fill=white, inner sep=5pt] {\parbox{2cm}{\tiny Broadcast \\ \small $[0,0,3,0]$}}; 
    
    \filldraw[black] (\Xtwo,\TickEighteen) circle (2pt) node[right=4pt, fill=white, inner sep=5pt] {\parbox{2cm}{\tiny Receive\\ \small $[1,0,3,0]$}};
    \draw[dashed, ->, >=triangle 45, shorten >=15pt, shorten <=15pt] (\Xone,\TickTwelve) -- (\Xtwo,\TickEighteen); 
    
    \filldraw[black] (\Xthree,\TickSeventeen) circle (2pt) node[right=4pt, fill=white, inner sep=5pt] {\parbox{2cm}{\tiny Receive\\ \small $[1,0,3,0]$}};
    \draw[dashed, ->, >=triangle 45, shorten >=15pt, shorten <=15pt] (\Xone,\TickTwelve) -- (\Xthree,\TickSeventeen); 
    
   % \filldraw[black] (\Xtwo,\TickSixteen) circle (2pt) node[left=4pt, fill=white, inner sep=5pt] {\parbox{2cm}{\raggedleft \tiny Receive\\ \small $[0,1,3,0]$}};
    %\draw[dashed, ->, >=triangle 45, shorten >=15pt, shorten <=15pt] (\Xthree,\TickEleven) -- (\Xtwo,\TickSixteen); 

     

\end{tikzpicture}
\end{center}




\subsection{Mathematical Background: Modeling Causal Direction with the von Mises--Fisher Distribution}

To ground our Keplerian analogy of causal directionality in mathematical tools, we introduce a formal model for representing and analyzing directional influence in distributed systems: the \\textbf{von Mises--Fisher (vMF) distribution}. This section outlines how vector clocks, inclusion maps, and vMF can be unified into a structured mathematical framework.

\subsubsection{Causal Geometry in Distributed Systems}
In a distributed high-frequency trading system:
\begin{itemize}
  \item \textbf{Vector clocks} assign partial temporal orderings to events, capturing their causal timestamps across machines.
  \item \textbf{Inclusion maps} (e.g., $e_1 \subseteq e_2$) encode structural containment: which events include or causally depend on others.
  \item \textbf{Each event} can be embedded as a high-dimensional vector, representing its direction of influence in a causality space.
\end{itemize}

These vectors describe \emph{direction}, not magnitude. We want to compare the \emph{angular similarity} between events: how aligned they are in their propagation across the system.

\subsubsection{The von Mises--Fisher Distribution}
The vMF distribution is the natural analog of the Gaussian for directional data on the unit hypersphere:
\[
  f(x \mid \mu, \kappa) = C_p(\kappa) \exp(\kappa \mu^T x)
\]
Where:
\begin{itemize}
  \item $x \in \mathbb{R}^p$, $\|x\| = 1$ is a data point on the unit hypersphere,
  \item $\mu$ is the mean direction (also a unit vector),
  \item $\kappa$ is the concentration parameter (higher = tighter cluster),
  \item $C_p(\kappa)$ is a normalization constant depending on dimension $p$.
\end{itemize}

\subsubsection{Causal Vector Construction}
To model causality:
\begin{itemize}
  \item Let $VC_i \in \mathbb{R}^N$ be the vector clock for event $i$.
  \item Compute causal gradients: $x_i = VC_i - VC_{\text{parent}(i)}$, where the parent is determined via inclusion maps.
  \item Normalize each $x_i$ to unit length: $x_i := \frac{x_i}{\|x_i\|}$.
\end{itemize}

These unit vectors represent the \textbf{direction of causal propagation}. The collection $\{x_i\}$ can be modeled using a vMF distribution.

\subsubsection{Why vMF?}
This distribution is ideal because:
\begin{itemize}
  \item It respects \textbf{rotational symmetry} and lives on the hypersphere.
  \item It captures \textbf{concentration of directionality} around a mean.
  \item It supports \textbf{anomaly detection}: vectors with low vMF likelihood may represent outliers or novel causal regimes.
\end{itemize}

\subsubsection{Application to Causal Dynamics}
Once the vMF is fit to the set of causal direction vectors $\{x_i\}$:
\begin{itemize}
  \item \textbf{Clusters} of aligned events emerge as concentrated regions on the sphere.
  \item \textbf{Divergence} from the mean direction signals shifts in the causal geometry (e.g., regime changes).
  \item \textbf{Directional flow} becomes quantifiable, interpretable, and comparable over time.
\end{itemize}

\subsubsection{Connecting Back to Kepler}
Just as Kepler’s laws interpreted planetary orbits via geometry and conserved quantities, this approach models causal motion via \emph{directional vectors} and \emph{angular distributions}. The hypersphere becomes our celestial map — and vMF becomes our algebraic telescope.

\subsubsection{Summary Table}
\begin{center}
\begin{tabular}{ll}
\toprule
\textbf{Concept} & \textbf{ML/Causal Modeling Equivalent} \\
\midrule
Vector clock deltas & Direction vectors in $\mathbb{R}^d$ \\
Inclusion maps & Structural encodings of causality \\
Direction of influence & Unit vectors on the hypersphere \\
Modeling orientation & von Mises--Fisher distribution \\
Goal & Cluster, track, or detect changes in causality direction \\
\bottomrule
\end{tabular}
\end{center}




\subsection{Keplerian Causality: An Analogy for Directional Data in Distributed Systems}

Modeling directional data in high-frequency trading systems using tools like vector clocks, inclusion maps, and the von Mises--Fisher (vMF) distribution can feel abstract and ungrounded. But there is a powerful analogy that can help make it intuitive: the celestial geometry of Kepler’s laws.

\subsubsection{The Setting: Causal Orbits in Event Space}

Imagine every event in a distributed trading system as a planet orbiting a central force. The "sun" is not a gravitational body but a shared source of causality: a market signal, a time sync, or a liquidity impulse. Events propagate outward, influencing other machines. Vector clocks give us the temporal location of these events; inclusion maps trace their structural influence.

\subsubsection{Kepler’s First Law: Causal Ellipses}

Kepler said: planets move in ellipses with the sun at one focus. In our analogy, events don’t always follow uniform or symmetric causal paths. Some propagate tightly and quickly (like perihelion), others meander and delay (like aphelion). Inclusion maps define these "orbits" — how far a causal effect extends through the network. Vector clocks help describe their shape.

\subsubsection{Kepler’s Second Law: Equal Causal Areas in Equal Time}

Kepler's second law is about conservation of angular momentum. Planets sweep equal areas in equal time. In the trading system, the rate at which influence propagates may vary, but there's a conserved quantity: the consistency of directional flow. Events propagate faster when closer to the causal core (tight consensus, fast messaging) and slower when farther (latency, fragmentation). 

The directional vectors of causality — gradients of vector clock changes — can be modeled as unit vectors on the hypersphere. The von Mises--Fisher distribution lets us statistically model the orientation of this flow, just as Kepler used geometry to understand motion.

\subsubsection{Kepler’s Third Law: Causal Radius and Influence Duration}

Kepler related the orbital period of a planet to its distance from the sun. Analogously, the "radius" of an event — how many machines it touches, how long its causal shadow persists — is related to its time footprint. High-impact trades have wide, slow orbits. Smaller trades stay tight. This structure defines the temporal geometry of causality.

\subsubsection{Directional Modeling as Angular Momentum}

In Kepler’s world, angular momentum gives you the direction and speed of rotation. In your system, vector clock deltas provide a direction of causal change. By normalizing these vectors and modeling them with the vMF distribution, we focus not on how long an influence is, but \emph{where} it points. 

\begin{quote}
Just as Kepler studied angular velocity and conserved areas, we study causal directionality and conserved influence. The hypersphere becomes our celestial map.
\end{quote}

\subsubsection{Summary Table: From Celestial Mechanics to Distributed Causality}

\begin{center}
\begin{tabular}{ll}
\textbf{Kepler Concept} & \textbf{Causal System Analogy} \\
\toprule
Orbit (ellipse) & Inclusion map: causal envelope of an event \\
Equal areas in equal time & Conserved rate of influence (angular flow) \\
Period-radius relation & Causal span vs. temporal footprint \\
Angular velocity & Vector clock delta (direction of change) \\
Angular momentum $L$ & vMF-modeled directional flow on hypersphere \\
\bottomrule
\end{tabular}
\end{center}

\subsubsection{Conclusion}

The causal geometry of your distributed system echoes the celestial mechanics of Kepler. Instead of gravity, you have causality. Instead of orbits, you have inclusion maps. And instead of planetary motion, you model directional flow with vector clocks and the von Mises--Fisher distribution. What Kepler saw in the heavens, we now trace in the microsecond paths of trading logic.

%\section{Avoiding Spurious Correlation: Using Inclusion Maps}

\subsection{The Problem of Timing: When Did What Happen?}

\vspace{0.5em}
In high-frequency trading, the difference between acting on a signal and reacting to one can come down to microseconds. But in a globally distributed system with machines operating asynchronously, that’s not an easy distinction to make. Two events might appear causally related simply because one is observed before the other. However, was it truly earlier, or just received sooner due to network latency?

This creates a fundamental challenge: \textbf{how can we determine whether one trade caused a market reaction, or merely appeared to?}

To answer that, we need more than statistical correlation. We need a structure that captures the \textbf{temporal flow of information}: one that respects both event order and system delays.

That’s where vector clocks come in.


\begin{figure}[H]
\centering
\begin{tikzpicture}[every node/.style={font=\footnotesize}]

% Panel 1 — Trader 1 thinks they caused a reaction
\comicpanel{0}{4}
  {Trader A}
  {Trader B}
  {That price spike right after my trade? You're welcome.}
  {(0,-0.6)}

% Panel 2 — Trader B is skeptical
\comicpanel{6.5}{4}
  {Trader A}
  {Trader B}
  {Pretty sure the spike hit our system before your trade even left the building.}
  {(0,-0.5)}

% Panel 3 — Trader A confused
\comicpanel{0}{0}
  {Trader A}
  {Trader B}
  {But... my logs say I traded first. Time is real, right?}
  {(0,0.5)}

% Panel 4 — Trader B drops the clock truth
\comicpanel{6.5}{0}
  {Trader A}
  {Trader B}
  {In distributed systems, time is just a polite suggestion. Use vector clocks.}
  {(0,0.6)}

\end{tikzpicture}
\caption{In high-frequency trading, causality is a microsecond mirage — unless you bring a vector clock.}
\end{figure}



\subsection{Vector Clocks: Capturing Temporal Causality in Trading}

\vspace{0.5em}
\noindent
Imagine a group of traders chatting in a messaging app, but everyone is on slightly different Wi-Fi. Each time a message (or trade) is sent, it’s timestamped locally. Now suppose one trader tries to argue that their message caused a market reaction; but another trader can show, based on message arrival times, that their system had already reacted before the first message even appeared. Clearly, the timeline doesn’t add up.

This kind of ambiguity is common in high-frequency trading, where machines operate asynchronously and respond to signals within microseconds. Determining what actually caused a price movement — and what merely reacted to one — becomes a serious challenge.

\textbf{Vector clocks} solve this problem by giving each participant not just their own local timestamp, but a full record of what they know about everyone else’s activity. Over time, this builds a consistent picture of event order across a distributed system.

\vspace{1em}
Formally, a vector clock assigns each trading machine a vector of timestamps \( V_i \), where each entry records the latest known activity of every other machine. Whenever machines exchange information (e.g., through trades or shared order book data), they update their clocks:

\[
V_i[j] = \max(V_i[j], V_j[j]) + 1.
\]

This update rule ensures that the knowledge each machine has about the rest of the system increases monotonically. If a trade at machine \( A \) happens before a price change at machine \( B \), then the causality constraint is:

\[
V_A < V_B.
\]

In other words, all components of \( V_A \) must be less than or equal to the corresponding components in \( V_B \), and at least one must be strictly less. This guarantees that no event is ever mistakenly attributed to something that happened \textbf{after} it.

Vector clocks provide a minimal, logical structure for enforcing temporal causality: a critical foundation for any model that hopes to distinguish meaningful signals from reactive noise in modern financial systems.

\vspace{1em}
\noindent
This causal structure naturally induces a \textbf{partially ordered set (poset)}. Each event — whether a trade, a message, or a price update — is an element in the set. The ordering relation is defined by the vector clock comparison: event \( e_1 \) precedes event \( e_2 \) if and only if \( V_{e_1} < V_{e_2} \). 

Since some events are concurrent (neither happened before the other), the system doesn’t form a total order — and that’s a feature, not a bug. It reflects the reality of distributed trading: not all events are causally connected, but the ones that are can be meaningfully arranged.

This poset representation is foundational in distributed systems theory, and in financial modeling, it gives us a principled way to reconstruct causal chains from asynchronous activity. When combined with inclusion maps and mutual information, these structures help isolate real causal pathways from reactive noise — sharpening the boundary between signal and illusion.

\begin{figure}[H]
\centering
\begin{tikzpicture}[
    node distance=1.2cm and 1.8cm,
    event/.style={circle, draw=black, fill=blue!10, thick, minimum size=1.2cm},
    arrow/.style={-Stealth, thick}
]

% Events
\node[event] (e1) at (0,0) {\( e_1 \)};
\node[event, above left=of e1] (e2) {\( e_2 \)};
\node[event, above right=of e1] (e3) {\( e_3 \)};
\node[event, above=of e2] (e4) {\( e_4 \)};
\node[event, above=of e3] (e5) {\( e_5 \)};

% Arrows (partial order)
\draw[arrow] (e1) -- (e2);
\draw[arrow] (e1) -- (e3);
\draw[arrow] (e2) -- (e4);
\draw[arrow] (e3) -- (e5);

% Optional: show concurrency
% No edge between e4 and e5, indicating they are concurrent

\end{tikzpicture}
\caption{A Hasse diagram of a poset induced by vector clock comparisons. Events \( e_4 \) and \( e_5 \) are concurrent — neither one causally precedes the other.}
\end{figure}

\vspace{1em}
\noindent
In short, vector clocks give us a way to tell who really hit the “go” button first — even if the network logs, exchange timestamps, and trader egos all insist otherwise. Because nothing says “mathematically defensible causality” like telling a billion-dollar trading desk their price spike came second.


\begin{figure}[H]
\centering
\begin{tikzpicture}[every node/.style={font=\footnotesize}]

% Panel 1 — Trader thinks they triggered the reaction
\comicpanel{0}{4}
  {Trader A}
  {Trader B}
  {My trade message caused that price move. Check the timestamp!}
  {(-0.2,-0.6)}

% Panel 2 — Trader B calls it out
\comicpanel{6.5}{4}
  {Trader A}
  {Trader B}
  {Funny, our system reacted before your message even arrived. Wi-Fi strikes again.}
  {(0,-0.5)}

% Panel 3 — Confused trader
\comicpanel{0}{0}
  {Trader A}
  {Trader C}
  {Wait... so whose timeline is the right one?}
  {(0,0.5)}

% Panel 4 — Trader C drops the solution
\comicpanel{6.5}{0}
  {Trader A}
  {Trader C}
  {None of them. Use a vector clock. It's like group chat with causality.}
  {(0,0.6)}

\end{tikzpicture}
\caption{In distributed trading systems, time is subjective — vector clocks give it structure.}
\end{figure}



\subsection{Defining Inclusion Maps in Financial Markets}

\vspace{0.5em}
\noindent
Now, imagine you’re analyzing a local high-frequency trading firm that operates on a regional exchange. This regional exchange has its own set of assets, trades, and internal microstructure — we’ll call this system \( X \). However, now you want to understand how this localized activity fits into the broader global market, where institutional players, international flows, and macroeconomic forces all intersect : this larger system is \( Y \).

\begin{figure}[H]
\centering
\begin{tikzpicture}[scale=1.2, every node/.style={font=\small}]
    % Global market Y
    \draw[thick, rounded corners=8pt] (0,0) rectangle (10,5);
    \node at (8.5,4.7) {\textbf{Global Market \(\boldsymbol{Y}\)}};

    % Enlarged Regional exchange X
    \draw[thick, fill=blue!10, rounded corners=6pt] (2,1.8) rectangle (7,4.4);
    \node at (4.5,3.1) {\textbf{Regional Exchange \(\boldsymbol{X}\)}};

    % Assets in X
    \node at (2.8,4.0) {\textit{Local Assets}};
    \node at (3.0,2.5) {\textit{Microstructure}};
    \node at (6.0,2.2) {\textit{Trade Data}};

    % Assets outside X (in Y)
    \node at (1.4,1.3) {\textit{Institutional Flows}};
    \node at (1.05,3.8) {\textit{Derivatives}};
    \node at (8.25,1.25) {\textit{Macroeconomic Events}};

    % Arrow to show embedding
    \draw[->, thick] (7.1,3.6) to[out=15,in=165] (9.2,3.5);
    \node at (8,3.9) {\(\phi: X \hookrightarrow Y\)};

\end{tikzpicture}
\caption{The regional exchange \( X \) is embedded within the global market \( Y \), preserving structure and measure through an inclusion map.}
\end{figure}



Crucially, when incorporating the regional data into the global market model, you don’t want to distort anything. A \$10 million trade on the local exchange should still count as a \$10 million trade in the global context. The events are not transformed or rescaled; they are simply recognized as part of a larger universe. This process is what an \textbf{inclusion map} formalizes: it preserves the structure and significance (i.e., the measure) of local information while embedding it into a wider context.

\begin{figure}[H]
\centering
\begin{tikzpicture}[scale=1.2, every node/.style={font=\small}]

    % Global Market Y
    \draw[thick, rounded corners=8pt] (0,0) rectangle (10,5);
    \node at (9.4,4.6) {\textbf{Global Market \(\boldsymbol{Y}\)}};

    % Regional Exchange X (larger inner rectangle)
    \draw[thick, fill=blue!10, rounded corners=6pt] (2,1.8) rectangle (7,4.4);
    \node at (4.5,3.1) {\textbf{Regional Exchange \(\boldsymbol{X}\)}};

    % Example Trade inside X
    \draw[fill=green!20, thick] (3.5,3.5) rectangle (5.5,4.0);
    \node at (4.5,3.75) {\$10M Trade};

    % Trade copy inside Y (mirrored)
    \draw[fill=green!20, thick] (7.8,2.0) rectangle (9.8,2.5);
    \node at (8.8,2.25) {\$10M Trade};

    % Arrow representing inclusion
    \draw[->, thick] (5.5,3.75) to[out=20,in=160] (7.8,2.25);
    \node[above] at (6.2,3.8) {\(\phi: X \hookrightarrow Y\)};

    % Annotation
    \node[align=center] at (5,1.2) {
        \textit{Inclusion preserves structure and value:} \\
        \textit{the trade is embedded without distortion.}
    };

\end{tikzpicture}
\caption{An inclusion map embeds local data into a global system without distortion: a \$10 million trade in the regional market remains a \$10 million trade in the global model.}
\end{figure}


\vspace{1em}
Formally, given two measure spaces \( (X, \mathcal{F}_X, \mu_X) \) and \( (Y, \mathcal{F}_Y, \mu_Y) \), an \textbf{inclusion map} is defined as:

\[
\phi: X \hookrightarrow Y, \quad \phi(x) = x, \quad \forall x \in X.
\]

This simply means that each element of \( X \) is mapped to itself in \( Y \), preserving identity. The key property is that the measure of any subset \( A \subset X \) remains unchanged:

\[
\mu_X(A) = \mu_Y(\phi(A)).
\]


\begin{figure}[H]
\centering
\begin{tikzpicture}[scale=1.2, every node/.style={font=\small}]

    % Global Market Y
    \draw[thick, rounded corners=8pt] (0,0) rectangle (10,5);
    \node at (9.4,4.6) {\textbf{Global Space \( Y \)}};

    % Regional Exchange X
    \draw[thick, fill=blue!10, rounded corners=6pt] (2,1.8) rectangle (7,4.4);
    \node at (4.5,4.2) {\textbf{Subset Space \( X \subset Y \)}};

    % Subset A inside X
    \draw[fill=orange!30, thick] (3,2.5) circle (0.6);
    \node at (3,2.5) {\( A \)};

    % Corresponding subset φ(A) in Y
    \draw[fill=orange!30, thick] (8,2.0) circle (0.6);
    \node at (8,2.0) {\( \phi(A) \)};

    % Arrow showing inclusion map
    \draw[->, thick] (3.6,2.6) to[out=15,in=165] (7.4,2.1);
    \node[above] at (5.7,3.1) {\(\phi: X \hookrightarrow Y\)};

    % Measure equivalence annotation
    \node at (5,1.2) {\(\mu_X(A) = \mu_Y(\phi(A))\)};

\end{tikzpicture}
\caption{An inclusion map preserves both identity and measure: a subset \( A \subset X \) retains its importance when mapped into the larger space \( Y \).}
\end{figure}

In other words, the importance (or probability, or volume) assigned to events in the smaller space \( X \) remains intact when viewed from the larger space \( Y \). For financial modeling, this allows us to combine local and global perspectives without distorting the quantitative meaning of the data.

\vspace{1em}
\noindent
To bridge the abstract concept of inclusion maps with a concrete setting, we model both the regional exchange \( X \) and the global market \( Y \) as partially ordered sets. Events in \( X \) — such as trades or price updates — are elements of a poset reflecting their causal relationships. When these events are embedded into \( Y \), they preserve their structure, but \( Y \) may contain additional events and interactions not visible within the scope of \( X \).

This Hasse diagram shows how a subset of global market events (in blue) corresponds to the local system \( X \), while additional global events (in green) reflect broader market dynamics. The inclusion map \( \phi: X \hookrightarrow Y \) embeds each event \( x_i \in X \) into its global counterpart \( y_i = \phi(x_i) \in Y \), maintaining causal structure without distorting the relationships or values.

\begin{figure}[H]
\centering
\begin{tikzpicture}[
    xevent/.style={circle, draw=black, fill=blue!10, thick, minimum size=1.2cm},
    yevent/.style={circle, draw=black, fill=green!10, thick, minimum size=1.2cm},
    arrow/.style={-Stealth, thick},
    dashedarrow/.style={->, thick, dashed, gray},
    node distance=1.5cm and 2cm,
    every node/.style={font=\small}
]

% Events in X (subset)
\node[xevent] (x1) at (0,0) {\( x_1 \)};
\node[xevent, above left=of x1] (x2) {\( x_2 \)};
\node[xevent, above=of x2] (x4) {\( x_4 \)};

% Arrows in X
\draw[arrow] (x1) -- (x2);
\draw[arrow] (x2) -- (x4);

% Events in Y (superset)
\node[yevent, right=5cm of x1] (y1) {\( y_1 \)};
\node[yevent, above left=of y1] (y2) {\( y_2 \)};
\node[yevent, above=of y2] (y4) {\( y_4 \)};
\node[yevent, above right=of y1, xshift=0.4cm] (y3) {\( y_3 \)};
\node[yevent, above=of y3] (y5) {\( y_5 \)};

% Arrows in Y
\draw[arrow] (y1) -- (y2);
\draw[arrow] (y2) -- (y4);
\draw[arrow] (y1) -- (y3);
\draw[arrow] (y3) -- (y5);

% Inclusion map arrows
\draw[dashedarrow] (x1) -- (y1);
\draw[dashedarrow] (x2) -- (y2);
\draw[dashedarrow] (x4) -- (y4);

% Labels
\node at (-1.5,3.5) {\textbf{Regional Exchange \( X \)}};
\node at (6.5,4.5) {\textbf{Global Market \( Y \)}};
\node at (0.7,2.5) {\( \phi: X \hookrightarrow Y \)};

\end{tikzpicture}
\caption{A subset of causally ordered events in the regional exchange \( X \) is embedded into the global market \( Y \) via the inclusion map \( \phi \). Events \( y_3 \) and \( y_5 \) are present in \( Y \) but not in \( X \), showing that \( X \subset Y \) captures only part of the full system.}
\end{figure}

\vspace{1em}
\noindent
So the next time someone insists their regional trading model "totally captures the global dynamics," you can smile, point to your inclusion map, and gently remind them: “Yes, your trades exist — but only as a proper subset.” Because in finance, as in mathematics, knowing where you fit in the larger $\sigma$-algebra is half the battle.



\begin{figure}[H]
\centering
\begin{tikzpicture}[every node/.style={font=\footnotesize}]

% Panel 1 — Analyst 1 confused
\comicpanel{0}{4}
  {Analyst 1}
  {Analyst 2}
  {So if we move a \$10M trade from the regional market to the global model... do we rescale it? Reclassify it? Normalize it?}
  {(0,-0.6)}

% Panel 2 — Analyst 2 clarifies
\comicpanel{6.5}{4}
  {Analyst 1}
  {Analyst 2}
  {Nope. It's still a \$10M trade. We’re not doing financial alchemy here.}
  {(0,-0.5)}

% Panel 3 — Analyst 1 leans in
\comicpanel{0}{0}
  {Analyst 1}
  {Analyst 2}
  {But surely the macro context gives it more... global aura? Maybe a currency glow effect?}
  {(0,0.6)}

% Panel 4 — Analyst 2 delivers punchline
\comicpanel{6.5}{0}
  {Analyst 1}
  {Analyst 2}
  {It’s an inclusion map. Not a magical realism filter.}
  {(0,0.6)}

\end{tikzpicture}
\caption{Inclusion maps: preserving meaning without market mysticism.}
\end{figure}







\subsection{Application to High-Frequency Trading}

\vspace{0.5em}
\noindent
Building on the idea of inclusion maps from the previous section, let’s zoom in on a high-frequency trading scenario. Imagine you’re watching a trading algorithm that repeatedly places rapid-fire trades. Each time it does, you observe a corresponding movement in the market price. It’s tempting to conclude that the trades are causing the price change — this would suggest a direct inclusion: the set of trades is embedded within the set of price changes.

But then you look closer. It turns out both the algorithm and the price movement tend to react to something else entirely ; like the release of government unemployment data or a sudden institutional sell-off. In this case, the trades and price movements are not causally linked to each other but are both effects of a third, hidden cause. If you mistakenly assume a direct inclusion, your model will identify spurious causality and could lead to catastrophic mispricing under stress.

This is where inclusion maps become more than just abstract math: they provide a framework to rigorously distinguish between real causation and mere coincidence in complex systems like financial markets.

\vspace{1em}
We model the entire market as a \textbf{measurable space} \( (\Omega, \mathcal{F}, \mu) \), where:

\begin{itemize}
    \item \( \Omega \) is the universe of all possible micro-events in the market — including trades, price updates, order book shifts, and macroeconomic announcements.
    
    \item \( \mathcal{F} \subset 2^\Omega \) is the \textbf{sigma-algebra} of measurable events: it contains all the collections of market events that are observable and well-defined in terms of market structure (e.g., "all trades within a 10ms window after a news release").
    
    \item \( \mu: \mathcal{F} \rightarrow [0, \infty] \) is a \textbf{measure} that assigns weight to each measurable set — such as trade volume, probability of occurrence, or economic significance. This allows us to compare and reason about the importance or likelihood of different events within the market.

    \item \( S \subset \Omega \) is the set of trades executed by a specific machine or algorithm.

    \item \( T \subset \Omega \) is the set of observed price changes across the market.

    \item \( C \subset \Omega \) is the \textbf{true causal set} — macroeconomic events such as central bank announcements, liquidity shocks, or institutional actions that influence both \( S \) and \( T \).
\end{itemize}

\begin{figure}[H]
\centering
\begin{tikzpicture}[scale=1.2, every node/.style={font=\small}]

    % Universal measurable space Omega
    \draw[thick, rounded corners=8pt] (0,0) rectangle (10,6);
    \node at (9.3,5.7) {\textbf{Event Space \(\Omega\)}};
    \node at (5,5.9) {\textit{Measurable space } \(\boldsymbol{(\Omega, \mathcal{F}, \mu)}\)};
    
    % Label sigma-algebra
    \node at (5,0.3) {\(\mathcal{F}\): \textit{All observable groupings of market events}};
    
    % Label measure
    \node at (5,0.0) {\(\mu\): \textit{Measure of impact, probability, or volume}};

    % Subset C — True causal set
    \draw[fill=yellow!30, draw=orange, thick, rounded corners=6pt] (1,4.2) rectangle (4,5.4);
    \node at (2.5,5.0) {\textbf{Causal Set \(C\)}};
    \node at (2.5,4.6) {\tiny (e.g., macro events)};

    % Subset S — Trades
    \draw[fill=blue!20, draw=blue!60!black, thick, rounded corners=6pt] (2,2.1) rectangle (5,3.4);
    \node at (3.5,3.0) {\textbf{Trades \(S\)}};

    % Subset T — Price Changes
    \draw[fill=red!20, draw=red!60!black, thick, rounded corners=6pt] (6,2.1) rectangle (9,3.4);
    \node at (7.5,3.0) {\textbf{Price Changes \(T\)}};

    % Arrows from C to S and T
    \draw[->, thick] (2.5,4.2) -- (3.5,3.4);
    \draw[->, thick] (3.5,4.2) -- (7.5,3.4);
    \node at (3.1,3.75) {\tiny reacts};
    \node at (5.8,3.75) {\tiny reacts};

    % Dashed arrow for false assumption
    \draw[->, thick, dashed, gray] (5,2.75) -- (6,2.75);
    \node[gray] at (5.5,3.0) {\tiny false causation};

\end{tikzpicture}
\caption{The market is modeled as a measurable space \( (\Omega, \mathcal{F}, \mu) \), where trades \( S \) and price changes \( T \) are observable events. Both may appear related, but are often reactions to a deeper causal set \( C \).}
\end{figure}


If trades genuinely cause price changes, we expect an inclusion map of the form:

\[
\phi: S \hookrightarrow T.
\]

\begin{figure}[H]
\centering
\begin{tikzpicture}[scale=1.2, every node/.style={font=\small}]

    % Universal space Omega
    \draw[thick, rounded corners=8pt] (0,0) rectangle (10,6);
    \node at (9.4,5.7) {\textbf{Event Space \(\Omega\)}};
    \node at (5,5.9) {\textit{Measurable Space } \((\Omega, \mathcal{F}, \mu)\)};

    % Subset T (Price Changes)
    \draw[fill=red!15, draw=red!60!black, thick, rounded corners=6pt] (1.5,1.5) rectangle (8.5,4.5);
    \node at (7.6,4.2) {\textbf{Price Changes \(T\)}};

    % Subset S inside T (Trades)
    \draw[fill=blue!20, draw=blue!60!black, thick, rounded corners=6pt] (3,2.3) rectangle (6.5,3.7);
    \node at (4.8,3.0) {\textbf{Trades \(S\)}};

    % Arrow annotation showing inclusion map
    \draw[->, thick] (6.6,3.6) to[out=20,in=160] (8.2,4.0);
    \node at (7.5,4.5) {\(\phi: S \hookrightarrow T\)};

    % Annotation
    \node at (5,0.7) {\textit{If trades cause price changes, then \( S \subset T \) and inclusion is valid.}};

\end{tikzpicture}
\caption{When trades \( S \) genuinely cause price changes \( T \), both are subsets of the market event space \( \Omega \), and the inclusion map \( \phi: S \hookrightarrow T \) correctly captures the causal structure.}
\end{figure}


This means the measure on trade activity naturally embeds within the measure on price movement. However, if both \( S \) and \( T \) are merely responding to \( C \), then the more accurate inclusion is:

\[
\phi: S \hookrightarrow \Omega \setminus C.
\]


\begin{figure}[H]
\centering
\begin{tikzpicture}[scale=1.2, every node/.style={font=\small}]

    % Full Event Space Omega
    \draw[thick, rounded corners=8pt] (0,0) rectangle (10,6);
    \node at (9.4,5.6) {\textbf{Event Space \(\Omega\)}};
    \node at (5,5.85) {\textit{Measurable Space } \((\Omega, \mathcal{F}, \mu)\)};

    % Subset C — True Causal Set (Should NOT be inside Omega \ C)
    \draw[fill=yellow!30, draw=orange, thick, rounded corners=6pt] (6.5,4.2) rectangle (9,5.3);
    \node at (7.8,5.0) {\textbf{Causal Set \(C\)}};
    \node at (7.8,4.6) {\tiny (macroeconomic triggers)};

    % Highlighted dashed box for Omega \ C (excludes C visually)
    \draw[dashed, thick, red!80!black] (0.4,0.6) rectangle (6.3,5.6);
    \node[red!80!black] at (1.3,5.3) {\(\Omega \setminus C\)};

    % Subset S — Trades (inside Omega \ C)
    \draw[fill=blue!20, draw=blue!60!black, thick, rounded corners=6pt] (1.2,1.6) rectangle (3.8,3.1);
    \node at (2.5,2.4) {\textbf{Trades \(S\)}};

    % Subset T — Price Changes (also inside Omega \ C)
    \draw[fill=red!20, draw=red!60!black, thick, rounded corners=6pt] (4.2,1.6) rectangle (6.0,3.1);
    \node at (5.1,2.4) {\textbf{Price Changes \(T\)}};

    % Inclusion arrow from S to Omega \ C (not pointing into C)
    \draw[->, thick] (3.8,2.8) to[out=20,in=200] (5.6,4.3);
    \node at (4.9,3.7) {\(\phi: S \hookrightarrow \Omega \setminus C\)};

    % Annotation
    \node at (5,0.7) {\textit{S and T are influenced by \(C\), but lie in \(\Omega \setminus C\).}};

\end{tikzpicture}
\caption{When both trades \( S \) and price changes \( T \) are effects of a hidden cause \( C \), the correct inclusion is \( \phi: S \hookrightarrow \Omega \setminus C \), which avoids misinterpreting correlation as causation.}
\end{figure}

\vspace{1em}
\noindent
\textbf{Why is \( T \) inside \( \Omega \setminus C \)?}

\vspace{0.5em}
When we write \( \phi: S \hookrightarrow \Omega \setminus C \), we are stating that the set of trades \( S \) belongs to the portion of the event space that excludes the causal set \( C \). If we believe that price changes \( T \) are also \emph{effects} of the true cause \( C \) — not directly caused by \( S \) — then \( T \) too should be located in \( \Omega \setminus C \). 

This means both \( S \) and \( T \) are valid elements of the same subset space, avoiding the erroneous assumption that one causes the other. However, they are still part of the full market space \( \Omega \). The diagram reflects this correctly.

\vspace{1em}
\begin{center}
\renewcommand{\arraystretch}{1.3}
\begin{tabular}{|c|p{8cm}|}
\hline
\textbf{Set} & \textbf{Meaning and Placement} \\
\hline
\( \Omega \) & The complete market event space; contains everything. Represented by the large outer rectangle. \\
\hline
\( C \subset \Omega \) & The true causal set (e.g., macroeconomic events). Placed inside \( \Omega \), but outside of \( \Omega \setminus C \). \\
\hline
\( \Omega \setminus C \) & The complement of the causal set within \( \Omega \); contains events not caused by \( C \). Represented by the dashed bounding box. \\
\hline
\( S, T \subset \Omega \setminus C \) & The sets of trades and price changes, both outside of \( C \) but still within \( \Omega \). Placed inside the dashed box and outside \( C \). \\
\hline
\end{tabular}
\end{center}

\vspace{1em}
\noindent
\textit{Conclusion:} To claim \( \phi: S \hookrightarrow \Omega \setminus C \), both \( S \) and \( T \) must be subsets of \( \Omega \setminus C \), not of \( C \). The diagram reflects this structure correctly.





This measure-theoretic perspective helps separate signal from noise — and more importantly, causation from correlation — in the rapid-fire, feedback-heavy world of algorithmic trading.


\subsubsection*{Concrete Example: Misinterpreting Causality in Event Ordering}

To make this abstract framework more tangible, consider a simple sequence of events in a high-frequency trading system. A machine places a trade. Shortly after, the price moves. At first glance, this seems like causation: the trade appears to drive the price.

However, suppose that just before both of these events, a macroeconomic indicator was released — triggering both the trade and the price movement. If we model this situation using event orderings, we see that the trade and price change are not causally linked to each other, but both follow from a third event.

The Hasse diagram below shows this structure clearly: the macroeconomic trigger \( c \) precedes both the trade \( s \) and the price update \( t \), but there's no direct causal path from \( s \) to \( t \). Mistaking this configuration as \( s \rightarrow t \) would be a case of false causation.


\begin{figure}[H]
\centering
\begin{tikzpicture}[
    node distance=1.5cm and 2.2cm,
    event/.style={circle, draw=black, fill=blue!10, thick, minimum size=1.2cm},
    arrow/.style={-Stealth, thick},
    every node/.style={font=\small}
]

% Nodes
\node[event] (c) at (0,2.5) {\( c \)};
\node[event, below left=of c] (s) {\( s \)};
\node[event, below right=of c] (t) {\( t \)};

% Arrows from cause to both
\draw[arrow] (c) -- (s);
\draw[arrow] (c) -- (t);

% Label
\node at (0,3.5) {\textbf{Causal Hasse Diagram}};

\end{tikzpicture}
\caption{Macroeconomic event \( c \) causes both the trade \( s \) and price change \( t \), but \( s \) and \( t \) are not causally linked. Misinterpreting this structure leads to false attribution of causality.}
\end{figure}


\vspace{1em}
\noindent
Models that ignore hidden causes often return elegant, confident conclusions — right up until the market reminds us that elegance without causality is just curve-fitting in a tuxedo. In high-frequency trading, where every microsecond invites misinterpretation, inclusion maps and measure-theoretic reasoning don’t just clarify structure — they keep us from hallucinating ghosts in the data.



\begin{figure}[H]
\centering
\begin{tikzpicture}[every node/.style={font=\footnotesize}]

% Panel 1 — Algo trader excited
\comicpanel{0}{4}
  {Algo A}
  {Algo B}
  {Did you see that? I placed a trade and boom — price moved. Clearly I caused it.}
  {(-0.3,-0.6)}

% Panel 2 — Algo B skeptical
\comicpanel{6.5}{4}
  {Algo A}
  {Algo B}
  {Or maybe we both reacted to that unemployment report. Ever think of that?}
  {(-0.2,-0.6)}

% Panel 3 — Algo A tries to justify
\comicpanel{0}{0}
  {Algo A}
  {Algo C}
  {I mean, sure... but the inclusion felt real. It was a very includable moment.}
  {(0,0.5)}

% Panel 4 — Algo C brings the math
\comicpanel{6.5}{0}
  {Algo A}
  {Algo C}
  {Check your inclusion map. You're inside \(\Omega \setminus C\), not inside price movement.}
  {(0,0.6)}

\end{tikzpicture}
\caption{Spurious causality: when your trades look important, but are just echoes of macro events.}
\end{figure}


\subsection{Mutual Information and Inclusion Maps}

\subsubsection{From Correlation to Causation: A Role for Mutual Information}

\vspace{0.5em}
\noindent
Continuing from the previous section, let’s say you've built an inclusion map that seems to link a trading algorithm’s actions to market price movements. You notice a pattern: whenever the algorithm places a flurry of trades, the price moves shortly afterward. At first glance, this looks like a strong signal that the trades might be driving the market.

But you're not convinced. You wonder: is this real causation, or are both just reacting to something else? Maybe a government report just dropped, or maybe a large institutional trader moved the market before your algorithm even fired. You want a tool that quantifies how strong the relationship is, and how much of it disappears once you account for external influences.

That tool is mutual information. It tells you how much knowing one variable (like trade activity) reduces uncertainty about another (like price changes). Even better, by conditioning on a potential confounder — say, macroeconomic news — you can test whether the relationship still holds or falls apart.

\subsubsection{Mutual Information as a Correction to Maximum Entropy}

\vspace{0.5em}
\noindent
We start with \textbf{maximum entropy}: a position of total uncertainty, assuming that trades, prices, and external events are all independent. Mutual information acts as a correction to this ignorance. If observing trades significantly reduces your uncertainty about price movements, you gain mutual information — a signal that a dependency exists. But if conditioning on a shared external cause eliminates that dependency, it reveals the relationship as spurious.

\begin{figure}[H]
\centering
\begin{tikzpicture}[
    node distance=2.2cm and 2.8cm,
    var/.style={circle, draw, thick, minimum size=1.3cm, align=center},
    info/.style={rectangle, draw, thick, rounded corners, fill=gray!10, inner sep=8pt, text width=6cm},
    arrow/.style={->, thick, >=Stealth}
]

% Nodes
\node[var] (trades) {Trades};
\node[var, right=of trades] (price) {Price};
\node[var, above=of $(trades)!0.5!(price)$] (external) {External\\Events};

% Arrows
\draw[arrow] (trades) -- (price);
\draw[arrow, dashed] (external) -- (trades);
\draw[arrow, dashed] (external) -- (price);

% Annotation box
\node[info, below=1.5cm of $(trades)!0.5!(price)$] (annotation) {
  \textbf{Mutual Information:}\\
  How much does knowing Trades reduce uncertainty about Price?\\
  \vspace{0.4em}
  \textbf{Conditioning:}\\
  If External Events explain both, the dependency might vanish.
};

\end{tikzpicture}
\caption{Illustration of mutual information between Trades and Price, with potential confounding from External Events.}
\end{figure}

\noindent
This diagram begins from a state of maximum entropy — a world where trades, prices, and external events are assumed independent. Mutual information measures how far reality deviates from that assumption. If observing trades helps predict price changes, mutual information reveals a dependency. But the presence of a shared external influence can create the illusion of causality. By conditioning on this influence, we can test whether the dependency is real or spurious.

\begin{center}
\renewcommand{\arraystretch}{1.4}
\begin{tabular}{|p{5.2cm}|p{7.4cm}|}
\hline
\textbf{Concept} & \textbf{Visual Representation} \\
\hline
Trades and Prices are linked & Solid arrow from Trades to Price \\
\hline
External Events influence both & Dashed arrows from External Events to both Trades and Price \\
\hline
Mutual Information detects a dependency & Shown as the implied flow from Trades to Price via information reduction \\
\hline
Conditioning reveals deeper structure & If the dashed arrows explain away the solid one, the dependency may be spurious \\
\hline
\end{tabular}
\end{center}

\vspace{1em}
\noindent
This sets the stage for using mutual information not only to detect potential links, but to interrogate them — separating causation from correlation in reactive environments like markets.


\subsubsection{Quantifying Dependence via Entropy}



\vspace{1em}
\noindent
Formally, mutual information between trades \( S \) and price changes \( T \) is defined as:

\[
I(S; T) = H(S) + H(T) - H(S, T),
\]

where \( H(S) \) and \( H(T) \) are the entropies of the two variables, and \( H(S, T) \) is their joint entropy. This measures how much uncertainty is reduced by observing both together, compared to treating them separately.

\begin{figure}[H]
\centering
\begin{tikzpicture}[
    entropy/.style={circle, minimum width=3.5cm, draw=black, thick, fill=blue!10, opacity=0.6},
    entropylabel/.style={font=\footnotesize\bfseries},
]

% Entropy circles with more overlap
\node[entropy, label=left:\( H(S) \)] (HS) at (0,0) {};
\node[entropy, label=right:\( H(T) \)] (HT) at (2,0) {};

% Mutual information label centered in overlap
\node[entropylabel] at (1, 0.3) {\( I(S; T) \)};

% Equation below
\node at (1, -2) {\( I(S; T) = H(S) + H(T) - H(S, T) \)};

\end{tikzpicture}
\caption{Visualizing mutual information as the overlap between the entropies of Trades \( S \) and Prices \( T \).}
\end{figure}

\noindent
This diagram shows how mutual information quantifies the reduction in uncertainty when observing two variables together. Here, trades \( S \) and price changes \( T \) are represented as overlapping uncertainty sets. Their mutual information is the part that is shared — the reduction in unpredictability one provides about the other.

\begin{center}
\renewcommand{\arraystretch}{1.4}
\begin{tabular}{|p{5.2cm}|p{7.4cm}|}
\hline
\textbf{Concept} & \textbf{Visual Representation} \\
\hline
Uncertainty in trades \( S \) & Entire area of circle labeled \( H(S) \) \\
\hline
Uncertainty in price changes \( T \) & Entire area of circle labeled \( H(T) \) \\
\hline
Joint uncertainty in observing both together & Union of both circles, corresponding to \( H(S, T) \) \\
\hline
Mutual information \( I(S; T) \) & Overlapping region — the shared information between \( S \) and \( T \) \\
\hline
\end{tabular}
\end{center}

\vspace{1em}
\noindent
This formalizes the intuition that mutual information tells us how much knowing one variable informs us about the other — and how far the world is from total independence.


\subsubsection{Testing for Causality with Conditional Mutual Information}


To test for causality, we condition on the hidden causal set \( C \). If the dependency between \( S \) and \( T \) is real, mutual information should remain intact:

\[
I(S; T \mid C) \approx I(S; T).
\]

\begin{figure}[H]
\centering
\begin{tikzpicture}[
    entropy/.style={circle, minimum width=3.5cm, draw=black, thick, fill=blue!10, opacity=0.6},
    condition/.style={ellipse, minimum width=6.2cm, minimum height=4cm, draw=red!70!black, thick, dashed, fill=red!10, opacity=0.3},
    labelstyle/.style={font=\footnotesize\bfseries}
]

% Entropy circles
\node[entropy, label=left:\( H(S) \)] (HS) at (0,0) {};
\node[entropy, label=right:\( H(T) \)] (HT) at (2,0) {};

% Conditioning set as a dashed ellipse around both
\node[condition, label={[labelstyle]above:\( C \)}] (C) at (1,0) {};

% Mutual information label in overlap
\node[labelstyle] at (1, 0.3) {\( I(S; T \mid C) \)};

% Equation below
\node at (1, -2.2) {\( I(S; T \mid C) \approx I(S; T) \Rightarrow \) likely causality};

\end{tikzpicture}
\caption{Conditional mutual information: If the dependency between \( S \) (Trades) and \( T \) (Price) persists when conditioned on \( C \) (Hidden Causes), it's likely causal.}
\end{figure}

\noindent
This diagram illustrates the scenario where the relationship between trades \( S \) and price movements \( T \) likely reflects a real causal influence. The visual elements and their meanings are summarized below:

\begin{center}
\renewcommand{\arraystretch}{1.4}
\begin{tabular}{|p{5.2cm}|p{7.4cm}|}
\hline
\textbf{Concept} & \textbf{Visual Representation} \\
\hline
\( S \) and \( T \) are both influenced by \( C \), but retain a direct link & The conditioning set \( C \) overlaps with both entropy sets \( H(S) \) and \( H(T) \), but does not eliminate their overlap \\
\hline
A persistent dependency remains between \( S \) and \( T \) after conditioning & The entropy circles \( H(S) \) and \( H(T) \) still overlap inside \( C \) \\
\hline
\( I(S; T \mid C) \approx I(S; T) \) & Conditioning on \( C \) does not reduce the mutual information — suggesting a likely causal connection \\
\hline
\end{tabular}
\end{center}

\vspace{1em}
\noindent
Mutual information, when preserved after conditioning on a shared cause, becomes strong evidence for a genuine interaction — a possible causal influence rather than a coincidental correlation.

\subsubsection{When Conditional Mutual Information Vanishes}



But if \( S \) and \( T \) are both simply reacting to \( C \), then conditioning removes the apparent dependency:

\[
I(S; T \mid C) \approx 0.
\]

This tells us that the correlation between trades and price movement was an illusion since both were echoes of something deeper. Mutual information, then, becomes a tool not just for detection, but for validation of causality in noisy, reactive systems like financial markets.

\begin{figure}[H]
\centering
\begin{tikzpicture}[
    entropy/.style={circle, minimum width=3.5cm, draw=black, thick, fill=blue!10, opacity=0.6},
    condition/.style={ellipse, minimum width=7cm, minimum height=4cm, draw=red!70!black, thick, dashed, fill=red!10, opacity=0.3},
    labelstyle/.style={font=\footnotesize\bfseries}
]

% Properly disjoint entropy circles
\node[entropy, label=left:\( H(S) \)] (HS) at (0,0) {};
\node[entropy, label=right:\( H(T) \)] (HT) at (4,0) {};

% Conditioning set
\node[condition, label={[labelstyle]above:\( C \)}] (C) at (2,0) {};

% Label indicating no mutual information
\node[labelstyle] at (2, 0.3) {\( I(S; T \mid C) \approx 0 \)};

% Explanation below
\node at (2, -2.2) {\parbox{8cm}{\centering Conditioning on \( C \) removes the apparent dependency.\\ No true causality — just a shared response to a hidden driver.}};

\end{tikzpicture}
\caption{When mutual information disappears after conditioning, the apparent link between Trades \( S \) and Price \( T \) was spurious.}
\end{figure}

\noindent
The illustration shows how conditioning on a hidden cause \( C \) affects the apparent relationship between trades \( S \) and price movements \( T \). The key visual elements and their meanings are summarized below:

\begin{center}
\renewcommand{\arraystretch}{1.4}
\begin{tabular}{|p{5.2cm}|p{7.4cm}|}
\hline
\textbf{Concept} & \textbf{Visual Representation} \\
\hline
\( S \) and \( T \) are influenced by \( C \) & The conditioning set \( C \) overlaps with both entropy sets \( H(S) \) and \( H(T) \) \\
\hline
No direct dependency between \( S \) and \( T \) after conditioning & \( H(S) \) and \( H(T) \) are disjoint (no overlap between circles) \\
\hline
\( I(S; T \mid C) \approx 0 \) & The mutual information region disappears; conditioning on \( C \) explains away the correlation \\
\hline
\end{tabular}
\end{center}

\vspace{1em}
\noindent
This illustrates how mutual information helps distinguish between causal influence and mere co-responsiveness to an external factor in reactive systems like financial markets.




\subsubsection{A Concrete Example: An Algorithm, a Price Spike, and a False Alarm}

\vspace{0.5em}
\noindent
Let’s say you're monitoring a high-frequency trading algorithm — call it \texttt{AlphaFlash}. It's designed to detect short-term inefficiencies in the spread between two correlated assets. On several occasions, you notice the following pattern:

\begin{itemize}
    \item \texttt{AlphaFlash} fires a rapid burst of trades around time \( t_0 \).
    \item Within milliseconds, the market price ticks upward.
\end{itemize}

It looks like textbook causality: a trade causes a price impact. Mutual information between the trade signals \( S \) and the price updates \( T \) is significantly positive:

\[
I(S; T) > 0.
\]

But then you add a new data stream into your analysis: a real-time economic feed. On closer inspection, you find that every instance of this pattern was preceded by the publication of a central bank policy announcement — a hidden variable \( C \) that affects both trade behavior and market prices.

After conditioning on \( C \), the mutual information nearly vanishes:

\[
I(S; T \mid C) \approx 0.
\]

This reveals a spurious correlation. The algorithm and the market weren’t responding to each other; they were both responding to \( C \).

\begin{figure}[H]
\centering
\begin{tikzpicture}[
    event/.style={circle, draw=black, thick, minimum size=1.2cm, align=center},
    arrow/.style={->, thick},
    dashedarrow/.style={->, thick, dashed, gray},
    node distance=2cm and 2.5cm,
    every node/.style={font=\small}
]

% Nodes
\node[event] (C) at (0,2.8) {\( C \)\\\tiny macro news};
\node[event, below left=of C] (S) {\( S \)\\\tiny trades};
\node[event, below right=of C] (T) {\( T \)\\\tiny prices};

% Arrows from C to S and T
\draw[arrow] (C) -- (S);
\draw[arrow] (C) -- (T);

% False assumption arrow
\draw[dashedarrow] (S) -- (T);

% Annotation
\node at (0,-1.5) {\textit{Both trades and prices are driven by macroeconomic news}};
\node at (0,-2.5) {\textit{— but falsely appear causally linked to each other.}};

\end{tikzpicture}
\caption{Trades \( S \) and price changes \( T \) appear correlated. But conditioning on \( C \) — macroeconomic data — reveals that both are responding to a shared external driver.}
\end{figure}

\vspace{1em}
\noindent
Mutual information doesn’t care how compelling the narrative sounds. It’s not impressed by timing coincidences, algorithmic swagger, or well-aligned candlesticks. If the dependency vanishes once you condition on the true cause, then congratulations — you’ve just mathematically disproven your own hype. In a world obsessed with fast signals, mutual information is a quiet, probabilistic voice reminding you: correlation is easy, causality is earned.





\begin{figure}[H]
\centering
\begin{tikzpicture}[every node/.style={font=\footnotesize}]

% Panel 1 — Algo thinks it cracked the market
\comicpanel{0}{4}
  {Algo A}
  {Algo B}
  {I found it! Every time I trade aggressively, the price moves. Clearly I'm the market now.}
  {(0,-0.6)}

% Panel 2 — Algo B is cautious
\comicpanel{6.5}{4}
  {Algo A}
  {Algo B}
  {Or maybe you just trade when news drops and everyone else moves first.}
  {(-0.2,-0.6)}

% Panel 3 — Algo A pulls out the math
\comicpanel{0}{0}
  {Algo A}
  {Algo C}
  {But look! High mutual information between my trades \( S \) and price moves \( T \)!}
  {(0,0.5)}

% Panel 4 — Algo C drops the bomb
\comicpanel{6.5}{0}
  {Algo A}
  {Algo C}
  {Condition on macro news \( C \). \( I(S; T \mid C) \approx 0 \). Sorry, you're just noise.}
  {(0,0.6)}

\end{tikzpicture}
\caption{Mutual information: because “I caused it” doesn’t mean anything until you check the conditioning.}
\end{figure}




\subsection{The Dirichlet Function Analogy: False Causation Vanishes Under Integration}

\vspace{0.5em}
\noindent

Imagine you're poring over high-frequency trading data and everywhere you look, there seems to be a pattern: a slight bump in trades followed by a tiny price movement, again and again. These coincidences are everywhere, like little rational blips in a sea of noise. It's tempting to believe that the pattern is real, and that you're uncovering a subtle but powerful signal.

However, what if you're just seeing the statistical equivalent of the Dirichlet function?

In real analysis, the Dirichlet function is defined as:

\[
D(x) =
\begin{cases}
1, & x \text{ is rational}, \\
0, & x \text{ is irrational}.
\end{cases}
\]

At first glance, it appears to have “value” everywhere: rational numbers are dense on the real line, and there’s one at every zoom level. But when we apply the Lebesgue integral to \( D(x) \), the entire function vanishes:

\[
\int_{0}^{1} D(x) \,dx = 0.
\]

Why? Because the rational numbers, while dense, have measure zero. They can’t “accumulate” any weight under a measure-theoretic lens.

\begin{figure}[H]
\centering
\begin{tikzpicture}[
    scale=1.2,
    every node/.style={font=\small},
    spike/.style={ultra thick, color=blue},
    axis/.style={->, thick},
    measure/.style={draw=none, fill=blue!10, opacity=0.4}
]

% Axes
\draw[axis] (-0.1,0) -- (5.2,0) node[right] {\( x \)};
\draw[axis] (0,-0.1) -- (0,1.5) node[above] {\( D(x) \)};

% Dots for rational spikes
\foreach \x in {0.3, 0.6, 1.2, 1.5, 1.8, 2.1, 2.6, 3.1, 3.5, 3.8, 4.1, 4.4, 4.7} {
    \draw[spike] (\x,0) -- (\x,1);
}

% Label on spike
\node at (2.6, 1.15) {\footnotesize Rational \( \Rightarrow D(x) = 1 \)};
\node at (2.6, -0.25) {\footnotesize Irrational \( \Rightarrow D(x) = 0 \)};

% Bracket (facing upward, placed lower below the axis)
\draw[decorate,decoration={brace,mirror,amplitude=6pt},yshift=-14pt]
  (0,0) -- (5,0);

% Text below bracket
\node at (2.5, -1.2) {\footnotesize Lebesgue measure over \([0,1]\): \quad \( \int_0^1 D(x)\, dx = 0 \)};

\end{tikzpicture}
\caption{Apparent patterns may be dense and tempting to interpret, but like the Dirichlet function, they can vanish under proper integration — exposing the illusion of structure.}
\end{figure}

This is the perfect analogy for false causation in high-frequency trading. Just like the Dirichlet function spikes at every rational, spurious correlations seem to occur constantly, but they vanish under rigorous integration via inclusion maps and measure theory. What appears to be a persistent signal is revealed to be noise: dense, but insignificant.

\begin{figure}[H]
\centering
\begin{tikzpicture}[
    every node/.style={font=\small},
    spike/.style={ultra thick, color=blue},
    axis/.style={->, thick},
    smoothline/.style={very thick, red!70!black},
    measure/.style={draw=none, fill=blue!10, opacity=0.4}
]

% Shift left graph to the left
\begin{scope}[xshift=-1.5cm]
    % Axes left: raw signal
    \draw[axis] (-0.1,0) -- (4.2,0) node[right] {\( t \)};
    \draw[axis] (0,-0.1) -- (0,1.5) node[above] {Signal};

    % Spikes (false signals)
    \foreach \x in {0.3, 0.6, 1.0, 1.2, 1.5, 1.8, 2.0, 2.2, 2.4, 2.7, 3.0, 3.3, 3.5, 3.7, 4.0} {
        \draw[spike] (\x,0) -- (\x,1.1);
    }

    % Caption below left graph
    \node at (2, -0.8) {\footnotesize Apparent signal: dense, everywhere, but vanishing under integration};
\end{scope}

% Arrow to right side (integration lens)
\draw[->, thick] (3.1,0.75) -- (4.9,0.75) node[midway, above] {\footnotesize integration };

% Right graph: flat line (no signal survives)
\begin{scope}[xshift=6.2cm]
    \draw[axis] (-0.1,0) -- (4.2,0) node[right] {\( t \)};
    \draw[axis] (0,-0.1) -- (0,1.5) node[above] {Signal};

    % Flat result after integration
    \draw[smoothline] (0,0.2) -- (4,0.2);
    \node at (2, 0.4) {\footnotesize Zero net influence};
\end{scope}

\end{tikzpicture}
\caption{False causation in high-frequency data is like the Dirichlet function: spikes are dense but disappear under rigorous integration. What looks like signal is often just noise.}
\end{figure}


In this sense, Lebesgue integration acts as a filter --- not for smoothing out spikes, but for exposing which of them actually matter when viewed through the lens of a consistent probability measure. And more often than not, the spurious ones disappear.


\subsubsection{Concrete Example: Integration Over Reactive Trades}

\vspace{0.5em}
\noindent
Suppose you observe a sequence of rapid trades by a high-frequency strategy during a volatile macroeconomic event — say, a central bank rate announcement. The algorithm reacts quickly and often: short bursts of trades scattered through a millisecond-scale window. Every time you zoom in, there’s another spike of activity. It looks like the trading algorithm is “doing something.”

But when you apply a consistent measure — one that considers the total weight or contribution of these trades to the overall market — their impact washes out. They're reacting to the same event, not producing meaningful new information. The total measure assigned to these micro-responses is negligible.

\[
\mu(\text{reactive trades}) \approx 0.
\]

Lebesgue integration over these trades reveals this insignificance. Despite being dense in time, they have no weight — just like the Dirichlet function's spikes at the rationals.

\vspace{1em}
\begin{figure}[H]
\centering
\begin{tikzpicture}[
    every node/.style={font=\small},
    axis/.style={->, thick},
    spike/.style={ultra thick, blue},
    integralarea/.style={fill=blue!10, opacity=0.4},
    dashedarrow/.style={->, thick, dashed},
    smoothline/.style={very thick, red!70!black}
]

% Left: set of reactive trades (dense spikes)
\begin{scope}[xshift=-1.8cm]
    \draw[axis] (-0.2,0) -- (4.2,0) node[right] {\( t \)};
    \draw[axis] (0,-0.1) -- (0,1.5) node[above] {\tiny Trade Activity};

    % Spikes (reactive trades)
    \foreach \x in {0.2, 0.5, 0.7, 0.9, 1.1, 1.4, 1.8, 2.0, 2.3, 2.5, 2.8, 3.0, 3.3, 3.5, 3.8} {
        \draw[spike] (\x,0) -- (\x,1);
    }

    % Label
    \node at (2, -0.8) {\footnotesize Reactive Trades};
\end{scope}

% Arrow indicating integration
\draw[dashedarrow] (3.3,0.7) -- (5.3,0.7) node[midway, above] {\footnotesize \textbf{Lebesgue integration}};

% Right: integrated result (area)
\begin{scope}[xshift=5.8cm]
    \draw[axis] (-0.2,0) -- (4.2,0) node[right] {\( t \)};
    \draw[axis] (0,-0.1) -- (0,1.5) node[above] {\tiny Measure};

    % Draw shaded area (tiny strip = ~0 measure)
    \draw[integralarea] (0,0) rectangle (4,0.1);
    \draw[smoothline] (0,0.1) -- (4,0.1);

    % Label
    \node at (2, 0.3) {\footnotesize \( \int \text{Reactive Trades} \approx 0 \)};
    \node at (2, -0.8) {\footnotesize Total impact: negligible};
\end{scope}

\end{tikzpicture}
\caption{Reactive trades during a macro event appear dense in time, but under integration their cumulative impact approaches zero — like measure-zero rational spikes.}
\end{figure}




\begin{figure}[H]
\centering
\begin{tikzpicture}[every node/.style={font=\footnotesize}]

% Panel 1 — Analyst is excited by the pattern
\comicpanel{0}{4}
  {Analyst A}
  {Analyst B}
  {Everywhere I look: a trade, a price twitch. I'm seeing structure! It's beautiful!}
  {(0,-0.6)}

% Panel 2 — Analyst B raises a math eyebrow
\comicpanel{6.5}{4}
  {Analyst A}
  {Analyst B}
  {Beautiful, yes. But have you tried integrating your discovery?}
  {(-0.2,-0.6)}

% Panel 3 — Analyst A hesitates
\comicpanel{0}{0}
  {Analyst A}
  {Analyst C}
  {Well... it looks a bit like the Dirichlet function. Spiky. Frequent. Very... rational.}
  {(0,0.5)}

% Panel 4 — Analyst C drops the measure-theoretic mic
\comicpanel{6.5}{0}
  {Analyst A}
  {Analyst C}
  {Exactly. Dense everywhere, measure zero. Integrate it — and poof, it’s gone.}
  {(0,0.6)}

\end{tikzpicture}
\caption{Some patterns are just Dirichlet functions in disguise — everywhere and worth nothing.}
\end{figure}






\subsection{Economic Impact of Ignoring Inclusion Maps}

\vspace{0.5em}
\noindent
Imagine a trading system made up of a thousand machines, each operating like an over-caffeinated intern on bad Wi-Fi. They're constantly scanning for signals --- trade bumps, price jitters, whatever seems to repeat --- and they act instantly, assuming causation where there may be none.

This is what happens when inclusion maps are ignored. The machines detect patterns that look like signals but are really statistical noise --- echoes of unrelated events or mutually triggered reactions to the same external cause. It’s like reading meaning into the spikes of the Dirichlet function: dense with activity, but empty when viewed through a proper lens.

Now let’s run the numbers.

\subsubsection{Case 1: Machines Trade on Spurious Correlations (No Inclusion Maps)}

\begin{itemize}
    \item Each machine executes 120 trades per second.
    \item The average loss per trade --- due to reacting to false signals --- is \$0.03.
    \item With 1,000 machines, this results in:

    \[
    120 \times 0.03 \times 1,000 = \text{\$3,600 lost per second}.
    \]
\end{itemize}

These machines aren't trading on insight: they’re chasing noise. And it costs them.

\subsubsection{Case 2: Machines Filter Out False Signals Using Inclusion Maps}

\begin{itemize}
    \item With inclusion maps in place, the machines ignore spurious correlations.
    \item Trades per machine drop to 90 per second: fewer, but smarter.
    \item The average profit per trade rises to \$0.08.
    \item With 1,000 machines, this becomes:

    \[
    90 \times 0.08 \times 1,000 = \text{\$7,200 earned per second}.
    \]
\end{itemize}

\textbf{Result:} By filtering out noise using inclusion maps — treating causality seriously rather than statistically — the system goes from losing \$3,600 every second to earning \$7,200. That’s a swing of \$10,800 per second, simply by shifting from reactive correlation to measured causation.

This isn’t just a philosophical point about math — it’s \$38.8 million per hour during market operations.


\begin{figure}[H]
\centering
\begin{tikzpicture}[every node/.style={font=\footnotesize}]

% Panel 1 — Frenzied trading bot
\comicpanel{0}{4}
  {Bot A}
  {Bot B}
  {I saw three bumps and a wiggle — deployed 120 trades per second. We're printing noise!}
  {(0,-0.6)}

% Panel 2 — Bot B checks the numbers
\comicpanel{6.5}{4}
  {Bot A}
  {Bot B}
  {You also lost \$3,600 per second. Congrats on outperforming entropy.}
  {(0,-0.5)}

% Panel 3 — Calm inclusion-mapped bot
\comicpanel{0}{0}
  {Bot C}
  {Bot D}
  {I filtered out spurious signals, traded less, and made \$7,200 per second.}
  {(0,0.6)}

% Panel 4 — Bot D delivers the verdict
\comicpanel{6.5}{0}
  {Bot A}
  {Bot D}
  {Turns out math isn't optional when money’s on the line.}
  {(0,0.6)}

\end{tikzpicture}
\caption{Inclusion maps: saving \$38.8 million an hour by not believing in trade wiggles.}
\end{figure}


\subsection{Final Takeaway: The Role of Mathematics in High-Frequency Trading}

\vspace{0.5em}
High-frequency trading operates in a world where milliseconds matter and patterns appear everywhere. But as we've seen, many of those patterns are like the Dirichlet function — dense, distracting, and ultimately meaningless when viewed through the right lens.

Without tools like inclusion maps, Lebesgue integration, and vector clocks:

\begin{itemize}
    \item We mistake coincidence for causation (acting on noise that just looks like signal).
    \item Our models chase false positives (reacting to patterns that vanish under proper integration).
    \item Entire trading systems amplify volatility, flooding the market with misguided decisions.
\end{itemize}

These mathematical tools form a kind of \textbf{epistemic firewall} — not just helping us model the market, but protecting us from ourselves. They give us a way to test whether a trade actually causes a price movement, or whether both are just reacting to the same macroeconomic tremor. They let us quantify how much we actually know, and whether that knowledge survives conditioning on reality.

\vspace{0.5em}
\noindent
\textbf{Bottom line:} Inclusion maps and vector clocks don’t just make our models cleaner; they make them safer. They ensure that our systems act on real signals, not illusions.

\begin{quote}
\textbf{Mathematics: the difference between chasing ghosts and capturing value; or between burning millions per second, and earning them.}
\end{quote}




\begin{figure}[H]
\centering
\begin{tikzpicture}[every node/.style={font=\footnotesize}]

% Panel 1 — Trader overwhelmed by patterns
\comicpanel{0}{4}
  {Trader A}
  {Trader B}
  {There are patterns everywhere! Look at all this market motion — it has to mean something.}
  {(0,-0.6)}

% Panel 2 — Trader B holds a book
\comicpanel{6.5}{4}
  {Trader A}
  {Trader B}
  {That's what the Dirichlet function said right before getting integrated into zero.}
  {(0,-0.5)}

% Panel 3 — Algorithm on fire
\comicpanel{0}{0}
  {Algo A}
  {Algo B}
  {Our model says BUY because of the third spike in the fifth harmonic. Also we’re on fire.}
  {(0,0.6)}

% Panel 4 — Calm analyst with math
\comicpanel{6.5}{0}
  {Algo A}
  {Analyst}
  {Inclusion maps. Vector clocks. Lebesgue. You know... the things that keep systems from hallucinating.}
  {(0,0.6)}

\end{tikzpicture}
\caption{Mathematics: the firewall between million-dollar models and million-dollar mistakes.}
\end{figure}

%
\section{Estimating Probability Distributions: Neural Networks as Measure-Theoretic Estimators}

In the previous section, we showed how to avoid spurious correlation by modeling financial systems using \textbf{inclusion maps}, \textbf{vector clocks}, and \textbf{Lebesgue integration} over causality sets. We laid out a framework for identifying real causal structure—not just patterns that appear to be meaningful but dissolve under proper measure-theoretic scrutiny.

But we left something unresolved.

We talked about integrating over sets of causally significant events—like reactive trades, macroeconomic triggers, and their effects—but we never actually computed that integral. Why?

Because we don't know the integrand.

\textbf{There exists, in theory, some function that assigns ``causal weight'' to each event in the system.} A function that tells us how strongly a trade, a price movement, or a macro shock contributes to market behavior. But we can’t write it down explicitly. We know it exists—because the mathematics of measure theory and causality tells us it must—but it’s as elusive as the actual distribution behind a real-world coin flip.

What we need, then, is a way to \textit{discover} this function.

\subsection{Estimators: From Probability Theory to Machine Learning}

In classical statistics, this idea has a name: an \textbf{estimator}. An estimator is a function that guesses at some underlying ``true'' function based on observed data. It assumes the existence of a hidden structure—a real, ideal function—and seeks to approximate it as closely as possible.

Deep learning is just the modern, high-powered, measure-theoretic version of this idea.

You can think of it like a limiting process from calculus. In calculus, if we can define a function and a process for refining our guess (via derivatives, limits, etc.), we can get closer and closer to the true value. In deep learning, we do something similar—but instead of working in Euclidean space with distance metrics, we're working in \textbf{information space}, using tools from probability, entropy, and measure theory.

\subsection{Neural Networks as Universal Function Estimators}

Neural networks are powerful because they act as \textbf{universal function approximators}. That means: given enough data and capacity, they can learn to approximate \textit{any} function—including the elusive, abstract one that integrates causal weight over market event space.

In our case, that function isn't defined by simple math or clean formulas. It's defined over a \textbf{measurable space} \( (\Omega, \mathcal{F}, \mu) \), where each point represents a micro-event in the market—a trade, a price change, a news release. And the function we want to estimate is the one that maps these events to their causal impact.

What deep learning gives us is a principled way to approximate this function:

\[
\hat{f} \approx f^*, \quad \text{where } f^* \text{ is the true causal density function over } \Omega.
\]

The process is iterative—just like an \(\varepsilon\)--\(\delta\) proof in calculus. If we define our loss function correctly (analogous to an error bound), and we train on representative data, we can ensure that our approximation gets arbitrarily close to the real function within a certain tolerance.

\subsection{From Causal Geometry to Learning Geometry}

This moves us from the realm of hand-crafted mathematical models into a new regime: \textbf{causal learning}.

We're no longer trying to specify everything ourselves. We're acknowledging that the true structure exists in a high-dimensional, non-Euclidean, information-theoretic space—and we're building machines that can estimate it, refine it, and learn from it over time.

And just as calculus needed limits to define smooth motion, deep learning needs measure theory to define smooth inference. As long as our sets are well-defined and our data encodes meaningful events in the proper \(\sigma\)-algebra, we can develop estimators that converge to the ``true'' causal function—not just correlations that look good on paper.

\begin{quote}
\textit{Inclusion maps gave us the structure. Lebesgue integration gave us the calculus. Now neural networks give us the machinery to estimate what we couldn’t write down—and to do it in information space.}
\end{quote}



\subsubsection{How Neural Networks Estimate Causal Functions}

In the previous section, we noted that the true causal function \( f^* \colon \Omega \rightarrow \mathbb{R} \), which maps market events to their impact, exists within a high-dimensional, measure-theoretic space — but cannot be written down explicitly. What we can do, however, is build an \textit{estimator} that approximates it.

This is where neural networks come in.

A neural network is a parameterized function \( f_\theta(X) \), where \( \theta \) represents the trainable weights. Given data drawn from a measurable event space \( X \subset \Omega \), the network learns to approximate the unknown mapping:

\[
Y \approx f_\theta(X)
\]

Here, \( Y \) could represent anything from the probability of a price movement to the causal weight assigned to a trade, or the expected impact of a macroeconomic event. The key difference from classical models is that we do not assume a fixed structure like \( P(Y \mid X) \). Instead, the network learns this structure implicitly through training — using data sampled from the event space \( \Omega \), shaped by a measure \( \mu \).

In essence, we are not fitting a curve in Euclidean space — we are estimating a causal function in \textbf{information space}, using neural networks as high-capacity approximators under a measure-theoretic framework.

\subsubsection{Illustration: A Neural Estimator Over Market Events}

\begin{center}
\begin{tikzpicture}[node distance=1.5cm, scale=1, every node/.style={scale=1}]
    % Input layer
    \node (x1) at (0,2) [draw, circle] {\small \(X_1\)};
    \node (x2) at (0,1) [draw, circle] {\small \(X_2\)};
    \node (x3) at (0,0) [draw, circle] {\small \(X_3\)};
    
    % Hidden layer
    \node (h1) at (2,1.5) [draw, circle] {\small \(Z_1\)};
    \node (h2) at (2,0.5) [draw, circle] {\small \(Z_2\)};
    
    % Output layer
    \node (y) at (4,1) [draw, circle] {\small \(Y\)};
    
    % Connections
    \foreach \i in {x1,x2,x3} {
        \foreach \j in {h1,h2} {
            \draw[->] (\i) -- (\j);
        }
    }
    \draw[->] (h1) -- (y);
    \draw[->] (h2) -- (y);
\end{tikzpicture}
\end{center}

\noindent In this simplified neural architecture:
\begin{itemize}
    \item Inputs \( X_1, X_2, X_3 \) might represent measurable features in our financial event space — such as trade size, volatility level, or recent price deltas.
    \item Hidden units \( Z_1, Z_2 \) capture intermediate structures — perhaps encoding temporal dependencies or interactions between causal sources.
    \item The output \( Y \) represents an estimated impact: for example, the expected future price change, or the inferred causal contribution of the event.
\end{itemize}

\textbf{Key Idea:} Instead of manually defining \( P(Y \mid X) \), we allow the network to learn a function that minimizes prediction error under a loss — effectively building a statistical estimator for the unknown causal mechanism.


\subsubsection{From Fixed Distributions to Adaptive Learning}

Up to this point, we've treated the causal function \( f^* \) as something that exists — hidden in the structure of the market — and neural networks as estimators that can approximate it. But there’s a deeper complication:

\begin{itemize}
    \item The function itself is not fixed.
    \item Financial markets are non-stationary: the distribution over events evolves as new information arrives.
    \item What appears causal in one regime (e.g., low volatility) may become irrelevant or even misleading in another (e.g., post-news shock).
\end{itemize}

In traditional models, we assume a static probability distribution — a fixed \( P(Y \mid X) \) — and hope it holds over time. But in high-frequency, data-rich environments, this assumption collapses. The true distribution is always shifting, always reacting — like the traders themselves.

\textbf{Therefore,} our estimators must not just approximate a fixed function — they must continuously \textit{adapt} to a changing measure space.

\vspace{0.5em}
Neural networks provide this adaptivity. Through online learning, gradient updates, or dynamic retraining, the parameterized function \( f_\theta \) evolves over time — effectively tracking a moving target within the larger measurable space \( (\Omega_t, \mathcal{F}_t, \mu_t) \). As new events reshape the space, the estimator learns to adjust:

\[
f_\theta^{(t+1)} \approx f^*_t \quad \text{for each new distribution } \mu_t.
\]

This turns the neural estimator into a kind of \textbf{information-theoretic compass} — constantly recalibrating itself to point toward the center of causal gravity, even as the terrain changes underneath.

\textit{But how do we ensure it adapts correctly?} That’s the question of convergence, generalization, and — crucially — how well we’ve defined the information geometry of the problem in the first place.


\subsection{Kullback-Leibler Divergence: Measuring Approximation Error in Information Space}

In the previous section, we framed neural networks as adaptive estimators operating over a shifting measure space. But even as these models update dynamically, there's a fundamental question we still need to answer:

\textbf{How close is our estimated function to the true causal structure?}

Mutual information helped us detect dependencies and isolate spurious correlations. But it doesn’t quantify the cost of approximation — the gap between what the model \textit{believes} and what the market actually \textit{does}.

This is where the \textbf{Kullback-Leibler (KL) divergence} enters the scene. It measures the information loss incurred when we replace a true distribution \( P \) with an estimated one \( Q \). In the context of deep learning, it tells us how far our learned distribution \( q(x) \) is from the underlying (and typically unknown) true distribution \( p(x) \):

\[
D_{KL}(P \,\|\, Q) = \int p(x) \log \frac{p(x)}{q(x)} \, d\mu(x).
\]

\vspace{0.5em}
Here:
\begin{itemize}
    \item \( p(x) \) is the true (but hidden) distribution of features or causal weights.
    \item \( q(x) \) is the model's current approximation — learned from data.
    \item \( d\mu(x) \) is the Lebesgue measure over the space \( \Omega \), ensuring the integration is valid across continuous event domains.
\end{itemize}

\textbf{KL divergence lives in information space.} It doesn't just measure numerical error — it quantifies epistemic drift: how much belief is lost when we trade ground truth for estimation.

As data flows through the layers of a neural network, the model refines its internal representations. But unless the KL divergence between the learned and true distributions shrinks over time, we're not converging — we're just reshaping our ignorance.

\vspace{0.5em}
\noindent
This gives us a concrete way to monitor learning as an information-theoretic process: not merely adjusting weights, but minimizing divergence in a high-dimensional probability landscape.

\begin{quote}
\textit{KL divergence is not just a loss function — it's a compass. It tells us how far our estimator has drifted from the truth in the topology of belief.}
\end{quote}

\subsubsection{Feature Selection in Decision Trees: A Discrete Case of Information Geometry}

To better understand how KL divergence operates in deep learning, it helps to first consider a simpler, discrete analogue — decision trees.

\paragraph{A Toy Model for Trade Decisions}

Imagine an automated trading system that uses a decision tree to determine whether to execute a trade based on observable features:

\begin{itemize}
    \item Trade Volume (\textbf{High} or \textbf{Low})
    \item Price Trend (\textbf{Up} or \textbf{Down})
\end{itemize}

The structure of the tree looks like this:

\begin{center}
\begin{tikzpicture}[
    every node/.style={draw, rounded corners, text width=3cm, align=center},
    sibling distance=4cm,
    level distance=2.5cm
]
  % Root Node
  \node (root) {Trade Volume?}
    child { node (left) {Price Trend?}
        child { node (leftleft) {Buy} }
        child { node (leftright) {Hold} }
    }
    child { node (right) {Sell} };
\end{tikzpicture}
\end{center}

At each branching point, the tree selects the feature that maximally reduces uncertainty — measured using Shannon entropy:

\[
\text{Information Gain} = H(Y) - H(Y \mid X),
\]

where:
\begin{itemize}
    \item \( H(Y) \) is the entropy of trade outcomes before the split.
    \item \( H(Y \mid X) \) is the conditional entropy after splitting on feature \( X \).
\end{itemize}

This process is, in essence, a discrete optimization over information space. The tree is looking for directions that maximally collapse uncertainty — isolating features that carve the space into more certain regions.

\paragraph{From Trees to Tensors: Generalizing to Neural Networks}

While decision trees operate in discrete feature spaces, deep neural networks extend this idea to continuous, high-dimensional, and non-linear domains. But the core logic remains the same:

\begin{itemize}
    \item Identify patterns that reduce uncertainty.
    \item Choose representations that best approximate the true distribution.
    \item Minimize divergence from the structure of the data-generating process.
\end{itemize}

\textbf{In decision trees, this is called information gain. In neural networks, this is formalized as KL divergence.}

Instead of selecting a single feature at a node, neural networks \textit{learn a basis} of features across layers — continuously adjusting them to minimize information loss as data flows forward. KL divergence then serves as the information-theoretic counterpart to entropy-based feature selection: it quantifies how much "structure" is lost when the model’s internal representation diverges from the true causal geometry.

\vspace{1em}
\begin{quote}
\textit{Decision trees split to minimize entropy. Neural networks flow to minimize divergence. Both are just travelers in information space — trying to get closer to the truth.}
\end{quote}


\subsection{Variational Information Bottleneck: Structured Compression in Information Space}

So far, we’ve viewed KL divergence as a measure of information loss — a way to quantify how far our learned distribution diverges from the true causal structure. But in modern deep learning, KL divergence does more than diagnose approximation error.

\textbf{It becomes part of the optimization objective itself.}

In particular, \textbf{Variational Information Bottleneck (VIB)} methods use KL divergence to explicitly control what information the model retains — and what it throws away. The goal is no longer just to match the true distribution, but to compress irrelevant structure while preserving what matters for prediction.

The VIB principle formalizes this balance:

\[
\max I(X; Y) - \beta \, D_{KL}\big(P(Z \mid X) \,\|\, Q(Z)\big),
\]

where:
\begin{itemize}
    \item \( I(X; Y) \) is the mutual information between the input \( X \) and the output \( Y \) — i.e., how much predictive signal is retained.
    \item \( P(Z \mid X) \) is the encoder’s learned distribution over latent representations.
    \item \( Q(Z) \) is a simplified, low-information reference distribution (often standard normal).
    \item \( \beta \) is a tradeoff parameter controlling how much compression is enforced.
\end{itemize}

This objective encourages the model to represent only the \textit{minimal sufficient statistics} of \( X \) for predicting \( Y \). In measure-theoretic terms: it learns a transformation \( \phi: X \rightarrow Z \subset \Omega \) such that the image of \( X \) under \( \phi \) retains maximal causal structure, while discarding measure-zero noise.

\textbf{KL divergence acts like an information regularizer:} it penalizes complexity and forces the network to operate in a compressed, low-dimensional region of information space — one where causality survives and overfitting doesn’t.

\vspace{0.5em}
\noindent
This is the deep learning analogue of financial signal filtering. Just as a trading algorithm must discard random price jitters while acting on macroeconomic structure, VIB forces a model to throw away features that don’t carry predictive weight — no matter how tempting their local fluctuations may appear.

\begin{quote}
\textit{Inclusion maps isolated true structure. KL divergence measured how far we drifted. Now the information bottleneck tells us what to keep — and what to forget.}
\end{quote}





\begin{figure}[H]
\centering
\begin{tikzcd}[row sep=large, column sep=huge, ampersand replacement=\&]
{(X, P_X)} \arrow[r, "\phi", dashed] \arrow[dr, "f", swap] \& {(Z, P_Z)} \arrow[d, "\psi"] \\
\& {(Y, P_Y)}
\end{tikzcd}
\caption{A commutative diagram of the Variational Information Bottleneck. The encoder \( \phi \) maps inputs \( X \) to compressed representations \( Z \), discarding irrelevant information via KL divergence. The decoder \( \psi \) reconstructs predictions \( Y \), approximating the direct map \( f: X \rightarrow Y \).}
\end{figure}










\subsection{Visualizing Compression: Learning as a Commutative Flow}

To generalize the learning pattern we’ve described — from feature extraction to causal filtering — we can represent the process as a commutative diagram in information space. This isn’t just a schematic of network layers: it’s a diagram of inference through compression, where each transformation refines our understanding of the world.










\[
\begin{tikzpicture}[>=latex, scale=1.1, every node/.style={scale=1.2}]
  
  % Define matrix for the true feature distributions P(Z | X)
  \matrix (m) [matrix of math nodes, row sep=3.5em, column sep=7em] {
      X \\ 
      Z_1 \\ 
      Z_2 \\ 
      Z_3 \\ 
      P \\ % Final Prediction
  };

  % Define the simplified model Q(Z | X) to the right
  \matrix (q) [matrix of math nodes, row sep=3.5em, column sep=7em, right of=m, node distance=8cm] {
      \\  % No simplified version for X
      Q(Z_1 | X) \\ 
      Q(Z_2 | Z_1) \\ 
      Q(Z_3 | Z_2) \\ 
      Q(P | Z_3) \\ % Simplified Prediction
  };

  % Feature Transformations (Left Column)
  \path[->] (m-1-1) edge node[left] {$P(Z_1 | X)$} (m-2-1);
  \path[->] (m-2-1) edge node[left] {$P(Z_2 | Z_1)$} (m-3-1);
  \path[->] (m-3-1) edge node[left] {$P(Z_3 | Z_2)$} (m-4-1);
  \path[->] (m-4-1) edge node[left] {$P(P | Z_3)$} (m-5-1);

  % Simplified Feature Representations (Right Column)
  \path[->] (q-2-1) edge node[right] {$Q(Z_2 | Z_1)$} (q-3-1);
  \path[->] (q-3-1) edge node[right] {$Q(Z_3 | Z_2)$} (q-4-1);
  \path[->] (q-4-1) edge node[right] {$Q(P | Z_3)$} (q-5-1);

  % KL Divergence Regularization (Dashed Arrows)
  \path[->, dashed] (m-2-1.east) edge node[above] {$D_{KL}$} (q-2-1.west);
  \path[->, dashed] (m-3-1.east) edge node[above] {$D_{KL}$} (q-3-1.west);
  \path[->, dashed] (m-4-1.east) edge node[above] {$D_{KL}$} (q-4-1.west);
  \path[->, dashed] (m-5-1.east) edge node[above] {$D_{KL}(P(P \mid Z_3) \, \|\, Q(P \mid Z_3))$} (q-5-1.west);
  
\end{tikzpicture}
\]

At each stage, the model constructs two parallel representations:

\begin{itemize}
    \item On the left: the \textbf{true conditional distributions} — ideally aligned with the structure of the data-generating process.
    \item On the right: the \textbf{compressed approximations} — constrained, regularized, and filtered through KL divergence.
\end{itemize}

The model progresses layer by layer, mapping input \( X \) into latent variables \( Z_i \), and eventually into a prediction \( P \). At every point, KL divergence acts as a tension between expressivity and parsimony — enforcing structure while penalizing unnecessary complexity.



\vspace{1em}
This diagram illustrates what learning looks like under a measure-theoretic lens: a flow through nested spaces of reduced uncertainty. Each \( D_{KL} \) is a checkpoint — a formal request for the model to justify its complexity, to compress responsibly, and to make its structure match the data’s causality.

\begin{quote}
\textit{Learning isn’t just moving forward — it’s narrowing. Each layer carves away irrelevance. Each KL divergence asks: “Are you closer to the truth, or just louder?”}
\end{quote}

The commutative diagram we just introduced isn’t merely a technical schematic. It visually encodes the logic of inference under constraint — a flow of information from raw observations to structured predictions, filtered through a series of transformations that shape and prune the model’s internal geometry.

KL divergence serves as the gatekeeper between expressive power and epistemic discipline: it pressures the model to compress aggressively, but not blindly.

\subsubsection{Nodes as Layers in Information Space}

Each node in the diagram represents a distinct stage in the evolution of knowledge:

\begin{itemize}
    \item \textbf{\( X \) (Raw Inputs):} The measurable substrate — bid-ask spreads, order book velocity, trade volume — from which the model begins constructing a representation.
    
    \item \textbf{\( Z_1, Z_2, Z_3 \) (Latent Representations):} Intermediate encodings in the model's internal space, successively transformed and refined. Each layer restructures the measure on \( \Omega \), ideally concentrating information relevant for prediction while discarding noise.
    
    \item \textbf{\( P \) (Final Prediction):} A probability distribution over future events (e.g., price movements) — the model’s best estimate of downstream outcomes based on its internal representation.
    
    \item \textbf{\( Q(Z_i | \cdot) \) (Regularized Approximations):} Constrained, compressed versions of each latent feature — simpler distributions with lower entropy, used as references during training to prevent overfitting and encourage generalization.
\end{itemize}


\[
\resizebox{\textwidth}{!}{
\begin{tikzpicture}[>=latex, scale=1.1, every node/.style={scale=1.2}, node distance=2.2cm]

  % Define matrix of math nodes (just the symbols)
  \matrix (m) [matrix of math nodes, row sep=3.8em, column sep=10em] {
      X \\
      Z_1 \\
      Z_2 \\
      Z_3 \\
      P \\
  };

  \matrix (q) [matrix of math nodes, row sep=3.8em, column sep=10em, right of=m, node distance=7.5cm] {
      \\ % X has no simplified version
      Q(Z_1|X) \\
      Q(Z_2|Z_1) \\
      Q(Z_3|Z_2) \\
      Q(P|Z_3) \\
  };

  % Arrows (structure only)
  \path[->] (m-1-1) edge (m-2-1);
  \path[->] (m-2-1) edge (m-3-1);
  \path[->] (m-3-1) edge (m-4-1);
  \path[->] (m-4-1) edge (m-5-1);

  \path[->] (q-2-1) edge (q-3-1);
  \path[->] (q-3-1) edge (q-4-1);
  \path[->] (q-4-1) edge (q-5-1);

  \path[->, dashed] (m-2-1.east) edge (q-2-1.west);
  \path[->, dashed] (m-3-1.east) edge (q-3-1.west);
  \path[->, dashed] (m-4-1.east) edge (q-4-1.west);
  \path[->, dashed] (m-5-1.east) edge (q-5-1.west);

  % Left-side annotations (closer spacing)
  \node[left=0.2cm of m-1-1] {\parbox{4cm}{\raggedleft\footnotesize \textit{Raw Inputs}\\\textit{(e.g., order book, volume)}}};
  \node[left=0.2cm of m-2-1] {\parbox{4cm}{\raggedleft\footnotesize \textit{Latent Layer 1}\\\textit{(initial encoding)}}};
  \node[left=0.2cm of m-3-1] {\parbox{4cm}{\raggedleft\footnotesize \textit{Latent Layer 2}\\\textit{(refined representation)}}};
  \node[left=0.2cm of m-4-1] {\parbox{4cm}{\raggedleft\footnotesize \textit{Latent Layer 3}\\\textit{(prediction-ready signal)}}};
  \node[left=0.2cm of m-5-1] {\parbox{4cm}{\raggedleft\footnotesize \textit{Final Prediction}\\\textit{(distribution over outcomes)}}};

  % Right-side annotations (closer spacing)
  \node[right=0.2cm of q-2-1] {\parbox{4cm}{\raggedright\footnotesize \textit{Approx. of $Z_1$}\\\textit{(lower entropy)}}};
  \node[right=0.2cm of q-3-1] {\parbox{4cm}{\raggedright\footnotesize \textit{Approx. of $Z_2$}\\\textit{(regularized)}}};
  \node[right=0.2cm of q-4-1] {\parbox{4cm}{\raggedright\footnotesize \textit{Approx. of $Z_3$}\\\textit{(compressed)}}};
  \node[right=0.2cm of q-5-1] {\parbox{4cm}{\raggedright\footnotesize \textit{Approx. Prediction}\\\textit{(reference dist.)}}};

\end{tikzpicture}
}
\]

\subsubsection*{Qualitative Interpretations of Diagram Nodes}

\vspace{1em}
\textbf{Left-Side (Main Path) Explanations}

\begin{itemize}
    \item \textbf{Raw Inputs (e.g., order book, volume)}: This is the raw substrate — the measurable phenomena the model observes. Think timestamped price ticks, trade volumes, order book changes. This layer is closest to the physical world and contains high entropy: it’s rich, unfiltered, and noisy.

    \item \textbf{Latent Layer 1 (initial encoding)}: The model’s first attempt at structure. This is where raw signals are projected into a lower-dimensional space that begins to highlight patterns — perhaps short-term momentum, microstructure signals, or statistical anomalies. Still close to the input space, but with noise partially filtered out.

    \item \textbf{Latent Layer 2 (refined representation)}: The middle abstraction. By now, the model is combining multiple signals into more complex, abstract concepts — e.g., implicit volatility regimes, mean-reverting behaviors, or early indicators of liquidity shifts. The information has been compacted and made more relevant for the final task.

    \item \textbf{Latent Layer 3 (prediction-ready signal)}: This is the final bottleneck before decision-making. At this point, most of the irrelevant entropy has been discarded, and what remains is a high-fidelity encoding optimized for forecasting — a minimal but sufficient statistic, so to speak, for making predictions.

    \item \textbf{Final Prediction (distribution over outcomes)}: The model’s output is not a single number, but a distribution over possible futures — often expressed as class probabilities or conditional expectations. This represents the model’s internal belief, formed from compressed and transformed versions of the raw inputs.
\end{itemize}

\vspace{1em}
\textbf{Right-Side (Regularized Approximations) Explanations}

\begin{itemize}
    \item \textbf{Approximation of \( Z_1 \) (lower entropy)}: A simplified version of the initial encoding. This approximation strips out some complexity to prevent overfitting and helps guide the model to learn generalizable features. It’s like teaching with a textbook summary instead of an encyclopedia.

    \item \textbf{Approximation of \( Z_2 \) (regularized)}: A soft constraint on the second latent layer. The idea here is to keep the model from encoding too much nuance in the middle layers by anchoring them to distributions that reflect simpler or smoothed behavior — effectively acting as a form of structural prior.

    \item \textbf{Approximation of \( Z_3 \) (compressed)}: This approximation enforces parsimony just before the prediction head. By regularizing the penultimate layer, we force the model to make decisions using only what’s strictly necessary — leading to better generalization under distributional shift.

    \item \textbf{Approximate Prediction (reference distribution)}: A reference prediction used during training — often a softened, distilled, or averaged version of the true output distribution. This acts like a "teacher signal" to stabilize learning and align the model’s confidence with broader trends rather than local quirks.
\end{itemize}





\subsubsection{Solid Arrows: The Flow of Representation}

The vertical arrows on the left trace the model’s expressive path — how it builds increasingly abstract representations through conditional transformations:

\[
P(Z_1 \mid X), \quad P(Z_2 \mid Z_1), \quad P(Z_3 \mid Z_2), \quad P(P \mid Z_3).
\]

On the right, the vertical arrows represent a simplified model — one that passes through lower-dimensional or constrained approximations:

\[
Q(Z_2 \mid Z_1), \quad Q(Z_3 \mid Z_2), \quad Q(P \mid Z_3).
\]

\[
\resizebox{\textwidth}{!}{
\begin{tikzpicture}[>=latex, scale=1.1, every node/.style={scale=1.2}]

  % Define matrix for the true feature distributions P(Z | X)
  \matrix (m) [matrix of math nodes, row sep=3.5em, column sep=7em] {
      X \\ 
      Z_1 \\ 
      Z_2 \\ 
      Z_3 \\ 
      P \\ % Final Prediction
  };

  % Define the simplified model Q(Z | X) to the right
  \matrix (q) [matrix of math nodes, row sep=3.5em, column sep=7em, right of=m, node distance=8cm] {
      \\  % No simplified version for X
      Q(Z_1 | X) \\ 
      Q(Z_2 | Z_1) \\ 
      Q(Z_3 | Z_2) \\ 
      Q(P | Z_3) \\ % Simplified Prediction
  };

  % Feature Transformations (Left Column) with parbox annotations
  \path[->] (m-1-1) edge node[left=0.2cm] {\parbox{4cm}{\raggedleft\footnotesize \textit{Model builds abstraction}\\[-0.2em] $P(Z_1 \mid X)$}} (m-2-1);
  \path[->] (m-2-1) edge node[left=0.2cm] {\parbox{4cm}{\raggedleft\footnotesize \textit{Refines structure}\\[-0.2em] $P(Z_2 \mid Z_1)$}} (m-3-1);
  \path[->] (m-3-1) edge node[left=0.2cm] {\parbox{4cm}{\raggedleft\footnotesize \textit{Distills signal}\\[-0.2em] $P(Z_3 \mid Z_2)$}} (m-4-1);
  \path[->] (m-4-1) edge node[left=0.2cm] {\parbox{4cm}{\raggedleft\footnotesize \textit{Final forecast}\\[-0.2em] $P(P \mid Z_3)$}} (m-5-1);

  % Simplified Feature Representations (Right Column) with parbox annotations
  \path[->] (q-2-1) edge node[right=0.2cm] {\parbox{4cm}{\raggedright\footnotesize \textit{Simplified structure}\\[-0.2em] $Q(Z_2 \mid Z_1)$}} (q-3-1);
  \path[->] (q-3-1) edge node[right=0.2cm] {\parbox{4cm}{\raggedright\footnotesize \textit{Compressed signal}\\[-0.2em] $Q(Z_3 \mid Z_2)$}} (q-4-1);
  \path[->] (q-4-1) edge node[right=0.2cm] {\parbox{4cm}{\raggedright\footnotesize \textit{Baseline prediction}\\[-0.2em] $Q(P \mid Z_3)$}} (q-5-1);

  % KL Divergence Regularization (Dashed Arrows)
  \path[->, dashed] (m-2-1.east) edge node[above] {} (q-2-1.west);
  \path[->, dashed] (m-3-1.east) edge node[above] {} (q-3-1.west);
  \path[->, dashed] (m-4-1.east) edge node[above] {} (q-4-1.west);
  \path[->, dashed] (m-5-1.east) edge node[above] {} (q-5-1.west);

  % Phantom annotations for nodes (preserve spacing)
  \node[left=0.2cm of m-1-1] {\phantom{\parbox{4cm}{\raggedleft\footnotesize \textit{Raw Inputs}\\\textit{(e.g., order book, volume)}}}};
  \node[left=0.2cm of m-2-1] {\phantom{\parbox{4cm}{\raggedleft\footnotesize \textit{Latent Layer 1}\\\textit{(initial encoding)}}}};
  \node[left=0.2cm of m-3-1] {\phantom{\parbox{4cm}{\raggedleft\footnotesize \textit{Latent Layer 2}\\\textit{(refined representation)}}}};
  \node[left=0.2cm of m-4-1] {\phantom{\parbox{4cm}{\raggedleft\footnotesize \textit{Latent Layer 3}\\\textit{(prediction-ready signal)}}}};
  \node[left=0.2cm of m-5-1] {\phantom{\parbox{4cm}{\raggedleft\footnotesize \textit{Final Prediction}\\\textit{(distribution over outcomes)}}}};

  \node[right=0.2cm of q-2-1] {\phantom{\parbox{4cm}{\raggedright\footnotesize \textit{Approx. of $Z_1$}\\\textit{(lower entropy)}}}};
  \node[right=0.2cm of q-3-1] {\phantom{\parbox{4cm}{\raggedright\footnotesize \textit{Approx. of $Z_2$}\\\textit{(regularized)}}}};
  \node[right=0.2cm of q-4-1] {\phantom{\parbox{4cm}{\raggedright\footnotesize \textit{Approx. of $Z_3$}\\\textit{(compressed)}}}};
  \node[right=0.2cm of q-5-1] {\phantom{\parbox{4cm}{\raggedright\footnotesize \textit{Approx. Prediction}\\\textit{(reference dist.)}}}};

\end{tikzpicture}
}
\]

\subsubsection*{Qualitative Interpretations of Transformational Arrows}

\vspace{1em}
\textbf{Left Column (True Representation Path)}

\begin{itemize}
    \item \textbf{Model builds abstraction ($P(Z_1 \mid X)$)}\\
    The first transformation learns to represent the high-dimensional, noisy raw input \( X \) into a structured latent form \( Z_1 \). This step begins the process of abstraction: reducing dimensionality and identifying signal amidst the noise.

    \item \textbf{Refines structure ($P(Z_2 \mid Z_1)$)}\\
    The second layer takes the initial encoding and deepens the model’s understanding. Here, the representation is made more abstract and structured — capturing higher-order patterns, dependencies, or temporal dynamics in the data.

    \item \textbf{Distills signal ($P(Z_3 \mid Z_2)$)}\\
    The third layer strips the representation down to its essential predictive content. Non-essential variance is discarded, concentrating the signal into a form that is maximally informative for downstream prediction.

    \item \textbf{Final forecast ($P(P \mid Z_3)$)}\\
    The model uses the final latent representation to make its ultimate prediction. This step projects the abstract signal into a concrete probability distribution over future events — i.e., the model’s belief.
\end{itemize}

\vspace{1em}
\textbf{Right Column (Simplified Reference Path)}

\begin{itemize}
    \item \textbf{Simplified structure ($Q(Z_2 \mid Z_1)$)}\\
    A lower-complexity approximation of the second latent transformation. This version intentionally reduces detail and expressiveness — guiding the model to learn smoother, more generalizable features at the cost of some precision.

    \item \textbf{Compressed signal ($Q(Z_3 \mid Z_2)$)}\\
    A compact representation of the third latent layer, designed to act as a bottleneck. It encourages the model to express relevant information with fewer bits — enforcing parsimony and robustness to noise.

    \item \textbf{Baseline prediction ($Q(P \mid Z_3)$)}\\
    A simplified predictive distribution derived from the compressed latent. Often serves as a regularization target — anchoring the model's final prediction toward a conservative, low-variance baseline during training.
\end{itemize}



\subsubsection{Dashed Arrows: Divergence as a Compression Constraint}

The horizontal dashed arrows represent the core regulatory mechanism of this system:

\[
D_{KL}\big(P(Z_i \mid \cdot) \, \| \, Q(Z_i \mid \cdot)\big),
\]

which measure how far the expressive model has deviated from its compressed baseline. These divergences are not errors — they are costs. And the model pays them only when the information gained is worth the entropy spent.

Each KL divergence acts like a tax on complexity. The model must “justify” every bit of additional structure by proving that it contributes to the predictive integrity of the system. Otherwise, the regularized approximation pulls the model back toward simplicity.

\begin{quote}
\textit{What emerges is a balance: expressive enough to model causality, but compressed enough to generalize. Each layer doesn't just learn — it negotiates.}
\end{quote}



\[
\resizebox{\textwidth}{!}{
\begin{tikzpicture}[>=latex, scale=1.1, every node/.style={scale=1.2}]

  % Define matrix for the true feature distributions P(Z | X)
  \matrix (m) [matrix of math nodes, row sep=3.5em, column sep=7em] {
      X \\ 
      Z_1 \\ 
      Z_2 \\ 
      Z_3 \\ 
      P \\ % Final Prediction
  };

  % Define the simplified model Q(Z | X) to the right
  \matrix (q) [matrix of math nodes, row sep=3.5em, column sep=7em, right of=m, node distance=8cm] {
      \\  % No simplified version for X
      Q(Z_1 | X) \\ 
      Q(Z_2 | Z_1) \\ 
      Q(Z_3 | Z_2) \\ 
      Q(P | Z_3) \\ % Simplified Prediction
  };

  % Feature Transformations (Left Column) — now phantom
  \path[->] (m-1-1) edge node[left=0.2cm] {\phantom{\parbox{4cm}{\raggedleft\footnotesize \textit{Model builds abstraction}\\[-0.2em] $P(Z_1 \mid X)$}}} (m-2-1);
  \path[->] (m-2-1) edge node[left=0.2cm] {\phantom{\parbox{4cm}{\raggedleft\footnotesize \textit{Refines structure}\\[-0.2em] $P(Z_2 \mid Z_1)$}}} (m-3-1);
  \path[->] (m-3-1) edge node[left=0.2cm] {\phantom{\parbox{4cm}{\raggedleft\footnotesize \textit{Distills signal}\\[-0.2em] $P(Z_3 \mid Z_2)$}}} (m-4-1);
  \path[->] (m-4-1) edge node[left=0.2cm] {\phantom{\parbox{4cm}{\raggedleft\footnotesize \textit{Final forecast}\\[-0.2em] $P(P \mid Z_3)$}}} (m-5-1);

  % Simplified Feature Representations (Right Column) — now phantom
  \path[->] (q-2-1) edge node[right=0.2cm] {\phantom{\parbox{4cm}{\raggedright\footnotesize \textit{Simplified structure}\\[-0.2em] $Q(Z_2 \mid Z_1)$}}} (q-3-1);
  \path[->] (q-3-1) edge node[right=0.2cm] {\phantom{\parbox{4cm}{\raggedright\footnotesize \textit{Compressed signal}\\[-0.2em] $Q(Z_3 \mid Z_2)$}}} (q-4-1);
  \path[->] (q-4-1) edge node[right=0.2cm] {\phantom{\parbox{4cm}{\raggedright\footnotesize \textit{Baseline prediction}\\[-0.2em] $Q(P \mid Z_3)$}}} (q-5-1);

  % KL Divergence Regularization (Dashed Arrows) with parbox annotations
  \path[->, dashed] (m-2-1.east) edge node[above=0.8em] {\parbox{6cm}{\centering\footnotesize \textit{KL penalty encourages simpler $Z_1$}\\$D_{KL}(P(Z_1 \mid X) \,\|\, Q(Z_1 \mid X))$}} (q-2-1.west);
  \path[->, dashed] (m-3-1.east) edge node[above=0.8em] {\parbox{6cm}{\centering\footnotesize \textit{Complexity tax for $Z_2$ refinement}\\$D_{KL}(P(Z_2 \mid Z_1) \,\|\, Q(Z_2 \mid Z_1))$}} (q-3-1.west);
  \path[->, dashed] (m-4-1.east) edge node[above=0.8em] {\parbox{6cm}{\centering\footnotesize \textit{Incentivizes compact $Z_3$}\\$D_{KL}(P(Z_3 \mid Z_2) \,\|\, Q(Z_3 \mid Z_2))$}} (q-4-1.west);
  \path[->, dashed] (m-5-1.east) edge node[above=0.8em] {\parbox{6cm}{\centering\footnotesize \textit{Tradeoff between fidelity and generalization}\\$D_{KL}(P(P \mid Z_3) \,\|\, Q(P \mid Z_3))$}} (q-5-1.west);

  % Phantom annotations for nodes (preserve spacing)
  \node[left=0.2cm of m-1-1] {\phantom{\parbox{4cm}{\raggedleft\footnotesize \textit{Raw Inputs}\\\textit{(e.g., order book, volume)}}}};
  \node[left=0.2cm of m-2-1] {\phantom{\parbox{4cm}{\raggedleft\footnotesize \textit{Latent Layer 1}\\\textit{(initial encoding)}}}};
  \node[left=0.2cm of m-3-1] {\phantom{\parbox{4cm}{\raggedleft\footnotesize \textit{Latent Layer 2}\\\textit{(refined representation)}}}};
  \node[left=0.2cm of m-4-1] {\phantom{\parbox{4cm}{\raggedleft\footnotesize \textit{Latent Layer 3}\\\textit{(prediction-ready signal)}}}};
  \node[left=0.2cm of m-5-1] {\phantom{\parbox{4cm}{\raggedleft\footnotesize \textit{Final Prediction}\\\textit{(distribution over outcomes)}}}};

  \node[right=0.2cm of q-2-1] {\phantom{\parbox{4cm}{\raggedright\footnotesize \textit{Approx. of $Z_1$}\\\textit{(lower entropy)}}}};
  \node[right=0.2cm of q-3-1] {\phantom{\parbox{4cm}{\raggedright\footnotesize \textit{Approx. of $Z_2$}\\\textit{(regularized)}}}};
  \node[right=0.2cm of q-4-1] {\phantom{\parbox{4cm}{\raggedright\footnotesize \textit{Approx. of $Z_3$}\\\textit{(compressed)}}}};
  \node[right=0.2cm of q-5-1] {\phantom{\parbox{4cm}{\raggedright\footnotesize \textit{Approx. Prediction}\\\textit{(reference dist.)}}}};

\end{tikzpicture}
}
\]


\subsubsection*{Qualitative Interpretation of KL Divergence Constraints}

\begin{itemize}
    \item \textbf{KL penalty encourages simpler $Z_1$}\\
    The first KL term, $D_{KL}(P(Z_1 \mid X) \,\|\, Q(Z_1 \mid X))$, measures how much complexity the model introduces when mapping from raw input to its initial encoding. A large divergence implies that $P(Z_1 \mid X)$ is significantly more expressive than the compressed approximation $Q(Z_1 \mid X)$. The model is therefore penalized for injecting unnecessary complexity early in the pipeline, incentivizing it to construct minimal yet sufficient initial representations.

    \item \textbf{Complexity tax for $Z_2$ refinement}\\
    As the model refines its representation into $Z_2$, the corresponding KL divergence $D_{KL}(P(Z_2 \mid Z_1) \,\|\, Q(Z_2 \mid Z_1))$ acts as a tax on the added structure. If $Z_2$ introduces subtle correlations or features beyond the scope of $Q(Z_2 \mid Z_1)$, the model must “pay” in terms of complexity. This constraint discourages overfitting and enforces that any refinement contributes genuine signal, not just noise fitting.

    \item \textbf{Incentivizes compact $Z_3$}\\
    The next divergence, $D_{KL}(P(Z_3 \mid Z_2) \,\|\, Q(Z_3 \mid Z_2))$, ensures that the most abstract latent layer, $Z_3$, remains compact and purpose-driven. It pushes the model to collapse redundant or noisy representations, retaining only information that directly contributes to downstream predictive power. This step is crucial for producing robust representations that generalize well beyond training data.

    \item \textbf{Tradeoff between fidelity and generalization}\\
    The final divergence, $D_{KL}(P(P \mid Z_3) \,\|\, Q(P \mid Z_3))$, governs the prediction layer itself. It enforces a balance between the model’s detailed internal belief $P(P \mid Z_3)$ and a more regularized output $Q(P \mid Z_3)$. In effect, this KL term serves as a check on the confidence and calibration of predictions — encouraging the model to be precise, but not brittle or overconfident. It reflects the ultimate tradeoff between expressivity and generalization.
\end{itemize}









\subsection{A Concrete Example: KL Divergence in High-Frequency Trading Decisions}

To see these abstract principles in action, let’s return to the domain where spurious causality and signal filtering matter most: high-frequency trading.

Imagine a trading system composed of thousands of machines, each attempting to forecast microsecond-level price movements using deep learning models. These models operate in a volatile, feedback-heavy environment — flooded with noisy data, evolving market structures, and ambiguous causal signals. The core challenge isn’t just prediction — it’s identifying which features truly matter, and discarding those that only appear to.

This is exactly where KL divergence becomes a tool for epistemic hygiene.

\subsubsection{Scenario: Trading on Volume Spikes}

Suppose an HFT model attempts to predict whether a stock’s price will increase in the next 10 milliseconds. It takes in the following features:

\begin{itemize}
    \item \( X_1 \): Trade volume over the past 10 milliseconds,
    \item \( X_2 \): Current bid-ask spread,
    \item \( X_3 \): Execution speed of the last 100 trades.
\end{itemize}

At first glance, trade volume spikes often precede price jumps. But sometimes they don't. They might be caused by unrelated bursts of algorithmic activity, liquidity mirages, or other traders reacting to hidden market signals. The model’s task isn’t just to identify these patterns — it’s to determine whether the pattern holds causally, in context.

\subsubsection{Step 1: Learning the Full Predictive Distribution}

The model initially estimates the full conditional distribution:

\[
P(Y \mid X) = P(\text{Price Increase} \mid X_1, X_2, X_3),
\]

learning from the joint statistical structure of recent trading activity. But the challenge is: which features actually carry signal, and which encode noise or reactive drift?

\subsubsection{Step 2: Measuring Feature Relevance with KL Divergence}

To isolate the importance of context, we construct a restricted baseline model:

\[
Q(Y \mid X_1) = P(\text{Price Increase} \mid X_1),
\]

which reflects a naive world where only volume matters. We then compute the KL divergence between the two distributions:

\[
D_{KL}(P(Y \mid X_1, X_2, X_3) \,\|\, Q(Y \mid X_1)).
\]

This measures how much explanatory power is lost when we compress the model and discard \( X_2 \) and \( X_3 \).

\textbf{Two interpretations:}
\begin{itemize}
    \item A high KL divergence implies that the bid-ask spread and execution speed provide essential causal structure. Dropping them would erase signal.
    \item A low KL divergence suggests those features are no longer informative — perhaps the market regime has changed, or the features were spurious to begin with.
\end{itemize}

\subsubsection{Step 3: Adaptive Estimation in a Shifting Measure Space}

Over time, as the market evolves, so does the underlying event space \( \Omega_t \). New information flows in, models retrain, and features that once mattered may no longer do so.

By tracking \( D_{KL} \) dynamically, the trading system learns which components of its internal information geometry are stable — and which are starting to drift.

\begin{itemize}
    \item If a surge in algorithmic trading floods the order book, past execution speed \( X_3 \) may become chaotic and irrelevant. KL divergence drops.
    \item If macroeconomic news affects liquidity, the bid-ask spread \( X_2 \) may become more predictive than it was before. KL divergence rises.
\end{itemize}

\textbf{In essence, the model performs online causal filtering.} It continuously compresses its feature space using KL divergence as a guide — keeping only what survives the pressure of real-world entropy.

\begin{quote}
\textit{Deep learning gives us the estimator. KL divergence tells us what it's still getting wrong. And in high-frequency markets, that’s the difference between trading on truth — and chasing noise.}
\end{quote}





\subsection{The Benefit: Trade Execution as a Consequence of Information Geometry}

Mathematics, when properly aligned with market structure, doesn’t just describe reality — it improves performance. By applying KL divergence as an adaptive filter within the model’s information space, we don’t merely fine-tune parameters — we reshape the estimator’s understanding of causality in a dynamic, noisy environment.

\textbf{But how does this translate economically?}

Let’s quantify the impact of disciplined, KL-regularized learning in a high-frequency trading system.

\begin{itemize}
    \item Without KL divergence, the model executes 10,000 trades per day at a modest 52\% success rate — slightly better than chance, but still prone to noise and false signals.
    \item After incorporating KL-based compression — pruning irrelevant features, isolating causal structure — the success rate climbs to 54\%.
    \item Assume each trade moves 100 shares, with an average profit of \$0.005 per share.
\end{itemize}

The gain from this small shift in predictive precision is:

\[
(0.54 - 0.52) \times 10{,}000 \times 100 \times 0.005 = \mathbf{\$1{,}000 \text{ additional profit per day}}.
\]

Over a standard trading year, this refinement adds up to nearly:

\[
\$1{,}000 \times 250 = \mathbf{\$250{,}000}.
\]

All from refining the internal structure of the model — not by adding data, not by increasing volume, but by compressing irrelevant information and converging on the features that carry weight under a shifting market measure.

\begin{quote}
\textit{KL divergence transforms learning into adaptive reasoning. The result isn’t just more accurate models — it’s more profitable trades.}
\end{quote}

\subsection{Extracting General Patterns: KL Divergence as the Geometry of Learning}

The high-frequency trading example was not just a technical case study — it was a microcosm of a much deeper principle. The role KL divergence plays in shaping the learning process reveals a general framework for intelligent behavior under uncertainty: a system evolves by continuously compressing information, pruning noise, and adapting its internal structure to remain aligned with a changing measure space.

\subsubsection{The General Pattern: Learning as Compression in Information Space}

Across deep learning models — and especially in real-time systems like algorithmic trading — the learning process follows a consistent structure:

\begin{enumerate}
    \item \textbf{Feature Extraction:} Raw input data is mapped into high-dimensional latent variables through successive transformations — each layer reshaping the measure on \( \Omega \).
    
    \item \textbf{Information Regularization:} KL divergence is applied as a constraint — limiting the complexity of the latent space and preventing overfitting to ephemeral patterns.
    
    \item \textbf{Dynamic Recalibration:} As the environment shifts, the model recalculates divergences — continuously aligning its internal geometry with the external distribution.
    
    \item \textbf{Causal Filtering:} KL divergence acts as a structural sieve, distinguishing stable causal pathways from reactive noise, and ensuring that what the model learns actually holds under transformation.
\end{enumerate}

\subsubsection{Why This Pattern Reflects Intelligence}

This pattern — compress, test, refine — is not just an engineering artifact. It mirrors how reasoning systems operate under uncertainty. Whether we’re optimizing a neural network or a trading algorithm, or even pruning synapses in the brain, intelligent systems evolve by discarding what does not help them predict or survive.

\begin{itemize}
    \item \textbf{Too little compression} (\( D_{KL} \approx 0 \)): The model retains nearly all information — including noise — and overfits to microstructure.
    
    \item \textbf{Too much compression} (\( D_{KL} \gg 0 \)): The model throws away signal, collapsing useful structure and underfitting the environment.
    
    \item \textbf{Optimal compression} (\( D_{KL} \approx \text{minimal sufficient} \)): The model retains only the structure necessary for causal prediction, balancing precision with generality.
\end{itemize}

\begin{quote}
\textit{In this view, KL divergence doesn’t just regularize learning — it traces the boundary between belief and overreach, between pattern and illusion.}
\end{quote}







\subsection{A Minimal VIB Estimator: Code as Information Geometry}

To ground the previous discussion in practice, here is a minimal implementation of a Variational Information Bottleneck neural network. This code operationalizes the concepts from earlier sections:

\begin{itemize}
    \item The encoder \( \phi \) maps inputs from event space \( X \subset \Omega \) into a latent distribution \( P(Z \mid X) \).
    \item A latent variable \( Z \) is sampled using the reparameterization trick, approximating the true causal map.
    \item The decoder \( \psi \) maps \( Z \) to predicted outcomes \( Y \), modeling the downstream effect.
    \item KL divergence \( D_{KL}(P(Z \mid X) \,\|\, \mathcal{N}(0, I)) \) penalizes excess complexity—enforcing compression in information space.
\end{itemize}


\begin{lstlisting}[language=Python, caption={Minimal Variational Information Bottleneck in PyTorch}, label={lst:vib_model}]
import torch
import torch.nn as nn
import torch.nn.functional as F

# KL divergence from N(mu, sigma^2) to standard normal
def kl_divergence(mu, logvar):
    return -0.5 * torch.sum(1 + logvar - mu.pow(2) - logvar.exp())

# Neural estimator under VIB framework
class VIBNet(nn.Module):
    def __init__(self, input_dim=3, hidden_dim=16, latent_dim=2, output_dim=1):
        super(VIBNet, self).__init__()

        # Encoder: raw features X --> latent mean and variance
        self.encoder = nn.Sequential(
            nn.Linear(input_dim, hidden_dim),
            nn.ReLU()
        )
        self.mu = nn.Linear(hidden_dim, latent_dim)
        self.logvar = nn.Linear(hidden_dim, latent_dim)

        # Decoder: latent Z --> predicted output Y
        self.decoder = nn.Sequential(
            nn.Linear(latent_dim, hidden_dim),
            nn.ReLU(),
            nn.Linear(hidden_dim, output_dim),
            nn.Sigmoid()  # For binary outcomes (e.g. price up/down)
        )

    def forward(self, x):
        h = self.encoder(x)
        mu = self.mu(h)
        logvar = self.logvar(h)

        # Sample Z ~ N(mu, sigma^2) via reparameterization trick
        std = torch.exp(0.5 * logvar)
        eps = torch.randn_like(std)
        z = mu + eps * std

        y_hat = self.decoder(z)
        kl = kl_divergence(mu, logvar)
        return y_hat, kl
\end{lstlisting}

\noindent
This model is trained by minimizing a loss function that combines:

\[
\mathcal{L} = \mathbb{E}_{X, Y}[\text{Prediction Loss}] + \beta \cdot D_{KL}(P(Z \mid X) \,\|\, \mathcal{N}(0, I)),
\]

where \( \beta \) controls the tradeoff between information retention and compression.

\begin{lstlisting}[language=Python, caption={Training loop with KL regularization}, label={lst:train_loop}]
model = VIBNet()
criterion = nn.BCELoss()
optimizer = torch.optim.Adam(model.parameters(), lr=1e-3)
beta = 1e-3  # controls compression strength

# Example training step
for batch_x, batch_y in dataloader:
    optimizer.zero_grad()
    pred_y, kl = model(batch_x)
    loss = criterion(pred_y, batch_y) + beta * kl
    loss.backward()
    optimizer.step()
\end{lstlisting}

\begin{quote}
\textit{This is not just code—it’s the operationalization of our information geometry.}
It estimates a function over \( (\Omega, \mathcal{F}, \mu) \), compresses irrelevant structure via KL divergence, and adapts to shifting measure spaces \( \mu_t \) as market regimes change.
\end{quote}




\subsubsection{Why This Matters in High-Frequency Trading}

In high-frequency trading, the difference between causality and coincidence can be measured in microseconds — and millions of dollars. The structured compression strategy we've outlined isn't just theoretical elegance; it's an operational necessity.

By embedding KL divergence into the learning architecture, trading models gain the ability to:

\begin{itemize}
    \item \textbf{Suppress reactive noise:} Irrelevant fluctuations and non-causal feedback are naturally pruned, reducing overfitting to transient artifacts in market microstructure.

    \item \textbf{Prioritize causal signal:} Only those latent features that contribute to predictive, high-impact decisions survive the compression process. The rest are discarded under information-theoretic pressure.

    \item \textbf{Adapt in real time:} As the market measure evolves — due to institutional flows, policy shocks, or shifting liquidity — KL divergence highlights when once-useful features begin to decay, prompting dynamic reconfiguration.
\end{itemize}

In essence, KL divergence is not just a regularizer — it's a memory mechanism, a filter, and a real-time map of which parts of the world the model should pay attention to.

\subsection{Final Takeaway: KL Divergence as the Geometry of Intelligence}

Throughout this section, we’ve seen KL divergence evolve from a statistical curiosity into a full-fledged epistemic compass — one that guides deep learning systems through the complexity of modern data environments.

\textbf{In high-frequency trading — and in learning systems more broadly — the process looks like this:}

\begin{enumerate}
    \item \textbf{Extract structure:} Begin with raw, unstructured inputs. Encode the world through conditional distributions that reflect causal flow.
    
    \item \textbf{Compress responsibly:} Apply KL divergence as a constraint — forcing the model to justify complexity and penalize false patterns.
    
    \item \textbf{Adapt continually:} Monitor divergence over time. Use it to detect regime shifts, concept drift, or decaying features. Let the model realign as the environment evolves.
\end{enumerate}

\begin{quote}
\textit{KL divergence is not just how models learn — it's how they remember what matters.} In markets, in machines, and even in minds, intelligence emerges not from seeing everything — but from knowing what to forget.
\end{quote}

%\section{How to Train a Neural Network (When the World is a Measure Space)}

Let’s bring it full circle.

Earlier, we saw how different signal decomposition methods — Fourier vs. Z-transform — produced different entropies, different interpretations, and ultimately, different answers. The method you choose frames what kind of structure you’re allowed to see.

\textbf{The same logic applies to neural networks.}

But now we’re not just breaking down signals. We’re integrating over complex, dynamic sets — causality sets, latent geometries, high-frequency data clouds shaped by real-world constraints.

So the question becomes:

\begin{quote}
\textit{If the world is a measure space, how do we train a neural network to navigate it?}
\end{quote}

We’re about to unpack that — piece by piece.

\subsection{Step 1: The Analogy — From Signal Decomposition to Information Geometry}

Signal processing has its transforms: Fourier, Z, wavelets.

Neural modeling has its architectures: feedforward nets, recurrent nets, variational encoders.

In signal land, the transformation determines the entropy landscape.

In measure-theoretic learning, the architecture determines the \textit{information topology} — what regions of your event space are prioritized, compressed, or expanded.

\begin{quote}
Choosing a neural network isn’t just about “what works.” It’s about choosing the right lens on your measure space.
\end{quote}

\subsection{Step 2: Define the Domain — What Are You Modeling?}

Every model begins with a domain — a space of events, inputs, or interactions.

Ask yourself:
\begin{itemize}
  \item Are you modeling physical dynamics? (e.g., fluid flows, particles)
  \item Are you modeling probabilistic uncertainty? (e.g., price outcomes, risk distributions)
  \item Are you generating new samples? (e.g., synthetic data, imagined futures)
\end{itemize}

Each case leads to a different type of estimator — and therefore, a different kind of neural net.

\subsection{Step 3: Choose Your Strategy — Architectures as Integrators}

Once the space is defined, the architecture becomes your integrator. Here’s where the fun begins.

\subsubsection*{Physics-Informed Neural Networks (PINNs)}

What if your domain is governed by physical laws?

PINNs embed differential equations \textit{into the loss function} — turning physical constraints into gradients. You’re not just fitting data — you’re respecting conservation, boundary conditions, and governing dynamics.

\subsubsection*{Generative Adversarial Networks (GANs)}

What if your task is generation?

GANs create a two-player game between generator and discriminator, forcing each to learn the structure of a hidden distribution. The generator estimates a measure-preserving transformation; the discriminator acts as an epistemic critic.

\subsubsection*{Bayesian and Maximum Entropy Neural Nets}

What if uncertainty is the star?

Here, the network estimates not just outputs but belief distributions. Maximum entropy priors enforce humility. Bayesian posteriors track what’s learned — and what still needs data.

\subsection{Step 4: Structure the Loss — What Are You Minimizing?}

Every neural network optimizes a loss. But in measure-theoretic terms, this loss is often an integral:

\[
\min_\theta \int \mathcal{L}(f_\theta(x), y) \, d\mu(x)
\]

This raises deep questions:
\begin{itemize}
  \item What is your \textit{true} measure space \( \mu \)?
  \item Are you under-sampling rare but high-impact events?
  \item Should your loss include information constraints (like KL divergence)?
\end{itemize}

Loss functions aren’t just recipes — they’re declarations of what matters.

\subsection{Step 5: Train, Adapt, Refine — All in Information Space}

Finally, training becomes an iterative flow through information space.

Each update shifts the model’s geometry:
\begin{itemize}
  \item Filtering entropy.
  \item Reshaping density.
  \item Learning causal flow.
\end{itemize}

What emerges is a neural estimator that doesn’t just fit — it reasons.

\begin{quote}
\textit{If data is drawn from a measure space, then training is a form of integration. The neural net becomes a calculus of belief.}
\end{quote}

\vspace{1em}
\noindent
In the next section, we’ll zoom in on each architecture, and show what it means to use them — not just as tools, but as \textit{lenses on uncertainty}.

%\section{How to Verify a Neural Network (When the World Bites Back)}

Training is an act of belief formation — an integration over data, a shaping of densities. But belief, once formed, invites scrutiny.

\begin{quote}
\textit{Verification is not about learning — it’s about accountability.}
\end{quote}

If the world is a measure space, then verifying a neural network means asking:  
\textit{Have we learned the right geometry? Are we safe under perturbation? What hides in the corners of our event space?}

\subsection{Step 1: Define the Threat Model — What Could Go Wrong?}

Before verification can begin, we need to define the kind of failure we fear:

\begin{itemize}
  \item \textbf{Adversarial perturbations} — Can tiny changes in the input lead to large semantic changes in the output?
  \item \textbf{Domain shift} — Does the model generalize to regions of the space it never trained on?
  \item \textbf{Specification violations} — Are there any inputs for which the model outputs something it fundamentally should not?
\end{itemize}

Verification begins with a question:  
\textit{What class of counterexamples are you trying to rule out?}

\subsection{Step 2: Express the Model as Logic — From Activation to Assertion}

Most neural networks are trained as numeric functions. But to verify them, we step outside that world — into logic.

Each layer becomes a constraint:
\begin{itemize}
  \item Linear transforms become affine relations.
  \item ReLU activations become piecewise cases: \( x \leq 0 \Rightarrow y = 0 \), \( x > 0 \Rightarrow y = x \).
  \item Output bounds become formal specifications.
\end{itemize}

You’re no longer optimizing — you’re solving.  
What you get is a logical system whose satisfiability corresponds to a safety violation.

\subsection{Step 3: Search the Space — Verification as a SAT/SMT Problem}

Now comes the heart of the process.

Verifiers like \textbf{Reluplex}, \textbf{Marabou}, and others turn the network into a hybrid of:
\begin{itemize}
  \item SAT solvers (Boolean search)
  \item SMT solvers (arithmetic constraints)
  \item Symbolic case-splitting engines (for ReLU and piecewise logic)
\end{itemize}

The system explores possible combinations of activation states, input bounds, and output specifications, asking:

\begin{quote}
\textit{Is there a path through the network that leads to contradiction?}
\end{quote}

If yes — you’ve found a counterexample.  
If no — the network holds under your defined constraints.

\subsection{Step 4: Complexity Lives Here — When Proofs Get Expensive}

The cost of verification grows quickly.

Each ReLU node doubles the number of regions to consider.  
Verifying a network with \( n \) ReLUs can, in the worst case, require checking up to \( 2^n \) activation patterns.

This is more than just slow — it’s a fundamental boundary:
\begin{itemize}
  \item Finding a counterexample is often \textbf{NP-complete}.
  \item Proving none exists may be \textbf{co-NP-complete} — or worse.
\end{itemize}

So we choose approximations:
\begin{itemize}
  \item Abstract interpretation to bound values.
  \item Convex relaxations to simplify regions.
  \item Incomplete solvers that catch most (but not all) bugs.
\end{itemize}

\subsection{Step 5: Interrogate the Measure — Verification as Risk Analysis}

Finally, we return to the measure space.

Your verification region is a subset \( S \subseteq \mathcal{X} \).  
But is that region:
\begin{itemize}
  \item Representative of the real-world data distribution?
  \item Weighted properly with respect to rare but catastrophic events?
  \item Inclusive of causal anomalies, out-of-distribution examples, and adversarial cases?
\end{itemize}

\begin{quote}
\textit{Verification isn’t just about finding bugs — it’s about mapping the tails of your uncertainty.}
\end{quote}

Just as training is integration over observed data, verification is **exploration of edge cases** — the dark matter of your input space.

---

\vspace{1em}
\noindent
In the next section, we’ll contrast different verification tools, show where their strengths lie, and explain how to choose the right one depending on the risks your network faces.

%\section{Choosing the Right Verifier (When Risk Has a Shape)}

Verification isn’t a one-size-fits-all task. The right tool depends on what kind of error you're guarding against, how fast you need an answer, and what complexity you're willing to entertain.

In this section, we’ll compare the major verification tools used for neural networks — each with its own philosophy, trade-offs, and domain of relevance.

\subsection{Reluplex — Precision in the Piecewise}

\textbf{Reluplex} was one of the first solvers designed specifically for ReLU-activated networks. Built on top of the classical simplex algorithm, it adds support for piecewise-linear constraints induced by ReLU activations.

\textbf{Strengths}:
\begin{itemize}
  \item Precise handling of linear and piecewise-linear logic.
  \item Sound and complete (for the networks it can handle).
\end{itemize}

\textbf{Limitations}:
\begin{itemize}
  \item Poor scalability — exponential blow-up with network size.
  \item Not well-suited for deep or convolutional networks.
\end{itemize}

Use Reluplex when:
\begin{quote}
You need exact verification on small, fully connected feedforward networks — especially in safety-critical systems where false negatives are unacceptable.
\end{quote}

\subsection{Marabou — Modular, Scalable, SMT Hybrid}

\textbf{Marabou} evolves Reluplex into a more modular SMT-style solver. It separates constraint solving from network architecture and adds better heuristics for case splitting.

\textbf{Strengths}:
\begin{itemize}
  \item More scalable than Reluplex.
  \item Modular design supports new activation functions and optimizations.
\end{itemize}

\textbf{Limitations}:
\begin{itemize}
  \item Still exponential in worst-case scenarios.
  \item Less precise in floating-point domains.
\end{itemize}

Use Marabou when:
\begin{quote}
You want a more flexible, extensible platform for verifying mid-sized networks — or when experimenting with different input domains and perturbation models.
\end{quote}

\subsection{AI\textsuperscript{2} and Planet — Overapproximation at Scale}

\textbf{AI\textsuperscript{2}} (Abstract Interpretation for AI) and \textbf{Planet} take a different approach: they use abstract domains or convex relaxations to overapproximate the behavior of ReLU networks.

\textbf{Strengths}:
\begin{itemize}
  \item Fast — suitable for deeper networks and industrial applications.
  \item Scales to larger architectures (e.g., CNNs).
\end{itemize}

\textbf{Limitations}:
\begin{itemize}
  \item Incomplete — may miss real counterexamples.
  \item Overapproximation can lead to false alarms.
\end{itemize}

Use AI\textsuperscript{2} or Planet when:
\begin{quote}
You care more about high-throughput screening of possible vulnerabilities than formal guarantees — or when you're running many checks in a continuous integration pipeline.
\end{quote}

\subsection{ERAN and DeepPoly — Hybrid Abstraction with Refinement}

\textbf{ERAN} uses zonotope and DeepPoly abstract domains to combine speed and relative precision. It is one of the most practical tools for verifying modern deep networks trained in PyTorch or TensorFlow.

\textbf{Strengths}:
\begin{itemize}
  \item Supports common network architectures.
  \item Can refine over-approximations dynamically.
\end{itemize}

\textbf{Limitations}:
\begin{itemize}
  \item Still limited on very deep or highly recurrent networks.
  \item Requires tuning of abstraction parameters.
\end{itemize}

Use ERAN when:
\begin{quote}
You want fast, approximate verification with feedback on uncertainty — particularly for evaluating robustness and safety in production-trained models.
\end{quote}

\subsection{So, Which One Do You Choose?}

Let the risk shape your verifier.

\begin{itemize}
  \item \textbf{Safety-critical system with small network?} Use \textbf{Reluplex} or \textbf{Marabou}.
  \item \textbf{Large image classifier?} Start with \textbf{AI\textsuperscript{2}}, \textbf{Planet}, or \textbf{ERAN}.
  \item \textbf{Do you care more about formal proof or high coverage?} That’s your tradeoff between soundness and scalability.
  \item \textbf{Need to run nightly regressions for edge-case bugs?} Use abstract verifiers as filters, and apply precise solvers only when needed.
\end{itemize}

\begin{quote}
\textit{There is no universal verifier — only verifiers appropriate to your universe.}
\end{quote}

\vspace{1em}
\noindent
In the next section, we’ll zoom in on how verification interacts with adversarial robustness — and how to design networks with verification in mind.

%\section{Processor Type and Ecosystem: Why Hardware Shapes the Network You Build}

We’ve talked about training cost, information geometry, and the economics of model design. But there’s an even more fundamental question:

\textbf{What kind of neural network should you build in the first place?}

The answer is not just “whatever gives good accuracy.” It depends — heavily — on the kind of processor you’re deploying to. And not just in terms of FLOPs per second. We're talking about deep architectural compatibility: whether your network plays to the strengths of the hardware, or fights it at every layer.
\subsection{A Brief History of Processors: From Von Neumann to Neural Engines}

Before diving into which processor is best for which model, it helps to understand how we got here. The story of hardware isn’t just about speed or size—it’s about architectural philosophy, economics, and even gaming culture.

\medskip

\noindent\textbf{It all began in the 1940s with two competing visions:} the \textbf{Von Neumann architecture} (1945) and the \textbf{Harvard architecture} (1944). The key difference lies in how they handle memory. Von Neumann systems use a single memory space for both instructions and data, while Harvard systems keep them separate. In theory, Harvard architectures can be faster and safer, but Von Neumann machines were simpler to build and more flexible.

Eventually, Von Neumann won out—especially in the world of general-purpose computing. Intel based its processors on this model, and by the 1990s, the \textbf{x86 instruction set architecture} had become the de facto standard for consumer and enterprise CPUs. Alternatives like SPARC (1987) and MIPS (1985) had their day, but market forces and software momentum favored x86. For decades, the future of computing looked like a straight line—smaller transistors, faster clocks, better CPUs.

\medskip

\noindent\textbf{Then came the smartphone boom in the late 2000s.} Suddenly, power consumption mattered. You couldn’t just make a chip faster—you had to make it efficient. That’s where the Harvard-style design returned, now reborn in \textbf{ARM processors}. These chips, with their low power draw and instruction simplicity, were perfect for mobile devices. ARM’s architecture began in the 1980s (Acorn RISC Machine, 1985) but came into dominance with the smartphone era around 2007–2010. Today, most smartphones—whether Android or iPhone—run on chips built around some variant of ARM’s Harvard-inspired architecture.

\medskip

\noindent\textbf{Meanwhile, another revolution was brewing—in pixels.} Graphics Processing Units (GPUs) were originally designed to render high-resolution visuals for video games. NVIDIA launched its first GPU in 1999 (the GeForce 256), and by the early 2000s, GPUs were becoming essential for gaming. At first, they were a niche product—mostly for gamers, many of whom were willing to spend lavishly for smoother frame rates and sharper textures.

\textbf{Then came crypto.} Around 2013–2017, the rise of cryptocurrency introduced a new motive: \textbf{squeezing every last drop of performance out of your hardware}. GPU sales exploded. And just as the crypto wave hit, another opportunity arrived: deep learning.

It turned out that what made GPUs good at graphics—\textbf{massively parallel matrix multiplication}—was also exactly what made them perfect for training neural networks. The deep learning boom of 2012 (with AlexNet) aligned perfectly with NVIDIA’s CUDA platform (introduced in 2006), which made GPUs programmable for general-purpose computing. NVIDIA, in particular, was well positioned to capitalize on this shift.

\medskip

\noindent\textbf{Then came Apple.} Seeking more control over their hardware and user experience, Apple began designing its own processors with the launch of the \textbf{A-series chips}. Starting with the \textbf{A11 Bionic} in 2017, Apple introduced the \textbf{Neural Engine}—a dedicated hardware block specifically optimized for machine learning tasks. This was designed to accelerate on-device intelligence like Face ID, Animoji, and Siri's voice recognition, all while preserving battery life.

\medskip

\noindent Each new chip iteration expanded the capabilities of the Neural Engine:

\begin{itemize}
    \item \textbf{A11 Bionic (2017):} First appearance of the Neural Engine (2 cores, up to 600 billion operations per second)
    \item \textbf{A12 Bionic (2018):} Improved Neural Engine (8 cores, 5 trillion operations per second)
    \item \textbf{A13 Bionic (2019):} Even faster inference, better integration with the CPU and GPU
    \item \textbf{A14 Bionic (2020):} 16-core Neural Engine, enabling real-time ML tasks across apps
    \item \textbf{A15–A17 (2021–2023):} Continued improvements in ML throughput, energy efficiency, and integration with custom ISP and Secure Enclave
\end{itemize}

\noindent These chips didn’t just boost performance—they marked the beginning of a new category of processor: one where machine learning wasn’t an add-on but a \textbf{first-class architectural concern}. Apple’s A-series Neural Engine sparked an industry-wide response. Soon, other manufacturers followed suit with processors explicitly designed for deep learning:

\begin{itemize}
    \item \textbf{Google Tensor SoC (2021–present):} Designed for Pixel smartphones, includes a custom TPU variant for on-device inference, enabling real-time translation, image segmentation, and voice recognition.
    
    \item \textbf{Huawei Ascend Series (2018–present):} Includes the Ascend 310 (edge) and Ascend 910 (data center), featuring dedicated AI cores optimized for matrix multiplication and low-precision arithmetic.
    
    \item \textbf{NVIDIA A100 (2020) and H100 (2022):} Built on the Ampere and Hopper architectures respectively, these GPUs include \textbf{Tensor Cores} specialized for deep learning, supporting mixed-precision training and massive parallelism.
    
    \item \textbf{AMD Instinct MI250 and MI300 (2021–2023):} High-performance accelerators built with AMD's CDNA architecture, targeting large-scale training workloads with support for BF16 and matrix-optimized pipelines.
\end{itemize}

\noindent For NVIDIA, the inflection point came in \textbf{2017} with the launch of the \textbf{Volta architecture}. This was the first time \textbf{Tensor Cores}—dedicated hardware units for deep learning—were introduced. While earlier GPUs had been repurposed for AI, Volta made deep learning acceleration a native hardware feature. It marked the transition from general-purpose compute devices to \textbf{AI-first silicon}.

\medskip

\noindent These processors weren’t just faster—they were smarter, architected from the ground up to accelerate the kinds of workloads that define modern machine learning: tensor operations, attention mechanisms, and large-scale dataflow graphs. What began as a design decision in mobile devices evolved into an industry-wide shift across edge, desktop, and data center hardware.

\noindent Looking ahead, Huawei is exploring even more radical architectural directions—including a \textbf{ternary logic processor}, which operates on three discrete states instead of the traditional binary “0” and “1.” If successful, ternary processors could offer advantages in both \textbf{energy efficiency} and \textbf{information density}, potentially representing values using fewer gates and reducing switching power. For machine learning workloads, this could mean faster low-precision inference, better support for quantized models, and even new forms of neural network architectures designed natively for three-state computation. While still experimental, it signals a future where the very logic underlying our chips might evolve to meet the demands of AI.


Machine learning had gone mainstream—into your camera, your keyboard, and your pocket.

\noindent The result? We’re no longer just choosing between “fast” and “cheap.” We’re choosing between entire ecosystems—each with its own compiler stack, memory layout, debugging tools, and architectural assumptions. And that’s why the neural network you design has to match the substrate it runs on.

In the next sections, we’ll walk through those substrates—CPU, GPU, TPU, and edge—and explore how to design networks that play to their strengths, not their bottlenecks.


% Insert this where you want the comic to appear
\begin{figure}[H]
\centering

% === First row ===
\begin{subfigure}[t]{0.45\textwidth}
\centering
\begin{tikzpicture}
  \comicpanel{0}{0}
    {Apple Engineer}
    {}
    {We need a to track facial expressions... for a talking poop emoji.}
    {(0,-1.2)}
\end{tikzpicture}
\caption*{Apple: Innovating for animated excrement.}
\end{subfigure}
\hfill
\begin{subfigure}[t]{0.45\textwidth}
\centering
\begin{tikzpicture}
  \comicpanel{0}{0}
    {Google Exec}
    {}
    {If Apple gets poop emojis, we get Tensor emojis. Launch the SoC division.}
    {(0,-0.8)}
\end{tikzpicture}
\caption*{Google: Never missing a branding opportunity.}
\end{subfigure}

\vspace{1em}

% === Second row ===
\begin{subfigure}[t]{0.45\textwidth}
\centering
\begin{tikzpicture}
  \comicpanel{0}{0}
    {NVIDIA Intern}
    {}
    {We optimized GAN latency so you can deepfake that girl you like.}
    {(0,-1.4)}
\end{tikzpicture}
\caption*{NVIDIA: Accelerating the porn industry.}
\end{subfigure}
\hfill
\begin{subfigure}[t]{0.45\textwidth}
\centering
\begin{tikzpicture}
  \comicpanel{0}{0}
    {Huawei R\&D}
    {}
    {Our new ternary chip predicts which TikTok filter you'll choose.}
    {(0,-1.2)}
\end{tikzpicture}
\caption*{Huawei: Destroying genZ one filter at a time.}
\end{subfigure}

\caption{The true reason your phone works faster than your laptop.}
\end{figure}



\subsection{CPUs: Good for Structured Models and Sparse Attention}

CPUs are general-purpose and memory-friendly. They're great at control flow, branching, and operations that depend on data locality.

\textbf{Networks that pair well with CPUs:}
\begin{itemize}
    \item Recurrent Neural Networks (RNNs), GRUs, LSTMs — especially when the sequence length is variable
    \item Transformer variants with sparse or local attention (e.g., Longformer, Reformer)
    \item Decision-tree hybrids (e.g., deep ensembles with GBDTs on tabular data)
\end{itemize}

\textbf{Why it works:}  
CPUs handle irregular workloads and cache-based memory access efficiently. Vectorized execution (AVX/NEON) is useful when the data layout can be controlled. RNN-style architectures can outperform transformers on CPUs when batching isn't feasible.

\subsection{GPUs: Ideal for Dense, Parallel Workloads}

GPUs want work that is embarrassingly parallel — batched, dense, regular, and with minimal branching.

\textbf{Networks that thrive on GPUs:}
\begin{itemize}
    \item CNNs — image models (ResNet, EfficientNet, UNet)
    \item Dense transformers (BERT, GPT)
    \item GANs with large convolutional or deconvolutional layers
\end{itemize}

\textbf{Why it works:}  
When data is batched and shape-regular, GPU tensor cores and memory coalescing can execute extremely efficiently. Layer fusions and mixed-precision training make this even more performant.

\subsection{TPUs: Specialized for Static Graphs and Giant Matmuls}

TPUs want your model to look like one giant matrix operation. If you can express it as a fused, static graph with predictable control flow, TPUs will love you.

\textbf{Best matches for TPUs:}
\begin{itemize}
    \item Transformer language models with fixed input sizes (T5, GPT-J, PaLM)
    \item Vision Transformers (ViTs) where tokenization and positional encodings are regular
    \item Physics-informed networks when they're expressed as fully vectorized tensor ops
\end{itemize}

\textbf{Where they struggle:}
\begin{itemize}
    \item Dynamic control flow (e.g., conditionals, data-dependent branches)
    \item Variable-length sequences
    \item Small batch sizes (underutilizes MXUs)
\end{itemize}

\subsection{ASICs, FPGAs, and Edge: When Customization Is the Feature}

Some models are built to run in environments where power and latency matter more than generality. That’s where application-specific hardware (ASICs) and field-programmable gate arrays (FPGAs) shine.

\textbf{Best-fit models:}
\begin{itemize}
    \item Quantized models (e.g., MobileNet, TinyML variants)
    \item Fixed-weight networks for inference-only workloads
    \item Neural ODE solvers for physical control systems
\end{itemize}

\textbf{Design constraint:}  
You must design for the hardware from the start — or be prepared to pay an extreme cost for conversion and debugging. Toolchains are narrow and fragile. Compilation to RTL or Verilog is nontrivial.

\subsection{Ecosystem Matters: Compiler Stack, Memory Hierarchy, and Debugging}

Choosing a processor isn't just about silicon. It's about the software and tooling around it:

\begin{itemize}
    \item \textbf{CUDA vs ROCm vs SYCL:} Determines your build system, profiler, and driver support
    \item \textbf{ONNX vs XLA vs TVM:} Affects what transformations you can do on the graph
    \item \textbf{Debugger quality:} JAX+TPU has a steep learning curve; PyTorch+CUDA is easier to inspect
    \item \textbf{Memory hierarchy:} L3 cache on CPUs, HBM on GPUs, shared memory on TPUs — your batch size and tensor layout must match
\end{itemize}

\subsection{A Decision Table: Which Model for Which Hardware?}

\begin{center}
\begin{tabular}{|l|l|l|}
\hline
\textbf{Model Type} & \textbf{Best On} & \textbf{Why} \\
\hline
CNN (e.g., ResNet) & GPU & Batching, spatial convolution, shared weights \\
\hline
Transformer (e.g., GPT) & TPU & Fused matmuls, static shapes, high throughput \\
\hline
Sparse Attention (e.g., Reformer) & CPU & Control flow, variable sequence handling \\
\hline
LSTM / RNN & CPU & Sequential ops, cache efficiency \\
\hline
Quantized MobileNet & Edge ASIC / FPGA & Power efficiency, fixed shape \\
\hline
Physics-Informed NN & TPU / GPU & Matrix solvers, statically compiled PDE kernels \\
\hline
\end{tabular}
\end{center}

\subsection{Final Thought: You Don’t Just Choose a Network. You Choose a Substrate.}

When you pick an architecture, you're also picking an ecosystem:

\begin{quote}
\textit{Do you want dynamic debugging or maximum throughput? Fine-grained control or massive parallelism? Static graphs or runtime freedom?}
\end{quote}

The choice of hardware sets constraints on how the information in your model flows — and how much that flow will cost, both computationally and economically. Optimization isn't just in the math. It’s in the silicon.

\vspace{1em}
In the next section, we’ll connect this insight to neural architecture design — and how to prototype with hardware-awareness from the very beginning.

%\section{The Economic Geometry of Training Neural Networks}

Let’s face it: training a neural network isn’t just a mathematical problem — it’s an economic one.

Every parameter costs time. Every pass burns electricity. Every decision in architecture, precision, and parallelization has a financial consequence. And in large-scale systems — from language models to trading models — this cost can determine the feasibility of an entire research pipeline.

\textbf{So how do we price intelligence?}

We’re going to treat neural network training as a measurable process with real-world economic consequences — and build a framework for estimating its cost using basic, composable variables.

\subsection{Step 1: Define the Training Objective}

Before we talk dollars, we talk design:

\begin{itemize}
    \item \textbf{What’s the target accuracy?}
    \item \textbf{How much data is needed to reach it?}
    \item \textbf{How many parameters will your network require?}
\end{itemize}

Each of these variables defines the shape of your training curve — and the budget required to climb it.

\subsection{Step 2: Break Training into Cost Components}

We model total training cost as:

\[
C_{\text{train}} = C_{\text{compute}} + C_{\text{memory}} + C_{\text{data}} + C_{\text{iteration}} + C_{\text{infrastructure}}
\]

Each component can be broken down as follows:

\begin{itemize}
  \item \textbf{Compute Cost:} 
    \[
    C_{\text{compute}} = t_{\text{train}} \times \text{Power Draw} \times \text{Cost per kWh}
    \]

  \item \textbf{Memory Cost:} 
    \[
    \text{VRAM usage} \times \text{Bandwidth bottlenecks} \times \text{Efficiency penalty}
    \]

  \item \textbf{Data Cost:} 
    \[
    \text{Data acquisition} + \text{Preprocessing time} + \text{I/O penalties}
    \]

  \item \textbf{Iteration Cost:} 
    \[
    \text{Epochs} \times \text{Pass Time} \times \text{Precision Factor} \times \text{Complexity Ratio}
    \]

  \item \textbf{Infrastructure Cost:} 
    \[
    \text{GPU rental} + \text{Storage} + \text{Cooling overhead}
    \]
\end{itemize}

\subsection{Step 3: Estimate Training Time Using Fermi Parameters}

Reuse your earlier estimation formula and expand it:

\[
t_{\text{train}} = \text{epochs} \times \text{pass time} \times \text{iterations/epoch} \times \text{data precision factor} \times \text{complexity ratio}
\]

Here’s how each term connects back to real-world cost:

\begin{itemize}
    \item \textbf{Epochs:} Determined by convergence behavior.
    \item \textbf{Pass Time:} Forward + backward + physics/residual components.
    \item \textbf{Iterations per Epoch:} Dataset size vs batch size.
    \item \textbf{Data Precision Factor:} More bits = more flops.
    \item \textbf{Complexity Ratio:} 
      \[
      \frac{\text{Network Depth} \times \text{Width}}{\text{Parallelism} \times \text{Memory Bandwidth}}
      \]
\end{itemize}

\subsection{Step 4: Assign Dollar Values (or Tokenized Resources)}

Now we turn the training pipeline into a financial model.

You can estimate:

\begin{itemize}
  \item \textbf{Cost per parameter trained}
  \item \textbf{Cost per inference sample}
  \item \textbf{Cost per accuracy point gained}
\end{itemize}

Bonus: if you're operating in a decentralized or tokenized system (e.g., edge compute, federated learning), you can express cost in:

\begin{itemize}
    \item Carbon tokens
    \item Bandwidth credits
    \item Smart-contract denominated incentives
\end{itemize}

\subsection{Step 5: Model Return on Training Investment (RTI)}

We close with a simple ratio:

\[
\text{RTI} = \frac{\text{Economic Value of Model}}{\text{Total Training Cost}}
\]

Where the numerator might be:

\begin{itemize}
    \item Performance gains (e.g. \$ value from HFT predictions)
    \item Efficiency improvements (e.g. throughput per watt)
    \item Strategic leverage (e.g. reduced human intervention)
\end{itemize}

\begin{quote}
\textit{Training a model isn’t just about achieving accuracy — it’s about making the economics work. The moment intelligence becomes measurable, it becomes accountable.}
\end{quote}

\vspace{1em}
\noindent
Next, we’ll walk through worked examples for different architectures — from compact classifiers to massive transformers — and see how to plug in real values.

%\section{Integrating Neuro-Symbolic GNNs with Inclusion Maps and Vector Clocks for High-Frequency Market Event Modeling}

This approach combines Graph Neural Networks (GNNs), symbolic temporal logic, and measure-theoretic constructs to model and analyze high-frequency market events. It treats market data---such as trades, price updates, and macroeconomic signals---as events embedded in a measurable event space, using \emph{inclusion maps} to structure their relationships and \emph{vector clocks} to capture partial orderings across distributed agents. The system is designed to address the limitations of purely numerical or purely symbolic systems by integrating the strengths of both paradigms.

\subsection{Modeling Framework}

We define a measurable event space $(\Omega, \mathcal{F})$, where:
\begin{itemize}
    \item $\Omega$: The sample space of all possible market events.
    \item $\mathcal{F}$: A $\sigma$-algebra over $\Omega$, containing measurable subsets such as the set of all trades at time $t$, or all events involving a specific security.
\end{itemize}

Each \emph{agent} (e.g., trader, exchange, algorithm) observes a subset of $\mathcal{F}$, denoted $\mathcal{F}_i \subseteq \mathcal{F}$, and each such subset is connected to others via \emph{inclusion maps} $i_{ij}: \mathcal{F}_i \hookrightarrow \mathcal{F}_j$, forming a presheaf over the category of agents. This structure models partial observability and information flow in the market.

\subsection{Temporal Logic with Vector Clocks}

Let each event $e$ carry a \emph{vector clock} $v(e) \in \mathbb{N}^n$, where $n$ is the number of agents. Define the partial ordering $e \rightarrow e'$ (i.e., ``event $e$ happened before $e'$'') as:
\[
v(e) < v(e') \iff \forall i,\ v(e)_i \leq v(e')_i \ \text{and} \ \exists j \ \text{s.t.} \ v(e)_j < v(e')_j
\]
This captures \emph{causality} in the system---crucial for determining arbitrage paths, information flow, or detecting violations of market fairness.

\subsection{Graph Neural Network Integration}

We define a graph $G_t = (V_t, E_t)$ at time $t$, where:
\begin{itemize}
    \item $V_t$: Vertices represent market agents or securities.
    \item $E_t$: Edges represent causal or informational dependencies between entities, computed via inclusion maps and vector clock comparisons.
\end{itemize}

Each node $v \in V_t$ has a feature vector $x_v$, incorporating:
\begin{itemize}
    \item Observed events (trades, quotes, news),
    \item Historical state (price velocity, momentum, etc.),
    \item Vector clock embeddings (relative to other agents).
\end{itemize}

The GNN message-passing function $\phi$ respects the partial ordering induced by the vector clocks:
\[
h_v^{(k+1)} = \text{UPDATE}^{(k)}\left(h_v^{(k)}, \sum_{\substack{u \in \mathcal{N}_v^{(k)} \\ v(u) < v(v)}} \text{MESSAGE}^{(k)}(h_u^{(k)}, h_v^{(k)}, e_{uv})\right)
\]
This ensures that information from temporally prior (but possibly concurrent) events is correctly propagated, enabling \emph{non-linear temporal reasoning} across the network.

\subsection{Symbolic-Probabilistic Fusion}

Each GNN layer outputs a distribution over measurable subsets of $\mathcal{F}$, representing updated \emph{beliefs} about latent variables like volatility spikes, manipulation risk, or systemic shocks. These subsets can be mapped back to symbolic expressions in first-order temporal logic, e.g.:

\begin{quote}
``There exists a trade sequence from agent A to B such that the price impact exceeds threshold $\delta$ within $\tau$ ms.''
\end{quote}

Thus, predictions are not just pointwise but \emph{measurable set-valued}, and interpretable.

\subsection{Applications}

\begin{itemize}
    \item \textbf{Arbitrage Detection}: Detect cycles in the event graph that violate causality or information asymmetry constraints.
    \item \textbf{Latency Arbitrage Immunization}: Identify vulnerable edges (slow links) and symbolically verify resilience via inclusion-preserving rewrites.
    \item \textbf{Market Manipulation Detection}: Capture spoofing or layering patterns as fixed points or anomalies in vector-clock-ordered subgraphs.
    \item \textbf{Macro-Micro Signal Fusion}: Merge top-down macroeconomic signals (symbolic events) with bottom-up tick-level GNN predictions via conditional inclusion.
\end{itemize}


\subsection{Belief Space Dynamics via Information-Geometric Lagrangians}

Extending beyond classical physics-informed models, we explore a formulation where the system evolves in a \emph{belief space}, guided not by physical laws, but by their formal structure---specifically, Lagrangian and Hamiltonian mechanics on information-geometric manifolds. In this setting, the market is interpreted as a distributed inference system, and each agent’s evolving internal model of the world (e.g., volatility estimates, risk exposure, latent order flow) is treated as a generalized coordinate \( q(t) \) evolving through time.

We define a Lagrangian over this belief manifold as:
\[
L(q, \dot{q}) = \frac{1}{2} \dot{q}^T G(q) \dot{q} - D_{\mathrm{KL}}(q \parallel q_{\mathrm{target}})
\]
where \( G(q) \) is the Fisher information metric and \( D_{\mathrm{KL}} \) penalizes divergence from a target belief distribution (e.g., equilibrium pricing or consensus sentiment). This mirrors the classical \( L = T - V \) structure, with ``kinetic energy’’ interpreted as the \emph{rate of inferential change}, and ``potential energy’’ as the \emph{informational tension} between an agent’s internal model and the observable market consensus.

The belief dynamics are governed by the Euler-Lagrange equation:
\[
\frac{d}{dt} \left( \frac{\partial L}{\partial \dot{q}} \right) - \frac{\partial L}{\partial q} = 0
\]
whose residual can be interpreted as the extent to which an agent's inference violates information-theoretic consistency over time. This residual defines a constraint loss:
\[
\mathcal{L}_{\text{lagrange}} = \sum_{t_i} \left\| \frac{d}{dt} \left( \frac{\partial L}{\partial \dot{q}} \right) - \frac{\partial L}{\partial q} \right\|^2
\]
which can be backpropagated through a neural network approximator \( f_\theta(t, x) \approx q(t) \). The neural architecture effectively learns a trajectory through belief space that respects both temporal consistency and information-geometric structure.

This formulation enables richer control-theoretic interpretations as well. If \( u(t) \) denotes the control input (e.g., trading decision, model adjustment), we can alternatively define:
\[
L(q, u) = \frac{1}{2} u^T R u + D(q \parallel \hat{q})
\]
with \( R \) penalizing effort and \( D \) quantifying divergence from a desired belief state \( \hat{q} \). This reflects a Hamilton-Jacobi-Bellman interpretation of inference under uncertainty: an agent minimizes informational regret and control cost over time.

In this setting, the symbolic-probabilistic fusion is enhanced: measurable subsets of market events are not just passively inferred, but actively \emph{navigated} in a way that balances computational effort, belief evolution, and consistency with macrostructure.

Applications of this Lagrangian-informed architecture include:
\begin{itemize}
    \item \textbf{Trajectory Forecasting in Belief Space}: Learn agent-specific inference paths aligned with latent structure in high-frequency data.
    \item \textbf{Interventional Market Modeling}: Model counterfactuals where control inputs \( u(t) \) simulate regulatory constraints, sentiment shifts, or liquidity injections.
    \item \textbf{Policy Synthesis via Optimal Belief Control}: Derive control policies (e.g., execution strategies) that minimize divergence from desired system behavior while respecting belief dynamics.
\end{itemize}



\subsection{High-Level PINN Architecture for Information-Geometric Belief Dynamics}

To implement the Lagrangian-based belief evolution described above, the Physics-Informed Neural Network (PINN) is structured around modular components that mirror the functional form of the Euler-Lagrange equations. These components work together to encode information-theoretic structure into the network's training dynamics.

\vspace{0.5em}
\noindent\textbf{(1) \texttt{compute\_derivative(q, t)}}

This function estimates the time derivative of the belief vector $q(t)$, either through:

\begin{itemize}
    \item \textbf{Automatic differentiation}: Using \texttt{torch::autograd::grad} or its C++ equivalent to compute $\dot{q}$ directly from the network output.
    \item \textbf{Finite differences}: For structured belief evolution over discrete time grids, a central or forward difference scheme can be applied.
\end{itemize}

\textbf{Purpose:} Captures the rate of inferential change, which serves as the kinetic term in the Lagrangian.

\vspace{0.5em}
\noindent\textbf{(2) \texttt{compute\_lagrangian(q, dq\_dt)}}

This function implements the Lagrangian:
\[
L(q, \dot{q}) = \frac{1}{2} \dot{q}^T G(q) \dot{q} - D(q \parallel q_{\text{target}})
\]

\begin{itemize}
    \item \textbf{Fisher metric term} $\dot{q}^T G(q) \dot{q}$: Represents information-theoretic motion over the manifold. $G(q)$ may be approximated empirically or learned as a differentiable module.
    \item \textbf{Divergence term} $D(q \parallel q_{\text{target}})$: Encourages alignment with an ideal or reference belief distribution (e.g., empirical market prior).
\end{itemize}

\textbf{Purpose:} Encodes belief dynamics into a unified scalar quantity, suitable for variational principles and gradient flow.

\vspace{0.5em}
\noindent\textbf{(3) \texttt{compute\_residual(L, q, dq\_dt, t)}}

This function applies the Euler-Lagrange operator:
\[
\mathcal{R}(t) = \frac{d}{dt} \left( \frac{\partial L}{\partial \dot{q}} \right) - \frac{\partial L}{\partial q}
\]

\begin{itemize}
    \item Uses automatic differentiation to compute both \texttt{dL/dq} and \texttt{dL/ddq}, and their time derivative.
    \item Supports backpropagation through time to enable learning of optimal inferential trajectories.
    \item Can be wrapped into a custom autograd function for better control and interpretability.
\end{itemize}

\textbf{Purpose:} Measures deviation from information-consistent dynamics, forming the constraint loss.

\vspace{0.5em}
\noindent\textbf{(4) \texttt{loss = mean(residual\^{}2)}}

The final scalar loss is computed as:
\[
\mathcal{L}_{\text{lagrange}} = \sum_{t_i} \left\| \mathcal{R}(t_i) \right\|^2
\]

\begin{itemize}
    \item Can be combined with other domain-specific losses (e.g., prediction error, classification loss).
    \item May be weighted by uncertainty or belief entropy to regularize against overconfident inference.
\end{itemize}

\textbf{Purpose:} Guides network training toward belief trajectories that are smooth, consistent, and information-conservative.

\vspace{0.5em}

Together, these components form a symbolic-neural system that enforces structured belief evolution across time. Unlike standard PINNs that enforce physical constraints, this framework imposes informational and inferential constraints, grounded in a variational and control-theoretic formalism. The result is a network that learns both to infer and to evolve in a way that respects its own logic of belief.


\subsection{Class and Function Signatures for Python Implementation}

The following outlines the high-level architecture of a Python implementation for the Lagrangian-based, information-geometric PINN. Each class and function is given as a structural stub, showing how the system might be modularized across components such as belief dynamics, metric learning, divergence calculation, and the training loop.

\begin{lstlisting}[language=Python]
import torch
import torch.nn as nn

# Main PINN model
class InfoGeoPINN(nn.Module):
    def __init__(self, network_config, metric_module, divergence_module):
        super().__init__()
        self.f_net = self.build_network(network_config)
        self.metric = metric_module
        self.divergence = divergence_module

    def build_network(self, config):
        pass

    def forward(self, t, x):
        pass

    def compute_derivative(self, q, t):
        pass

    def compute_lagrangian(self, q, dq_dt):
        pass

    def compute_residual(self, L, q, dq_dt, t):
        pass

    def compute_loss(self, residual):
        pass

# Fisher information metric or Riemannian metric module
class FisherMetric(nn.Module):
    def __init__(self, input_dim):
        super().__init__()
    
    def forward(self, q):
        pass

# Divergence module (e.g., KL divergence)
class Divergence(nn.Module):
    def __init__(self, target_distribution):
        super().__init__()

    def forward(self, q):
        pass

# Utility for time integration or time sampling
def generate_time_grid(t_min, t_max, steps):
    pass

# Loss wrapper for Euler-Lagrange residuals
def euler_lagrange_residual(L, q, dq_dt, t):
    pass

# Training loop
def train(model, dataloader, optimizer, num_epochs):
    pass

# Dataset wrapper for time-dependent inputs
class TimeSeriesDataset(torch.utils.data.Dataset):
    def __init__(self, time_points, features):
        pass

    def __len__(self):
        pass

    def __getitem__(self, idx):
        pass
\end{lstlisting}

This modular layout allows each component---from metric evaluation to divergence estimation and residual computation---to be independently tested, optimized, or swapped with alternate formulations. The structure supports experimentation with both classical and learned metrics, symbolic divergences, and multi-scale time dynamics.



%\section{There Is No Final Model: The Pipeline as a Living System}


\subsection{Gödel’s Revenge: The Limits of Financial Models}

Gödel’s incompleteness theorem tells us that no mathematical system can ever be both complete and consistent; which, frankly, is the perfect way to describe finance. Every time we think we’ve built the perfect trading model, the market changes, invalidating our assumptions.  

\begin{quote}
The moment we prove we’ve found the best strategy, it’s already wrong.
\end{quote}  

Just ask the hedge funds that got blindsided by the GameStop saga. A bunch of Redditors on r/WallStreetBets found a loophole in the system, executed a short squeeze, and for a brief moment, it looked like the retail investors had hacked finance itself. But markets, like mathematics, have a way of self-correcting. The banks adapted, changed margin requirements, limited trading, and ultimately ensured that the house still wins.  

And that’s why Gödel always wins: any hack you find will eventually be met with a hack to undo it. The moment you think you’ve broken the system, the system rewrites itself.The market proved that every winning strategy eventually stops working. And the No Free Lunch Theorem? It seals the deal:

\begin{quote}
\textit{No model is optimal for every world. So as the world changes, so must the model.}
\end{quote}

That brings us to the uncomfortable truth: \textbf{you will never be done.} A machine learning model is not a static artifact. It is a \textit{dynamical system} embedded in another — the real world. And when that outer system drifts, the inner one must follow, or it fails.

So if your model is going to survive in the wild, it needs more than accuracy. It needs a nervous system.

\subsection{The Canonical ML Pipeline (Revisited)}

Most practitioners know the steps. But they usually think of them as one-time stages in a project:

\begin{enumerate}
    \item \textbf{Data Collection}
    \item \textbf{Data Preprocessing}
    \item \textbf{Model Selection}
    \item \textbf{Model Training}
    \item \textbf{Model Evaluation}
    \item \textbf{Hyperparameter Tuning}
    \item \textbf{Deployment}
\end{enumerate}

But in reality, this is just a snapshot. A freeze-frame of an inherently cyclical process. The real picture looks more like this:

\begin{center}
\begin{tikzcd}[row sep=3em, column sep=3em]
  \text{Deployed Model} \arrow[r, "\text{Monitor}"] & \text{Drift Detection} \arrow[d, "\text{Trigger Update}"] \\
  \text{Model Evaluation} \arrow[u, "\text{Compare Performance}", dashed] & \text{Retrain or Tune} \arrow[d, "\text{Validate}"] \\
  \text{Preprocessing Update} \arrow[u, "\text{Data Feedback}"] & \text{Deployment}
\end{tikzcd}
\end{center}

This diagram doesn’t just represent a set of tools. It represents \textbf{an organizational responsibility}. Because once your model is live, you’ve made a promise — that it will keep learning, adapting, and behaving in line with reality.

And that promise? It’s enforced not by a one-time training script, but by software engineering.

\subsection{ML Is Not a Model. It's a System.}

This is the part people miss: machine learning is not the model you trained last quarter. It’s the \textit{system} you build around that model — to keep it alive, responsive, and honest.

\begin{quote}
\textit{A model without monitoring is a rumor. A pipeline without retraining is a time bomb.}
\end{quote}

The canonical pipeline isn’t a waterfall. It’s a loop. And every time the world shifts — in data, behavior, or infrastructure — the loop spins again.

\vspace{1em}

Let’s quantify this. Suppose your deployed model handles 10,000 predictions per day — a fairly conservative estimate for even modest-scale applications. If your average cost per prediction error is \$0.10 (whether from lost revenue, bad UX, or misrouted alerts), then:

\[
\text{Daily Loss} = 10{,}000 \times \text{Error Rate} \times \$0.10
\]

Let’s say your model silently drifts to a 20\% error rate before anyone notices (because no monitoring):

\[
\text{Daily Loss} = 10{,}000 \times 0.2 \times \$0.10 = \$200
\]

Now multiply that by 30 days of unnoticed drift: \boxed{\$6,000} down the drain — and that’s just direct costs, not reputation, opportunity cost, or executive side-eyes.

\vspace{1em}

\subsubsection*{A Game-Theoretic Perspective: Pipeline as a Strategy}

Treating ML as a one-time artifact (train and forget) vs.\ an adaptive system (continuous loop) creates different strategic payoffs. Here's the game:

\begin{center}
\begin{tabular}{r|c|c}
\textbf{You / Reality} & \textbf{World Stays Static} & \textbf{World Changes} \\
\hline
\textbf{You Use Static Pipeline} & Low Cost, OK Accuracy & \textcolor{red}{\textbf{High Risk of Silent Failure}} \\
\textbf{You Use Adaptive Loop} & Slightly Higher Cost & \textcolor{green!60!black}{\textbf{Resilient, Self-Correcting}} \\
\end{tabular}
\end{center}

If you assume the world never changes, static ML pipelines look efficient. But in practice, data shifts constantly — user behavior evolves, infrastructure gets updated, your labels drift. That bottom-right cell? That’s the only one where you survive long-term.


\subsection{Case Study: TSMC and the Feedback Loop of Mastery}

What does a semiconductor foundry have to do with machine learning pipelines?

Everything... if you think in systems.

Back in the 1990s, Taiwan Semiconductor Manufacturing Company (TSMC) wasn’t even close to a household name. It was a contract manufacturer — a third-tier shop that fabless startups like nVidia used because they couldn’t afford their own foundries. The heavyweights of chipmaking — Intel, AMD, Texas Instruments — owned their own fabs. They saw contract manufacturing as a niche sideshow.

But TSMC saw something else: a system with a feedback loop.

The more chips you make, the more you learn how to make chips. Every fab run is a signal. Every yield issue is an opportunity for process tuning. Every new client design is a source of variation that drives iteration. Unlike vertically integrated players who mostly manufactured their own designs, TSMC became a platform; one where \textbf{every customer became a feedback node} in a massive, evolving learning loop.

\begin{quote}
\textit{Chipmaking isn’t just manufacturing. It’s iteration at scale.}
\end{quote}

While other companies saw fabless manufacturing as low-margin contract work, TSMC turned it into a strategy. They committed to building not just chips — but systems for making chips better. Faster. Smaller. Cleaner. More repeatable.

And as the fabless model exploded — first in gaming GPUs, then mobile, then crypto, then AI — the learning loop accelerated. More designs meant more runs. More runs meant more data. More data meant tighter feedback, faster improvements, and fewer defects.

By the time Apple (i.e. the 800lb gorilla of high-volume consumer silicon) arrived on the scene, the difference was clear. TSMC’s feedback loops were tighter. Their defect rates were lower. Their learning velocity was unmatched. Samsung might compete, but TSMC never tried to steal your design. That trust, combined with relentless process improvement, made them the go-to partner for bleeding-edge silicon.

\begin{quote}
\textit{Trust compounds. So does process mastery.}
\end{quote}

Today, TSMC produces 90\% of the world’s most advanced chips. But it didn’t happen by accident. It happened because they built a system that learns. A pipeline that gets better the more it runs.

The parallels to machine learning pipelines are not coincidental:

\begin{itemize}
    \item \textbf{Feedback loops improve outcomes.} Just as TSMC uses each chip run to refine its processes, every deployed model should feed metrics back into evaluation and retraining.
    \item \textbf{Volume reveals edge cases.} TSMC learned from hundreds of clients; ML teams learn from production data, failure cases, and user behavior.
    \item \textbf{Systems thinking outperforms one-off optimization.} The winning strategy wasn’t building the perfect fab once — it was building a fab system that improves with every iteration.
    \item \textbf{Trust and adaptability win in dynamic environments.} TSMC’s reputation is built on non-interference and fast adaptation. ML pipelines must earn trust in the same way — by staying aligned with changing data and real-world goals.
\end{itemize}

In short: your ML pipeline should behave more like TSMC than Intel. Not an artifact frozen in time, but a living system that compounds its own experience.

TSMC didn’t win because it built the perfect fab once. It won because it treated its entire infrastructure like a learning organism. Every chip wasn’t just a product: it was a lesson. Every delay, every failure, every unexpected voltage curve was a data point. And most importantly, they \textit{used} those lessons. They folded that information back into the process. They built systems that watched themselves.

Now contrast that with Intel during the same era. Intel optimized for control. Closed fabs. Top-down planning. Rigid design flows. Brilliant engineers, no doubt. However, they had little room for the kind of messy, real-world variability that teaches you how to adapt. Their fabs were temples: elegant, powerful, and fragile. When the market shifted, their process couldn’t shift with it. TSMC’s could... because it was built to listen.

Your ML pipeline is no different.

If you treat it like a temple — something you perfect once and then enshrine in production — it will become brittle the moment reality shifts. The data will drift. The labels will decay. The product will pivot. And suddenly, your beautiful model becomes a liability, quietly hemorrhaging value while everyone assumes it's “still working.”

But if your pipeline behaves more like TSMC — if it’s built to notice changes, to adapt, to learn — then every deployment makes it stronger. Every failure feeds back into process improvements. Every weird edge case becomes a catalyst for tightening the loop. Over time, your system becomes not just accurate, but resilient.

\begin{quote}
\textit{Accuracy wins today. Feedback wins forever.}
\end{quote}




%\section{Conclusion: From Pythagoras to High-Frequency Trading (One Crisis at a Time)}

\subsection{How We Broke Math, Then Monetized It}

Somehow, we've gone from Pythagoras freaking out over irrational numbers to hedge funds running their own private arms race, executing trades faster than the SEC can blink. And, it wasn’t because mathematicians sat around sipping tea and contemplating the beauty of pure abstraction. No, it was because each era hit a mathematical crisis so baffling that we had no choice but to invent new math just to survive.

\begin{itemize}
    \item Galileo couldn't describe forces? We got calculus.
    \item Fourier broke math with infinite sums? We got real analysis.
    \item Cantor made infinity weird? We got set theory.
    \item Riemann integration couldn't handle chaotic functions? Enter Lebesgue.
    \item Big banks couldn't make money fast enough? We got high-frequency trading.
\end{itemize}

So, if you've made it this far, \textbf{congratulations, you’re now a quant.}

\begin{figure}[H]
\centering
\begin{tikzpicture}[every node/.style={font=\footnotesize}]

% Panel 1 — Pythagoras
\comicpanel{0}{4}
  {Pythagoras}
  {}
  {\textbf{Pythagoras:} What is the nature of number? Can the universe be measured?}
  {(0,-0.6)}

% Panel 2 — Newton
\comicpanel{6.5}{4}
  {Newton}
  {}
  {\textbf{Newton:} Does motion follow divine law, and can we measure its mechanics?}
  {(0,-0.6)}

% Panel 3 — Cantor
\comicpanel{0}{0}
  {Cantor}
  {}
  {\textbf{Cantor:} Is infinity a single concept, or a landscape of sizes and hierarchies?}
  {(0,0.8)}

% Panel 4 — Quant
\comicpanel{6.5}{0}
  {Quant}
  {}
  {\textbf{Quant:} If I embed Lebesgue measures in a trading model, can I short causality?}
  {(0,0.8)}

\end{tikzpicture}
\caption{Math broke, we fixed it then used it to outpace reality.}
\end{figure}



\subsection{Why the Smartest Mathematicians Work in Finance (and Not for You)}

The real secret of modern finance isn’t hidden in some dusty textbook or sealed away in a Swiss vault. It’s right there in the open — pulsing through server farms in New Jersey, whispered in nanoseconds between exchanges, dressed up like math but walking like a heist.

And I’ve just gone and spilled it.  So if I suddenly disappear after publishing this... you'll know why. 

\begin{quote}
High-frequency trading firms are not analysts, and they’re not investors. They’re financial hitmen with PhDs. They don’t play the market; they execute it. Arbitrage is their bullet, speed is their silencer, and you never see the trade until it’s already settled... and you’re already bleeding slippage.
\end{quote}

But really, explaining how this is like revealing that "tax efficiency" really means "financially engineering all your profits to go offshore," or that "integrating with investors" really means "lie to them, but make it legal."

That is, unless you’ve drunk the Kool-Aid: then "complex derivatives" are just innovative financial solutions, "structured products" are purely about risk management, and "market-making" has nothing to do with front-running clients.

\begin{figure}[H]
\centering
\figureseparator
\
\begin{minipage}{\linewidth}
\centering
\begin{tikzpicture}[node distance=5cm, every node/.style={align=center, font=\small}]
  % Market Maker node
  \node[draw, rounded corners, fill=blue!10, minimum width=4cm, minimum height=2cm] (maker) 
    {Market Maker\\ \textit{“Adds Liquidity”}};

  % Front-runner node (use relative positioning via node distance)
  \node[right=of maker, draw, rounded corners, fill=red!10, minimum width=4cm, minimum height=2cm] (runner)
    {Front Runner\\ \textit{“Acts Swiftly”}};

  % Centered Arrow with Label
  \draw[->, thick] (maker) -- node[above, midway] {\textbf{Receives Client Order}} (runner);

  % Dashed box implying identity
  \node[fit={(maker) (runner)}, draw=gray, dashed, thick, inner sep=0.4cm] (box) {};

  % Label just above the box
  \node[anchor=south] at ([yshift=4pt]box.north) 
    {\itshape \textbf{Just two completely unrelated roles... often performed on the same infrastructure.}};

  % Label just below the box
  \node[anchor=north, text width=11cm, align=center, font=\footnotesize] 
    at ([yshift=-4pt]box.south) 
    {Some people say this looks like front-running.\\\ \\ Others say it’s just efficient market participation.\\\ \\
    \textit{We say: it depends who's asking.}};
\end{tikzpicture}
\end{minipage}

\vspace{0.5em}
\caption{Market Making vs. Front Running: A Totally Innocent Overlap}
\caption*{\textit{These two things are obviously very different.\\ Don’t ask too many questions.}}
\figureseparator
\end{figure}


\vspace{1em}

And in case you think finance is just for MBAs and Wall Street bros, let me remind you: the industry that hires the most mathematicians and physicists isn’t academia: it’s finance. There's a reason why Wall Street poaches talent from theoretical physics and number theory faster than universities can replace them. Even Bill Gates once admitted that the company he feared the most wasn’t another tech giant: it was Goldman Sachs. 

And honestly? He had a point. Because at the end of the day, \textbf{mathematics has always been about power.}

So… what’s next? 

We’ve patched up the old gaps, cleaned up the messy edges, and wrapped everything in a rigorous framework. However, when math gets too comfortable, it gets boring. Maybe what we really need is another crisis: something so baffling, so frustrating, that it forces us to throw everything we know out the window and start fresh. 

Because if Thomas Kuhn taught us anything, it’s that desperation is the only true source of good ideas. And if history is any guide, the next great mathematical breakthrough won’t come from academia: it’ll come from some poor soul staring at their screen at 3 AM, realizing that everything they thought they knew is wrong. Again. 

And if that crisis happens to involve finance? Well, let’s just hope Goldman Sachs doesn’t get there first. 


\begin{figure}[H]
\centering

% === First row ===
\begin{subfigure}[t]{0.45\textwidth}
\centering
\begin{tikzpicture}
  \comicpanel{0}{0}
    {Grad Student}
    {}
    {I just proved a lemma that generalizes a corollary of a theorem that no one uses.}
    {(0,-0.6)}
\end{tikzpicture}
\caption*{Academia: Proving things no one reads.}
\end{subfigure}
\hfill
\begin{subfigure}[t]{0.45\textwidth}
\centering
\begin{tikzpicture}
  \comicpanel{0}{0}
    {Quant}
    {}
    {I used measure theory to front-run \$40M of trades. Before lunch.}
    {(0,-0.6)}
\end{tikzpicture}
\caption*{Finance: Theorems with immediate cash flow.}
\end{subfigure}

\vspace{1em}

% === Second row ===
\begin{subfigure}[t]{0.45\textwidth}
\centering
\begin{tikzpicture}
  \comicpanel{0}{0}
    {Professor}
    {}
    {Want to research pure math? Here's a stipend and a coffee machine.}
    {(0,0.8)}
\end{tikzpicture}
\caption*{The university incentive package.}
\end{subfigure}
\hfill
\begin{subfigure}[t]{0.45\textwidth}
\centering
\begin{tikzpicture}
  \comicpanel{0}{0}
    {Goldman Sachs}
    {}
    {Here's \$500k, and unlimited cocaine and hookers.}
    {(0,0.8)}
\end{tikzpicture}
\caption*{The Goldman Sachs incentive package.}
\end{subfigure}

\caption{Why the smartest mathematicians work in finance (and not for you).}
\end{figure}



%\part{Appendix}
%\section{The Real ML Pipeline (a.k.a. The Sisyphus Workflow)}

%\subsection{You Probably Don’t Believe How Bad It Gets}

\subsubsection{Flexible ML Pipelines: Engineering for Change}

Most people don’t believe how dysfunctional real-world machine learning pipelines are...  until they’ve been paged at 3am because the model broke and no one knows why.


Let’s borrow a lesson from traditional software systems: good systems are designed to evolve. So what makes an ML pipeline flexible enough to keep up with a changing world?

Here’s the checklist — feel free to expand on each:

\begin{itemize}
    \item \textbf{Modularity}: Each stage (ingestion, preprocessing, training, evaluation, deployment) should be separable and swappable. Think DAGs, not monoliths.

    \item \textbf{Versioning}: Data, models, and code should all be versioned — and tightly coupled. Reproducibility is non-negotiable.

    \item \textbf{Configurable Interfaces}: Avoid hardcoding parameters. Use configs or schemas that let you inject new components without rewriting core logic.

    \item \textbf{Observability}: Logs, metrics, and alerts for every step. Drift isn’t always visible in accuracy. You need to watch data distributions, latency, confidence, and edge cases.

    \item \textbf{Continuous Integration + Deployment (CI/CD)}: Yes, for ML. Automated tests, retraining hooks, deployment gates based on metrics. MLOps isn’t buzz—it’s ops.

    \item \textbf{Feedback Loops}: Human-in-the-loop or automatic feedback ingestion. Label drift and concept drift are not the same — your system should distinguish them.

    \item \textbf{Fail-Safes}: Fallback models, circuit breakers, and shadow deployments help contain damage when things go wrong — and they will.
\end{itemize}


\vspace{1em}

\subsubsection{The Checklist as a Risk Offset}

Each item in the ML pipeline checklist acts as a hedge against systemic collapse:

\begin{itemize}
    \item \textbf{Modularity} isolates failure domains.
    \item \textbf{Versioning} creates rollback points.
    \item \textbf{Configurable Interfaces} allow rapid pivoting.
    \item \textbf{Observability} turns drift from a mystery into a metric.
    \item \textbf{CI/CD} ensures changes propagate safely.
    \item \textbf{Feedback Loops} make the system antifragile.
    \item \textbf{Fail-Safes} contain blast radius when everything goes sideways.
\end{itemize}

Each is a small tax you pay now to avoid catastrophic debt later. If your pipeline isn't engineered like a nervous system — with reflexes, memory, and fallbacks — you're not doing ML. You're cosplaying it.

\vspace{1em}

\begin{quote}
\textit{Machine learning without systems thinking is just an expensive way to overfit to the past.}
\end{quote}



So let’s be clear: this checklist isn’t just a nice-to-have. It’s a set of survival tools. You need to invest in each of them, because the kinds of disasters that follow? They’re not hypotheticals. They’re Tuesday.

The rest of this section is a tour through exactly why each piece matters (with real, horrifying examples). If you don’t already have a healthy fear of ML in production, you will.




%\subsection{Modularity: Because the Intern Deployed a Hardcoded Model to Prod}

\subsubsection{Act I: The Intern and the Ancient Logs}

It always starts the same way.

An intern joins a big company: the kind with seventeen internal platforms named after birds, seven ways to deploy code, and exactly zero onboarding documentation that's less than two fiscal years out of date.

Their mission? Train a “quick linear model” for a low-risk, low-visibility feature.

You know the type: technically in production, but nobody cares unless it’s broken. It’s the kind of feature that lives in the shadowy basement of the product stack: somewhere between “critical to compliance” and “nobody’s sure who owns this anymore.” 

It doesn’t drive revenue. It doesn’t get showcased at all-hands. It’s the ML equivalent of changing the default font size in legal disclaimers. The only time it gets attention is when it silently fails, and suddenly a VP is asking why the "numbers feel off" in a dashboard no one has looked at since the last reorg.

And yet --- paradoxically --- this model has to be robust, fast, explainable, and deployable by Friday. It’s the Schrödinger’s cat of machine learning: simultaneously irrelevant and vital, invisible and on fire.


Now here's the catch: No one could tell them where the data came from.  Or where it went.  Or what shape it was in.  Or if it even existed outside of a stale JIRA ticket and someone’s memory of a meeting from last October.

\begin{quote}
In other words: institutional memory had evaporated.
\end{quote}

What was once a functioning data pipeline was now a digital ruin — still running, but no one dared touch it. The current team treated it like an ancient relic from a lost civilization. No documentation, no dashboards, no tests. Just a handful of cryptic cron jobs duct-taped together with bash scripts and hope.

Trying to debug it felt like decoding the Dead Sea Scrolls — except half the scrolls are in YAML, and the other half are missing entirely. Ask around, and you’d hear whispers of “someone in infra might remember the original config,” or “maybe try grepping old Git commits from 2019?”

You don’t maintain this kind of system. You pray it doesn’t break during daylight hours — because if it does, it’ll summon a war room full of confused engineers, each discovering in real time that they are now the de facto owner.

So the intern did what anyone would do under the circumstances:

\begin{itemize}
	\item They grep’ed.
	\item They found an old log folder with timestamps from the pre-pandemic era.
	\item And they wrote \texttt{pandas.read\_csv()} like it was a sacred ritual passed down from Stack Overflow itself.
\end{itemize}

And miraculously, it ran... sort of.

There were nulls in half the columns, timestamps in three different formats, and a mysterious field called \texttt{v2FlagTemp} that no one ever defined.

Nobody knew what \texttt{v2FlagTemp} actually did. Some said it was a deprecation marker. Others claimed it toggled a legacy feature that hadn’t existed since the Kubernetes migration of ’19. One engineer swore it flipped to \texttt{TRUE} during solar flares.

Meanwhile, the timestamps were a small chaos engine unto themselves — sometimes ISO, sometimes American short date, occasionally slashed Y/M/D, and once, just once, a format that appeared to be Morse code.

Parsing the logs felt less like data engineering and more like digital paleontology. You weren’t cleaning data. No, you were performing forensic analysis on a crime scene left behind by a team that vanished during a global pandemic.

\begin{lstlisting}[
    caption={Sample log entries recovered from the pre-pandemic era},
    label={lst:ancientlogs},
    basicstyle=\ttfamily\small,
    frame=single,
    numbers=left,
    numberstyle=\tiny,
    breaklines=true,
    breakatwhitespace=false,
    postbreak=\mbox{\textcolor{gray}{$\hookrightarrow$}\space}
  ]
  2020-01-03 10:15:32 INFO  | feature1=12.3 feature2=7.1 target=? v2FlagTemp=TRUE
  Jan 03 2020 10:15:33 INFO | feature1=13.1 feature2=8.0 target=4.0
  2020/01/03 10:15:34 INFO  | feature1=nan feature2=6.9 target=?
  2020-01-03 10:15:35 WARN  | v2FlagTemp=FALSE dropped packet from node17
  2020-01-03 10:15:36 INFO  | feature1=11.8 feature2=7.3 target=3.9 v2FlagTemp=1
  ??? 10:15:37 --- | featurX=?? fea2=error targt=missing
  2020-01-03 10:15:38 INFO  | feature1=13.0 feature2=8.1 target=4.1
  \end{lstlisting}
  

But the intern pressed on because a few \texttt{dropna()} calls and some light string parsing would bring order to the chaos.  Hope, after all, is the most dangerous optimizer.

Armed with Stack Overflow tabs and unshakable faith in pandas, he began the sacred rite of modern data cleaning: (a) removing nulls without asking why they were there, (b)coercing all timestamps into ISO format (with silent fallback), and (c) mapping suspicious Boolean-like strings to something vaguely coherent. \texttt{v2FlagTemp}? Treated as a categorical, obviously.  Unused features? Dropped with the righteous confidence of someone who had never seen a postmortem.  \textbf{Because nothing says “production ready” like blindly transforming orphaned telemetry logs from an undocumented cronjob.}

\begin{quote}
By the time the Jupyter notebook was done, the data \textit{looked} ``clean''.  Not correct. Not meaningful. But clean enough to pass a code review; and that, in the intern’s defense, is often good enough to get a green checkmark in a quarterly slide deck.
\end{quote}


\begin{figure}[H]
\centering

% === First row ===
\begin{subfigure}[t]{0.45\textwidth}
\centering
\begin{tikzpicture}
  \comicpanel{0}{0}
    {Engineer 1}
    {Engineer 2}
    {\footnotesize Let's just log it like this for now. I'll document it later.}
    {(0,-0.6)}
\end{tikzpicture}
\caption*{Prologue: It begins, as always, with good intentions.}
\end{subfigure}
\hfill
\begin{subfigure}[t]{0.45\textwidth}
\centering
\begin{tikzpicture}
  \comicpanel{0}{0}
    {Engineer 1}
    {Engineer 2}
    {\footnotesize v2FlagTemp? Yeah, just set it to TRUE if the system feels unstable.}
    {(0,-0.6)}
\end{tikzpicture}
\caption*{The design philosophy: chaos with confidence.}
\end{subfigure}

\vspace{1em}

% === Second row ===
\begin{subfigure}[t]{0.45\textwidth}
\centering
\begin{tikzpicture}
  \comicpanel{0}{0}
    {Executive 1}
    {Executive 2}
    {\footnotesize We’re preparing a strategic reorg. Most of the original team will be gone by Friday.}
    {(0,-0.6)}
\end{tikzpicture}
\caption*{The culling: efficiencies must be unlocked.}
\end{subfigure}
\hfill
\begin{subfigure}[t]{0.45\textwidth}
\centering
\begin{tikzpicture}
  \comicpanel{0}{0}
    {Executive 1}
    {Executive 2}
    {\footnotesize But I’m not worried. I’ve been assured everything is fully documented.}
    {(0,-0.6)}
\end{tikzpicture}
\caption*{The punchline: spoken like someone who’s never used Confluence.}
\end{subfigure}

\caption{The Birth of a Legacy System: High hopes and zero documentation.}
\end{figure}


\subsubsection{Act II: The Birth of the Monolith}

Fresh off the intoxicating high of successfully reading a CSV the intern embarked on their next great odyssey.

He did what every unsupervised intern with vague requirements, root access, and a deeply misplaced sense of destiny inevitably does: he wrote a script.

Not a pipeline. Not a collection of well-documented, reusable modules.

No. This was different.

It was a single, sprawling, glorious file—an unbroken stream of code that took data from cradle to grave. From loading to preprocessing to training to evaluation to deployment to celebratory print statement. All in one script. All in one breath.

\lstset{
  basicstyle=\ttfamily\small,
  keywordstyle=\color{blue},
  commentstyle=\color{gray},
  stringstyle=\color{teal},
  breaklines=true,
  breakatwhitespace=false,
  postbreak=\mbox{\textcolor{gray}{$\hookrightarrow$}\space},
  showstringspaces=false,
  frame=single,
  caption={The Monolith Script in All Its Glory},
  label={lst:monolith},
  numbers=left,
  numberstyle=\tiny,
  language=Python
}

\begin{lstlisting}[language=Python]
# the_monolith.py

import os
import re
import pickle
import pandas as pd
from sklearn.linear_model import LinearRegression
from sklearn.model_selection import train_test_split

# Step 1: Parse logs (from a hardcoded path)
log_path = "/mnt/interns_box_of_mystery/logs/training_data.log"
with open(log_path, "r") as file:
    lines = file.readlines()

# Step 2: Extract data using regex and vibes
data = []
for line in lines:
    match = re.match(r".*INFO\s+\|\s+feature1=(\d+\.\d+)\s+feature2=(\d+\.\d+)\s+target=(\d+\.\d+)", line)
    if match:
        data.append(tuple(map(float, match.groups())))

df = pd.DataFrame(data, columns=["feature1", "feature2", "target"])

# Step 3: Clean data using chained one-liners
df = df.dropna().reset_index(drop=True)  # works... until it doesn't

# Step 4: Train model (default linear regression)
X = df[["feature1", "feature2"]]
y = df["target"]
X_train, X_test, y_train, y_test = train_test_split(X, y, test_size=0.2)

model = LinearRegression()
model.fit(X_train, y_train)

# Step 5: Save model with pickle (because why not)
with open("model.pkl", "wb") as f:
    pickle.dump(model, f)

# Step 6: Deploy it via SCP (hardcoded credentials? you bet)
os.system("scp model.pkl prod-server:/var/www/html/models/model.pkl")

print("Model deployed to production. Nothing could possibly go wrong.")
\end{lstlisting}


It pulled packages from four different ecosystems, redefined variables mid-loop, and had logging statements that doubled as performance metrics. It learned, it inferred, it emailed the results to itself at 3am.

There were no functions... only faith.  
Tehre was no error handling... only optimism.

The monolith had been born.

It was ugly. It was fragile. It was unreadable to everyone, including the intern who wrote it.

But—miraculously—it worked.

\begin{quote}
And for one brief, cursed moment, the system was in balance.
\end{quote}



\vspace{1em}
\subsubsection{Act III: The Fall}

Weeks passed.

No one touched the script. No one asked about the model.

It sat in its digital corner, whirring quietly inside a Docker container no one dared to rebuild, delivering predictions with the eerie consistency of a haunted Roomba—faithful, silent, and faintly menacing.

Until, of course, the day it didn’t.

It was a Friday afternoon. The kind of Friday afternoon when half the team had already mentally clocked out and the other half was pretending to write documentation.

Then it happened.

\textbf{The logs changed.}

Not a dramatic change—just enough to be invisible to humans, but fatal to regex. A new column had quietly appeared, unnamed and smug. One old column, long relied upon, was gone without a trace. The date format had subtly shifted—no longer slashes, but dashes. Or was it the timezone? No one knew. The changes crept in like code gremlins, tiptoeing past CI pipelines.

And the script—poor, unsuspecting monolith that it was—choked.

The model broke.

The service failed.

Metrics flatlined. Dashboards turned a shade redder than anyone was comfortable with.

PagerDuty screamed into the void.

On-call phones vibrated on desks and nightstands with all the fury of a thousand missed deadlines. Slack channels lit up like a Christmas tree in a lightning storm. Messages began with \texttt{"Hey..."} and ended with \texttt{"@here."}

The intern, who had last touched the code three rotations ago, was summoned like a wizard in exile. Somewhere, their laptop opened slowly.

A senior SRE, already running on two Red Bulls and a personal grudge against YAML, was seen storming through Grafana panels, muttering ancient incantations like \texttt{"rollback,"} \texttt{"rollback now,"} and \texttt{"who approved this?"}

The autopsy hadn’t started, but the blame game had.

And deep in a log folder, timestamped exactly 15 minutes before failure, the script had left one final message:

\texttt{INFO | Prediction succeeded. target=4.0}

Its last words.

\begin{lstlisting}[caption={Slack transcript, 2:43 PM on a Friday}, label={lst:slackpanic}, basicstyle=\ttfamily\small, frame=single]
#sre-oncall

[2:43 PM] @alertbot: [ALERT] PROD Model Service Failure - HTTP 500s detected
[2:44 PM] @senior_sre: who owns this??
[2:44 PM] @eng_manager: wasn't this the intern's thing? 
[2:45 PM] @intern: hi yes uh give me 5 min
[2:46 PM] @senior_sre: what changed?
[2:46 PM] @intern: the logs have... evolved?
[2:47 PM] @intern: feature2 is now feature3
[2:47 PM] @intern: also the timestamps are in ISO format? I think?
[2:48 PM] @senior_sre: where's the source? 
[2:48 PM] @intern: it is not in Git
[2:48 PM] @senior_sre:  what
[2:49 PM] @intern: I think I SCPed it from my VM last summer?
[2:49 PM] @eng_manager: do we have any docs?
[2:49 PM] @intern: there was a notebook... on my old laptop.
[2:50 PM] @alertbot: [ALERT] PROD Model Service Crash Loop Restarting 
[2:51 PM] @senior_sre: this is why we don't YOLO deploy models 
[2:51 PM] @intern: understood
[2:52 PM] @eng_manager: action item: rewrite everything
\end{lstlisting}


They searched for the source code.

Desperation mounting, they combed through old repos, abandoned branches, shared drives with names like \texttt{archive\_final\_bkp\_DO\_NOT\_DELETE}. They unzipped tarballs nested within tarballs like digital matryoshka dolls. They even searched Confluence.

They found nothing.

No Git history. No documentation. Not even a rogue Jupyter notebook half-filled with markdown and regret.

Just a single, mysterious artifact sitting in a forgotten S3 bucket, untouched for months:  
\texttt{final\_model\_v2\_new\_new.pkl}

The filename inspired no confidence. It had clearly been renamed at least three times. Possibly by committee.

No one knew how it had been trained.  
No one knew what data it had seen.  
No one knew what “new\_new” meant.

No random seed. No environment file. No config. Just a pickle file — full of secrets.

The only logs available were the ones that had just broken.

They hadn’t deployed a model.  
They had released an unsupervised cryptid into production.

An eldritch pipeline, stitched together by interns long since graduated, silently shaping the fate of a product roadmap.

And now it was angry.

\begin{lstlisting}[caption={The log that broke the build}, label={lst:falllogs}, basicstyle=\ttfamily\small, frame=single]
2020-01-03 10:15:36 INFO  | feature1=13.0 feature2=8.1 target=4.1

% One week later...
2020-01-10 10:15:36 INFO  | feature1=13.0 feature3=foo status=OK timestamp=2020-01-10T10:15:36Z
\end{lstlisting}

\subsubsection{Act IV: The Redemption}

Eventually, the ghost was found.

Not quickly, and not cleanly, but found---buried in an old user directory, hidden behind a filename that looked like keyboard spam and timestamped in a timezone no one could identify.

The model was reverse-engineered.

The service was patched.

A new script---slightly more modular, slightly more commented---was written under duress.

\begin{lstlisting}[caption={The patched script, written under duress}, label={lst:patchscript}, basicstyle=\ttfamily\small, frame=single]
# semi_structured_pipeline.py

import os
import re
import pickle
import pandas as pd
from sklearn.linear_model import LinearRegression
from sklearn.model_selection import train_test_split

LOG_PATH = "/mnt/data_logs/new_logs_2020b/"  # not hardcoded, just... less hardcoded
MODEL_OUTPUT = "final_model_v3_recovery.pkl"

def parse_log_line(line):
    match = re.match(
        r".*INFO\s+\|\s+feature1=(\d+\.\d+)\s+(?:feature2|feature3)=(\d+\.\d+)\s+target=(\d+\.\d+)",
        line
    )
    return tuple(map(float, match.groups())) if match else None

def load_data_from_logs(log_path):
    entries = []
    for filename in os.listdir(log_path):
        if filename.endswith(".log"):
            with open(os.path.join(log_path, filename), "r") as file:
                for line in file:
                    parsed = parse_log_line(line)
                    if parsed:
                        entries.append(parsed)
    return pd.DataFrame(entries, columns=["feature1", "featureX", "target"])

def train_model(df):
    df = df.dropna().reset_index(drop=True)
    X = df[["feature1", "featureX"]]
    y = df["target"]
    X_train, X_test, y_train, y_test = train_test_split(X, y, test_size=0.2)
    model = LinearRegression()
    model.fit(X_train, y_train)
    return model

def save_model(model, path):
    with open(path, "wb") as f:
        pickle.dump(model, f)

def main():
    print("[INFO] Recovery Mode Initiated")
    df = load_data_from_logs(LOG_PATH)
    if df.empty:
        print("[ERROR] No usable data found. Aborting.")
        return
    model = train_model(df)
    save_model(model, MODEL_OUTPUT)
    print("[INFO] Model saved to", MODEL_OUTPUT)

if __name__ == "__main__":
    main()
\end{lstlisting}


Postmortems were held.

Slides were made.

The phrase “we should really build a pipeline for this” was said no fewer than five times, then quietly ignored.

The intern lived to tell the tale.

And, somehow, was offered a return offer.

Because deep down, everyone knew the truth:

\begin{quote}
The problem wasn’t the intern.  It was the system that let a one-off script become infrastructure.  (Also, the intern was the only one who remembered how \texttt{v2FlagTemp} was derived. So they kind of had to.)
\end{quote}

\subsubsection{Act V: The Pipeline That Could Have Been}

In a better world—one with deadlines that flex and teams that plan—this story could’ve gone very differently.

Instead of a one-shot pickle file hurled into prod like a cursed artifact, they could have built a proper pipeline.

Not a monolith. Not a notebook duct-taped into a cron job. A real, honest-to-goodness data pipeline. One with stages. One with structure. One that didn’t cause PagerDuty to cry in the middle of a sprint demo.

\textbf{Enter: Airflow.}

With Airflow, each step of the intern’s unholy monolith could’ve been a task in a Directed Acyclic Graph (DAG). Each task could’ve had retries. Timeouts. Logging. Alerts that didn’t rely on Slack sleuthing.

\begin{itemize}
  \item A task for loading logs (with schema checks, no less!).
  \item A task for parsing and validating those logs (with actual error handling!).
  \item A task for cleaning and transforming the data (instead of blindly \texttt{dropna()}-ing your sins away).
  \item A task for training the model (with versioning, reproducibility, and metrics).
  \item And a final task for deployment (ideally, not with \texttt{scp}).
\end{itemize}

Each step modular. Each failure traceable. Each retry logged, monitored, and contained.

\begin{figure}[H]
    \centering
    \begin{tikzpicture}[
      node distance=1.1cm and 1.8cm,
      dagnode/.style={
        draw, rounded corners, minimum width=2.4cm, minimum height=0.8cm,
        font=\scriptsize, align=center, fill=blue!5
      },
      errnode/.style={
        draw, dashed, minimum width=2.4cm, minimum height=0.8cm,
        font=\scriptsize, align=center, fill=red!10
      },
      groupbox/.style={
        draw, rounded corners, thick, inner sep=0.5em, fill=blue!2!white
      },
      arrow/.style={->, thick}
    ]
    
    % Main flow nodes
    \node[dagnode, fill=gray!10] (start) {Start};
    \node[dagnode, below=of start] (load) {Load Logs};
    \node[dagnode, below=of load] (validate) {Validate Schema};
    \node[dagnode, below=of validate] (parse) {Parse Logs};
    \node[dagnode, below=of parse] (clean) {Clean Data};
    \node[dagnode, below=of clean] (train) {Train Model};
    \node[dagnode, below=of train] (evaluate) {Evaluate};
    \node[dagnode, below=of evaluate] (deploy) {Deploy Model};
    \node[dagnode, fill=gray!10, below=of deploy] (end) {End};
    
    % Error handler node
    \node[errnode, right=4.5cm of clean] (error) {Handle Error\\ (TriggerRule.ONE\_FAILED)};
    
    % Group box in the background layer
    \begin{pgfonlayer}{background}
      \node[groupbox, fit=(load)(validate)(parse)(clean)(train)(evaluate)(deploy), label={[align=center]above:\textbf{Execution Block}}] (taskgroup) {};
    \end{pgfonlayer}
    
    % Main path arrows
    \draw[arrow] (start) -- (load);
    \draw[arrow] (load) -- (validate);
    \draw[arrow] (validate) -- (parse);
    \draw[arrow] (parse) -- (clean);
    \draw[arrow] (clean) -- (train);
    \draw[arrow] (train) -- (evaluate);
    \draw[arrow] (evaluate) -- (deploy);
    \draw[arrow] (deploy) -- (end);
    
    % Shared error handler arrow
    \draw[arrow, dashed] (taskgroup.east) -- (error.west);
    
    \end{tikzpicture}
    \caption{Airflow DAG with a linear execution block and a shared error handler.}
    \label{fig:airflowdag_grouped}
\end{figure}
    
    
    


Instead of a haunted Roomba running in silence, it could’ve been a well-lit assembly line with guardrails, dashboards, and dignity.

Better yet, Airflow could’ve scheduled the job, rather than depending on a Bash script tied to someone’s user crontab. If the logs changed? The schema check would’ve failed. The DAG would’ve paused. Someone would’ve been notified before the VP’s dashboard went full red alert.

And most importantly—every DAG run would’ve left breadcrumbs: metadata, timestamps, artifact hashes. Something future interns could follow without needing a séance.

\begin{itemize}
    \item \textbf{Functions?} Written once, reused across tasks.  
    \item \textbf{Secrets?} Stored in a vault, not hardcoded in a script named \texttt{lolmodel.py}.  
    \item \textbf{Logs?} Structured, queryable, archived.  
    \item \textbf{Documentation?} Okay, maybe that’s still wishful thinking. But everything else? Achievable.
\end{itemize}

Sure, it might’ve taken an extra week to set up.  Sure, it wouldn’t fit in a single slide.  But it would’ve survived the reorg.  It would’ve survived Friday.

And no one would’ve had to grep for salvation.


\begin{lstlisting}[caption={Load logs from a directory.}, label={lst:load_logs}]
    def load_logs(log_dir: str) -> List[str]:
        import os
        log_lines = []
        for filename in os.listdir(log_dir):
            if filename.endswith(".log"):
                with open(os.path.join(log_dir, filename), "r") as f:
                    log_lines.extend(f.readlines())
        if not log_lines:
            raise FileNotFoundError("No logs found in directory.")
        return log_lines
\end{lstlisting}
    

\begin{lstlisting}[caption={Parse structured fields from log lines.}, label={lst:parse_logs}]
    def parse_logs(lines: List[str]) -> List[Tuple[float, float, float]]:
        import re
        parsed = []
        for line in lines:
            match = re.search(r"feature1=(\d+\.\d+).*?(feature2|feature3)=(\d+\.\d+).*?target=(\d+\.\d+)", line)
            if match:
                f1 = float(match.group(1))
                fx = float(match.group(3))
                target = float(match.group(4))
                parsed.append((f1, fx, target))
            else:
                print(f"[WARN] Failed to parse line: {line.strip()}")
        return parsed
\end{lstlisting}



\begin{lstlisting}[caption={Clean and standardize the parsed DataFrame.}, label={lst:clean_data}]
    def clean_data(df: pd.DataFrame) -> pd.DataFrame:
        df = df.rename(columns={"feature2": "featureX", "feature3": "featureX"})
        df = df.dropna().reset_index(drop=True)
        if df.empty:
            raise ValueError("No clean data left after dropping nulls.")
        return df
\end{lstlisting}



\begin{lstlisting}[caption={Train a linear model using scikit-learn.}, label={lst:train_model}]
    def train_model(df: pd.DataFrame) -> LinearRegression:
        from sklearn.linear_model import LinearRegression
        from sklearn.model_selection import train_test_split
    
        X = df[["feature1", "featureX"]]
        y = df["target"]
        X_train, X_test, y_train, y_test = train_test_split(X, y, test_size=0.2)
    
        model = LinearRegression()
        model.fit(X_train, y_train)
        return model
\end{lstlisting}


\begin{lstlisting}[caption={Deploy the trained model by pickling it.}, label={lst:deploy_model}]
    def deploy_model(model, output_path="final_model_v4.pkl"):
        import pickle
        with open(output_path, "wb") as f:
            pickle.dump(model, f)
        print(f"[INFO] Model saved to {output_path}")
        # Simulate deployment
        # subprocess.run(["scp", output_path, "prod-server:/var/www/html/models/"])
\end{lstlisting}


\begin{lstlisting}[caption={Airflow DAG with linear tasks and error handling flow.}, label={lst:airflowdag_error_handling}]
    from airflow.decorators import dag, task
    from airflow.operators.empty import EmptyOperator
    from airflow.operators.python import PythonOperator
    from airflow.utils.trigger_rule import TriggerRule
    from datetime import datetime
    
    @dag(schedule_interval="@daily", start_date=datetime(2023, 1, 1), catchup=False, tags=["modular_pipeline"])
    def resilient_model_pipeline():
    
        start = EmptyOperator(task_id="start")
        end = EmptyOperator(task_id="end")
    
        # Fallback error handler task
        error_handler = PythonOperator(
            task_id="handle_error",
            python_callable=lambda: print("[ERROR] Pipeline failure detected. Escalating."),
            trigger_rule=TriggerRule.ONE_FAILED
        )
    
        @task()
        def load_logs_task():
            return load_logs("/mnt/logs")
    
        @task()
        def validate_schema_task(logs):
            if not logs:
                raise ValueError("Log validation failed: Empty or missing.")
            return logs
    
        @task()
        def parse_logs_task(validated_logs):
            return parse_logs(validated_logs)
    
        @task()
        def clean_data_task(parsed):
            import pandas as pd
            df = pd.DataFrame(parsed, columns=["feature1", "featureX", "target"])
            return clean_data(df)
    
        @task()
        def train_model_task(df):
            return train_model(df)
    
        @task()
        def evaluate_model(model):
            print("[INFO] Evaluating model...")
            return "evaluation complete"
    
        @task()
        def deploy_model_task(eval_result, model):
            deploy_model(model)
    
        # DAG wiring
        logs = load_logs_task()
        validated = validate_schema_task(logs)
        parsed = parse_logs_task(validated)
        cleaned = clean_data_task(parsed)
        model = train_model_task(cleaned)
        eval_result = evaluate_model(model)
        deploy = deploy_model_task(eval_result, model)
    
        # Dependencies
        start >> logs >> validated >> parsed >> cleaned >> model >> eval_result >> deploy >> end
    
        # Attach error handler to all main tasks (except start/end)
        for t in [logs, validated, parsed, cleaned, model, eval_result, deploy]:
            t >> error_handler
    
    dag = resilient_model_pipeline()
    \end{lstlisting}





\subsubsection{version 2}



\begin{figure}[H]
    \centering
    \begin{tikzpicture}[
      node distance=1.1cm and 1.8cm,
      dagnode/.style={
        draw, rounded corners, minimum width=2.6cm, minimum height=0.8cm,
        font=\scriptsize, align=center, fill=blue!5
      },
      errnode/.style={
        draw, dashed, minimum width=2.6cm, minimum height=0.8cm,
        font=\scriptsize, align=center, fill=red!10
      },
      groupbox/.style={
        draw, rounded corners, thick, inner sep=0.5em, fill=blue!2!white
      },
      arrow/.style={->, thick}
    ]
    
    % Main flow
    \node[dagnode, fill=gray!10] (start) {Start};
    \node[dagnode, below=of start] (load) {Load Logs};
    \node[dagnode, below=of load] (validate) {Validate Schema};
    \node[dagnode, below=of validate] (parse) {Parse Logs};
    
    % Parallel tasks
    \node[dagnode, below left=1.2cm and 1.5cm of parse] (quality) {Validate Quality};
    \node[dagnode, below right=1.2cm and 1.5cm of parse] (features) {Feature Engineering};
    
    % Join and continue
    \node[dagnode, below=1.8cm of parse] (train) {Train Model};
    \node[dagnode, below=of train] (evaluate) {Evaluate};
    \node[dagnode, below=of evaluate] (deploy) {Deploy Model};
    \node[dagnode, fill=gray!10, below=of deploy] (end) {End};
    
    % Error handler
    \node[errnode, right=5.3cm of train] (error) {Handle Error\\ (TriggerRule.ONE\_FAILED)};
    
    % Group box in background
    \begin{pgfonlayer}{background}
      \node[groupbox, fit=(load)(validate)(parse)(quality)(features)(train)(evaluate)(deploy), 
            label={[align=center]above:\textbf{Execution Block}}] (taskgroup) {};
    \end{pgfonlayer}
    
    % Main arrows
    \draw[arrow] (start) -- (load);
    \draw[arrow] (load) -- (validate);
    \draw[arrow] (validate) -- (parse);
    \draw[arrow] (parse) -- (quality);
    \draw[arrow] (parse) -- (features);
    \draw[arrow] (quality) -- (train);
    \draw[arrow] (features) -- (train);
    \draw[arrow] (train) -- (evaluate);
    \draw[arrow] (evaluate) -- (deploy);
    \draw[arrow] (deploy) -- (end);
    
    % Error path from group
    \draw[arrow, dashed] (taskgroup.east) -- (error.west);
    
    \end{tikzpicture}
    \caption{Airflow DAG with parallel validation and feature engineering inside the execution block, and a shared error handler.}
    \label{fig:airflowdag_parallel}
\end{figure}




\begin{lstlisting}[caption={Airflow DAG with parallel data quality validation and feature engineering before training.}, label={lst:airflowdag_parallel}]
    from airflow.decorators import dag, task
    from airflow.operators.empty import EmptyOperator
    from airflow.operators.python import PythonOperator
    from airflow.utils.trigger_rule import TriggerRule
    from datetime import datetime
    
    @dag(schedule_interval="@daily", start_date=datetime(2023, 1, 1), catchup=False, tags=["modular_pipeline"])
    def enhanced_pipeline():
    
        start = EmptyOperator(task_id="start")
        end = EmptyOperator(task_id="end")
    
        # Shared error handler
        error_handler = PythonOperator(
            task_id="handle_error",
            python_callable=lambda: print("[ERROR] Pipeline failure detected."),
            trigger_rule=TriggerRule.ONE_FAILED
        )
    
        @task()
        def load_logs_task():
            return load_logs("/mnt/logs")
    
        @task()
        def validate_schema_task(logs):
            if not logs:
                raise ValueError("Log validation failed: Empty or missing.")
            return logs
    
        @task()
        def parse_logs_task(validated_logs):
            return parse_logs(validated_logs)
    
        @task()
        def validate_quality_task(parsed_data):
            # e.g., check missing values, outliers, etc.
            print("Validating data quality...")
            return parsed_data
    
        @task()
        def generate_features_task(parsed_data):
            # e.g., feature transformation, normalization
            print("Engineering features...")
            return parsed_data
    
        @task()
        def train_model_task(inputs1, inputs2):
            import pandas as pd
            # Merge cleaned inputs
            df = pd.DataFrame(inputs1, columns=["feature1", "featureX", "target"])
            return train_model(df)
    
        @task()
        def evaluate_model(model):
            print("[INFO] Evaluating model...")
            return "evaluation complete"
    
        @task()
        def deploy_model_task(eval_result, model):
            deploy_model(model)
    
        # Wiring
        logs = load_logs_task()
        validated = validate_schema_task(logs)
        parsed = parse_logs_task(validated)
    
        quality = validate_quality_task(parsed)
        features = generate_features_task(parsed)
    
        model = train_model_task(quality, features)
        eval_result = evaluate_model(model)
        deploy = deploy_model_task(eval_result, model)
    
        start >> logs >> validated >> parsed
        parsed >> [quality, features]
        [quality, features] >> model >> eval_result >> deploy >> end
    
        for t in [logs, validated, parsed, quality, features, model, eval_result, deploy]:
            t >> error_handler
    
    dag = enhanced_pipeline()
\end{lstlisting}

    


\subsubsection{version 3}


\begin{lstlisting}[caption={Airflow DAG with parallel data quality validation and feature engineering before training.}, label={lst:airflowdag_parallel}]
from airflow.decorators import dag
from airflow.operators.empty import EmptyOperator
from airflow.operators.python import PythonOperator
from airflow.utils.trigger_rule import TriggerRule
from datetime import datetime

from tasks.load_logs import load_logs_task
from tasks.validate_schema import validate_schema_task
from tasks.parse_logs import parse_logs_task
from tasks.validate_quality import validate_quality_task
from tasks.generate_features import generate_features_task
from tasks.train_model import train_model_task
from tasks.evaluate_model import evaluate_model_task
from tasks.deploy_model import deploy_model_task

@dag(schedule_interval="@daily", start_date=datetime(2023, 1, 1), catchup=False, tags=["modular_pipeline"])
def enhanced_pipeline():
    start = EmptyOperator(task_id="start")
    end = EmptyOperator(task_id="end")

    error_handler = PythonOperator(
        task_id="handle_error",
        python_callable=lambda: print("[ERROR] Pipeline failure detected."),
        trigger_rule=TriggerRule.ONE_FAILED
    )

    logs = load_logs_task()
    validated = validate_schema_task(logs)
    parsed = parse_logs_task(validated)

    quality = validate_quality_task(parsed)
    features = generate_features_task(parsed)

    model = train_model_task(quality, features)
    eval_result = evaluate_model_task(model)
    deploy = deploy_model_task(eval_result, model)

    start >> logs >> validated >> parsed
    parsed >> [quality, features]
    [quality, features] >> model >> eval_result >> deploy >> end

    for t in [logs, validated, parsed, quality, features, model, eval_result, deploy]:
        t >> error_handler

dag = enhanced_pipeline()
\end{lstlisting}




\lstdefinestyle{tree}{
  basicstyle=\ttfamily\small,
  backgroundcolor=\color{gray!5},
  frame=single,
  columns=fullflexible,
  showstringspaces=false
}

\begin{lstlisting}[style=tree, caption={Modular DAG file structure (ASCII-safe)}, label={lst:dag_tree_ascii}]
    dags/
    |- enhanced_pipeline_dag.py         # DAG definition & task wiring
    |- tasks/
       |- __init__.py
       |- load_logs.py
       |- validate_schema.py
       |- parse_logs.py
       |- validate_quality.py
       |- generate_features.py
       |- train_model.py
       |- evaluate_model.py
       |- deploy_model.py
\end{lstlisting}



\subsubsection{version 4}

\begin{lstlisting}[style=tree, caption={Modular DAG file structure (ASCII-safe)}, label={lst:dag_tree_ascii}]
    dags/
    |- enhanced_pipeline/
    |  |- __init__.py
    |  |- enhanced_pipeline_dag.py         # DAG definition + @dag wrapper
    |  |- task_factory.py                  # contains create_tasks()
    |  |- task_wiring.py                   # contains wire_tasks()
    |  |- orchestrator.py                  # contains orchestrate_dag()
    |
    |- tasks/
    |  |- __init__.py
    |  |- load_logs.py
    |  |- validate_schema.py
    |  |- parse_logs.py
    |  |- validate_quality.py
    |  |- generate_features.py
    |  |- train_model.py
    |  |- evaluate_model.py
    |  |- deploy_model.py
    |
    |- utils/
       |- __init__.py
       |- io.py                            # e.g., load_logs()
       |- ml.py                            # e.g., train_model(), etc.
    
\end{lstlisting}

\begin{lstlisting}[caption={Airflow DAG with parallel data quality validation and feature engineering before training.}, label={lst:airflowdag_parallel}]
    from airflow.decorators import dag
    from airflow.operators.empty import EmptyOperator
    from airflow.operators.python import PythonOperator
    from airflow.utils.trigger_rule import TriggerRule
    from datetime import datetime
    
    from tasks.load_logs import load_logs_task
    from tasks.validate_schema import validate_schema_task
    from tasks.parse_logs import parse_logs_task
    from tasks.validate_quality import validate_quality_task
    from tasks.generate_features import generate_features_task
    from tasks.train_model import train_model_task
    from tasks.evaluate_model import evaluate_model_task
    from tasks.deploy_model import deploy_model_task
    
    
    def create_tasks():
        """Return a dictionary of all tasks with dependencies left unwired."""
        start = EmptyOperator(task_id="start")
        end = EmptyOperator(task_id="end")
    
        error_handler = PythonOperator(
            task_id="handle_error",
            python_callable=lambda: print("[ERROR] Pipeline failure detected."),
            trigger_rule=TriggerRule.ONE_FAILED
        )
    
        logs = load_logs_task()
        validated = validate_schema_task(logs)
        parsed = parse_logs_task(validated)
    
        quality = validate_quality_task(parsed)
        features = generate_features_task(parsed)
    
        model = train_model_task(quality, features)
        eval_result = evaluate_model_task(model)
        deploy = deploy_model_task(eval_result, model)
    
        return {
            "start": start,
            "end": end,
            "error_handler": error_handler,
            "logs": logs,
            "validated": validated,
            "parsed": parsed,
            "quality": quality,
            "features": features,
            "model": model,
            "eval_result": eval_result,
            "deploy": deploy
        }
    
    
    def wire_tasks(t):
        """Connect tasks in the desired DAG structure."""
        t["start"] >> t["logs"] >> t["validated"] >> t["parsed"]
        t["parsed"] >> [t["quality"], t["features"]]
        [t["quality"], t["features"]] >> t["model"] >> t["eval_result"] >> t["deploy"] >> t["end"]
    
        for task in [
            t["logs"], t["validated"], t["parsed"],
            t["quality"], t["features"],
            t["model"], t["eval_result"], t["deploy"]
        ]:
            task >> t["error_handler"]
    
    
    def orchestrate_dag():
        """High-level orchestration of task creation and wiring."""
        tasks = create_tasks()
        wire_tasks(tasks)
    
    
    @dag(schedule_interval="@daily", start_date=datetime(2023, 1, 1), catchup=False, tags=["modular_pipeline"])
    def enhanced_pipeline():
        orchestrate_dag()
    
    
dag = enhanced_pipeline()
\end{lstlisting}
    
\begin{table}[H]
    \centering
    \begin{tabular}{|l|p{10cm}|}
    \hline
    \textbf{File} & \textbf{Responsibility} \\
    \hline
    \texttt{enhanced\_pipeline\_dag.py} & Contains \texttt{@dag} decorator and calls \texttt{orchestrate\_dag()} \\
    \hline
    \texttt{task\_factory.py} & Defines and returns all task objects (\texttt{create\_tasks()}) \\
    \hline
    \texttt{task\_wiring.py} & Wires the task dependencies (\texttt{wire\_tasks(tasks)}) \\
    \hline
    \texttt{orchestrator.py} & Calls both functions (\texttt{orchestrate\_dag()}) \\
    \hline
    \texttt{tasks/*.py} & Each contains a single \texttt{@task} Airflow function \\
    \hline
    \texttt{utils/io.py} & Low-level data access helpers (e.g., log loading) \\
    \hline
    \texttt{utils/ml.py} & Model logic, feature engineering, and evaluation helpers \\
    \hline
    \end{tabular}
    \caption{Modular DAG code responsibilities by file}
    \label{tab:dag_structure}
\end{table}
    

    





\subsubsection*{Epilogue: Why Modularity Matters}

Let’s be honest.

This story didn’t end in disaster because the intern was reckless.  It ended in disaster because the system had no guardrails.

When a one-off training script, duct-taped together with assumptions, globals, and good intentions, is all it takes to deploy to production, the issue isn’t code quality---it’s architecture.

Modularity isn’t just a buzzword consultants put in slide decks.  It’s a survival strategy.

If the ingestion logic had been its own component---versioned, tested, isolated---a format change wouldn’t have brought down the system.\\
If training and deployment had been separate stages with inputs, outputs, and contracts, we wouldn’t have been staring at a mystery \texttt{.pkl} file like it was an alien artifact.

Modularity gives you reproducibility.  It gives you traceability. It lets you swap a broken piece without disassembling the entire machine.

Without it?

You get fragile scripts, ghost models, and institutional knowledge stored in a single intern’s memory.

Which is fine\ldots until they graduate.


\vspace{1em}
\textbf{A Visual, in Your Mind}

Picture a Jenga tower.

A modular system lets you pull out one block and put in a better one.\\
A monolithic system? One wrong move, and the whole thing comes crashing down.

Your move.









\end{document}
